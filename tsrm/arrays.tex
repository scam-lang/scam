\chapter{Arrays}
\label{Arrays}

In this chapter, we will study {\it arrays}, a device
used for bringing related bits of data together underneath one
roof, so to speak. Such a device is known as a data structure;
an array is one of the most basic of data structures.

Unfortunately, arrays are not built in to Scam and
third party array modules are practically or semantically 
deficient. Therefore,
this chapter is based on a custom array module you can
download (sorry, Linux only).

\section{Getting the array and list modules}

If you are running Linux on a 32-bit Intel machine,
run these commands to download the modules:

\begin{verbatim}
    wget beastie.cs.ua.edu/cs150/array-i386/nakarray.so
\end{verbatim}

If you are running Linux on a 64-bit machine, use this
command instead:

\begin{verbatim}
    wget beastie.cs.ua.edu/cs150/array-amd64/nakarray.so
\end{verbatim}

Place the {\it nakarray.so} file
in the same directory as
your program that uses arrays. One would
import the array module
with the following line:

\begin{verbatim}
    from nakarray import *
\end{verbatim}

\section{Arrays as data structures}

A {\it data}
{\it structure} is simply a collection of
bits of information\footnote{
Bits in the informal sense, not zeros and ones.
}
that are
somehow glued together into a single whole. Each of these bits
can be be accessed individually. Usually, the bits are somehow
related, so the data structure is a convenient way
to keep all these related bits nicely packaged together.

At its most basic, a data structure supports the following actions:

\begin{itemize}
\item
       creating the structure
\item
       putting data into the structure
\item
       taking data out of the structure
\end{itemize}

We will study these actions with an eye to how long each
action is expected to take.

An array is a data structure with the property that each
individual piece of data can be accessed just as quickly as any of the
others. In other words, the process of putting data in and
taking data out takes a constant amount of time regardless of
the size of the array and where the data being accessed is
placed in the array\footnote{
Not all data structures have this
property (but they do have other advantages over arrays.}.

\section{Array creation}

To create an array, one uses the {\it Array} function:

\begin{verbatim}
    a = Array(10)
\end{verbatim}

The function call \verb!Array(10)!
creates an array with room for ten items or {\it elements}.
The variable {\it a} is created and set to point to
the array.

In many languages, array creation can take anywhere from {\it constant} time
(if elements are not initialized) or {\it linear} time (if they are).
What do we mean by constant time? We mean, in this particular
case, the amount of time it takes to create an array of 10 elements
takes the same amount of time it takes to create an array of 
1,000,000 elements. In other words, the time it takes to create an
array is independent of the size of the array.
By the same token, linear time means the time it takes to create
an array is proportional to the size of the array. If array creation
takes linear time, we would expect the time it takes to create
a 1,000,000 element array would be, roughly, 100,000,000 times
longer than the time to create a 10 element array. 

At the end of the chapter, we will attempt to figure out whether
array creation takes constant or linear time.

\section{Setting and getting array elements}

Most data structures allow you to both add and remove data.
For arrays, adding data means to {\it set} the value of an array element,
while removing data means to {\it get} the value of an array element.
It does not mean that elements themselves are being created and
destroyed (although this can be the case in other data structures).

To set an individual element, one uses {\it square bracket notation}.
For example, here's how to set the
first item in the array

\begin{verbatim}
    >>> a[0] = 42
\end{verbatim}

The number between the square brackets is known as an
{\it index}.
As with Scam lists, note that the index of the first element is zero.
This means
that the index
of the second element is one, and so on.
In other words, zero-based counting is used to
number array elements.

Once an element has been set, you can retrieve it from the array
using a similar notation. The following code illustrates the
retrieval of the first (index zero) element.

\begin{verbatim}
    >>> e = a[0]

    >>> e
    42
\end{verbatim}

Of course, there is no need to assign the value of an array
element to a variable; you can access it directly:

\begin{verbatim}
    >>> a[0]
    42
\end{verbatim}

The elements of an array are initialized to \verb!None!, so if you
access an element that hasn't been set, you get \verb!None! as a
result:

\begin{verbatim}
    >>> print(a[1])
    None

    >>>print(a)
    [42, None, None, None, None, None, None, None, None, None]
\end{verbatim}

To find out the number of elements in an array, you can use
the {\it len} function:

\begin{verbatim}
    >>> len(a)
    10
\end{verbatim}

Because of zero-based counting, the index of the last element
is always one less the number of elements in the array. For
a ten element array then, the last legal index is nine:

\begin{verbatim}
    >>> a[9] = 13

    >>>print(a)
    [42, None, None, None, None, None, None, None, None, 13]
\end{verbatim}

Trying to access an element beyond the $9^{th}$ results in
an error:

\begin{verbatim}
    >>> a[10] = 99
    Traceback (most recent call last):
      File "<stdin>", line 1, in <module>
      IndexError: Index '10' is out of range.
\end{verbatim}

\section{Limitations on our arrays}

Unlike built-in Scam strings and lists, the {\it nakarray} module
does not allow:

\begin{itemize}
\item
        `slicing' an array.
\item
        negative indices
\item
        concatenation two arrays using the plus operator
\end{itemize}

The reasons for these restrictions is that other common languages
do not support these features. In those languages, you need
to implement these features yourself. The superior student
will attempt to add these features by defining special purpose
functions for our arrays.

\section{Advantages of our arrays}

The array module that comes with the
Scam distribution has two major disadvantages:

\begin{itemize}
\item
    you cannot create an array of a given size - you must
    instead create an empty array and then repeatedly append
    elements to the array.
\item
    you must specify the type of element to  be stored in
    the array at array creation time - such arrays are known
    as {\it homogeneous} arrays.
\end{itemize}
        
As seen in the previous sections,
arrays made with the {it nakarray} can
be made of any size quite easily.
As to the second disadvantage,
we feel that arrays, 
like Scam lists, should be
{\it heterogeneous}.
Homogeneous, in Computer Science speak, means ``all of one type''
so a homogeneous array can only hold all integers or all strings,
but not both. Heterogeneous arrays can hold a mixture of types.

\section{The utility of arrays}

Due to their constant time properties, arrays are
a useful data structure for many applications involving
large amounts of data. One of the most useful applications
is searching for presence of certain elements in an array.
The next chapter discusses how arrays can make this
search process quite efficient.
