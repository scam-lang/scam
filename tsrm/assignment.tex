\chapter{Assignment}
\label{Assignment}

Once a variable has been created, it is possible to change its value,
or {\it binding},
Consider the following interaction with
the Scam interpreter:

\begin{verbatim}
    (define eyeColor 'black)    ; creation
    
    (inspect eyeColor)          ; reference
    -> eyeColor is black
    
    (set! eyeColor 'green)      ; assignment
    
    (eq? eyeColor 'black)       ; equality
    -> #f
    
    (eq? eyeColor 'green)       ; equality
    -> #t

    (assign eyecolor BROWN)     ; assignment (alternate)
\end{verbatim}

The assignment function is not like the arithmetic operators.
Recall that {\tt +} evaluates all its arguments
before performing the addition.
For {\it set!} and {\it assign},
the leftmost operand is not evaluated:
If it were, the assignment

\begin{verbatim}
    (define x 1)
    (set! x 3)
\end{verbatim}
    
would be equivalent to:

\begin{verbatim}
    (set! 1 3)
\end{verbatim}

In general, an operator which does not evaluate
all its arguments is known as a {\it special form}\footnote{
In Scam, there are no special forms. As such, {\it assign} is
a true function and can be given a new meaning.}.
For {\it assign}, the evaluation of the first argument
is suppressed.

\section{Other functions for changing the value of a variable}

Scam has another functions for changing the value of a
varible:

\begin{verbatim}
    (set 'x 5)
\end{verbatim}

which changes the current value of {\it x} to 5.
It is equivalent to the {\it set!} function,
except that it evaluates all its arguments. This
is why the variable name was quoted in the example.
The reason for this behavior is that, sometimes,
it is useful to derive the variable name
to be modified programmatically. The {\it set} function
allows for this while the {\it set!} function does not.

\section{Assignment and Environments/Objects}

The assignment functions can take an environment as an
optional third argument.
Because the predefined variable
{\it this} always points to the current environment,
the following
four expressions are equivalent:

\begin{verbatim}
        (assign x 5)
        (assign x 5 this)
        (set! x 5 this)
        (set (symbol "x") 5 this)
\end{verbatim}

The {\it symbol} function is used to create a variable name from
a string.
Since environments form the basis for objects in Scam,
{\it assign}, {\it set!}, and {\it set} 
can be used to update the instance variables
of objects.

\section{Setting elements of a collection}
\label{ListsStringsArraysSetting}

The {\it set-car!} function can be used to set the first element of
a collection:

\begin{verbatim}
    (define a (list 3 5 7))

    (inspect a)
    -> a is (3 5 7)

    (set-car! a 11)

    (inspect a)
    -> a is (11 5 7)
\end{verbatim}

More generally, the {\it setElement} function can be used to set a new value
at any legal index. The first argument to {\it setElement} is the collection
to be modified, the second is in index at which to place the new value,
the third argument.
Indices are numbered using zero-based counting:

\begin{verbatim}
    (define a (list 3 5 7))
    (setElement a 1 44)         ;index 1 refers to the 2nd element
    (inspect a)
    -> a is (3 44 7)
\end{verbatim}

For strings, the new value must be a non-empty string. If the
value is composed of multiple characters, the characters after
the first character are ignored.

For lists, it is possible to set the tail of a list. 
The tail of a list is the list of all elements in the list 
except the first element. Here, we set the tail of list {\it a}
to be list {\it b}, using {\it set-cdr!}:

\begin{verbatim}
    (define a (list 1 2 3))
    (define b (list "two" "three" "four" "five"))

    (set-cdr! a b)
    
    (inspect a)
    -> a is (1 "two" "three" "four" "five")

    (set-cdr! (cdr (cdr a)) (list 6))

    (inspect a)
    -> a is (1 "two" "three" 6)
\end{verbatim}

The last two interactions show that any expression that resolves
to a list can be passed as the first and second arguments.
Note that you cannot set the tail of either an array or a string.
