\chapter{Assignment}
\label{Assignment}

Once a variable has been created, it is possible to change its value,
or {\it binding},
using the assignment operator. Consider the following interaction with
the interpreter:

\begin{verbatim}
    (define BLACK 1)            ; creation
    (define BROWN 2)            ; creation
    (define GREEN 3)            ; creation
    
    (define eyeColor BLACK)     ; creation
    
    (inspect eyeColor)          ; reference
    -> (eyeColor is 1)
    
    (assign eyeColor GREEN)     ; assignment
    
    (eq? eyeColor BLACK)        ; equality
    -> #f
    
    (== eyeColor BROWN)         ; equality (alternate)
    -> #f

    (eq? eyeColor GREEN)        ; equality
    -> #t

    (set! 'eyecolor BROWN)      ; assignment (alternate)
\end{verbatim}

The operator/variable {\tt\codesize =} (equals sign) is bound to the
{\it assignment} function.
The assignment function, however, is not like the operators
{\tt\codesize +} and {\tt\codesize *}.
Recall that {\tt\codesize +} and the like evaluate the things on either
side (recall that those things on either side are generically known as
operands) before combining them. For {\tt\codesize =},
the left operand is not evaluated:
(if it were, the assignment

\color{CodeGreen}
\begin{codesize}
\begin{verbatim}
    eyeColor = GREEN
\end{verbatim}
\end{codesize}
\color{black}
    
would attempt to assign the value of 1 to be 3.
In general, an operator which does not evaluate
all its arguments is known as a {\it special form}.

The last two expressions given to the interpreter in the previous
interaction refer to the {\tt\codesize ==} (equality) operator.
This {\tt\codesize ==} operator returns
true if its operands refer to the same thing and false otherwise.

Another thing to note in the above interaction is
that the
variables {\sf BLACK}, {\sf GREEN}, and {\sf BROWN}
are not meant
to change from their initial values.
We denote variables whose values aren't supposed to change
by naming the
variable using (mostly) capital letters (this convention is borrowed
from earlier programming languages).
The use of caps emphasizes the constant nature of the (not too) variable.

In the above interaction with the interpreter, we use the integers 1, 2,
and 3 to represent the colors black, brown, and green. By abstracting 1,
2, and 3 and giving them meaningful names (i.e., {\sf BLACK}, {\sf BROWN},
and {\sf GREEN}) we find it easy to read code that
assigns and tests eye color.
We do this because it is difficult to remember which integer
is assigned to which color. Without the variables {\sf BLACK}, {\sf BROWN}, and
{\sf GREEN}, we have to keep little notes somewhere to remind ourselves what's
what. Here is an equivalent interaction with the interpreter without
the use of the variables {\sf BLACK}, {\sf GREEN}, and {\sf BROWN}.

\color{CodeGreen}
\begin{codesize}
\begin{verbatim}
    >>> eyeColor = 1
    1
    
    >>> eyeColor
    1
    
    >>> eyeColor = 3
    3
    
    >>> eyeColor == 2
    False
    
    >>> eyeColor == 3
    True
\end{verbatim}
\end{codesize}
\color{black}
    
In this interaction, the meaning of {\it eyeColor}
is not so obvious. We know its a 3, but what eye color
does 3 represent? When numbers appear directly in code,
they are referred to as {\it magic numbers} because they
obviously mean something and serve some purpose,
but how they make the code work correctly
is not always readily apparent,
much like a magic trick.
Magic numbers are to be avoided. Using well-name constants
(or variables if constants are not part of the programming
langue) is considered stylistically superior.

\section{Precedence and Associativity of Assignment}

Assignment has the lowest precedence among the binary operators. It is
also right associative. The right associativity allows for statements like

\color{CodeGreen}
\begin{codesize}
\begin{verbatim}
    a = b = c = d = 0
\end{verbatim}
\end{codesize}
\color{black}

which conveniently assigns a zero to four variables at once and,
because of the right associative nature of the operator, is 
equivalent to:

\color{CodeGreen}
\begin{codesize}
\begin{verbatim}
    (a = (b = (c = (d = 0))))
\end{verbatim}
\end{codesize}
\color{black}

The resulting value of an assignment operation is the value assigned,
so the assignnment {\tt\codesize d == 0} returns 0, which is,
in turned,
assigned to
{\it c} and so on.

\section{Assignment Patterns}

The art of writing programs lies in the ability to
recognize and use patterns that have appeared since the
very first programs were written. In this text, we take
a pattern approach to teaching how to program. For the
topic at hand, we will give a number of patterns that
you should be able to recognize to use or avoid as
the case may be.

\subsection{The Transfer Pattern}

The {\it transfer} pattern is used to  change the value of a
variable based upon the value of another variable.
Suppose we have a variable named {\it alpha} which is initialized
to 3 and a variable {\it beta} which is initialized to 10:

\color{CodeGreen}
\begin{codesize}
\begin{verbatim}
    alpha = 3
    beta = 10
\end{verbatim}
\end{codesize}
\color{black}

Now consider the statement:

\color{CodeGreen}
\begin{codesize}
\begin{verbatim}
    alpha = beta
\end{verbatim}
\end{codesize}
\color{black}

This statement is read like this:
make the new value of {\it alpha} equal to the value
of {\it beta}, throwing away the old value of {\it alpha}.
What is the value of {\it alpha} after that statement
is executed?
\W Highlight the following line to see the answer:

\begin{quote}
    The new value of {\it alpha} is 
    {
    \T\color{black}
    \W\color{white}
    10
    }.
\end{quote}

The {\it transfer} pattern tells us that value of {\it beta} is
imprinted on {\it alpha} at the moment of assignment
but in no case are {\it alpha} and {\it beta} conjoined in anyway in
the future. Think of it this way. Suppose your friend
spray paints her bike neon green. You like the color
so much you spray paint your bike neon green as well.
This is like assignment: you made the value (color) of
your bike the same value (color) as your friend's bike.
Does this mean your bike and your friend's bike will
always have the same color forever? Suppose your friend
repaints her bike. Will your bike automatically become
the new color as well? Or suppose you repaint your bike.
Will your friend's bike automatically assume the color
of your bike?

To test your understanding, what happens if the following
code is executed:

\color{CodeGreen}
\begin{codesize}
\begin{verbatim}
    alpha = 4
    beta = 13
    alpha = beta
    beta = 5
\end{verbatim}
\end{codesize}
\color{black}

What are the final values of {\it alpha}  and {\it beta}?
\W Highlight the following line to see the answer:

\begin{quote}
    The value of {\it alpha} is 
    {
    \T\color{black}
    \W\color{white}
    13
    }
    and the value of {\it beta} is 
    {
    \T\color{black}
    \W\color{white}
    5
    }.
\end{quote}

To further test your understanding, what happens if the following
code is executed:

\color{CodeGreen}
\begin{codesize}
\begin{verbatim}
    alpha = 4
    beta = 13
    alpha = beta
    alpha = 42
\end{verbatim}
\end{codesize}
\color{black}

What are the final values of {\it alpha}  and {\it beta}?
\W Highlight the following line to see the answer:

\begin{quote}
    The value of {\it alpha} is 
    {
    \T\color{black}
    \W\color{white}
    42
    }
    and the value of {\it beta} is 
    {
    \T\color{black}
    \W\color{white}
    13
    }.
\end{quote}

\subsection{The Update Pattern}

The {\it update} pattern is used to  change the value of a
variable based upon the original value of the variable.
Suppose we have a variable named {\it counter} which is initialized
to zero:

\color{CodeGreen}
\begin{codesize}
\begin{verbatim}
    counter = 0
\end{verbatim}
\end{codesize}
\color{black}

Now consider the statement:

\color{CodeGreen}
\begin{codesize}
\begin{verbatim}
    counter = counter + 1
\end{verbatim}
\end{codesize}
\color{black}

This statement is read like this:
make the new value of counter equal to the old value
of counter plus one. Since the old value is zero, the
new value is one.
What is the value of counter after the following
code is executed?

\color{CodeGreen}
\begin{codesize}
\begin{verbatim}
    counter = 0
    counter = counter + 1
    counter = counter + 1
    counter = counter + 1
    counter = counter + 1
    counter = counter + 1
\end{verbatim}
\end{codesize}
\color{black}

Highlight the following line to see the answer:

\begin{center}
    The answer is {\color{white} 5}
\end{center}

The {\it update} pattern can be used to sum a number of
variables. Suppose we wish to compute the sum of the
variables {\it a}, {\it b}, {\it c}, {\it d}, and {\it e}.
The obvious way to do this is with one statement:

\color{CodeGreen}
\begin{codesize}
\begin{verbatim}
    sum = a + b + c + d + e
\end{verbatim}
\end{codesize}
\color{black}

However, we can use the {\it update} pattern as well:

\color{CodeGreen}
\begin{codesize}
\begin{verbatim}
    sum = 0
    sum = sum + a
    sum = sum + b
    sum = sum + c
    sum = sum + d
    sum = sum + e
\end{verbatim}
\end{codesize}
\color{black}

If {\it a} is 1, {\it b} is 2, {\it c} is 3, {\it d} is 4, and {\it e} is 5,
then the value of sum in both cases is 15.
Why would we ever want to use the {\it update} pattern for
computing a sum when the
first version is so much more compact and readable? The
answer is...you'll have to wait until we cover a programming
concept called a {\it loop}. With loops, the {\it update} pattern
is almost always used to compute sums, products, etc.

\subsection{The Throw-away Pattern}

The {\it throw-away} pattern is a mistaken attempt to use
the {\it update} pattern. In the {\it update} pattern, we use
the original value of the variable to compute the
new value of the variable. Here again is the
classic example of incrementing a counter:

\color{CodeGreen}
\begin{codesize}
\begin{verbatim}
    counter = counter + 1
\end{verbatim}
\end{codesize}
\color{black}

In the {\it throw-away} pattern, the new value is computed
but it the variable is not reassigned, nor is the
new value stored anywhere. Many novice programmers
attempt to update a counter simply by computing
the new value:

\color{CodeGreen}
\begin{codesize}
\begin{verbatim}
    count + 1       # throw-away!
\end{verbatim}
\end{codesize}
\color{black}

Scam does all the work to compute the new value, but
since the new value is not assigned to any variable,
the new value is thrown away.

\subsection{The Throw-away Pattern and Functions}

The {\it throw-away} pattern applies to function calls
as well.
We haven't discussed functions much, but the following
example is easy enough to understand. First we define
a function that computes some value:

\color{CodeGreen}
\begin{codesize}
\begin{verbatim}
    def inc(x):
        return x + 1
\end{verbatim}
\end{codesize}
\color{black}

This function returns a value one greater than the value
given to it, but what the function actually does is irrelevant
to this discussion. That said, we want to start indocrinating
you on the use of functions. Repeat this ten times:

\begin{quote} We always do three things with functions: define them (we just
did that!), call them, and save the return value.
\end{quote}

To call the function, we use the function name followed by
a set of parentheses. Inside the parentheses, we place the
value we wish to send to the function. Consider this code,
which includes a call to the function {\it inc}:

\color{CodeGreen}
\begin{codesize}
\begin{verbatim}
    y = 4
    y = inc(y)
    print("y is",y)
\end{verbatim}
\end{codesize}
\color{black}

If we were to run this code, we would see the following 
output:

\color{CodeGreen}
\begin{codesize}
\begin{verbatim}
    y is 5
\end{verbatim}
\end{codesize}
\color{black}

The value of {\it y}, 4, is sent to the function which adds one to
the given value and returns this new value. This new value, 5, is
assigned to {\it y}. Thus we see that {\it y} has a new value of 5.

Suppose, we run the following code instead:

\color{CodeGreen}
\begin{codesize}
\begin{verbatim}
    y = 4
    inc(y)
    print("y is",y)
\end{verbatim}
\end{codesize}
\color{black}
    
Note that the
return value of the function {\it inc} is not assigned to any
variable. Therefore, the return value is thrown away and the
output becomes:

\color{CodeGreen}
\begin{codesize}
\begin{verbatim}
    y is 4
\end{verbatim}
\end{codesize}
\color{black}

The variable {\it y} is unchanged because it was never reassigned.

\section{About Patterns}

As you can see from above, not all patterns are good ones. However,
we often mistakenly use bad patterns when programming. If we 
can recognize those bad patterns more readily, our job of producing
a correctly working program is greatly simplified.

\section{Assignment and Lists}

You can change a particular element of a list by
assigning a new value to the index of that element
by using bracket notation:

\color{CodeGreen}
\begin{codesize}
\begin{verbatim}
    >>> items = ['a', True, 7]

    >>> items[0] = 'b'

    >>> items
    ['b', True, 7]`
\end{verbatim}
\end{codesize}
\color{black}

As expected, assigning to index 0 replaces the first element. In
the example, the first element {\tt\codesize 'a'}
is replaced with {\tt\codesize 'b'}.

What bracket notation would you use to change the 7 in the list
to a 13? The superior student will experiment with the Scam
interpreter to verify his or her guess.
