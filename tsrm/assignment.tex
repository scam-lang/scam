\chapter{Assignment}
\label{Assignment}

Once a variable has been created, it is possible to change its value,
or {\it binding},
We have already seen a form of assignment in
Section {\link ListsArraysStringsSetting};
Assignment to variables proceeds in a similar fasion.
Consider the following interaction with
the Scam interpreter:

\begin{verbatim}
    (define eyeColor 'black)    ; creation
    
    (inspect eyeColor)          ; reference
    -> (eyeColor is black)
    
    (assign eyeColor 'green)    ; assignment
    
    (eq? eyeColor 'black)       ; equality
    -> #f
    
    (== eyeColor 'brown)        ; equality (alternate)
    -> #f

    (eq? eyeColor 'green)       ; equality
    -> #t

    (set! eyecolor BROWN)       ; assignment (alternate)
\end{verbatim}

The assignment function is not like the arithmetic operators.
Recall that {\tt +} evaluates all its arguments
before performing the addition.
For {\it assign},
the leftmost operand is not evaluated:
If it were, the assignment

\begin{verbatim}
    (define x 1)
    (assign x 3)
\end{verbatim}
    
would be equivalent to:

\begin{verbatim}
    (assign 1 3)
\end{verbatim}

In general, an operator which does not evaluate
all its arguments is known as a {\it special form}\footnote{
In Scam, there are no special forms. As such, {\it assign} is
a true function and can be given a new meaning.}.
So, for assign, the evaluation of the first argument
is suppressed. One can force an evaluation be
passing a function call as the first argument.
Suppose we have the function {\it symbolX} which returns
the symbol \verb!x!.

\begin{verbatim}
    (define (symbolX) 'x)
\end{verbatim}

With this definition, these two calls to assign are equivalent:

\begin{verbatim}
    (assign x 3)
    (assign (symbolX) 3)
\end{verbatim}

The same is true for {\it set!}.

\section{Assignment and Collections}

In the chapter on \link{lists and other collections}{ListsStringsArraysSetting},
we saw how to change an element in a collection. Recall the
the functions for doing so:

\begin{verbatim}
    (setElement collection index newValue)
    (set-car! collection newValue)
\end{verbatim}

Also, it is possible to reset the tail of a list:

\begin{verbatim}
    (set-cdr! items newTail)
\end{verbatim}

\section{Assignment and Environments/Objects}

The assign function can take an environment as an
optional third argument.
Since the predefined variable
{\it this} always points to the current environment,
the following
three expressions are equivalent:

\begin{verbatim}
        (assign x 5)
        (assign x 5 this)
        (set! x 5 this)
\end{verbatim}

Since environments form the basis for objects in Scam,
{\it assign} and {\it set!} are used to update the instance variables
of objects.
