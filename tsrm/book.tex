\documentclass{book}  
\usepackage[usenames]{color}
\usepackage{graphicx}
\usepackage{hyperlatex}
\usepackage{upquote}
\usepackage[margin=1.0in]{geometry}
\usepackage[compact,small]{titlesec}
\usepackage{booktabs}

\W\newcommand{\not}{\xmlent{##172}}
\W\newcommand{\pm}{\xmlent{##177}}

\W\newcommand{\alpha}{\xmlent{##945}}
\W\newcommand{\beta}{\xmlent{##946}}
\W\newcommand{\gamma}{\xmlent{##947}}
\W\newcommand{\delta}{\xmlent{##948}}
\W\newcommand{\epsilon}{\xmlent{##949}}
\W\newcommand{\varepsilon}{\xmlent{##949}}
\W\newcommand{\zeta}{\xmlent{##950}}
\W\newcommand{\eta}{\xmlent{##951}}
\W\newcommand{\theta}{\xmlent{##977}}
\W\newcommand{\vartheta}{\xmlent{##952}}
\W\newcommand{\iota}{\xmlent{##953}}
\W\newcommand{\kappa}{\xmlent{##954}}
\W\newcommand{\lambda}{\xmlent{##955}}
\W\newcommand{\mu}{\xmlent{##956}}
\W\newcommand{\nu}{\xmlent{##957}}
\W\newcommand{\xi}{\xmlent{##958}}
\W\newcommand{\pi}{\xmlent{##960}}
\W\newcommand{\varpi}{\xmlent{##982}}

\W\newcommand{\rho}{\xmlent{##961}}
\W\newcommand{\varrho}{\xmlent{##961}}
\W\newcommand{\sigma}{\xmlent{##963}}
\W\newcommand{\varsigma}{\xmlent{##963}}
\W\newcommand{\tau}{\xmlent{##964}}
\W\newcommand{\upsilon}{\xmlent{##965}}
\W\newcommand{\phi}{\xmlent{##966}}
\W\newcommand{\varphi}{\xmlent{##966}}
\W\newcommand{\chi}{\xmlent{##967}}
\W\newcommand{\psi}{\xmlent{##968}}
\W\newcommand{\omega}{\xmlent{##969}}


\W\newcommand{\Gamma}{\xmlent{##915}}
\W\newcommand{\Delta}{\xmlent{##916}}
\W\newcommand{\Theta}{\xmlent{##920}}
\W\newcommand{\Gamma}{\xmlent{##915}}
\W\newcommand{\Lambda}{\xmlent{##923}}
\W\newcommand{\Xi}{\xmlent{##926}}
\W\newcommand{\Pi}{\xmlent{##928}}
\W\newcommand{\Sigma}{\xmlent{##931}}
\W\newcommand{\Upsilon}{\xmlent{##933}}
\W\newcommand{\Phi}{\xmlent{##934}}
\W\newcommand{\Omega}{\xmlent{##937}}

\W\newcommand{\Re}{\xmlent{##8476}}
\W\newcommand{\Im}{\xmlent{##8465}}
\W\newcommand{\aleph}{\xmlent{##8501}}
\W\newcommand{\partial}{\xmlent{##8706}}
\W\newcommand{\forall}{\xmlent{##8704}}
\W\newcommand{\partial}{\xmlent{##8706}}
\W\newcommand{\exists}{\xmlent{##8707}}
\W\newcommand{\emptyset}{\xmlent{##8709}}

\W\newcommand{\nabla}{\xmlent{##8711}}
\W\newcommand{\in}{\xmlent{##8712}}
\W\newcommand{\ni}{\xmlent{##8715}}

\W\newcommand{\prod}{\xmlent{##8719}}
\W\newcommand{\sum}{\xmlent{##8721}}
\W\newcommand{\sqrt}{\xmlent{##8730}}
%\newcommand{\htmlsqrt}[1]{\xmlent{##8730} (#1)}

\W\newcommand{\propto}{\xmlent{##8733}}
\W\newcommand{\infin}{\xmlent{##8734}}
\W\newcommand{\angle}{\xmlent{##8736}}
\W\newcommand{\wedge}{\xmlent{##8743}}
\W\newcommand{\vee}{\xmlent{##8744}}
\W\newcommand{\cup}{\xmlent{##8745}}
\W\newcommand{\cap}{\xmlent{##8746}}
\W\newcommand{\int}{\xmlent{##8747}}
\W\newcommand{\sim}{\xmlent{##8764}}
\W\newcommand{\approx}{\xmlent{##8776}}
\W\newcommand{\neq}{\xmlent{##8800}}
\W\newcommand{\equiv}{\xmlent{##8801}}
\W\newcommand{\leq}{\xmlent{##8804}}
\W\newcommand{\geq}{\xmlent{##8805}}


\W\newcommand{\cdot}{\xmlent{##8901}}


\W\usepackage{rhxpanel}
\W\newcommand{\HlxPanelHome}{0}
\htmlcss{steely.css}
\newcommand{\HlxIcons}{}
\htmlpanelfield{Contents}{hlxcontents}
\W\newcommand\codesize \small
\T\newcommand\codesize \normalsize
\W\newcommand\sf \bf
\W\newcommand\sc \it

\definecolor{CodeGreen}{rgb}{0.0,0.45,0.2}
\W\newcommand\highlight{white}
\T\newcommand\highlight{black}

\htmldepth{2}
\htmltitle{The Scam Reference Manual}
\htmladdress{\xlink{lusth@cs.ua.edu}{mailto:lusth@cs.ua.edu}}

\T\setlength\parskip{10 pt}
\T\setlength\parindent{0 in}

\title{The Scam Reference Manual}

\author{John C. Lusth}

\begin{document}

\maketitle

\W\htmlrule
\W\xlink{Printable Version}{book.pdf}
\W\htmlrule

\begin{rawxml}
<script type="text/javascript">

// Google Internal Site Search script-
// By JavaScriptKit.com (http://www.javascriptkit.com)
// For this and over 400+ free scripts,
// visit JavaScript Kit- http://www.javascriptkit.com/
// This notice must stay intact for use

//Enter domain of site to search.
var domainroot="beastie.cs.ua.edu/cs150/book/"

function Gsitesearch(curobj){
curobj.q.value="site:"+domainroot+" "+curobj.qfront.value
}

</script>


<form
    action="http://www.google.com/search"
    method="get" onSubmit="Gsitesearch(this)">

    <p>Search the textbook: 
    <input name="q" type="hidden" />
    <input name="qfront" type="text" style="width: 180px" />
    <input type="submit" value="Search" /></p>

    </form>

<p style="font: normal 11px Arial">This free script provided by<br />
<a href="http://www.javascriptkit.com">JavaScript Kit</a></p>
\end{rawxml}

\tableofcontents
\setcounter{tocdepth}{2} 

\chapter{Starting Out}
\label{StartingOut}

A word of warning: if you require fancy graphically oriented development
environments, you will be sorely disappointed in the Scam Programming
Language. Scam is programming at its simplest: a prompt at which you type
in expressions which the Scam interpreter evaluates. Of course, you can
create Scam programs using your favorite text editor (mine is {\it vim}).
Scam can be used as a scripting language, as well.

\section{Running Scam}

In a terminal window, simply type the command:

\begin{verbatim}
    scam <fileName>
\end{verbatim}

where <fileName> is replaced by the name of the file contining your
Scam program ({\it i.e.} Scam source code).
You should be rewarded with the output from your Scam program.
At this point, you are ready to proceed to next chapter.

\chapter{Literals}
\label{Literals}

Scam works by figuring out the meaning or value of some code.
This is true for the tiniest pieces of code to the largest
programs. The process of finding out the meaning of code
is known as {\it evaluation}.

The things whose values are the things themselves are known as
{\it literals}. The literals of Scam can be categorized by the following
types:
{\it integers}, {\it real} {\it numbers}, {\it strings}, {\sc Booleans},
{\it symbols}, and {\it lists}.

Scam (or more correctly, the Scam interpreter) responds to literals
by echoing back the literal itself.
Here are examples of each of the types:

\begin{verbatim}
    (inspect 3)
    -> 3 is 3
     
    (inspect -4.9
    -> -4.900000 is -4.900000
     
    (inspect "hello")
    -> hello is hello
     
    (inspect #t)
    -> #t is #t

    (inspect (list 3 -4.9 "hello"))
    -> (list 3 -4.9 "hello") is (3, -4.9, "hello")
\end{verbatim}
Let's examine the
five types in more detail.

\section{Integers}

Integers are numbers without any fractional parts.
Examples of integers are:

\begin{verbatim}
    (inspect 3)
    -> 3 is 3
    
    (inspect -5)
    -> -5 is -5
    
    (inspect 0)
    -> 0 is 0
\end{verbatim}

Integers must begin with a digit or a minus sign. The initial minus sign
must immediately be followed by a digit.

\section{Real Numbers}

Reals are numbers that do have a fractional part (even if that fractional
part is zero!). Examples of real numbers are:

\begin{verbatim}
    (inspect 3.2)
    -> 3.200000 is 3.200000
    
    (inspect 4.0)
    -> 4.000000 is 4.000000
       
    (inspect 5.)
    -> 5.000000 is 5.000000
       
    (inspect 0.3)
    -> 0.300000 is 0.300000
       
    (inspect .3)
    -> 0.300000 is 0.300000
    
    (inspect 3.0e-4)
    -> 0.000300 is 0.000300
    
    (inspect 3e4)
    -> 30000.000000 is 30000.000000
    
    (inspect .000000987654321)
    -> 0.000001 is 0.000001
\end{verbatim}

Real numbers must start with a digit or a minus sign or a decimal
point. An initial minus sign must immediately be followed by a digit or a
decimal point. An initial decimal point must immediately be followed by
a digit. Scam accepts real numbers in scientific notation. For example,
$3.0 * 10^{-11}$ would be entered as 3.0e-11. The `e' stands for exponent and
the 10 is understood, so e-11 means multiply whatever precedes the
e by $10^{-11}$.

The Scam interpreter can hold huge numbers,
limited by only the amount of memory available to the
interpreter,
but holds only 15 digits after the decimal point:

\begin{verbatim}
    (inspect 1.2345678987654329
    -> 1.234568 is 1.234568
\end{verbatim}

Note that Scam rounds up or rounds down, as necessary.

Numbers greater than $10^6$ and
less than $10^{-6}$ are displayed in
scientific notation.

\section{Strings}

Strings are sequences of characters delineated by double quotation marks:

\begin{verbatim}
    (println "hello, world!")
    -> hello, world!
    
    (println "x\nx")
    -> x
       x
    
    (println "\"z\"")
    -> "z" 
\end{verbatim}

Scam accepts both double quotes and single quotes to
delineate strings. In this text, we will use the convention
that double quotes are used for strings of multiple 
characters and single quotes for strings consisting of
a single character.

Characters in a string can be {\it escaped} (or quoted)
with the backslash character,
which changes the meaning of some characters. For example, the character
{\it n}, in a string refers to the letter {\it n} while the character sequence
{\it $\backslash$n}
refers
to the {\it newline} character. A backslash also changes the meaning of the
letter {\it t},
converting it into a tab character.
You can also quote single and double quotes with backslashes.
When other characters are escaped,
it is assumed the backslash is a character of the
string and it is escaped (with a backslash) in the result:

\begin{verbatim}
    (println "\z")
    -> z
\end{verbatim}

Note that Scam, when asked
the value of strings that contain newline and tab characters, displays
them as escaped characters. When newline and tab characters in a string
are printed in a program, however, they are displayed as actual newline
and tab characters, respectively.
As already noted,
double and single quotes can be embedded in a
string by quoting them with backslashes. A string with no characters
between the double quotes is known as an empty string.

Unlike some languages, there is no character type in Scam. A single
character {\verb+a+}, for example, is entered as the string
{\verb+"a"+}.

\section{Symbols}

A symbol is a set of characters, much like a string. Like strings,
symbols evaluate to themselves. Unlike strings,
symbols are not formed using a beginning quotation mark and an
ending quotation mark. They are also limited
in the characters that compose them. For example, a symbol cannot
contain a space character while a string can. A symbol is introduced
with a single quotation mark:

\begin{verbatim}
    (print 'a)
    -> a

    (print 'hello)
    -> hello
\end{verbatim}

We we learn more about symbols and their relationship to entities
called {\it variables} in a later chapter.

\section{True, False, and nil}

There are two special literals, \verb!#t!
and \verb!#f!.
These literals are known as the {\sc Boolean} values;
\verb!#t! is true and \verb!#f! is false.
Boolean values are used to guide the flow of a program.
The term {\sc Boolean} is derived from the last name of George Boole, who,
in his 1854 paper {\it An Investigation of the Laws of Thought, on which are
founded the Mathematical Theories of Logic and Probabilities}, laid one
of the cornerstones of the modern digital computer. The so-called {\sc Boolean}
logic or {\sc Boolean} algebra is concerned with the rules of combining truth
values (i.e., true or false). As we will see, knowledge of such rules will
be important for making Scam programs behave properly. In particular,
{\sc Boolean} expressions will be used to control conditionals and loops.

Another special literal is \verb!nil!.
This literal is used to
indicate an empty list or an empty string; it also is used
to indicates something that has not yet been
created. More on \verb!nil! when we cover lists and
objects.

\section{Lists}

Lists are just collections of entities.
The simplest list is the empty list:

\begin{verbatim}
    (inspect ())
    -> nil
\end{verbatim}

Since the empty list looks kind of strange, Scam uses the symbol \verb!nil!
to represent an empty list.

One creates non-empty list by
using the built-in {\it list function}. 
Here, we make a list containg the numbers
10, 100, and 1000:

\begin{verbatim}
    (list 10 100 1000)
    -> (10 100 1000)
\end{verbatim}

Lists can contain values besides numbers:

\begin{verbatim}
    (list 'a "help me" length)
    -> (a "help me" <built-in length(items)>)
\end{verbatim}

The first value is an integer, the second a string,
and the third item is also a function. 
The built-in {\it length} function
is used to tell us how many items are in a list:

\begin{verbatim}
    (length (list 'a "help me" length))
    -> 3
\end{verbatim}

As expected, the {\it length} function tells us that the list
\verb!('a "help me" length)! has three items in it.

Lists can even contain lists!

\begin{verbatim}
    (list 0 (list 3 2 1) 4)
    -> (0 (3 2 1) 4)
\end{verbatim}

A list is something known as a {\it data structure};
data structures are extremely important in writing
sophisticated programs.

\section{Indexing into Lists}

You can pull out an item from a list by using another function
named {\it getElement}.
With the {\it getElement} function,
you specify exactly which element (or elements)
you wish to extract from the list. This specification is called an
{\it index}. The first element of the list has index 0, the second
index 1, and so on. This concept of having the first element having
an index of zero is known as {\it zero-based counting}, a common concept
in Computer Science. Here is some code that extracts the first element
of a list:

\begin{verbatim}
    (getElement (list "a" #t 7) 0)
    -> "a"

    (getElement (list "a" #t 7) 1)
    -> #t

    (getElement (list "a" #t 7) 2)
    -> 7
\end{verbatim}

What happens if our index is too large?

\begin{verbatim}
    (getElement (list "a" #t 7) 3)
    EXCEPTION: generalException
    index (3) is too large
\end{verbatim}

Not surprisingly, we get an error.
In Scam, as with many programming languages, an error is known
as an {\it exception}.

As with Scheme, the built-in {\it car} and {\it cdr} functions
returns the first item and the tail of a given list, respectively.

We will see more of lists in a later chapter.

\chapter{Combining Literals}
\label{CombiningLiterals}

Like the literals themselves, combinations of literals are also
expressions. For example, suppose you have forgotten your times table
and aren't quite sure whether 8 times 7 is 54 or 56. We can ask Scam,
presenting the interpreter with the expression:

\begin{verbatim}
    (* 8 7)
    -> 56
\end{verbatim}

The
multiplication sign * is known as an {\it operator}, as it {\it operates} on the 8
and the 7, producing an equivalent literal value.
As with all LISP/Scheme-like languages, operators like \verb!*!
are true functions and thus prefix notation is used in function calls.

The 8 and the 7 in the above expression are
known as {\it operands}. It seems that the actual names of various operands are
not being taught anymore, so for nostalgia's sake, here they are. The
operand to the left of the multiplication sign (in this case the 8) is
known as the {\it multiplicand}. The operand to the right (in this case the 7)
is known as the {\it multiplier}. The result is known as the {\it product}.

The operands of the other basic operators
have special names too. For addition of two operands, the left operand is known as the
{\it augend} and the right operand is known as the {\it addend}.
The result is known as the {\it sum}.
For subtraction,
the left operand is the {\it minuend}, the right the {\it subtrahend}, and
the result as the {\it difference}.
Finally
for division (and I think this is still taught), the left operand is
the {\it dividend}, the right operand is the {\it divisor}, and the 
result is the {\it quotient}.

We separate
operators from their operands by
spaces, tabs,
or newlines, collectively known as {\it whitespace}.\footnote{
Computer Scientists, when they have to write their annual reports,
often refer to the things they are reporting on as
{\it darkspace}. It's always good to have a lot of darkspace in
your annual report!
}

Scam always takes in an expression and returns an equivalent
literal expression ({\it e.g.}, integer or real). All Scam operators are
variadic, meaning they operate on exactly on any number of operands:

\begin{verbatim}
    (+ 1 2 3 4 5)
    -> 15
\end{verbatim}

\section{Numeric operators}

If it makes sense to add two things together, you can probably do it in
Scam using the + operator. For example:

\begin{verbatim}
    (+ 2 3)
    -> 5
    
    (+ 1.9 3.1)
    -> 5.000000
\end{verbatim}
    
One can see that if one adds two integers, the result is an integer. If
one does the same with two reals, the result is a real.
Things get more interesting when
you add things having different types. Adding an integer and a real (in
any order) always yields a real.

\begin{verbatim}
    (+ 2 3.3)
    -> 5.300000
    
    (+ 3.3 2)
    -> 5.300000
\end{verbatim}
    
Adding an string to an integer
(with an augend integer) yields an error;
the types are not `close' enough, like they are with
integers and reals:

\begin{verbatim}
    (+ 2 "hello")
    -> EXCEPTION: generalException
       wrong types for '+': INTEGER and STRING
\end{verbatim}

In general, when adding two things,
the types must match or nearly match.
    
Of special note is the division operator with respect to integer
operands. Consider evaluating the following expression:

\begin{verbatim}
    15 / 2
\end{verbatim}

If one asked the Scam interpreter to perform this task, the result
will not be 7.5, as expected, but rather 7, as the division operator
performs {\it integer division}:

\begin{verbatim}
    (/ 15 2)
    -> 7
\end{verbatim}

However, we wish for a real result, we can convert one of
the operands to a real, as in:

\begin{verbatim}
    (/ (real 15) 2)
    -> 7.500000
\end{verbatim}

Note that Scheme would produce the rational number $\frac{15}{2}$ in this
case. Scam does not have rationals, but they can be added to
the language if one desires more Scheme compatibility.

The complement to integer division is the modulus operator \%. While the
result of integer division is the quotient, the result of the modulus
operator is the remainder. Thus

\begin{verbatim}
    (% 14 5)
    -> 4
\end{verbatim}

evaluates to 4 since 4 is left over when 5 is divided into 14. To check
if this is true, one can ask the interpreter to evaluate:

\begin{verbatim}
    (== (+ (* (/ 14  5) 5) (% 14 5)) 14)
    -> #t
\end{verbatim}

This complicated expression asks the question `is it true that the
quotient times the divisor plus the remainder is equal to the original
dividend?'. The Scam interpreter will respond that, indeed, it is
true. 

\section{Combining strings}

To concatenate strings together, one uses the {\it string+} operator.
Like \verb!+! and \verb!-,! {\it string+} is variadic;
we can concatenate any number of
strings at one time. Strings are concatenated from left to right.
For examle, the expression:

\begin{verbatim}
    (string+ "a" "b" "c")
\end{verbatim}

produces the new string \verb!"abc"!. Note that the strings passed
to {\it string+} are
unmodified.

\section{Comparing things}

Remember the {\sc Boolean} literals, {\tt \#t} and {\tt \#t}?
We can use the {\sc Boolean}
comparison operators to generate such values. For example, we can ask
if 3 is less than 4:

\begin{verbatim}
    (< 3 4)
    -> #t
\end{verbatim}

Evaluating this expression shows that, indeed, 3 is less than 4. If it were
not, the result would be {\tt \#f}.
Besides
{\tt <}
(less than),
there are other {\sc Boolean} comparison operators:
{\tt <=}
(less than or equal to),
{\tt >}
(greater than),
{\tt >=}
(greater than or equal to),
{\tt ==}
(equal to), and
{\tt !=}
(not equal to).

The comparison operators are variadic:

\begin{verbatim}
    (< 1 2 3)
    -> #t

    (< 1 2 2)
    -> #f
\end{verbatim}

Any Scam type can be compared with any other type with the
{\tt ==}
and
{\tt !=}
comparison operators.
If an integer is compared with a real with these
operators, the integer is converted into a real before the comparison
is made. In other cases, comparing different types with
{\tt ==}
will yield
a value of {\tt \#f}. Conversely, comparing different types with
{\tt !=}
will yield
{\tt \#t}
(the exception, as above, being integers compared with reals).
If the types match,
{\tt ==}
will yield true only if the values
match as well. The operator
{\tt !=}
behaves accordingly.

\section{Combining comparisons}

We can combine comparisons with the {\sc Boolean} logical connectives
{\tt and} and {\tt or}:

\begin{verbatim}
    (and (< 3 5) (< 4 5))
    -> #t
    
    (or (< 3 4) (< 4 5))
    -> #t
    
    (and (< 3 4) (< 5 4))
    -> #f
    
    (or (< 3 4) (< 5 4))
    -> #t
\end{verbatim}

The first interaction asks if both the expression
{\tt (< 3 4)} and the expression
{\tt (< 4 5)} are true. Since both are, the
interpreter responds with true. The second interaction
asks if at least one of the expressions is true. Again, the
interpreter responds with true. The difference between {\tt and}
and {\tt or} is illustrated in the last two interactions. Since
only one expression is true (the latter expression being
false) only the {\tt or} operator yields a true value.
If both expressions are false, both \verb!and! and \verb!or! returns false.

There is one more {\sc Boolean} logic operation, called
{\it not}. It simply reverses the value of the given expression.

\begin{verbatim}
    (not (and (< 3 4) (< 4 5)))
    -> #f
\end{verbatim}

\chapter{Precedence and Associativity}
\label{PrecedenceAndAssociativity}

\section{Precedence}

Precedence (partially) describes the order in which operators, in
an expression involving different operators, are evaluated. In Scam,
the expression

\begin{verbatim}
    3 + 4 < 10 - 2
\end{verbatim}

evaluates to true. In particular, {\tt 3 + 4} and {\tt 10 - 2}
are evaluated before the {\tt <}, yielding {\tt 7 < 8},
which is indeed true. This implies that {\tt +} and
{\tt -} have higher precedence than {\tt <}. If
{\tt <} had higher precedence, then {\tt 4 < 10}
would be evaluated first, yielding {\tt 3 + :true - 2}, which
is nonsensical.

Note that precedence is only a partial ordering. We cannot tell, for
example whether {\tt 3 + 4} is evaluated before the {\tt 10 - 2},
or vice versa. Upon close examination, we see that it does not
matter which is performed first as long as both are performed before
the expression involving {\tt <} is evaluated.

It is common to
assume that the left operand is evaluated before the right operand. For
the {\sc boolean} connectives {\tt and} and {\tt or},
this is indeed true. But for other operators, such an assumption
can lead you into trouble. You will learn why later. For now, remember
never, never, never depend on the order in which operands are evaluated!

The lowest precedence operator in Scam is the assignment operator which
is described later. Next come the {\sc boolean} connectives {\tt and}
and {\tt or}.  At the next higher level are the {\sc boolean}
comparatives, {\tt <}, {\tt <=}, {\tt >},
{\tt >=}, {\tt ==}, and {\tt !=}.  After
that come the additive arithmetic operators {\tt +}
and{\tt -}. Next comes the
multiplicative operators
{\tt *}, {\tt /} and {\tt \%}.
Higher still is the exponentiation operator {\tt **}.
Finally, at the highest level of precedence is the selection,
or {\it dot}, operator (the dot operator is a period or full-stop). Higher
precedence operations are performed before lower precedence operations.
Functions which are called with operator syntax have the same precedence
level as the mathematical operators.

\section{Associativity}

Associativity describes how multiple expressions connected by operators at the same
precedence level are evaluated. All the operators, with the exception of
the assignment and exponentiation operators, are left associative.
For example the expression
5 - 4 - 3 - 2 - 1 is equivalent to ((((5 - 4) - 3) - 2) - 1). For
a left-associative structure, the equivalent, fully parenthesized,
structure has open parentheses piling up on the left. If the minus
operator was right associative, the equivalent expression would be
(5 - (4 - (3 - (2 - 1)))), with the close parentheses piling up on
the right. For a commutative operator, it does not matter whether it
is left associative or right associative. Subtraction, however, is not
commutative, so associativity does matter. For the given expression, the
left associative evaluation is -5. If minus were right associative, the
evaluation would be 3. 

\chapter{Variables}
\label{Variables}

One defines variables with the {\it define} function:

\begin{verbatim}
    (define x 13)
\end{verbatim}

The above expression creates a variable named {\it x} in the current
scope and initializes it to the value 13.
If the initializer is missing:

\begin{verbatim}
    (define y)

    (inspect y)
    -> y is nil
\end{verbatim}

the variable is initialized to nil.

\section{Defining functions}

There are two ways to define a function. The first is through a 
regular variable definition, where the initializer is a lambda
expression:

\begin{verbatim}
    (define square (lambda (x) (* x x)))

    (square 3)
    -> 9

    (inspect square)
    -> square is <function square(x)>
\end{verbatim}

The above expression defines a function that returns the square of
its argument. The second method uses a special syntax:

\begin{verbatim}
    (define (square x) (* x x))
    
    (inspect square)
    -> square is <function square(x)>
\end{verbatim}

The two forms are equivalent, so regardless of the method, the
name of the function is simply a variable. The only difference
is its value.

\section{Environments and Objects}

When one defines a variable, the variable name and value are
stored in a table called an {\it environment}. The predefined
variable this always points to the current environment.
For example, consider this interaction:

\begin{verbatim}
    (define n 10)

    (inspect this)
    -> this is <object 8393>
                   label  : environment
                 context  : <object 4495>
                   level  : 0
             constructor  : nil
                    this  : <object 8393>
                       n  : 10
\end{verbatim}

Among other information stored in the current environment,
we see an entry for {\it n} and its value is indeed 10.

The Scam object system is based upon environments. We will
learn about objects in a later chapter.

\section{Defining Variables Programatically}

The function addSymbol is used to define variables on the 
fly. For example, to define a variable named x in the current
scope and to initialized it to 13, one might use the
following expression:

\begin{verbatim}
    (addSymbol 'x 13 this)
\end{verbatim}

You can also define functions this way:

\begin{verbatim}
    (addSymbol 'square (lambda (x) (* x x)) this)
\end{verbatim}

Since {\it addSymbol} evaluates all its arguments, the first
argument can be any expression that resolves to a symbol,
the second argument can be any expression that resolves
to an appropriate value, and the third argument can
be any expression that resolves to an environment or
object.

\section{Variable naming}

Unlike many languages,
Scam is quite liberal in regards to legal variable names.
A variable can't begin with any of the these characters:
\verb!0123456789;,'"()! nor whitespace and cannot contain any of these
characters: \verb!;,'"()! nor whitespace. Typically,
variable names start with a letter or underscore, but
they do not have to. This flexibility allows Scam programmers
to easily define new functions that have appropriate names.
Here is a function that increments the value of its argument:

\begin{verbatim}
    (define (+1 n) (+ n 1))
\end{verbatim}

While Scam lets you name variables in wild ways:

\begin{verbatim}
    (assign $#1_2!3iiiiii@ 7)
\end{verbatim}

you should temper your
creativity if it gets out of hand.
While the name \verb-$#1_2!3iiiiii@-
is a perfectly good variable name from Scam's point of
view,
it is a particularly poor name from the point of making your Scam
programs readable by you and others.
It is important that your variable
names reflect their purpose.



\chapter{Assignment}
\label{Assignment}

Once a variable has been created, it is possible to change its value,
or {\it binding},
using the assignment operator. Consider the following interaction with
the interpreter:

\begin{verbatim}
    (define BLACK 1)            ; creation
    (define BROWN 2)            ; creation
    (define GREEN 3)            ; creation
    
    (define eyeColor BLACK)     ; creation
    
    (inspect eyeColor)          ; reference
    -> (eyeColor is 1)
    
    (assign eyeColor GREEN)     ; assignment
    
    (eq? eyeColor BLACK)        ; equality
    -> #f
    
    (== eyeColor BROWN)         ; equality (alternate)
    -> #f

    (eq? eyeColor GREEN)        ; equality
    -> #t

    (set! 'eyecolor BROWN)      ; assignment (alternate)
\end{verbatim}

The operator/variable {\tt\codesize =} (equals sign) is bound to the
{\it assignment} function.
The assignment function, however, is not like the operators
{\tt\codesize +} and {\tt\codesize *}.
Recall that {\tt\codesize +} and the like evaluate the things on either
side (recall that those things on either side are generically known as
operands) before combining them. For {\tt\codesize =},
the left operand is not evaluated:
(if it were, the assignment

\color{CodeGreen}
\begin{codesize}
\begin{verbatim}
    eyeColor = GREEN
\end{verbatim}
\end{codesize}
\color{black}
    
would attempt to assign the value of 1 to be 3.
In general, an operator which does not evaluate
all its arguments is known as a {\it special form}.

The last two expressions given to the interpreter in the previous
interaction refer to the {\tt\codesize ==} (equality) operator.
This {\tt\codesize ==} operator returns
true if its operands refer to the same thing and false otherwise.

Another thing to note in the above interaction is
that the
variables {\sf BLACK}, {\sf GREEN}, and {\sf BROWN}
are not meant
to change from their initial values.
We denote variables whose values aren't supposed to change
by naming the
variable using (mostly) capital letters (this convention is borrowed
from earlier programming languages).
The use of caps emphasizes the constant nature of the (not too) variable.

In the above interaction with the interpreter, we use the integers 1, 2,
and 3 to represent the colors black, brown, and green. By abstracting 1,
2, and 3 and giving them meaningful names (i.e., {\sf BLACK}, {\sf BROWN},
and {\sf GREEN}) we find it easy to read code that
assigns and tests eye color.
We do this because it is difficult to remember which integer
is assigned to which color. Without the variables {\sf BLACK}, {\sf BROWN}, and
{\sf GREEN}, we have to keep little notes somewhere to remind ourselves what's
what. Here is an equivalent interaction with the interpreter without
the use of the variables {\sf BLACK}, {\sf GREEN}, and {\sf BROWN}.

\color{CodeGreen}
\begin{codesize}
\begin{verbatim}
    >>> eyeColor = 1
    1
    
    >>> eyeColor
    1
    
    >>> eyeColor = 3
    3
    
    >>> eyeColor == 2
    False
    
    >>> eyeColor == 3
    True
\end{verbatim}
\end{codesize}
\color{black}
    
In this interaction, the meaning of {\it eyeColor}
is not so obvious. We know its a 3, but what eye color
does 3 represent? When numbers appear directly in code,
they are referred to as {\it magic numbers} because they
obviously mean something and serve some purpose,
but how they make the code work correctly
is not always readily apparent,
much like a magic trick.
Magic numbers are to be avoided. Using well-name constants
(or variables if constants are not part of the programming
langue) is considered stylistically superior.

\section{Precedence and Associativity of Assignment}

Assignment has the lowest precedence among the binary operators. It is
also right associative. The right associativity allows for statements like

\color{CodeGreen}
\begin{codesize}
\begin{verbatim}
    a = b = c = d = 0
\end{verbatim}
\end{codesize}
\color{black}

which conveniently assigns a zero to four variables at once and,
because of the right associative nature of the operator, is 
equivalent to:

\color{CodeGreen}
\begin{codesize}
\begin{verbatim}
    (a = (b = (c = (d = 0))))
\end{verbatim}
\end{codesize}
\color{black}

The resulting value of an assignment operation is the value assigned,
so the assignnment {\tt\codesize d == 0} returns 0, which is,
in turned,
assigned to
{\it c} and so on.

\section{Assignment Patterns}

The art of writing programs lies in the ability to
recognize and use patterns that have appeared since the
very first programs were written. In this text, we take
a pattern approach to teaching how to program. For the
topic at hand, we will give a number of patterns that
you should be able to recognize to use or avoid as
the case may be.

\subsection{The Transfer Pattern}

The {\it transfer} pattern is used to  change the value of a
variable based upon the value of another variable.
Suppose we have a variable named {\it alpha} which is initialized
to 3 and a variable {\it beta} which is initialized to 10:

\color{CodeGreen}
\begin{codesize}
\begin{verbatim}
    alpha = 3
    beta = 10
\end{verbatim}
\end{codesize}
\color{black}

Now consider the statement:

\color{CodeGreen}
\begin{codesize}
\begin{verbatim}
    alpha = beta
\end{verbatim}
\end{codesize}
\color{black}

This statement is read like this:
make the new value of {\it alpha} equal to the value
of {\it beta}, throwing away the old value of {\it alpha}.
What is the value of {\it alpha} after that statement
is executed?
\W Highlight the following line to see the answer:

\begin{quote}
    The new value of {\it alpha} is 
    {
    \T\color{black}
    \W\color{white}
    10
    }.
\end{quote}

The {\it transfer} pattern tells us that value of {\it beta} is
imprinted on {\it alpha} at the moment of assignment
but in no case are {\it alpha} and {\it beta} conjoined in anyway in
the future. Think of it this way. Suppose your friend
spray paints her bike neon green. You like the color
so much you spray paint your bike neon green as well.
This is like assignment: you made the value (color) of
your bike the same value (color) as your friend's bike.
Does this mean your bike and your friend's bike will
always have the same color forever? Suppose your friend
repaints her bike. Will your bike automatically become
the new color as well? Or suppose you repaint your bike.
Will your friend's bike automatically assume the color
of your bike?

To test your understanding, what happens if the following
code is executed:

\color{CodeGreen}
\begin{codesize}
\begin{verbatim}
    alpha = 4
    beta = 13
    alpha = beta
    beta = 5
\end{verbatim}
\end{codesize}
\color{black}

What are the final values of {\it alpha}  and {\it beta}?
\W Highlight the following line to see the answer:

\begin{quote}
    The value of {\it alpha} is 
    {
    \T\color{black}
    \W\color{white}
    13
    }
    and the value of {\it beta} is 
    {
    \T\color{black}
    \W\color{white}
    5
    }.
\end{quote}

To further test your understanding, what happens if the following
code is executed:

\color{CodeGreen}
\begin{codesize}
\begin{verbatim}
    alpha = 4
    beta = 13
    alpha = beta
    alpha = 42
\end{verbatim}
\end{codesize}
\color{black}

What are the final values of {\it alpha}  and {\it beta}?
\W Highlight the following line to see the answer:

\begin{quote}
    The value of {\it alpha} is 
    {
    \T\color{black}
    \W\color{white}
    42
    }
    and the value of {\it beta} is 
    {
    \T\color{black}
    \W\color{white}
    13
    }.
\end{quote}

\subsection{The Update Pattern}

The {\it update} pattern is used to  change the value of a
variable based upon the original value of the variable.
Suppose we have a variable named {\it counter} which is initialized
to zero:

\color{CodeGreen}
\begin{codesize}
\begin{verbatim}
    counter = 0
\end{verbatim}
\end{codesize}
\color{black}

Now consider the statement:

\color{CodeGreen}
\begin{codesize}
\begin{verbatim}
    counter = counter + 1
\end{verbatim}
\end{codesize}
\color{black}

This statement is read like this:
make the new value of counter equal to the old value
of counter plus one. Since the old value is zero, the
new value is one.
What is the value of counter after the following
code is executed?

\color{CodeGreen}
\begin{codesize}
\begin{verbatim}
    counter = 0
    counter = counter + 1
    counter = counter + 1
    counter = counter + 1
    counter = counter + 1
    counter = counter + 1
\end{verbatim}
\end{codesize}
\color{black}

Highlight the following line to see the answer:

\begin{center}
    The answer is {\color{white} 5}
\end{center}

The {\it update} pattern can be used to sum a number of
variables. Suppose we wish to compute the sum of the
variables {\it a}, {\it b}, {\it c}, {\it d}, and {\it e}.
The obvious way to do this is with one statement:

\color{CodeGreen}
\begin{codesize}
\begin{verbatim}
    sum = a + b + c + d + e
\end{verbatim}
\end{codesize}
\color{black}

However, we can use the {\it update} pattern as well:

\color{CodeGreen}
\begin{codesize}
\begin{verbatim}
    sum = 0
    sum = sum + a
    sum = sum + b
    sum = sum + c
    sum = sum + d
    sum = sum + e
\end{verbatim}
\end{codesize}
\color{black}

If {\it a} is 1, {\it b} is 2, {\it c} is 3, {\it d} is 4, and {\it e} is 5,
then the value of sum in both cases is 15.
Why would we ever want to use the {\it update} pattern for
computing a sum when the
first version is so much more compact and readable? The
answer is...you'll have to wait until we cover a programming
concept called a {\it loop}. With loops, the {\it update} pattern
is almost always used to compute sums, products, etc.

\subsection{The Throw-away Pattern}

The {\it throw-away} pattern is a mistaken attempt to use
the {\it update} pattern. In the {\it update} pattern, we use
the original value of the variable to compute the
new value of the variable. Here again is the
classic example of incrementing a counter:

\color{CodeGreen}
\begin{codesize}
\begin{verbatim}
    counter = counter + 1
\end{verbatim}
\end{codesize}
\color{black}

In the {\it throw-away} pattern, the new value is computed
but it the variable is not reassigned, nor is the
new value stored anywhere. Many novice programmers
attempt to update a counter simply by computing
the new value:

\color{CodeGreen}
\begin{codesize}
\begin{verbatim}
    count + 1       # throw-away!
\end{verbatim}
\end{codesize}
\color{black}

Scam does all the work to compute the new value, but
since the new value is not assigned to any variable,
the new value is thrown away.

\subsection{The Throw-away Pattern and Functions}

The {\it throw-away} pattern applies to function calls
as well.
We haven't discussed functions much, but the following
example is easy enough to understand. First we define
a function that computes some value:

\color{CodeGreen}
\begin{codesize}
\begin{verbatim}
    def inc(x):
        return x + 1
\end{verbatim}
\end{codesize}
\color{black}

This function returns a value one greater than the value
given to it, but what the function actually does is irrelevant
to this discussion. That said, we want to start indocrinating
you on the use of functions. Repeat this ten times:

\begin{quote} We always do three things with functions: define them (we just
did that!), call them, and save the return value.
\end{quote}

To call the function, we use the function name followed by
a set of parentheses. Inside the parentheses, we place the
value we wish to send to the function. Consider this code,
which includes a call to the function {\it inc}:

\color{CodeGreen}
\begin{codesize}
\begin{verbatim}
    y = 4
    y = inc(y)
    print("y is",y)
\end{verbatim}
\end{codesize}
\color{black}

If we were to run this code, we would see the following 
output:

\color{CodeGreen}
\begin{codesize}
\begin{verbatim}
    y is 5
\end{verbatim}
\end{codesize}
\color{black}

The value of {\it y}, 4, is sent to the function which adds one to
the given value and returns this new value. This new value, 5, is
assigned to {\it y}. Thus we see that {\it y} has a new value of 5.

Suppose, we run the following code instead:

\color{CodeGreen}
\begin{codesize}
\begin{verbatim}
    y = 4
    inc(y)
    print("y is",y)
\end{verbatim}
\end{codesize}
\color{black}
    
Note that the
return value of the function {\it inc} is not assigned to any
variable. Therefore, the return value is thrown away and the
output becomes:

\color{CodeGreen}
\begin{codesize}
\begin{verbatim}
    y is 4
\end{verbatim}
\end{codesize}
\color{black}

The variable {\it y} is unchanged because it was never reassigned.

\section{About Patterns}

As you can see from above, not all patterns are good ones. However,
we often mistakenly use bad patterns when programming. If we 
can recognize those bad patterns more readily, our job of producing
a correctly working program is greatly simplified.

\section{Assignment and Lists}

You can change a particular element of a list by
assigning a new value to the index of that element
by using bracket notation:

\color{CodeGreen}
\begin{codesize}
\begin{verbatim}
    >>> items = ['a', True, 7]

    >>> items[0] = 'b'

    >>> items
    ['b', True, 7]`
\end{verbatim}
\end{codesize}
\color{black}

As expected, assigning to index 0 replaces the first element. In
the example, the first element {\tt\codesize 'a'}
is replaced with {\tt\codesize 'b'}.

What bracket notation would you use to change the 7 in the list
to a 13? The superior student will experiment with the Scam
interpreter to verify his or her guess.

\chapter{Conditionals}
\label{Conditionals}

Conditionals implement decision points in a computer program.
Suppose you have a program that performs some task on an
image. You may well have a point in the program where you
do one thing if the image is a JPEG and quite another
thing if the image is a GIF file. Likely, at this point,
your program will include a conditional expression to make
this decision.

Before learning about conditionals, it is important to
learn about logical expressions. Such expressions are the
core of conditionals and loops.\footnote{
We will learn about loops in the next chapter.
}

\section{Logical expressions}

A logical expression evaluates to a truth value, in essence true or
false. For example, the expression $x > 0$
resolves to true if {\it x} is positive
and false if {\it x} is negative or zero. In Scam, truth is represented by
the symbol {\tt True} and falsehood
by the symbol {\tt False}.
Together,
these two symbols are known as {\it {\sc boolean}} values.

One can assign truth values to variables:

\begin{verbatim}
      >>> c = -1
      >>> z = c > 0;

      >>> z
      False
\end{verbatim}

Here, the variable {\it z} would be assigned a value of {\tt True}
if {\it c} is positive;
since {\it c} is negative, it is assigned a value of {\tt False}.

\section{Logical operators}

Scam has the following logical operators.
 
\begin{tabular}{cl}
    {\tt ==}    & equal to \\
    {\tt !=}    & not equal to \\
    {\tt >=}    & greater than or equal to \\
    {\tt >}     & greater than \\
    {\tt <}     & less than \\
    {\tt <=}    & less than or equal to \\
    {\tt and}   & and \\
    {\tt or}    & or \\
\end{tabular}

The first five operators are used for comparing two things,
while the last two operators are the glue that joins up simpler
logical expressions into more complex ones.

\section{Short circuiting}

When evaluating a logical expression,
Scam evaluates the expression from left to right and
stops evaluating as soon as it finds out that the expression
is definitely true or definitely false.
For example, when encountering the expression:

\begin{verbatim}
      x != 0 and y / x > 2
\end{verbatim}

if {\it x} has a value of 0, the subexpression on the left side of the
{\tt and}
connective resolves to false. At this point, there is no way for the
entire expression to be true (since both the left hand side and the right
hand side must be true for an
{\tt and}
expression to be true), so the right
hand side of the expression is not evaluated. Note that this expression
protects against a divide-by-zero error.

\section{If expressions}

Scam's {\it if} expressions are used to conditionally execute code,
depending on the truth value of what is known as the
{\it test} expression. One version of {\it if} has a block of
code following the test expression:

Here is an example:

\begin{verbatim}
    if (name == "John"):
        print("What a great name you have!")
\end{verbatim}

In this version, when the test expression is true ({\it i.e.}, 
the string {\tt "John"} is bound to the variable {\it name}), 
then the code that is indented under the {\tt if} is evaluated 
(i.e., the compliment
is printed). The indented code is known as a {\it block}.
If the test expression is false, however the
block is not evaluated.
In this text, we will enclose the test expression in parentheses
even though it is not required by Scam. We do that because
some important programming languages require the parentheses
and we want to get you into the habit.

Here is another form of {\it if}:

\begin{verbatim}
    if (major == "Computer Science"):
        print("Smart choice!")
    else:
        print("Ever think about changing your major?")
\end{verbatim}

In this version, {\it if} has two blocks, one following the
test expression and one following the {\tt else} keyword.
Note the colons that follow the test expression and the else;
these are required by Scam.
As before, the first block is evaluated if the test expression
is true. If the test expression is false, however,
the second block is evaluated instead.

\section{if-elif-else chains}

You can chain {\tt if} statements together, as in:

\begin{verbatim}
    if (bases == 4):
        print("HOME RUN!!!")
    elif (bases == 3):
        print("Triple!!")
    elif (bases == 2):
        print("double!")
    elif (bases == 1) 
        print("single")
    else:
        print("out")
\end{verbatim}

The block that is eventually evaluated is
directly underneath the first test expression
that is true, reading from top to bottom.
If no test expression is true, the block associated
with the else is evaluated.

What is the difference between {\tt if-elif-else}
chains and a sequence of
unchained {\it if}s? Consider this rewrite of the
above code:

\begin{verbatim}
    if (bases == 4):
        print("HOME RUN!!!");
    if (bases == 3):
        print("Triple!!");
    if (bases == 2):
        print("double!");
    if (bases == 1):
        print("single");
    else:
        print("out");
\end{verbatim}

In the second version, there are four if statements and
the else belongs to the last if. Does this behave exactly
the same? The answer is, it depends. Suppose the value
of the variable {\it bases} is 0. Then both versions print:

\begin{verbatim}
    out
\end{verbatim}

However, if the value of {\it bases} is 3, for example, the first
version prints:

\begin{verbatim}
    triple!!
\end{verbatim}

while the second version prints:

\begin{verbatim}
    triple!!
    out
\end{verbatim}

Why the difference? In the first version, a subsequent test
expression is evaluated {\it only} if all previous test expressions
evaluated to false. Once a test expression evaluates to true in
an {\tt if-elif-else} chain, the associated block is evaluated and
no more processing of the chain is performed. Like the {\tt and} and
{\tt or} {\sc boolean} connectives,
an {\tt if-elif-else} chain short-circuits.

In contrast, the sequences of {\tt if}s are independent; there is no
short-circuiting. When the test expression of the first if
fails, the test expression of the second {\tt if} succeeds
and {\tt triple!!} is
printed. Now the test expression of the third if is tested and
fails as well as the test expression of the fourth if. But since
the fourth if has an else, the {\tt else} block is evaluated and
{\tt out}
is printed.

It is important to know when to use an {\tt if-elif-else} chain and
when to use a sequence of independent {\tt if}s.
If there should be only
one outcome, then use an {\tt if-elif-else} chain. Otherwise,
use a sequence of {\tt if}s.

\chapter{Functions}
\label{Functions}

%Recall from 
%\link*{the chapter on assignment}[Chapter~\Ref]{Assignment}
Consider the equation to find
the {\it y}-value of a point on the line:

\begin{verbatim}
    y = 5x - 3
\end{verbatim}
    
First, we assigned values to the slope,
the {\it x}-value, and the {\it y}-intercept:

\begin{verbatim}
    (define m 5)
    (define x 9)
    (define b -3)
\end{verbatim}

Once those variables have been created,
we can compute the value of {\it y}:

\begin{verbatim}
    (define y (+ (* m x) b))

    (inspect y)
    -> y is 42
\end{verbatim}

Now, suppose we wished to find the {\it y}-value corresponding to
a different {\it x}-value or, worse yet, for a different {\it x}-value
on a different line. All the work we did would have to be
repeated. A {\it function} is a way to encapsulate all these operations
so we can repeat them with a minimum of effort.

\section{Encapsulating a series of operations}

First, we will define a not-too-useful function that
calculates {\it y} give a slope of 5,
a {\it y}-intercept of -3, and an
{\it x}-value of 9 (exactly
as above). We do this by wrapping a function around
the sequence of operations above.
The return value of this function is the computed {\it y} value:

\begin{verbatim}
    (define (y)
        (define m 5)
        (define x 9)
        (define b -3)
        (+ (* m x) b) ;the value of this expression is the return value
        )
\end{verbatim}

There are a few things to note. The call to the {\it define}
function indicates
that a variable definition is occurring. The fact that the first
argument to {\it define} looks like a list indicates that the variable
being defined will be bound to a function and that the variable/function
name is {\it y}, as it is the first member of that list.
The formal parameters of the function follow the function name;
since there is nothing after the {\it y}, we don't
need to send any information to this function when we call it.
Together, the first line is known as the {\it function signature},
which tells you the name of the function and how many values
it expects to be sent when called.

The expressions after the function name and formal parameters
are called the {\it function body}; the body
is the code that will be evaluated (or executed) when the
function is called. You must remember this: {\it the function body
is not evaluated until the function is called}.

Finally, the return value of a function is the value of the
last expression evaluated. In the above case, the
expression is:

\begin{verbatim}
    (+ (* m x) b)
\end{verbatim}

Once the function is defined, we can find the value of {\it y} repeatedly.

\begin{verbatim}
    (y)
    -> 42

    (y)
    -> 42
\end{verbatim}

Because we designed the
function to take no values when called, we do not place any
values between the parentheses.

Note that when we call the {\it y} function again,
we get the exact same answer.

The {\it y} function, as written,
is not too useful in that we cannot use it to compute
similar things, such as the {\it y}-value for a different value of
{\it x}.
This is because we `hard-wired' the values of {\it b}, {\it x}, and {\it m},
We can improve this function by passing in the value of {\it x}
instead of hard-wiring the value to 9.

\section{Passing arguments}

A hallmark of a good function is that it lets you compute
more than one thing. We can modify our {\it y} function to {\it take in} the
value of {\it x} in which we are interested.
In this way,
we can compute more than one value of {\it y}.
We do this by {\it passing} in 
an {\it argument}\footnote{
The information that is passed into a function is collectively
known as {\it arguments}.}, in this case, the value of {\it x}.

\begin{verbatim}
    (define (y x)
        (define m 5)
        (define b -3)
        (+ (* m x) b)
        )
\end{verbatim}

Note that we have moved {\it x} from the body of the function
to after the function name. We have also refrained from
giving it a value since its value is to be sent to the function
when the function is called.
What we have done is to {\it parameterize} the function to make it more
general and more useful. The variable {\it x} is now called a
{\it formal parameter}.

Now we can compute {\it y} for an infinite number of {\it x}'s:

\begin{verbatim}
    (y 9)
    -> 42
    
    (y 0)
    -> -3
    
    (y -2)
    -> -13
\end{verbatim}

What if we wish to
compute a {\it y}-value for a given {\it x} for a different
line? One approach would be to pass in the {\it slope} and {\it intercept}
as well as {\it x}:

\begin{verbatim}
 (define (y x m b)
        (+ (* m x) b)
        )
\end{verbatim}

Now we need to pass even more information to {\it y} when we call it:
    
\begin{verbatim}
    (y 9 5 -3)
    -> 42
     
    (y 0 5 -3)
    -> -3
\end{verbatim}

If we wish to calculate using a different line, we just pass in the
new {\it slope} and {\it intercept} along with our value of {\it x}.
This certainly works as intended, but is not the best way. One problem
is that we keep on having to type in the slope and intercept even if
we are computing {\it y}-values on the same line. Anytime you
find yourself doing the same tedious thing over and over,
be assured that
someone has thought of a way to avoid that particular tedium.
If so, how do we
customize our function so that we only have to enter the slope
and intercept once per particular line? We will explore
one way for doing this. In reading further,
it is not important if you understand all that is going on.
What is important is that you know you can use functions
to run similar code over and over again.

\section{Creating functions on the fly}

Since creating functions is hard work (lots of typing) and
Computer Scientists avoid unnecessary work like the plague, somebody
early on got the idea of writing a function that itself 
creates functions! Brilliant! We can do this for our line problem.
We will tell our creative function to create a {\it y} function
for a particular slope and intercept! While we are at it,
let's change the variable names {\it m} and {\it b} to {\it slope}
and {\it intercept}, respectively:

\begin{verbatim}
    (define (createLine slope intercept)
        (define (y x)
            (+ (* slope x) intercept)
            )
        y    ; the value of y is returned, y is NOT CALLED!
        )
\end{verbatim}

The {\it createLine} function creates a {\it y} function
and then returns it. Note that this returned function {\it y} takes
one value when called, the value of {\it x}.

So our creative {\it createLine} function
simply defines a {\it y} function and then
returns it. Now we can create a bunch of different lines:

\begin{verbatim}
    (define a (createLine 5 -3))
    (define b (createLine 6 2))

    (a 9)
    -> 42
    
    (b 9)
    -> 56

    (a 9)
    -> 42
\end{verbatim}

Notice how lines {\it a} and {\it b} remember
the slope and intercept supplied
when they were created.\footnote{
The local function {\it y} does not really remember these values,
but at this point in time, this is a good enough explanation.}
While this is decidedly cool, the problem is many languages (
for example C, C++, and Java\footnote{
C++ and Java, as well as Scam, give you another approach, {\it objects}.
We will discuss objects in a later chapter.})
do not allow you to define functions that create other functions.
Fortunately, Scam, Python, and most functional languages allow this.

While this might seem a little mind-boggling, don't worry. The
things you should take away from this are:

\begin{itemize}
\item
    functions encapsulate calculations
\item
    functions can be parameterized
\item
    functions can be called
\item
    functions return values
\end{itemize}

\chapter{Scam Programs and Using Files}
\label{ScamPrograms}

After a while, it gets rather tedious to
cut and paste into the Scam interpreter.
A more efficient method is to store
your program in a text file and then 
load the file.

I use {\it vim} as my text editor. {\it Vim} is
an editor that was written by programmers
for programmers ({\it emacs} is another such
editor) and serious Computer Scientists
and Programmers should learn {\it vim} (or {\it emacs}).

\section{Your first program}

Create a text file named {\it hello.py}. The
name really doesn't matter and doesn't
have to end in {\it .py} (the {\it .py} is a convention
to remind us this file contains Scam source
code). Place in the file:

\begin{verbatim}
print("hello, world!");
\end{verbatim}

Save your work and exit the text editor.

Now execute the following command at the
system prompt (not the Scam interpreter prompt!):

\begin{verbatim}
    python3 hello.py
\end{verbatim}

You should see the phrase:

\begin{verbatim}
    hello, world!
\end{verbatim}

displayed on your console. Here's a trace using Linux:

\begin{verbatim}
    lusth@warka:~$ python3 hello.py
    hello, world!
    lusth@warka:~$
\end{verbatim}

The {\tt lusth@warka:~\$} is my system prompt.

\section{Vim and Scam}

Move into your home directory and list the files found there
with this command:

\begin{verbatim}
    ls -al
\end{verbatim}

If you see the file .exrc, then all is well and good.
If you do not, then run the following command to
retrieve it:

\begin{verbatim}
    (cd ; wget beastie.cs.ua.edu/cs150/.exrc)
\end{verbatim}

This configures {\tt vim} to understand Scam syntax and to color
various primitives and keywords in a pleasing manner.

\section{A Neat Macro}

One of the more useful things you can do is set up a {\it vim} 
macro. Edit the file {\it .exrc} in your home directory and
find these lines:

\begin{verbatim}
    map @ :!python %^M
    map # :!python % 
    set ai sm sw=4
\end{verbatim}

If you were unable to download the file in the previous section,
just enter the lines above in the {\it .exrc} file.

The first line makes the {\tt '@'} key,
when pressed,
run the Scam interpreter on
the file you are currently editing (save your work first before
tapping the {\tt @} key). The \verb+^M+ part of the macro
is not a two character sequence (\verb+^+ followed by {\tt M}),
but a single character made by typing \verb+<Ctrl>-v+ followed by
\verb+<Ctrl>-m+.
It's just when you type \verb+<Ctrl>-v <Ctrl>-m+, it will display as
\verb+^M+.
The second line defines a similar macro that pauses to let you enter
command-line arguments to your Scam program.
The third line sets some useful parameters:
{\it autoindent} and {\it showmatch}.
The expression {\tt sw=4} sets the indentation to four spaces.

\section{Writing Scam Programs}

A typical Scam program is composed of two sections. The first
section is composed of variable and function definitions.
The next section is composed of statements, which are Scam
expression. Usually the latter section is reduced to
a single function call (we'll see an example in a bit).

The {\it hello.py} file above was a program with no
definitions and a single statement. A Scam program composed
only of definitions will usually run with no output to the
screen. Such programs are usually written for the express
purpose of being included
into other programs.

Typically, one of the function definitions is a function
named {\it main} (by convention); this function takes no
arguments.  The last line of the program (by convention)
is a call to {\it main}.
Here is a rewrite of {\it hello.py} using that convention.

\begin{verbatim}
def main():
    println("hello, world!")

main()
\end{verbatim}

This version's output is exactly the same as the previous version.
We also can see that {\it main} implements the {\it procedure patterrn} since
it has no explicit return value.

\section{Order of definitions}

A function (or variable) must be created or defined\footnote{
From now on, we will use the word defined.}
before it is used. This 
program will generate an error:

\begin{verbatim}
main()           #undefined variable error

def main():
    y = 3
    x = y * y
    print("x is",x)

\end{verbatim}

since {\it main} can't be called  until it is defined.
This program is legal, however:

\begin{verbatim}
def main():
    x = f(3)
    print("x is",x)
def f(z):
    return z * z
main()
\end{verbatim}

because even though the body of {\it main} refers to
function {\it f} before function {\it f} is defined,
function {\it f} is defined by the time
function {\it main} is called (the last statement of the program).

Don't be alarmed if you don't understand what is going on with
this program. But you should be able to type this program into
a file and then run it. If you do so, you should see the
output:

\begin{verbatim}
    x is 9
\end{verbatim}

\section{Importing code}

One can include one file of Scam code into another.
The included file is known as a module. The {\it import}
statement is used to include a module

\begin{verbatim}
from moduleX.py import *
\end{verbatim}

where {\it moduleX.py} is the name of the file containing the Scam definitions
you wish to include.
Usually import statements are placed at the top of the file.
Including a module imports all the code
in that module, as if you had written it 
the file yourself.

If {\it moduleX.py} has import statements, those modules will be
included in the file as well.

Import statements often are used include the standard Scam libraries.

\chapter{Input and Output}
\label{InputAndOutput}

Scam uses a {\it port} system for input and output.
When Scam starts up, the current input port defaults to
{\it stdin} (the keyboard) and the current output
port defaults to {\it stdout} (the screen).

To change these ports, one first creates new port
and then sets the port.
For example, to read from a file (say "data")
instead of the keyboard,
first create a file port:

\begin{verbatim}
    (define p (open "data" 'read))   ; p points to a port
    (define oldInput (setPort p))
    ...                              ; read stuff from the file data
    (setPort oldInput)               ; restore the old input port
\end{verbatim}

Once the port is set, all input will come from the new port.
The {\it setPort} function, in addition to setting the port, returns
the old port so that it eventually can be restored.

To change the output port, the procedure is similar, except
the symbol \verb!'write! is used instead.

\begin{verbatim}
    (define p (open "data" 'write))   ; p points to a port
    (define oldOutput (setPort p))
    ...                               ; write stuff to the file data
    (setPort oldOutput)               ; restore the old output port
\end{verbatim}

Opening a file in \verb!'write! mode overwrites the file;
to append content to an existing file, use the \verb!'append!
symbol instead.

Scam only allows a limited number of ports to be open at
any given time. If you no longer need a port, close it with
the built-in function {\it close}, which takes a port as its
sole argument:

\begin{verbatim}
    (close p)
\end{verbatim}

\section{Reading}

Scam supplies built-in functions for reading characters,
integers, reals,
strings, and whitespace delimited tokens:

\begin{verbatim}
    (assign s (readChar))
    (assign i (readInt))
    (assign r (readReal))
    (assign s (readString))
    (assign t (readToken))
    (assign s (readRawChar))
    (assign u (readUntil stopCharacterString))
    (assign w (readWhile continueCharacterString))
\end{verbatim}

The first five functions listed skip any whitespace preceeding the
entity they are to read. The last three functions do not skip whitespace.

Both the {\it readChar} and the {\it readToken} functions return strings.
Scam uses the same rules as the C programming language
for what characters constitute an integer and a real.
None of these functions take an argument; they use the current
input port.

To read a symbol, use the {\it symbol} function in conjunction
with the {\it readToken} function:

\begin{verbatim}
    s = symbol(readToken());
\end{verbatim}

To read a line of text, use the built-in {\it readLine} function:

\begin{verbatim}
    (assign l (readLine))
\end{verbatim}

The {\it readLine} function reads up to, and including, the next
newline character, but the newline is not part of the
returned string.

The {\it pause} function always reads from {\it stdin},
regardless of the current input port.
It reads (and discards) a line of text (up to and including the newline).
Its purpose is to pause execution of a program for debugging
purposes.

Three other reading functions are useful for scanning text.
The first is {\it readRawChar}, which returns a string containing
the next character in the file, regardless of whether that
character is whitespace or not.
The second is {\it readUntil}, which is passed a string of characters
that is used to control the read. For example,

\begin{verbatim}
    (readUntil " \t\n")
\end{verbatim}

will start reading at the current point in the file
and return a string of all characters read up to point
where a character in a given string is encountered.
The character that caused the read to stop is pushed
back into the input stream and will be the next character
read.

The {\it readWhile} function is analogous, stopping when
a character not in the given string is encountered.

\section{Writing}

Most output functions write to the current output port.

The simplest output function is {\it display}. It takes a single
argument, which can be any Scam object:

\begin{verbatim}
    (display "Hello, world!\n")
\end{verbatim}

The character sequence \verb!\! followed by
\verb!n! indicate that
a newline is to be displayed.

More useful than display are the functions {\it print} and
{\it println} in that they take any number of arguments:

\begin{verbatim}
    (print "(f x) is " (f x) "\n")
    (println "(f x) is " (f x))
\end{verbatim}

The {\it println} function is just like {\it print}, except it
outputs a newline after the displaying the last argument.
Thus, the two calls above produce the same output.

When a string is printed, the quote marks are not displayed.
Likewise, when a symbol is printed, the quote mark is not displayed.

The {\it inspect} function 
prints out the unevaluated
version of its argument followed by the arguments evaluation value:

\begin{verbatim}
    (inspect (f x))
    -> (f x) is 3
\end{verbatim}

The {\it inspect} function always prints to {\it stdout},
regardless of the current output port.

\section{Pretty printing}

The function {\it pp} acts much like {\it display}, unless it is
passed an environment/object. In such cases, it prints out a table
listing the variables defined in that scope.
Since functions, thunks, exceptions, and errors are all encoded
as objects, pp can be used to inspect them in greater detail.
For example, consider this definition of square:

\begin{verbatim}
    (define (square x)
        (* x x)
        )
\end{verbatim}

Printing the value of square using \verb!(print square)! yields:

\begin{verbatim}
    <function square(x)>
\end{verbatim}

In contrast, using \verb!(pp square)! yields:

\begin{verbatim}
    <object 8573>
               label  : closure
             context  : <object 8424>
                name  : square
          parameters  : (x)
                code  : (begin (* x x))
\end{verbatim}

\section{Formatting}

The {\it fmt} function can be used to format numbers and strings
if the default formatting is not acceptable. It uses the C
programming language formatting scheme, taking a formatting
specification as a string, and the item to be formatted.
The function returns a string.

For example,

\begin{verbatim}
    sway>"<" + fmt("%6d",3) + ">";
    STRING: "<     3>"

    sway>"<" + fmt("%-6d",3) + ">";
    STRING: "<3     >"
\end{verbatim}

A format specification begins with a percent sign and is usually followed
by a number representing the width (in number of characters)
of the resulting string. If the width is positive, the
item is right justified in the resulting string. If the width
is negative, the item is left justified.
After any width specification is a character specifying the
type of the item to be formatted:
\verb!d! for an integer,
\verb!f! for a real number, and
\verb!s! for a string.

The format specification is quite a bit more sophisticated
than shown here. You can read more on a Linux system by
typing the command \verb!man 3 printf! at the system prompt.

\section{Testing for end of file}

The {\it eof?} function can be used to test whether the last
read was successful or not. The function is NOT used to
test if the {\it next} read would be successful. Here is a typical
use of {\it eof?} in tokenizing a file:

\begin{verbatim}
    (define t (readToken))
    (while (not(eof?))
        (store t)
        (assign t (readToken))
        )
\end{verbatim}

\section{Pushing back a character}

Sometimes, it is necessary to read one character too
many from the input. This happens in cases like
advancing past whitespace in the input.
Here is a typical whitespace-clearing loop:

\begin{verbatim}
    (define ch (readRawChar))
    (while (space? ch))
        (assign ch (readRawChar))
        )

    ; last character read wasn't whitespace
    ; so push it back to be read again later

    (pushBack ch)
\end{verbatim}

The {\it pushBack} function takes a string as its
sole argument, but only pushes back the first
character of the string; subsequent characters in
the string are ignored.

\chapter{More about Functions}
\label{MoreAboutFunctions}

We have already seen some examples of functions,
some user-defined and some built-in.
For example, we have used the built-in functions,
such as 
{\tt *} and defined our own functions,
such as {\it square}.
In reality, {\it square} is not a function, per se, but a variable
that is bound to the function that multiplies two numbers
together. It is tedious to say `the function bound to
the variable {\it square}', however,
so we say the more concise (but technically incorrect)
phrase `the {\it square} function'.

\section{Built-in Functions}

Scam has many built-in, or {\it predefined}, functions.
No one, however,
can anticipate all possible tasks that someone might want to perform,
so most programming languages allow the user to define new functions.
Scam is no exception and provides
for the creation of new and novel functions.
Of course,
to be useful,
these functions should be able to call
built-in functions as well as other programmer created
functions.

For example, a function that determines whether a given
number is odd or even is not built into Scam but can be
quite useful in certain situations.
Here is a definition
of a function named {\it isEven} which returns true if the
given number is even, false otherwise:

\begin{verbatim}
    >>> def isEven(n):
    ...     return n % 2 == 0
    ...
    >>>

    >>> isEven(42)
    True

    >>> isEven(3)
    False

    >>> isEven(3 + 5)
    True
\end{verbatim}

We could spend days talking about about what's going on in these
interactions with the interpreter. First, let's talk
about the syntax of a function definition. Later, we'll
talk about the purpose of a function definition. Finally,
will talk about the mechanics of a function definition
and a function call.

\section{Function syntax}

Recall that the words of a programming language include its
primitives, keywords and variables. A function definition
corresponds to a sentence in the language in that it is
built up from the words of the language. And like human
languages, the sentences must follow a certain form. This
specification of the form of a sentence is known as its
{\it syntax}. Computer Scientists often use a special way
of describing syntax of a programming language called the
Backus-Naur form (or {\sc bnf}). Here is a high-level description
of the syntax of a Scam function definition using {\sc bnf}:

\begin{verbatim}
    functionDefinition : signature ':' body

    signature : 'def' variable '(' optionalParameterList ')'

    body : block

    optionalParameterList : *EMPTY*
                          | parameterList
    
    parameterList : variable
                  | variable ',' parameterList
    
    block:  definition 
          | definition block
          | statement
          | statement block
\end{verbatim}

The first {\sc bnf} {\it rule} says that a function definition is
composed of two pieces, a signature and a body, separated
by the colon character
(parts of the rule that appear verbatim appear within single quotes).
The signature starts
with the keyword {\it def}
followed by a variable,
followed by an open parenthesis, followed by something
called an {\it optionalParameterList}, and finally followed by a close
parenthesis.
The body of a function 
something called a {\it block},
which is composed of {\it definitions} and {\it statements}.
The {\it optionalParameterList} rule tells us that
the list of formal parameters can possibly be empty,
but if not, is composed of a list of variables
separated by commas.

As we can see from the {\sc bnf} rules,
parameters are variables that will be bound
to the values supplied in the function call.
In the particular case of {\it isEven}, from the
previous section,
the variable {\it x} will be bound to the number whose
evenness is to be determined. As noted earlier,
it is customary to call {\it x}
a {\it formal parameter} of the function {\it isEven}.
In function calls, the values to be bound to the
formal parameters are called {\it arguments}.

\section{Function Objects}

Let's look more closely at the body of {\it isEven}:

\begin{verbatim}
    def isEven(x):
        return x % 2 == 0
\end{verbatim}

The \% operator is bound to the remainder or modulus
function. The {\tt ==} operator is bound to the equality function
and determines whether the value of the left operand
expression is equal to the value of the right operand
expression, yielding true or false as appropriate. The
{\sc boolean} value produced by {\tt ==} is then immediately returned as the
value of the function.

When given a function definition like that above, Scam
performs a couple of tasks. The first is to create the
internal form of the function, known as a {\it function object},
which holds the function's signature and body.
The second task is to bind
the function name to the function object so that it
can be called at a later time.
Thus, the name
of the function is simply a variable that happens to be
bound to a function object. As noted before, we often say
'the function {\it isEven}' even though we really mean 'the
function object bound to the variable {\it even}?'.

The value of a function definition is the
function object; you can see this by
printing out the value of {\it isEven}:

\begin{verbatim}
    >>> print(isEven)
    <function isEven at 0x9cbf2ac>

    >>> isEven = 4
    >>> print(isEven)
    4

\end{verbatim}

Further interactions with the interpreter provide evidence
that {\it isEven} is indeed a variable; we can reassign its value,
even though it is considered in poor form to do so.

\section{Calling Functions}

Once a function is created,
it is used by {\it calling} the
function with {\it arguments}.
A function is called by supplying
the name of the function followed by a parenthesized,
comma separated, list of expressions.
The arguments are
the values that the formal parameters will receive.
In Computer Science speak, we say that the values
of the arguments are to be bound to
the
formal parameters.
In general, if there are {\it n} formal parameters,
there should be {\it n} arguments.\footnote{
For {\it variadic} functions, which Scam
allows for, the number of arguments
may be more or less than the number of formal parameters.
}
Furthermore, the value of the
first argument is bound to the first formal parameter, the
second argument is bound to the second formal parameter,
and so on. Moreover, all the arguments are evaluated
before being bound to any of the parameters.

Once the evaluated arguments are bound to the parameters,
then the body of the function is evaluated. Most times,
the expressions in the body of the function will reference
the parameters. If so, how does the interpreter find the
values of those parameters? That question is answered in
the next chapter.

\section{Returning from functions}

The return value of a function is the value of the expression
following the {\tt return} keyword.
For a function to return this  expression, however,
the return has to be {\it reached}.
Look at this example:

\begin{verbatim}
def test(x,y):
    if (y == 0):
        return 0
    else:
        print("good value for y!")
        return x / y

    print("What?")
    return 1
\end{verbatim}

Note that the {\it ==} operator returns true if the
two operands have the same value.
In the function, if {\it y} is zero, then the
\begin{verbatim}
    return 0
\end{verbatim}

statement is reached.
This causes an
immediate return from the function and
no other expressions in the function body are evaluated.
The return value, in this case, is zero.
If {\it y} is not equal to zero,
a message is printed and 
the second return is reached, again causing
an immediate return. In this case,
a quotient is returned.

Since both parts of the if statement have
returns, then the last two lines of the
function:

\begin{verbatim}
    print("What?")
    return 1
\end{verbatim}

are {\it unreachable}. Since they
are unreachable, they cannot be
executed under any conditions and
thus serve no purpose and can be deleted.

\chapter{Scope}
\label{Scope}

A {\it scope} holds the current set of variables and their values.
In Scam, there is something called the {\it global scope}.
The global scope holds all the values of the built-in variables
and functions (remember, a function name is just a variable).

When you enter the Scam interpreter, either by running it
interactively or by using it to evaluate a program in
a file, you start out in the global scope.
As you define variables, they and their values are added
to the global scope.

This interaction adds the variable {\it x} to the global scope:

\begin{verbatim}
    >>> x = 3
\end{verbatim}

This interaction adds two more variables to the global scope:

\begin{verbatim}
    >>> y = 4
    >>> def negate(z):
    ...     return -z;
    ...
    >>>
\end{verbatim}

What are the two variables?
\T The two variables added to the global scope are {\it y} and {\it negate}.
\W Highlight the following line to see the answer:

\W\begin{quote}
\W {\it The two variables added to the global scope are} {\color{white} y and negate.}
\W\end{quote}

Indeed, since the name of a function is a variable and the variable {\it negate}
is being bound to a function object, it becomes clear that
this binding is occurring in the
global scope, just like {\it y} being bound to 4.

Scopes in Scam can be identified by their indentation level.
The global scope holds all variables defined with an indentation
level of zero.
Recall that when functions are defined, the body of the function
is indented. This implies that variables defined in the function body
belong to a different scope and this is indeed the case.
Thus we can identify to which scope a variable belongs by
looking at the pattern of indentations.
In particular,
we can label variables as 
either {\it local} or {\it non-local}
with respect to a particular scope.
Moreover, non-local variables may be {\it in scope} or
or {\it out of scope}.

\section{In Scope or Out}

The indentation pattern of a program can tells us where
variables are visible (in scope) and where they are
not (out of scope).
We begin by first learning to recognizing the scopes in which variables
are defined.

\subsection{The Local Variable Pattern}

All variables {\it defined} at a particular
indentation level or scope are considered
{\it local} to that indentation level or scope.
In Scam, if one assigns a value to a variable, that variable
must be local to that scope.
The only exception is if the variable was
explicitly declared {\it global} (more on that later).
Moreover, the formal parameters of a function definition
belong to the scope that is identified with
the function body. 
So within a function body, the local variables are the formal
parameters plus any variables defined in the function body.

Let's look at an example.
Note,
you do not need to completely understand the examples presented in
the rest of the chapter in order 
to identify the local and non-local variables.

\begin{verbatim}
def f(a,b):
    c = a + b
    c = g(c) + X
    d = c * c + a
    return d * b
\end{verbatim}

In this example, we can immediately say the 
formal parameters,
{\it a} and {\it b},
are local with respect to the scope of the body of function {\it f}.
Furthermore, variables
{\it c} and {\it d}
are defined in the function body
so they are local as well,
with respect to the scope of the body of function {\it f}.
It is rather wordy to say ``local with respect to the
scope of the body of the function {\it f}'', so Computer Scientists
will almost always shorten this to ``local with respect to {\it f}''
or just ``local''
if it is clear the discussion is about a particular function or scope.
We will use this shortened phrasing from here on out.
Thus {\it a}, {\it b}, {\it c}, and {\it d} are local with respect to {\it f}.
The variable {\it f} is local to the global scope since the function
{\it f} is defined in the global scope.

\subsection{The Non-local Variable Pattern}

In the previous section, we determined the local
variables of the function.
By the process 
of elimination, that means the variables
{\it g}, and {\it X} are non-local.
The name of function itself is non-local with respect
to its body, {\it f} is non-local as well.

Another way of making this determination is
that
neither {\tt g} nor {\tt X} are assigned values
in the function body. Therefore, they must be non-local.
In addition, should a variable be explicitly declared {\it global},
it is non-local even if it is assigned a value.
Here again is an example:

\begin{verbatim}
def h(a,b):
    global c
    c = a + b
    c = g(c) + X
    d = c * c + a
    return d * b
\end{verbatim}

In this example, variables {\it a}, {\it b}, and {\it d} are local with respect
to {\it h} while {\it c}, {\it g}, and {\it X} are non-local.
Even though {\it c} is assigned a value, the declaration:

\begin{verbatim}
    global c
\end{verbatim}

means that {\it c} belongs to a different scope (the global scope) and thus
is non-local.

\subsection{The Accessible Variable Pattern}

A variable is accessible with respect to
a particular scope if it is {\it in scope}.
A variable is in scope if it is local or
was defined in a scope that
{\it encloses} the particular scope.
Some scope
{\it A} encloses some other scope {\it B}
if, by moving (perhaps repeatedly) leftward from
scope B, scope A can be reached.
Here is example:

\begin{verbatim}
Z = 5

def f(x):
   return x + Z

print(f(3))
\end{verbatim}

The variable {\it Z} is local with respect to the global scope
and is non-local with respect to {\it f}. However, we can
move leftward from the scope of {\it f} one indentation level and
reach the global scope where {\it Z} is defined.
Therefore, the global scope encloses the scope of {\it f} and
thus {\it Z} is accessible from {\it f}.
Indeed, the global scope encloses all other scopes and this
is why the built-in functions are accessible at any indentation
level.

Here is another example that has two enclosing scopes:

\begin{verbatim}
X = 3
def g(a)
   def m(b)
      return a + b + X + Y
   Y = 4
   return m(a % 2)

print(g(5))
\end{verbatim}

If we look at function {\it m}, we see that there is only
one local variable, {\it b}, and that {\it m} references three
non-local variables,
{\it a}, {\it X}, and {\it Y}. 
Are these non-local variables accessible?
Moving leftward from the body of {\it m}, we reach the body of {\it g},
so the scope of {\it g} encloses the scope of {\it m}. The local variables
of {\it g} are {\it a}, {\it m}, and {\it Y}, so both {\it a} and {\it Y}
are accessible in the scope of {\it m}.
If we move leftward again, we reach the global scope,
so the global scope encloses the scope of {\it g}, which in turn encloses
the scope of {\it m}. By transitivity, the global scope encloses
the scope of {\it m}, so {\it X}, which is defined in the global scope
is accessible to the scope of {\it m}.
So, all the non-locals of {\it m} are accessible to {\it m}.

In the next section, we explore how a variable can be
inaccessible.

\subsection{The Tinted Windows Pattern}

The scope of local variables is like a car with tinted
windows, with the variables defined within riding in
the back seat.
If you are outside the scope, you cannot
peer through the car windows  and see those variables.
You might try and buy some x-ray glasses, but they
probably wouldn't work.
Here is an example:

\begin{verbatim}
    z = 3

    def f(a):
        c = a + g(a)
        return c * c

    print("the value of a is",a) #x-ray!
    f(z);
\end{verbatim}

The print statement causes an error:

\begin{verbatim}
    Traceback (most recent call last):
      File "xray.py", line 7, in <module>
          print("the value of a is",a) #x-ray!
          NameError: name 'a' is not defined
\end{verbatim}

If we also tried to print the value of {\it c},
which is a local variable of function {\it f}, at that
same point in the program, we would get a similar error.

The rule for figuring out which variables are in scope and
which are not is:
{\it you} {\bf cannot} {\it see into an enclosed scope}.
Contrast this with the non-local pattern:
{\it you} {\bf can} {\it see variables
declared in enclosing outer scopes}.

\subsection{Tinted Windows with Parallel Scopes}

The tinted windows pattern also applies to parallel scopes.
Consider this code:

\begin{verbatim}
    z = 3

    def f(a):
        return a + g(a)

    def g(x):
        # starting point 1
        print("the value of a is",a) #x-ray!
        return x + 1

    f(z);
\end{verbatim}

Note that the global scope encloses both the scope of {\it f} and
the scope of {\it g}. However, the scope of {\it f} does
not enclose the scope of {\it g}. Neither does
the scope of {\it g} enclose the scope of {\it f}.

One of these functions references a variable that is not in scope.
Can you guess which one?
\T The function {\it g} references a  variable not in scope.
\W Highlight the following line to see the answer:

\W\begin{quote}
\W {\it The function} {\color{white} g} {\it references variable not in scope.}
\W\end{quote}

Let's see why by first examining {\it f} to see whether or
not its non-local references are in scope.
The only local variable of function {\it f} is
{\it a}. The only referenced non-local is {\it g}.
Moving leftward from the body of {\it f}, we reach the
global scope where where both {\it f} and {\it g} are defined.
Therefore, {\it g}
is visible with respect to {\it f} since it is defined in a scope
(the global scope) that encloses {\it f}.

Now to investigate {\it g}. The only local variable of
{\it g} is {\it x}
and the only non-local that {\it g} references is {\it a}.
Moving outward to the global scope, we see that there is
no variable {\it a} defined there,
therefore the variable {\it a} is not in scope with
respect to {\it g}.

When we actually run the code,
we get an error similar to the following when running this program:

\begin{verbatim}
    Traceback (most recent call last):
      File "xray.py", line 11, in <module>
        f(z);
      File "xray.py", line 4, in f
        return a + g(a)
      File "xray.py", line 8, in g
        print("the value of a is",a) #x-ray!
    NameError: global name 'a' is not defined
\end{verbatim}

The lesson to be learned here is
that we cannot see into
the local scope of the body of function {\it f},
{\it even if we are at a similar nesting level}.
Nesting level doesn't matter. We can only see variables
in our own scope and those in {\it enclosing} scopes.
All other variables cannot be seen.

Therefore, if you ever see a variable-not-defined error,
you either have spelled the variable name wrong, you haven't
yet created the variable, or you are trying to use x-ray vision
to see somewhere you can't. 

\section{Alternate terminology}

Sometimes, enclosed scopes are referred to as {\it inner} scopes while
enclosing scopes are referred to as {\it outer} scopes. In addition,
both locals and any non-locals found in enclosing scopes are considered
{\it visible} or {\it in scope}, while non-locals that are not
found in an enclosing scope are considered {\it out of scope}.
We will use all these terms in the remainder of the text book.

\section{Three Scope Rules}

Here are three simple rules you can use to help you
figure out the scope of a particular variable:

\begin{itemize}
\item
        Formal parameters belong in
\item
        The function name belongs out
\item
        You can see out but you can't see in (tinted windows).
\end{itemize}

The first rule is shorthand for the fact that formal parameters
belong to the scope of the function body. Since the function body
is `inside' the function definition, we can say the formal parameters
belong in.

The second rule reminds us that as we move outward from a function body,
we find the enclosing scope holds the function definition. That is to
say, the function name is bound to a function object in the scope
enclosing the function body.

The third rule tells us all the variables that belong to 
ever-enclosing scopes are accessible and therefore
can be referenced by the innermost scope. The opposite is
not true. A variable in an enclosed scope can not be referenced
by an enclosing scope. If you forget the directions of this
rule, think of tinted windows. You can see out of a tinted
window, but you can't see in.

\section{Shadowing}

The formal parameters of a function can be thought of
as variable definitions that are only in effect when
the body of the function is being evaluated. That is,
those variables are only visible in the body and no where
else. This is why formal parameters are considered to be
{\it local} variable definitions, since they only have local
effect (with the locality being
the function body). Any direct reference to those
particular variables outside the body of the function
is not allowed (Recall that you can't see in).
Consider the following interaction with
the interpreter:

\begin{verbatim}
    >>> def square(a):
    ...     return a * a
    ...
    >>>
     
    >>> square(4)
    16
     
    >>> a
    NameError: name 'a' is not defined
\end{verbatim}


In the above example, the scope of variable {\it a} is restricted
to the body of the function {\it square}.
Any reference to
{\it a} other than in the context of {\it square} is invalid. Now
consider a slightly different interaction with the
interpreter:

\begin{verbatim}
    >>> a = 10
    >>> def almostSquare(a):
    ...     return a * a + b
    ...
    >>> b = 1
     
    >>> almostSquare(4)
    17

    >>> a
    10
\end{verbatim}

In this dialog, the global scope has three
variables added, {\it a}, {\it almostSquare} and
{\it b}.
In addition, the variable serving as the formal parameter
of {\it almostSquare} has the same name as the first variable
defined in the dialog. Moreover, the body of {\it almostSquare}
refers to both variables {\it a} and {\it b}. To which {\it a} does
the body of almostSquare refer? The global {\it a} or the local {\it a}?
Although it seems confusing at first,
the Scam interpreter has no difficulty in figuring out
what's what. From the responses of the interpreter,
the {\it b} in the body must refer to the variable that was
defined with an initial value of one. This is consistent with
our thinking, since {\it b} belongs to the enclosing scope
and is accessible within the body of {\it almostSquare}.
The {\it a} in the function body must refer to the
formal parameter whose value was set to 4 by the call to
the function (given the output of the interpreter).

When a local variable has the same name as a non-local variable
that is also in scope, the local variable is said to
{\it shadow} the non-local version. The term shadowed refers to the fact
that the other variable is in the shadow of the local
variable and cannot be seen. Thus, when the
interpreter needs the value of the variable, the value of
the local variable is retrieved.
It is also possible for a non-local variable to shadow
another non-local variable. In this case, the variable
in the nearest outer scope shadows the variable in
the scope further away.

In general, when a variable is referenced,
Scam first looks in the local
scope.
If the variable is not found there,
Scam looks in the enclosing scope.
If the variable is not there,
it looks in the scope enclosing the enclosing scope, 
and so on.

In the particular example,
a reference to {\it a} is made when the body of
{\it almostSquare} is executed. The value of {\it a}
is immediately found in the local scope.
When the value of {\it b} is required,
it is not found in the local scope. The interpreter
then searches the enclosing scope (which in this case happens
to be the global scope).
The global scope does hold {\it b} and its value, so a value
of 1 is retrieved.

Since {\it a} has a value of 4 and {\it b} has a value of 1, the
value of 17 is returned by the function. Finally, the
last interaction with the interpreter illustrates the
fact that the initial binding of {\it a} was unaffected by the
function call.

\section{Modules}

Often, we wish to use code that has already been written.
Usually, such code contains handy functions that have
utility for many different projects. In Scam, such
collections of functions are known as modules.
We can include modules into our current project with
the {\it import} statement, which we saw in
\link*{the chapter on functions}[Chapters~\Ref]{Functions}
and
\link*{the chapter on programs and files}[\Ref]{ScamPrograms}.

The import statement has two forms. The first is:

\begin{verbatim}
    from ModuleX import *
\end{verbatim}

This statement imports all the definitions from ModuleX and places
them into the global scope. At this point, those definitions
look the same as the built-ins, but if any of those definitions
have the same name as a built-in, the built-in is shadowed.

The second form looks like this:

\begin{verbatim}
    import ModuleX
\end{verbatim}

This creates a new scope that is separate from the global scope
(but is enclosed by the global scope).
Suppose
{\it ModuleX} has a definition for variable {\it a}, with a value
of 1.
Since {\it a} is in a scope
enclosed by the global scope, it is inaccessible from the global
scope (you can't see in):

\begin{verbatim}
    >>> import ModuleX
    >>> a
    NameError: name 'a' is not defined
\end{verbatim}

The direct reference to {\it a} failed, as expected.
However, one can get to {\it a} and its value {\it indirectly}:

\begin{verbatim}
    >>> import ModuleX
    >>> ModuleX . a
    1
\end{verbatim}

This new notation is known as {\it dot} notation and is commonly
used in object-oriented programming systems to references pieces
of an object. For our purposes, {\it ModuleX} can be thought
of as
a {\it named} scope and the {\it dot} operator is used to look
up variable {\it a} in the scope named ModuleX.

This second form of import is used when the possibility that
some of your functions or variables have the same name as
those in the included module. Here is an example:

\begin{verbatim}
    >>> a = 0
    >>> from ModuleX import *

    >>> a
    1
\end{verbatim}

Note that the {\it ModuleX}'s variable {\it a} has showed the previously
defined {\it a}. With the second form of import, the two versions
of {\it a} can each be referenced:

\begin{verbatim}
    >>> a = 0
    >>> import ModuleX

    >>> a
    0
    >>> ModuleX . a
    1
\end{verbatim}

\chapter{Recursion}
\label{Recursion}

In 
\link*{the chapter on conditionals}[Chapter~\Ref]{Conditionals},
we learned about conditionals.
When we combine conditionals with functions that call
themselves, we obtain a powerful programming paradigm
called {\it recursion}.

Recursion is a form of looping; when we loop,
we evaluate code over and over again. To stop the loop,
we use a conditional.
Recursive loops are often
easier to write and understand, as compared to the iterative loops
such as {\it while}s and {\it for}s, which you will learn
about in the next chapter.

Many mathematical functions are easy to implement recursively, so
we will start there. Recall that the factorial of a number
{\it n} is:

    \[ n! = n * (n - 1) * (n - 2) * ... * 2 * 1 \]

Consider writing a function which computes the factorial of a positive
integer. For example, if the function were passed the value of 4, it
should return the value of 24 since 4! is 4*3*2*1 or 24.
To apply
recursion to solve this problem or any problem, for that matter,
it must be possible to state the solution
of a problem so that it references a simpler version
of the problem. For factorial, the factorial of a
number can be stated in terms of a simpler factorial.

    \[ 0! = 1 \]
    \[ n! = n * (n - 1)! \] otherwise

This second formulation states that the factorial of zero is one\footnote{
Mathematicians, being an inclusive bunch, like to invite
zero to the factorial party.
}
and that the factorial of any other (positive) number is obtained by
multiplying the that number by the factorial of one less than that
number. After some study, you should be able to see that both
the first formulation (with the ellipses ...) and this new
formulation are equivalent.\footnote{
The first formulation gets the basic idea of factorial
across but is not very precise. For example, how would you compute
the factorial of three using the first formulation?
}
The second form is particularly well suited for implementation
as a computer program:

\begin{verbatim}
      def factorial(n):
          if (n == 0):
              return 1
          else:
              return n * factorial(n - 1)
\end{verbatim}

Note how  the {\it factorial} function
precisely implements the second formulation.

Convince yourself that the function really works by tracing the function call:

\begin{verbatim}
    factorial(3)
\end{verbatim}

Decomposing the call, we find that:

\begin{verbatim}
    factorial(3) is 3 * factorial(2)
\end{verbatim}

since {\it n}, having a value of 3, is not equal to 0.
and so the second block of the {\it if} is evaluated.
We can replace \verb!factorial(2)!
by \verb!2 * factorial(1)!, yielding:

\begin{verbatim}
    factorial(3) is 3 * 2 * factorial(1)
\end{verbatim}

since {\it n}, now having a value of 2, is still not zero.
Continuing along this vein, we can replace
\verb!factorial(1)! by \verb!1 * factorial(0)!, yielding:

\begin{verbatim}
    factorial(3) is 3 * 2 * 1 * factorial(0)
\end{verbatim}

Now in this call to factorial,
{\it n} does have a value of zero, so we can replace
\verb!factorial(0)! with its immediate return value of zero:

\begin{verbatim}
    factorial(3) is 3 * 2 * 1 * 1
\end{verbatim}

Thus, \verb!factorial(3)! has a value of six:

\begin{verbatim}
    >>> factorial(3)
    6
\end{verbatim}

as expected.

\section{The parts of a recursive function}

Recursive approaches rely on the fact that it is
usually simpler to solve a smaller problem than a
larger one. In the factorial problem, trying to
find the factorial $n - 1$ is a little bit simpler
than finding the factorial of $n$. If finding the
factorial of $n - 1$ is still too hard to solve easily,
then find the factorial of $n - 2$ and so on until
we find a case where the solution is easy.
With regards to factorial, this is when $n$ is equal to
zero. The {\it easy-to-solve} code (and the values that get
you there) is known as the {\it base} case. The
{\it find-the-solution-using-a-simpler-problem} code (and the values
that get you there) is known as the {\it recursive} case.
The recursive case usually contains a call to the very
function being executed. This call is known as a
{\it recursive} call.

Most well-formed recursive functions are composed of
at least one {\it base} case and at least one {\it recursive} case.

\section{The greatest common divisor}

Consider finding the greatest common divisor (gcd) of two numbers. One
The Ancient Greek Philosopher Euclid devised a solution involving
repeated division.
The first division divides the two
numbers in question, saving the remainder. Now the divisor becomes
the dividend and the
remainder becomes the divisor. This process is repeated until
the remainder becomes zero.
At that
point, the current divisor is the gcd.
We can specify this as a recurrence equation, with this last bit about
the remainder becoming zero as our base case:

\begin{center}
\begin{tabular}{lcll}
\T\toprule
    $gcd$($a$,$b$) & is & $b$ & if $a$ divided by $b$ has a remainder of zero\\
    $gcd$($a$,$b$) & is & $gcd$($b$,$a$ \% $b$) & otherwise \T\\
\T\bottomrule
\end{tabular}
\end{center}

where {\it a} and {\it b} are the dividend and the divisor, respectively.
Recall that the modulus operator \verb!%! returns the remainder.
Using the recurrence equation as a guide, we can easily implement
a function for computing the gcd of two numbers.

\begin{verbatim}
    def gcd(dividend,divisor):
        if (dividend % divisor == 0):
            return divisor
        else:
            gcd(divisor,dividend % divisor)
\end{verbatim}

We can improve this function slightly, by noting that
the remainder is computed again to make the recursive call.
Rather than compute it twice, we compute it straight off
and save it in an aptly named variable:

\begin{verbatim}
    def gcd(dividend,divisor):
        remainder = dividend % divisor
        if (remainder == 0):
            return divisor
        else:
            return gcd(divisor,remainder)
\end{verbatim}

Look at how the recursive version turns the {\it divisor} into 
the {\it dividend}
by passing {\it divisor} as the first argument in the recursive
call.
By the same token, {\it remainder} becomes {\it divisor} by nature
of being the second argument in the recursive call.
To convince yourself that the routine really works,
modify {\it gcd} to `visualize' the arguments. On simple
way of visualizing the action of a function is to
add a print statement:

\begin{verbatim}
    def gcd(dividend,divisor):
        remainder = dividend % divisor
        print("gcd:",dividend,divisor,remainder)
        if (remainder == 0):
            return divisor
        else:
            return gcd(divisor,remainder)
\end{verbatim}

After doing so, we get the following output:

\begin{verbatim}
    >>> gcd(66,42)
    gcd: 66 42 24
    gcd: 42 24 18
    gcd: 24 18 6
    gcd: 18 6 0
    INTEGER: 6
\end{verbatim}

Note, how the first remainder, 24, keeps shifting
to the left.
In the first recursive call, the remainder becomes 
{\it second}, so the 24 shifts one spot to the left. On
the second recursive call, the current {\it divisor},
which is 24,
becomes the {\it dividend},
so the 24 shifts once again to
the left.

\section{The Fibonacci sequence}

A third example of recursion is the computation of the
$n^{th}$ Fibonacci number.
The Fibonacci series looks like this:
 
\begin{verbatim}
    n       0   1   2   3   4   5   6   7   8    ...
    Fib(n)  0   1   1   2   3   5   8  13  21    ...
\end{verbatim}

and is found in nature again and again.\footnote{
Pineapples, the golden ratio, chambered nautilus, etc.
}
In general, a Fibonacci number is equal to the sum of the previous
two Fibonacci numbers. The exceptions are the zeroeth and the first
Fibonacci numbers which are equal to 0 and 1 respectively. Voila! The
recurrence case and the two base cases have jumped right out at us! Here,
then is a recursive implementation of a function which computes the $n^{th}$
Fibonacci number.

\begin{center}
\begin{tabular}{lcll}
\T\toprule
    {\it fib}({\it n}) & is & 0 & if {\it n} is zero\\
    {\it fib}({\it n}) & is & 1 & if {\it n} is one\\
    {\it fib}({\it n}) & is & {\it fib}({\it n} - 1) + {\it fib}({\it n} - 2) & otherwise\\
\T\bottomrule
\end{tabular}
\end{center}

Again, it is straightforward to convert the recurrence equation
into a working function:
 
\begin{verbatim}
    # compute the nth Fibonacci number
    # n must be non-negative!
    
    def fibonacci(n):
        if (n == 0):
            return 0
        elif (n == 1):
            return 1
        else:
            return fibonacci(n-1) + fibonacci(n-2)
\end{verbatim}

Our implementation is straightforward and elegant. Unfortunately, it's
horribly inefficient. Unlike our recursive version of
{\it factorial} which
recurred about as many times as the size of the number sent to the function,
our Fibonacci version will recur many, many more times than the size of
its input.  Here's why.

Consider the call to \verb!fib(6)!.
Tracing all the recursive calls to {\it fib}, we get:

\begin{verbatim}
    fib(6) is fib(5) + fib(4)
\end{verbatim}

Replacing \verb!fib(5)! with \verb!fib(4) + fib(3)!,
we get:

\begin{verbatim}
    fib(6) is fib(4) + fib(3) + fib(4)
\end{verbatim}

We can already see a problem, we will compute \verb!fib(4)! twice,
once from the original call to \verb!fib(6)! and again when we
try to find \verb!fib(5)!.
If we write down all the recursive calls generated by \verb!fib(6)!
with each recursive call indented from the previous, we
get a structure that looks like this:

\begin{verbatim}
    fib(6)
        fib(5)
            fib(4)
                fib(3)
                    fib(2)
                        fib(1)
                        fib(0)
                    fib(1)
                fib(2)
                    fib(1)
                    fib(0)
            fib(3)
                fib(2)
                    fib(1)
                    fib(0)
                fib(1)
        fib(4)
            fib(3)
                fib(2)
                    fib(1)
                    fib(0)
                fib(1)
            fib(2)
                fib(1)
                fib(0)
\end{verbatim}

We would expect, based on how the Fibonacci sequence is
generated,
to take about six 'steps' to calculate \verb!fib(6)!. 
Instead,
ultimately there were 13 calls to either
\verb!fib(1)! or \verb!fib(0)!.\footnote{
13 is $7^{th}$ Fibonacci number. Curious!
}
There was a tremendous amount of duplicated
and, therefore, wasted effort. An important part
of Computer Science is how to reduce the wasted effort.
One way is to cache previously computed values.\footnote{
Another way is to use an iterative loop. You will learn
about loops in the next chapter.}

\section{Manipulating lists with recursion}

Recursion and lists go hand-in-hand. What follows
are a number of recursive patterns involving lists
that you should be able to recognize and implement.

For the following discussion, assume the 
{\it head} function returns the first item in the given list,
while the
{\it tail}
function returns a list composed of all the items
in the given list except for the first element.
If the list is empty, it will have a value of 
\verb![]!.

By the way, the head and tail functions are easily implemented
in Scam:

\begin{verbatim}
    def head(items): return items[0]
    def tail(items): return items[1:]  #slicing!
\end{verbatim}

\section{The {\it counting} pattern}

The {\it counting} pattern is used to count the number of items
in a collection. If a list is empty, then its count of items
is zero.
The following function
counts and ultimately returns the number of items in the list:

\begin{verbatim}
    def count(items):
        if (items == []):
            return 0
        else:
            return 1 + tail(items)
\end{verbatim}

The functions works on the observation that if you count
the number of items in the tail of a list, then the number
of items in the entire list is one plus that number. The
extra one accounts for the head item that was not counted when the
tail was counted.

\section{The {\it accumulate} pattern}

The {\it accumulate} pattern is used to combine all the values in
a list.
The following function performs a summation of the list values:

\begin{verbatim}
    def sum(items):
        if (items == []):
            return 0
        else:
            return head(items) + sum(tail(items))
\end{verbatim}

Note that the only difference between the {\it count} function and
the {\it sum} function is the recursive case adds in the value of
the head item, rather than just counting the head item.
That the function count and the function sum fact look similar
is no coincidence. In fact, most recursive functions, especially
those working on lists, look very
similar to one another.

\section{The {\it filtered-count} and {\it filtered-accumulate} patterns}

A variation on the {\it counting} and {\it accumulate} 
patterns involves {\it filtering}. When filtering,
we use an additional \verb!if! statement to decide whether 
or not we should count the item, or in the case of accumulating,
whether or not the item ends up in the accumulation.

Suppose we wish to count the number of even items in
a list:

\begin{verbatim}
    def countEvens(items):
        if (items == []):
            return 0
        elif (head(items) % 2 == 0):
            return 1 + countEvens(tail(items))
        else:
            return 0 + countEvens(tail(items))
\end{verbatim}

The base case states that there are zero even numbers
in an empty list.
The first recursive case simply counts the head item if
it is even and so adds 1 to the count of even items in
the remainder of the list. The second recursive case
does not count the head item as even (because it is not)
and so adds in a 0 to the count of the remaining items.
Of course, the last return would almost always be written
as:

\begin{verbatim}
    return countEvens(tail(items))
\end{verbatim}

As another example of filtered counting, we can
pass in a value and count how many times that
value occurs:

\begin{verbatim}
    def occurrences(target,items):
        if (items == []):
            return 0
        elif (head(items) == target):
            return 1 + occurrences(target,tail(items))
        else:
            return occurrences(target,tail(items))
\end{verbatim}

An example of a filtered-accumulation would be 
to sum the even-numbered integers in a list:

\begin{verbatim}
    def sumEvens(items):
        if (items == []):
            return []
        elif (isEven(head(items))):
            return head(items) + sumEvens(tail(items))
        else:
            return sumEvens(tail(items))
\end{verbatim}

where the {\it isEven} function is defined as:

\begin{verbatim}
    def isEven(x):
        return x % 2 == 0
\end{verbatim}

\section{The {\it filter} pattern}

A special case of a filtered-accumulation is called {\it filter}.
Instead of summing the filtered items (for example), we collect
the filtered items into a list. The new list is said to be
a {\it reduction} of the original list.

Suppose we wish to extract the even numbers from a list. The
code looks very much like the {\it sumEvens} function in the previous
section, but instead of adding in the desired item,
we make a list out of it and concatenate to it the
reduction of the tail of the list:

\begin{verbatim}
    def extractEvens(items):
        if (items == []):
            return []
        elif (isEven(head(items))):
            return [head(items)] + extractEvens(tail(items))
        else:
            return extractEvens(tail(items))
\end{verbatim}

Given a list of integers, {\it extractEvens} returns a (possibly empty)
list of the even numbers:

\begin{verbatim}
    >>> extractEvens([4,2,5,2,7,0,8,3,7])
    [4, 2, 2, 0, 8]

    >>> extractEvens([1,3,5,7,9])
    []
\end{verbatim}

\section{The {\it map} pattern}

{\it Mapping} is a task closely coupled with that
of reduction, but rather than collecting
certain items, as with the {\it filter} pattern, we
collect all the items. As we collect, however,
we transform each item as
we collect it. The basic pattern looks like this:

\begin{verbatim}
    def map(f,items):
        if (items == []): 
            return []
        else:
            return [f(head(items))] + map(f,tail(items))
\end{verbatim}

Here, function {\it f} is used to transform each item in the
recursive step.

Suppose we wish to subtract one from each element in a list.
First we need a transforming function that reduces its argument
by one:

\begin{verbatim}
    def decrement(x): return x - 1
\end{verbatim}

Now we can ``map'' the {\it decrement} function over a list of numbers:

\begin{verbatim}
    >>> map(decrement,[4,3,7,2,4,3,1])
    [3, 2, 6, 1, 3, 2,  0]
\end{verbatim}

\section{The {\it search} pattern}

The {\it search} pattern is a slight variation of {\it filtered-counting}.
Suppose we wish to see if a value is present in a list. We can
use a filtered-counting approach and if the count is greater than
zero, we know that the item was indeed in the list.

\begin{verbatim}
    def find(target,items):
        return occurrences(target,items) > 0
\end{verbatim}

In this case, {\it occurrences} helps {\it find} do its job. We call
such functions, naturally, {\it helper functions}.
We can improve the efficiency of {\it find} by having it
perform the search, but short-circuiting the
search once the target item is found. We do this
by turning the first recursive case into a second base case:

\begin{verbatim}
    def find(target,items):
        if (items == []):
            return False
        elif (head(items) == target):    # short-circuit!
            return True
        else:
            return find(target,tail(items))
\end{verbatim}

When the list is empty, we return false because if
the item had been list, we would have hit the second
base case (and returned true) before hitting the first.
If neither base case hits, we simple search the remainder of
the list (the recursive case).
If the second base case never hits, the first base case 
eventually will.

\section{The {\it shuffle} pattern}

Sometimes, we wish to combine two lists into a third list,
This is easy to do with the concatenation operator, \verb!+!.

\begin{verbatim}
   list3 = list1 + list2
\end{verbatim}

This places the first element in the second list after the last
element in the first list.
However, many times we wish to intersperse the elements from the
first list with the elements in the second list. This is known
as a {\it shuffle}, so named since it is similar to shuffling a deck of
cards. When a deck of cards is shuffled,
the deck is divided in two halves (one half is akin to the first
list and the other half is akin to the second list). Next the 
two halves are interleaved back into a single deck (akin to
the resulting third list).

We can use recursion to shuffle two lists. If both lists are exactly
the same length, the recursive function is easy to implement
using the {\it accumulate} pattern:

\begin{verbatim}
    def shuffle(list1,list2):
        if (list1 == []):
            return []
        else:
            return [head(list1),head(list2)] + shuffle(tail(list1),tail(list2))
\end{verbatim}

If {\it list1} is empty
(which means {\it list2} is empty since they have the same number of elements),
the function returns the empty, since shuffling nothing together
yields nothing.
Otherwise, we take the first elements of each list and make a list
out of the two elements, then appending the shuffle of
the remaining elements to that list.

If you have ever shuffled a deck of cards, you will know that it is
rare for the deck to be split exactly in half prior to the shuffle.
Can we amend our shuffle function to deal with this problem? Indeed,
we can,
by simply placing the extra cards (list items) at the end
of the shuffle. We don't know which list ({\it list1} or {\it list2})
will go empty first,
so we test for each list becoming empty in turn:

\begin{verbatim}
    def shuffle2(list1,list2):
        if (list1 == []):
            return list2
        elif (list2 == []):
            return list1
        else:
            return [head(list1),head(list2)] + shuffle(tail(list1),tail(list2))
\end{verbatim}

If either list is empty, we return the other. Only if both are
not empty do we execute the recursive case.
       
\section{The {\it merge} pattern}

With the {\it shuffle} pattern, we always took the head elements from both
lists at each step in the shuffling process.
Sometimes, we wish to place a constraint of the choice of elements.
For example, suppose the two lists to be combined are sorted
and we wish the resulting list to be sorted as well. The
following example shows that shuffling does not always work:

\begin{verbatim}
    >>> a = [1,4,6,7,8]
    >>> b = [2,3,5,9]

    >>> c = shuffle2(a,b)
    [1, 2, 4, 3, 6, 5, 7, 9, 8]
\end{verbatim}

The {\it merge} pattern is used to ensure the resulting list is
sorted and is based upon the {\it filtered-accumulate} pattern.
the merge. We only accumulate an item {\it if} it is the smallest item
in the two lists:

\begin{verbatim}
    def merge(list1,list2):
        if (list1 == []):
            return list2
        elif (list2 == []):
            return list1
        elif (head(list1) < head(list2)):
            return [head(list1)] + merge(tail(list1),list2)
        else:
            return [head(list2)] + merge(list1,tail(list2))
\end{verbatim}

As with {\it shuffle2}, we don't know which list will become empty
first, so we check both in turn.

In the first recursive case, the first element of the first list
is smaller than the first element of the second list.
So we accumulate the first element of the first list and
recur, sending the tail of the first list because we have
used/accumulated the head of that list. The second list we
pass unmodified, since we did not use/accumulate an element
from the second list.

In the second recursive case, we implement the symmetric version
of the first recursive case, focusing on the second list rather
than the first.

\section{The {\it generic} {\it merge} pattern}

The {\it merge} function in the previous section hard-wired the
comparison operator to \verb!<!. Many times, the elements
of the lists to be merged cannot be compared with \verb!<! or, if
they can, a different operator, such as \verb!>!, might be desired.
The generic merge solves this problem by allowing the caller to
pass in a comparison function as a third argument:

\begin{verbatim}
    def genericMerge(list1,list1,pred):
\end{verbatim}
        
where {\it pred} is the formal parameter that
holds a predicate function\footnote{Recall that a predicate
function returns {\it True} or {\it False}.}.
Now we replace the \verb!<! in merge with a call to {\it pred}.

\begin{verbatim}
    def genericMerge(list1,list1,pred):
        if (list1 == []):
            return list2
        elif (list2 == []):
            return list1
        elif (pred(head(list1),head(list2))):
            return [head(list1)] + genericMerge(tail(list1),list2,pred)
        else:
            return [head(list2)] + genericMerge(list1,tail(list2),pred)
\end{verbatim}

The {\it pred} function, which is passed the two head elements, returns
\verb!True!, if the first element should be accumulated,
and \verb!False!, otherwise.

We can still use {\it genericMerge} to merge two sorted lists of numbers
(which can be compared with <) by using the {\it operator} module. The
{\it operator} module provides function forms of the
operators \verb!+!, \verb!-!,
\verb!<!, and so on.

\begin{verbatim}
    >>> import operator
    >>> a = [1,4,6,7,8]
    >>> b = [2,3,5,9]

    >>> c = genericMerge(a,b,operator.lt)
    [1, 2, 3, 4, 5, 6, 7, 8, 9]
\end{verbatim}

The {\it genericMerge} function is a {\it generalization} of {\it merge}.
When we generalize a function,  we modify it so it can do what it
did before plus new things that it could not. Here, we can still
have it (genericMerge) do what it (merge) did before, by passing
in the correct comparison operator.

\section{The {\it fossilized} pattern}

If a recursive function mistakenly never makes the problem
smaller, the problems is said to be {\it fossilized}.
Without ever smaller problems, the base case is never reached
and the function recurs\footnote{The word is {\it recurs}, not {\it recurses}!}
forever.
This condition is known as an
{\it infinite recursive loop}. Here is an example:

\begin{verbatim}
    def factorial(n):
        if (n == 0):
            return 1
        else:
            return n * factorial(n)
\end{verbatim}

Since factorial is solving the same problem over and over, {\it n}
never gets smaller so it never reaches zero.
Fossilizing the problem is a common error made by both novice
and expert programmers alike.

\section{The {\it bottomless} pattern}

Related to the {\it fossilized} pattern is the {\it bottomless} pattern.
With the {\it bottomless} pattern, the problem gets smaller, but the base
case is never reached. Here is a function that attempts to
divide a positive number by two, by seeing how many times
you can subtract two from the number:\footnote{
Yes, division is just repeated subtraction, just like
multiplication is repeated division.}

\begin{verbatim}
    def div2(n):
        if (n == 0):
            return 0
        else:
            return 1 + div2(n - 2)
\end{verbatim}

Things work great for a while:

\begin{verbatim}
    >>> div2(16)
    8

    >>> div2(6)
    3

    >>> div2(134)
    67
\end{verbatim}

But then, something goes terribly wrong:

\begin{verbatim}
    >>> div2(7)
    RuntimeError: maximum recursion depth exceeded in cmp
\end{verbatim}

What happened? To see, let's {\it visualize} our function,
as we did with the {\it gcd} function previously,
by adding a {\it print} statement:

\begin{verbatim}
    def div2(n):
        print("div2: n is",n)
        if (n == 0):
            return 0
        else:
            return 1 + div2(n - 2)
\end{verbatim}

Now every time the function is called, both originally and
recursively, we can see how the value of {\it n} is changing:

\begin{verbatim}
    >>div2(7)
    div2: n is 7
    div2: n is 5
    div2: n is 3
    div2: n is 1
    div2: n is -1
    div2: n is -3
    div2: n is -5
    div2: n is -7
    ...
    RuntimeError: maximum recursion depth exceeded in cmp
\end{verbatim}

Now we can see why things went wrong, the value of {\it n}
skipped over the value of zero and just kept on going.
The solution is to change the base case to catch odd
(and even) numbers:

\begin{verbatim}
    def div2(n):
        if (n < 2):
            return 0
        else:
            return 1 + div2(n - 2)
\end{verbatim}
    
Remember, when you see a recursion depth exceeded error,
you likely have implemented either
the fossilized or the bottomless pattern.

\chapter{Loops}
\label{Loops}

In the previous chapter, you learned how recursion can
solve a problem by breaking it in to smaller versions
of the same problem. Another approach is to use
{\it iterative} {\it loops}. In some programming
languages, loops are preferred as they use much
less computer memory as compared to recursions.
In other languages, this is not the case at all.
In general, there
is no reason to prefer recursions over loops or vice versa,
other than this memory issue.
Any loop can be written as a recursion and any recursion
can be written as a loop.
Use a recursion if that makes the implementation
more clear, otherwise, use an iterative loop.

The most
basic loop structure in Scam is the {\it while} loop, an example of
which is:

\begin{verbatim}
    while (i < 10):
        print(i,end="")
        i = i + 1
\end{verbatim}

We see a {\tt while} loop looks much like an {\tt if}
statement.
The difference
is that blocks belonging to {\tt if}s are evaluated at most once whereas
blocks associated with loops may be evaluated many many times.
Another difference in nomenclature is that the block of a loop
is known as the {\it body} (like blocks associated with function
definitions). Furthermore, the loop test expression is known
as the {\it loop condition}.

As Computer Scientists hate to type extra characters if they can
help it, you will often see:

\begin{verbatim}
    i = i + 1
\end{verbatim}

written as

\begin{verbatim}
    i += 1
\end{verbatim}

The latter version is read as ``increment {\it i}''.

A {\it while} loop tests its condition before the body of the loop is
executed. If the initial test fails, the body is not executed at all. For
example:

\begin{verbatim}
    i = 10
    while (i < 10):
        print(i,end="")
        i += 1
\end{verbatim}

never prints out anything since the test immediately fails. In this example,
however:

\begin{verbatim}
    i = 0;
    while (i < 10):
        print(i,end="")
        i += 1
\end{verbatim}

the loop prints out the digits 0 through 9:

\begin{verbatim}
    0123456789
\end{verbatim}
    
A {\tt while} loop repeatedly evaluates its body
as long as the loop condition remains true.

To write an infinite loop, use {\tt :true} as the condition:

\begin{verbatim}
    while (True):
        i = getInput()
        print("input is",i)
        process(i)
        }
\end{verbatim}

\section{Other loops}

There are many kinds of loops in Scam, in this text
we will only refer to {\tt while} loops and {\tt for}
loops that count,
as these are commonly found in other programming languages.
The {\tt while} loop we have seen; here is an example of a counting
{\tt for} loop:

\begin{verbatim}
    for i in range(0,10,1):
        print(i)
\end{verbatim}

This loop is exactly equivalent to:

\begin{verbatim}
    i = 0
    while (i < 10):
        print(i)
        i += 1
\end{verbatim}

In fact, a while loop of the general form:

\begin{verbatim}
    i = INIT
    while (i < LIMIT):
        # body
        ...
        i += STEP
\end{verbatim}

can be written as a \\verb!for! loop:

\begin{verbatim}
    for i in range(INIT,LIMIT,STEP):
        # body
        ...
\end{verbatim}

The {\it range} function counts from
{\tt INIT} to {\tt LIMIT} (non-inclusive)
by {\tt STEP} and these
values are assigned
to {\it i}, in turn. After each assignment to {\it i},
the loop body is evaluated.
After the last value is assigned to {\it i} and the
loop body evaluated on last time, the \\verb!for! loop ends.

In Scam, the {\it range} function assumes 1 for the step
if the step is omitted and assumes 0 for the initial
value and 1 for the step if both the initial value and
step are omitted.
However, in this text, we will always give the initial
value and step of the \\verb!for! loop explicitly.

For loops are commonly used to sweep through each element of an list:

\begin{verbatim}
     for i in range(0,len(items),1):
         print(items[i]) 
\end{verbatim}

Recall the items in a list of $n$ elements are located at
indices $0$ through $n - 1$. These are exactly the values
produced by the {\it range} function. So, this loop accesses
each element, by its index, in turn, and thus prints out
each element, in turn.
Since using an index of {\it n} in a list of {\it n} items produces an
error, the {\it range} function conveniently makes its given
limit non-inclusive.

As stated earlier, there are other kinds of loops in
Scam, some of which, at times, are more convenient
to use than a {\tt while} loop or a counting {\tt for} loop. However,
anything that can be done with those other loops can be
done with the loops presented here.
Like recursion and lists, loops and lists go very well
together.
The next sections detail some common loop patterns involving
lists.

\section{The {\it counting} pattern}

The counting pattern is used to count the number of items
in a collection. Note that the built-in function len already
does this for Scam lists, but many programming languages
do not have lists as part of the language; the programmer
must supply lists. For this example, assume that the {\it start}
function gets the first item in the given list,
returning {\tt None}
if there are no items in the list. The {\it next}
function returns the next item in the given list,
returning {\tt None}
if there are no more items. This {\tt while} loop counts the number
of items in the list:

\begin{verbatim}
    count = 0
    i = start(items)
    while (i != None)
         count += 1
         i = next(items)
\end{verbatim}

When the loop finishes, the variable count holds the number of
items in the list.

The counting pattern increments a counter everytime the loop
body is evaluated.

\section{The {\it filtered-count} pattern}

A variation on counting pattern involves filtering. When {\it filtering},
we use an {\tt if} statement to decide whether we should count an item
or not. Suppose we wish to count the number of even items in
a list:

\begin{verbatim}
    count = 0
    for i in range(0,len(items),1):
        if (items[i] % 2 == 0):
            count += 1
\end{verbatim}

When this loop terminates, the variable {\it count} will hold the
number of even integers in the list of items since the count
is incremented only when the item of interest is even.

\section{The {\it accumulate} pattern}

Similar to the counting pattern, the {\it accumulate} pattern
updates a variable, not by increasing its value by one, but by the value of
an item. This loop, sums all the values in a list:

\begin{verbatim}
    total = 0
    for i in range(0,len(items),1):
        total += items[i]
\end{verbatim}

By convention, the variable {\it total} is used to accumulate the item
values. When accumulating a sum, total is initialized to zero. When
accumulating a product, total is initialized to one.

\section{The {\it filtered-accumulate} pattern}

Similar to the {\it accumulate} pattern, the {\it filtered-accumulate} pattern
updates a variable only if some test is passed.
This function sums all the even values in a given list, returning
the final sum:

\begin{verbatim}
    def sumEvens(items):
        total = 0
        for i in range(0,len(items),1):
            if (items[i] % 2 == 0)
                total += items[i]
        return total
\end{verbatim}

As before, the variable {\it total} is used to accumulate the item
values. As with a regular accumulating, {\it total} is initialized to zero when
accumulating a sum. The initialization value is one when
accumulating a product and the initialization value is
the empty list when accumulating a list (see {\it filtering} below).

\section{The {\it search} pattern}

The {\it search} pattern is a slight variation of {\it filtered-counting}.
Suppose we wish to see if a value is present in a list. We can
use a filtered-counting approach and if the count is greater than
zero, we know that the item was indeed in the list.

\begin{verbatim}
    count = 0
    for i in range(0,len(items),1):
        if (items[i] == target):
            count += 1
    found = count > 0
\end{verbatim}

This pattern is so common, it is often encapsulated in a function.
Moreover, we can improve the efficiency by short-circuiting the
search. Once the target item is found, there is no need to 
search the remainder of the list:

\begin{verbatim}
    def find(target,items):
        found = False:
        i = 0
        while (not(found) and i < len(items)):
            if (items[i] == target):
                found = True
            i += 1
        return found
\end{verbatim}

We presume the target item is not in the list and 
as long as it is not found, we continue to search the list.
As soon as we find the item, we set the variable found
to True and then the loop condition fails, since
not true is false.

Experienced programmers would likely define this function
to use an immediate return once the target item is found:

\begin{verbatim}
    def find(target,items):
        for i in range(0,len(items),1):
            if (items[i] == target):
                return True
        return False
\end{verbatim}

As a beginning programmer, however, you should avoid returns
from the body of a loop. The reason is most beginners end up
defining the function this way instead:

\begin{verbatim}
    def find(target,items):
        for i in range(0,len(items),1):
            if (items[i] == target):
                return True
            else:
                return False
\end{verbatim}

The behavior of this latter version of {\it find} is incorrect,
but unfortunately, it appears to work correctly under some
conditions. If you cannot figure out why this version
fails under some conditions and appears to succeed under
others, you most definitely should stay away from placing
returns in loop bodies.
        
\section{The {\it filter} pattern}

Recall that a special case of a filtered-accumulation is the {\it filter}
pattern.
A loop version  of filter starts out by initializing an accumulator variable to
an empty list. In the loop body, the accumulator variable
gets updated with those items from the original list that 
pass some test.

Suppose we wish to extract the even numbers from a list.
Our test, then, is to see if the current element is even.
If so, we add it to our growing list:

\begin{verbatim}
    def extractEvens(items):
        evens = []
        for i in range(0,len(items),1):
            if (items[i] % 2 == 0):
                evens = evens + [items[i]]
        return evens
\end{verbatim}

Given a list of integers, {\it extractEvens} returns a (possibly empty)
list of the even numbers:

\begin{verbatim}
    >>> extractEvens([4,2,5,2,7,0,8,3,7])
    [4, 2, 2, 0, 8]

    >>> extractEvens([1,3,5,7,9])
    []
\end{verbatim}

\section{The {\it extreme} pattern}

Often, we wish to find the largest or smallest value
in a list. Here is one approach, which assumes that the
first item is the largest and then corrects that assumption
if need be:

\begin{verbatim}
    largest = items[0]
    for i in range(0,len(items),1):
        if (items[i] > largest):
            largest = items[i]
\end{verbatim}

When this loop terminates, the variable {\it largest} holds the
largest value. We can improve the loop slightly by noting
that the first time the loop body evaluates, we compare the
putative largest value against itself, which is a worthless
endeavor. To fix this, we can start the index variable {\it i}
at 1 instead:

\begin{verbatim}
    largest = items[0]
    for i in range(1,len(items),1): #start comparing at index 1
        if (items[i] > largest):
            largest = items[i]
\end{verbatim}

Novice programmers often make the mistake of initialing setting
{\it largest} to zero and then comparing all values against {\it largest},
as in:

\begin{verbatim}
    largest = 0
    for i in range(0,len(items),1):
        if (items[i] > largest):
            largest = items[i]
\end{verbatim}

This code appears to work in some cases, namely if the largest
value in the list is greater than or equal to zero. If not,
as is the case when all values in the list are negative,
the code produces an erroneous result of zero as the largest
value.

\section{The {\it extreme-index} pattern}

Sometimes, we wish to find the index of the most extreme
value in a list rather than the actual extreme value.
In such cases, we assume index zero holds the extreme value:

\begin{verbatim}
    ilargest = 0
    for i in range(1,len(items),1):
        if (items[i] > items[ilargest]):
            ilargest = i
\end{verbatim}

Here, we successively store the index of the largest value
see so far in the variable {\it ilargest}.

\section{The {\it shuffle} pattern}

Recall, the {\it shuffle} pattern from the previous chapter.
Instead of using recursion, we can use a version of
the loop accumulation pattern instead. As before,
let's assume the lists are exactly the same size:

\begin{verbatim}
    def shuffle(list1,list2):
        list3 = []
        for i in range(0,len(list1),1):
            list3 = list3 + [list1[i],list2[i]]
        return list3
\end{verbatim}

Note how we initialized the resulting list {\it list3} to
the empty list. Then, as we walked the first list, we
pulled elements from both lists, adding them into the
resulting list.

When we have walked past the end of {\it list1} is empty,
we know we have also walked past the end of {\it list2}, since the
two lists have the same size.

If the incoming lists do not have the same length,
life gets more complicated:

\begin{verbatim}
    def shuffle2(list1,list2):
        list3 = []
        if (len(list1) < len(list2)):
            for i in range(0,len(list1),1):
                list3 = list3 + [list1[i],list2[i]]
            return list3 + list2[i:]
        else:
            for i in range(0,len(list2),1):
                list3 = list3 + [list1[i],list2[i]]
            return list3 + list1[i:]
\end{verbatim}

We can also use a {\it while} loop that goes until one of the lists
is empty. This has the effect of removing the redundant code
in {\it shuffle2}:

\begin{verbatim}
    def shuffle3(list1,list2):
        list3 = []
        i = 0
        while (i < len(list2) and i < len(list2)):
            list3 = [list1[i],list2[i]]
            i = i + 1
        ...
\end{verbatim}

When the loop ends, one or both of the lists have been exhausted,
but we don't know which one or ones. A simple solution is to
add both remainders to {\it list3} and return.

\begin{verbatim}
    def shuffle3(list1,list2):
        list3 = []
        i = 0
        while (i < len(list2) and i < len(list2)):
            list3 = [list1[i],list2[i]]
            i = i + 1
        return list3 + list1[i:] + list2[i:]
\end{verbatim}

Suppose {\it list1} is empty. Then the expression {\tt list1[i:]} will
generate the empty list. Adding the empty list to {\it list3} will
have no effect, as desired. The same is true if {\it list2}
(or both {\it list1} and {\it list2} are empty).

\section{The {\it merge} pattern}

We can also merge using a loop. Suppose we have two
ordered lists (we will assume increasing order)
and we wish to merge them into one ordered
list. We start by keeping two index variables,
one pointing to the smallest element in {\it list1} and
one pointing to the smallest element in {\it list2}.
Since the lists are ordered, we know the that the smallest
elements are at the head of the lists:

\begin{verbatim}
   i = 0  # index variable for list1
   j = 0  # index variable for list2
\end{verbatim}

Now, we loop, similar to {\it shuffle3}:

\begin{verbatim}
    while (i < len(list1) and j < len(list2)):
\end{verbatim}

Inside the loop, we test to see if the smallest element
in list1 is smaller than the smallest element in list2:

\begin{verbatim}
        if (list1[i] < list2[j]):
\end{verbatim}

If it is, we add the element from {\it list1} to {\it list3} and increase
the index variable {\it i} for {\it list1} since we have `used up' the value
at index {\it i}.

\begin{verbatim}
            list3 = list3 + [list1[i]]
            i = i + 1
\end{verbatim}

Otherwise, {\it list2} must have the smaller element and we do likewise:

\begin{verbatim}
            list3 = list3 + [list2[j]]
            j = j + 1
\end{verbatim}

Finally, when the loop ends ({\it i} or {\it j} has gotten too large),
we add the remainders of both lists to {\it list3} and return:

\begin{verbatim}
    return list3 + list1[i:] + list2[j:]
\end{verbatim}

In the case of merging, one of the lists will be exhausted
and the other will not. As with shuffle3, we really don't
care which list was exhausted.

Putting it all together yields:

\begin{verbatim}
    def merge(list1,list2):
        list3 = []
        i = 0
        j = 0
        while (i < len(list1) and j < len(list2)):
            if (list1[i] < list2[j]):
                list3 = list3 + [list1[i]]
                i = i + 1
            else:
                list3 = list3 + [list2[j]]
                j = j + 1
        return list3 + list1[i:] + list2[j:]
\end{verbatim}

\section{The {\it fossilized} Pattern}

Sometimes, a loop is so ill-specified that it never ends. This
is known as an {\it infinite loop}. Of the
two loops we are investigating, the {\it while} loop is the most
susceptible to infinite loop errors. One common mistake
is the {\it fossilized} pattern, in which the index variable never
changes so that the loop condition never becomes false:

\begin{verbatim}
    i = 0
    while (i < n):
        print(i)
\end{verbatim}

This loop keeps printing until you terminate the program with
prejudice. The reason is that {\it i} never changes; presumably a
statement to increment {\it i} at the bottom of the loop body has
been omitted.

\section{The {\it missed-condition} pattern}

Related to the bottomless pattern of recursive functions
is the missed condition pattern of loops.
With missed condition, the index variable is updated, but
it is updated in such a way that the loop condition is
never evaluates to false.

\begin{verbatim}
    i = n
    while (i > 0):
        print(i)
        i += 1
\end{verbatim}

Here, the index variable {\it i} needs to be decremented rather than
incremented. If {\it i} has an initial value greater than zero,
the increment pushes {\it i} further and further above zero.
Thus, the loop condition never fails and the loop
becomes infinite.

\chapter{Comparing Recursion and Looping}
\label{RecursionLoop}

In the previous two chapters, we learned about repeatedly
evaluating the same code using both recursion and loops.
Now we compare and contrast the two techniques by
implementing the three mathematical functions from
\link*{the chapter on assignment}[Chapter~\Ref]{Recursion}:
{\it factorial}, {\it fibonacci}, and {\it gcd}, with loops.

\section{Factorial}

Recall that the factorial function, written recursively,
looks like this:

\begin{verbatim}
    def factorial(n):
        if (n == 0):
            return 1
        else:
            return n * factorial(n - 1)
\end{verbatim}

We see that is a form of the {\it accumulate} pattern. So our factorial
function using a loop should look something like this:

\begin{verbatim}
    def factorial(n):
        total = ???
        for i in range(???):
            total *= ???
        return total
\end{verbatim}

Since we are accumulating a product, total should
be initialized to 1.

\begin{verbatim}
    def factorial(n):
        total = 1
        for i in range(???):
            total *= ???
        return total
\end{verbatim}

Also, the loop variable should take on all values in
the factorial, from 1 to {\it n}:

\begin{verbatim}
    def factorial(n):
        total = 1
        for i in range(1,n+1,1):
            total *= ???
        return total
\end{verbatim}

Finally, we accumulate {\it i} into the total:

\begin{verbatim}
    def factorial(n):
        total = 1
        for i in range(1,n+1,1):
            total *= i
        return total
\end{verbatim}

The second argument to range is set to $n+1$ instead of $n$ because
we want $n$ to be included in the total.

Now, compare the loop version to the recursive version. Both contain
about the same amount of code, but the recursive version is easier
to ascertain as correct.

\section{The greatest common divisor }

Here is a slightly different version of the gcd function, built using
the following recurrence:

\begin{center}
\begin{tabular}{lcll}%
\T\toprule
    $gcd$($a$,$b$) & is & $a$ & if $b$ is zero\\
    $gcd$($a$,$b$) & is & $gcd$($b$,$a$ \% $b$) & otherwise \T\\
\T\bottomrule
\end{tabular}
\end{center}

The function allows one more recursive call than the other. By doing
so, we eliminate the need for the local variable {\it remainder}. Here is
the implementation:

\begin{verbatim}
    def gcd(a,b):
        if (b == 0):
            return a
        else:
            return gcd(b,a % b)
\end{verbatim}

Let's turn it into a looping function. This style of
recursion doesn't fit any of the patterns we know, so
we'll have to start from scratch. We do know that 
{\it b} becomes the new value of {\it a} and {\it a} \% {\it b} becomes 
the new value of {\it b}
on every recursive call,
so the same thing must happen on every evaluation of
the loop body.
We stop when {\it b} is equal to zero so we should continue looping
while {\it b} is  not equal to zero. These observations lead us
to this implementation:

\begin{verbatim}
    def gcd(a,b):
        while (b != 0):
            a = b
            b = a % b
        return a
\end{verbatim}

Unfortunately, this implementation is faulty, since we've lost
the original value of {\it a} by the time we perform the modulus 
operation. Reversing the two statements in the body of the loop:

\begin{verbatim}
    def gcd(a,b):
        while (b != 0):
            b = a % b
            a = b
        return a
\end{verbatim}

is no better; we lose the original value of {\it b} by the time we
assign it to {\it a}. What we need to do is temporarily save the
original value of {\it b} before we assign {\it a}'s value. Then
we can assign the saved value to {\it a} after {\it b} has been reassigned:

\begin{verbatim}
    def gcd(a,b):
        while (b != 0):
            temp = b
            b = a % b
            a = temp
        return a
\end{verbatim}

Now the function is working correctly. But why did we temporarily
need to save a value in the loop version and not in the recursive
version? The answer is that the recursive call does not perform
any assignments so no values were lost. On the recursive call,
new versions of the formal parameters {\it a} and {\it b} received the
computations performed for the function call. The old versions
were left untouched.

It should be noted that Scam allows simultaneous assignment that
obviates the need for the temporary variable:

\begin{verbatim}
    def gcd(a,b):
        while (b != 0):
            a,b = b,a % b
        return a
\end{verbatim}

While this code is much shorter, it is a little more difficult to
read. Moreover, other common languages do not share this feature
and you are left using a temporary variable to preserve needed
values when using those languages.

\section{The Fibonacci sequence}

Recall the recursive implementation of Fibonacci:

\begin{verbatim}
    def fib(n):
        if (n < 2)
            return n
        else
            return fib(n - 1) + fib(n - 2)
\end{verbatim}

For brevity, we have collapsed the two base cases into
a single base case. If {\it n} is zero, zero is returned and if
{\it n} is one, one is returned, as before.

Let's So let's try to compute
using an iterative loop. As before, this doesn't seem
to fit a pattern, so we start by reasoning about this.
If we let {\it a} be the first Fibonacci number, zero, and {\it b}
be the second Fibonacci number, one, then the third fibonacci
number would be $a + b$, which we can save in a variable 
named {\it c}.
At  this point, the fourth Fibonacci number would be $b + c$,
but since we are using a loop, we need to have the code be
the same for each iteration of the loop. If we let $a$ have the
value of $b$ and $b$ have the value of $c$, then the fourth Fibonacci
number would be $a + b$ again.
This leads to our implementation:

\begin{verbatim}
    def fib(n):
        a = 0    # the first Fibonacci number
        b = 1    # the second Fibonacci number
        for i in range(0,n,1):
            c = a + b
            a = b
            b = c
        return a
\end{verbatim}

In the loop body, we see that {\it fib} is much like {\it gcd};
the second number becomes the first number and some combination of
the first and second number becomes the second number.
In the case of {\it gcd}, the combination was the remainder and, in the
case of {\it fib}, the combination is sum.
A rather large question remains, why does the function return {\it a}
instead of {\it b} or {\it c}? The reason is, suppose {\it fib} was
called with a value of 0, which is supposed to generate
the first Fibonacci number. The loop does not run in this case
and the value of {\it a} is returned, zero, as required.
If a value of 1 is passed to {\it fib}, then the loop runs exactly
once and {\it a} gets the original value of {\it b}, one. The loop expects and
this time, one is returned, as required. So, empirically, it
appears that the value of a is the correct choice of return value.
As with factorial, hitting on the right way to proceed iteratively
is not exactly straightforward, while the recursive version
practically wrote itself.

\section{CHALLENGE: Transforming loops into recursions}

To transform an iterative loop into a
recursive loop, one first identifies those variables
that exist outside the loop but are changing in the loop body;
these variable will become formal parameters in the recursive
function.
For example, the {\it fib} loop above has three (not two!)
variables that
are being changed during each iteration of the loop:
{\it a}, {\it b}, and {\it i}.\footnote{The loop variable
is considered an outside variable changed by the loop.} 
The variable {\it c} is used only inside the loop and thus is
ignored.

Given this, we start out
our recursive function like so:

\begin{verbatim}
    def loop(a,b,i):
        ...
\end{verbatim}

The loop test becomes an {\it if} test in the body of
the {\it loop} function:

\begin{verbatim}
    def loop(a,b,i)
        if (i < n):
            ...
        else:
            ...
\end{verbatim}

The {\it if-true} block becomes the recursive call.
The arguments to the recursive call encode the updates
to the loop variables 
The {\it if-false} block becomes the value the loop attempted to
calculate:

\begin{verbatim}
    def loop(a,b,i):
        if (i < n):
            return loop(b,a + b,i + 1)
        else:
            return a
\end{verbatim}

Remember, a gets the value of b and b gets the value of
{\it c} which is $a + b$. Since we are performing recursion
with no assignments, we don't need the variable {\it c} anymore.
The loop variable {\it i} is incremented
by one each time.

Next, we replace the loop with the the {\it loop} function in the
function
containing the original loop. That way, any non-local variables
referenced in the test or body of the original loop will
be visible to the {\it loop} function:

\begin{verbatim}
    def fib(n):
        def (a,b,i):
            if (i < n)
                return loop(b,a + b,i + 1)
            else:
                return a
        ...
\end{verbatim}

Finally, we call the {\it loop} function with the initial
values of the loop variables:

\begin{verbatim}
    def fib(n):
        def (a,b,i):
            if (i < n)
                return loop(b,a + b,i + 1)
            else:
                return a
        return loop(0,1,0)
\end{verbatim}

Note that this recursive function looks nothing like our
original {\it fib}. However, it suffers from none of the inefficiencies
of the original version and yet it performs no assignments.\footnote{
A style of programming that uses no assignments is called {\it functional}
programming and is very important in theorizing about the nature
of computation.} The reason for its efficiency is that it performs
that exact calculations and number of calculations as the
loop based function.

For more practice, let's convert the iterative version of
{\it factorial} into a recursive function using this method.
We'll again end up with a different recursive function
than before. For convenience, here is the loop version:

\begin{verbatim}
    def fact(n):
        total = 1
        for i in range(1,n+1,1):
            total *= i
        return total
\end{verbatim}

We start, as before, by working on the {\it loop} function.
In this case,
only two variables are changing in the loop:
{\it total} and {\it i}.

\begin{verbatim}
    def loop(total,i):
        ...
\end{verbatim}

Next, we write the {\it if} expression:

\begin{verbatim}
    def loop(total,i):
        if (i < n + 1):
            return loop(total * i,i + 1)
        else:
            return total
\end{verbatim}

Next, we embed the {\it loop} function and call it:

\begin{verbatim}
    def fact(n):
        def loop(total,i):
            if (i < n + 1):
                return loop(total * i,i + 1)
            else:
                return total
        return loop(1,1)
\end{verbatim}

The moral of this story is that any iterative loop can be rewritten
as a recursion and any recursion can be rewritten as
an iterative loop. Moreover, in {\it good} languages,\footnote{
Scam is one of these good languages!}
there is no reason to prefer one way over the other,
either in terms of the time it takes or the space used
in execution. To reiterate,
use a recursion if that makes the implementation
more clear, otherwise, use an iterative loop.

\chapter{More on Input}
\label{MoreOnInput}

Now that we have learned how to loop, we can perform
more sophisticated types of input.

\section{Converting command line arguments en mass}

Suppose all the command-line arguments are numbers that
need to be converted from their string versions stored
in {\it sys.argv}.
We can use a loop and the accumulate pattern to accumulate
the converted string elements:

\begin{verbatim}
    def convertArgsToNumbers():
        total = []
        # start at 1 to skip over program file name
        for i in range(1,len(sys.argv),1):
            num = eval(sys.argv[i])
            total = total + [num]
        return total
\end{verbatim}

The accumulator, total, starts out as the empty list. For each
element of sys.argv beyond the program file name, we convert
it and store the result in num. We then turn that number
into a list (by enclosing it in brackets) and then add it
to the growing list.

With a program file named {\it convert.py} as follows:

\begin{verbatim}
    import sys

    def main():
        ints = convertArgsToNumbers()
        print("original args are",sys.argv[1:])
        print("converted args are",ints)

    def convertArgsToNumbers():
        ...

    main()
\end{verbatim}

we get the following behavior:

\begin{verbatim}
   $ python convert.py 1 34 -2
   original args are ['1', '34', '-2']
   converted args are [1, 34, -2]
\end{verbatim}

Note the absence of quotation marks in the converted list,
signifying that the elements are indeed numbers.

\section{Reading individual items from files}

Instead of reading all of the file at once using the {\it read} function,
we can read it one item at a time. When we read an
item at a time, we always follow this pattern:

\begin{verbatim}
    open the file
    read the first item
    while the read was good
        process the item
        read the next item
    close the file
\end{verbatim}

In Scam, we tell if the read was good by checking the
value of the variable that points to the value read.
Usually, the empty string is used to indicate the
read failed.

\section*{Processing files a line at a time}

Here is another version of the {\it copyFile} function from
\link*{the chapter on input and output}[Chapter~\Ref]{InputAndOutput}.
This version reads and writes
one line at a time. In addition, the function returns
the number of lines processed:

\begin{verbatim}
    def copyFile(inFile,outFile):
        in = open(inFile,"r")
        out = open(outFile,"w")
        count = 0
        line = in.readline()
        while (line != ""):
            out.write(line)
            count += 1
            line = in.readline()
        in.close()
        out.close()
        return count
\end{verbatim}
            
Notice we used the counting pattern.

\section*{Using a Scanner}

A scanner is a reading subsystem that allows you
to read whitespace-delimited tokens from a file.
To get a scanner for Scam, issue this command:

\begin{verbatim}
    wget beastie.cs.ua.edu/cs150/projects/scanner.py
\end{verbatim}

To use a scanner, you will need to import it into your program:

\begin{verbatim}
   from scanner import *
\end{verbatim}

Typically, a scanner is used with a loop. Suppose we wish to count the
number of 
short tokens (a token is a series of characters surrounded by empty space)
in a file. Let's assume a short token is one whose length is less
than or equal to some limit.
Here is a loop that does that:

\begin{verbatim}
    def countShortTokens(fileName):
        s = Scanner(fileName)              #create the scanner
        count = 0
        token = s.readtoken()              #read the first token
        while token != "":                 #check if the read was good
            if (len(token) <= SHORT_LIMIT):
                count += 1
            token = s.readtoken()          #read the next token
        s.close()                          #always close the scanner when done
        return count
\end{verbatim}

Note that the use of the scanner follows the standard reading pattern:
opening (creating the scanner),
making the first read, testing if the read was good, processing
the item read (by counting it), reading the next item, and finally
closing the file (by closing the scanner) after the loop terminates.
Using a scanner always means performing the five steps as given
in the comments.
This code also incorporates the filtered-counting pattern, as expected.

\section{Reading Tokens into a List}

Note that the {\it countShortTokens} function is doing two things, reading
the tokens and also counting the number of short tokens. It is said that this
function has two {\it concerns}, reading and counting. A
fundamental principle of Computer Science is {\it separation of
concerns}.
To separate the concerns, we have one function read the tokens,
storing them into a list (reading and storing is considered to
be a single concern).
We then have another function count the tokens. Thus, we will
have separated the two concerns into separate functions, each
with its own concern.
Here is the reading (and storing) function, which implements
the accumulation pattern:

\begin{verbatim}
    def readTokens(fileName):
        s = Scanner(fileName)              #create the scanner
        items = []
        token = s.readtoken()              #read the first token
        while token != "":                 #check if the read was good
            items = items + [token]
            token = s.readtoken()          #read the next token
        s.close()                          #always close the scanner when done
        return items
\end{verbatim}

Next, we implement the filtered-counting function. Instead of passing
the file name, as before, we pass the list of tokens that
were read:

\begin{verbatim}
    def  countTokens(items):
        count = 0
        for i in range(0,len(items),1)
            if (len(items[i]) <= SHORT_LIMIT):
                count += 1
        return count
\end{verbatim}

Each function is now simpler than the original function. This makes
it easier to fix any errors in a function since you can concentrate
on the single concern implemented by that function.

\section{Reading Records into a List}

Often, data in a file is organized as {\it records}, where
a record is just a collection of consecutive tokens.
Each token in a record is known as a {\it field}.
Suppose every four tokens in a file comprises a record:

\begin{verbatim}
    Smith    President 32  87000
    Jones    Assistant 15  99000
    Thompson    Hacker  2 147000
\end{verbatim}

Typically, we define a function to read one collection of tokens
at a time.
Here is a function that reads a single record:

\begin{verbatim}
    def readRecord(s):                   # we pass the scanner in
        name = s.readtoken()
        if name == "":
            return None                  # no record, returning None
        title = s.readtoken()
        service = eval(s.readtoken())
        salary = eval(s.readtoken())
        return [name,title,service,salary]
\end{verbatim}

Note that we return either a record as a list or None if
no record was read.
Since years of service and salary are numbers, we
convert them appropriately with {\it eval}.

To total up all the salaries, for example, we can use an accumulation
loop (assuming the salary data resides in a file named
{\it salaries}).
We do so by repeatedly calling {\it readrecord}:

\begin{verbatim}
    function totalPay(fileName):
        s = Scanner(fileName)
        total = 0
        record = readRecord(s)
        while (record != None):
            total += record[3]
            record = readRecord(s)
        s.close()            
        print("total salaries:",total)
\end{verbatim}

Note that it is the job of the caller of {\it readRecord} to
create the scanner, repeatedly send the scanner to
{\it readRecord}, and close
the scanner when done.
Also note that we tell if the read was good by checking
to see if {\it readRecord} return {\tt None}.

The above function has two stylistic flaws. It uses those
magic numbers we read about in 
\link*{the chapter on assignment}[Chapter~\Ref]{Assignment}.
It is not clear from the code that the field at index
three is the salary.
To make the code more readable, we can set up some ``constants''
in the global scope (so that they will be visible everywhere):
The second issue is that
that the function has two concerns (reading and accumulating).
We will fix the magic number problem first.

\begin{verbatim}
    NAME = 0
    TITLE = 1
    SERVICE = 2
    SALARY = 3
\end{verbatim}

Our accumulation loop now becomes:

\begin{verbatim}
    total = 0
    record = readRecord(s)
    while record != None:
        total += record[SALARY]
        record = readRecord(s)
\end{verbatim}

We can also rewrite our {\it readRecord} function so
that it only needs to know the number of fields:

\begin{verbatim}
    def readRecord(s):                   # we pass the scanner in
        name = s.readtoken()
        if name == "":
            return None                  # no record, returning None
        title = s.readtoken()
        service = eval(s.readtoken())
        salary = eval(s.readtoken())

        # create an empty record

        result = [0,0,0,0]               

        # fill out the elements

        result[NAME] = name
        result[TITLE] = title
        result[SERVICE] = service
        result[SALARY] = salary

        return result
\end{verbatim}

Even if someone changes the constants to:

\begin{verbatim}
    NAME = 3
    TITLE = 2
    SERVICE = 1
    SALARY = 0
\end{verbatim}

The code still works correctly. Now, however, the salary resides
at index 0, but the accumulation loop is still accumulating the
salary due to its use of the constant to access the salary.

\section{Creating a List of Records}

We can separate the two concerns of the {\it totalPay} function
by having one function read the records into a list
and having another total up the salaries.
A list of
records is known as a {\it table}.
Creating the table
is just like accumulating the salary, but instead
we accumulate the entire record into a list:

\begin{verbatim}
    def readTable(fileName):
        s = Scanner(fileName)
        table = []
        record = readRecord(s)
        while record != None:
            table += [record]       #brackets around record!
            record = readRecord(s)
        s.close()            
\end{verbatim}

Now the table holds all the records in the file.
We must remember to enclose the record in square brackets 
before we accumulate it into the growing table. The
superior student will try this code without the brackets
and ascertain the difference.

The accumulation function is straightforward:

\begin{verbatim}
    def totalPay(fileName):
        table = readTable(fileName)
        total = 0
        for i in range(0,len(table),1):
            record = table[i]
            total += record[SALARY]
        return total
\end{verbatim}

We can simply this function by removing the temporary variable
{\it record}:

\begin{verbatim}
    def totalPay(fileName):
        table = readTable(fileName)
        total = 0
        for i in range(0,len(table),1):
            total += table[i][SALARY]
        return total
\end{verbatim}

Since a table is just a list, so we can walk it, accumulate
items in each record (as we just did with salary), filter it and so on.

\section{Other Scanner Methods}

A scanner object has other methods for reading. They are

\begin{description}
\item[{\tt readline()}]
    read a line from a file, like Scam's {\it readline}.
\item[{\tt readchar()}]
    read the next non-whitespace character
\item[{\tt readrawchar()}]
    read the next character, whitespace or no
\item[{\tt readstring()}]
    read a string - if a string is not pending, '' is returned
\item[{\tt readint()}]
    read an integer - if an integer is not pending, '' is returned
\item[{\tt readfloat()}]
    read a floating point number - if an float is not pending, '' is returned
\end{description}

You can also use a scanner to read from the keyboard. Simply
pass an empty string as the file name:

\begin{verbatim}
    s = Scanner("")
\end{verbatim}

You can scan tokens and such from a string as well by first
creating a keyboard scanner, and then setting the input
to the string you wish to scan:

\begin{verbatim}
    s = Scanner("")
    s.fromstring(str)
\end{verbatim}

\include{arrays-and-lists}
\chapter{Sorting}
\label{Sorting}

\section*{Sorted Tables}

Recall that a table is a list of records where
each record is a list of the fields incorporating
the record.

Sometimes, you need to merge two sorted tables into
one table that remains sorted. First, you have
to decide which field is used for the
sorting. In our example, the records in the data file
could be sorted on NAME or on SALARY or any other
field.

Suppose we had two data files that are sorted on SALARY,
{\it salaries.1} and {\it salaries.2}.  We wish to merge the data
in both files, printing  out the merged data, again in
sorted order.

First, we need to read the data into tables:

\begin{verbatim}
    table1 = readTable("salaries.1")
    table2 = readTable("salaries.2")
\end{verbatim}

Our strategy is to compare the first unaccumulated
record in {\it table1} to the first unaccumulated record
in {\it table2}. Let's call these records {\it r1} and {\it r2}, respectively.
If the salary of {\it r1} is less than that of {\it r2}, we accumulate
{\it r1}. Otherwise we accumulate {\it r2}.
We will repeat this process using a loop.

It is clear we need two variables, the first points to
the index of the first unaccumulated record in the table1,
while the second variable points to the first unaccumulated
record in the second table.
We start out both variables at zero, meaning no records
have been accumulated yet:

\begin{verbatim}
    index1 = 0
    index2 = 0
\end{verbatim}

How do we know when to stop accumulating? When we
run out of records to compare. This happens
when {\it index1} has passed
the index of the last record in {\it table1} or {\it index2} has passed the
index of the last record in {\it table2}.
We reverse that logic for a while loop, because it runs
while the test condition is true. The reversed logic is
``as long as index1 has not passed the {\it index} of the last record
in {\it table1} AND {\it index2} has not passed the index of the last
record in {\it table2}''.

\begin{verbatim}
    total = []
    while (index1 < len(table1) and index2 < len(table2):
        r1 = table1[index1]
        r2 = table2[index2]
        ...
\end{verbatim}

We also must advance {\it index1} and {\it index2} to that the loop will
finally end. When do we advance {\it index1}? When we accumulate
a record from {\it table1}. When do we advance {\it index2}? Likewise,
when we accumulate a record from {\it table2}.


\begin{verbatim}
    total = []
    while (index1 < len(table1) and index2 < len(table2):
        r1 = table1[index1]
        r2 = table2[index2]
        if (r1[SALARY] < r2[SALARY]):
           total = total + [r1]
           index1 += 1
        else:
           total = total + [r2]
           index2 += 1
\end{verbatim}

When will this loop end? When one of the indices gets too high\footnote{
Only one will be too high. Why is that?}.
This means we will have accumulated all the
records from one of the tables, but we don't know which one.
So, we add two more loops to accumulate any left over records:

\begin{verbatim}
    for i in range(index1,len(table1),1):
        total = total + [table1[i]]

    for i in range(index2,len(table2),1):
        total = total + [table2[i]]
\end{verbatim}

Finally, we encapsulate all of our merging code into
a function, passing in the index of the field that was
used to sort the data. This field is
known as the {\it key}:

\begin{verbatim}
    def merge(table1,table2,key):
        total = []
        while (index1 < len(table1) and index2 < len(table2):
            r1 = table1[index1]
            r2 = table2[index2]
            if (r1[key] < r2[key]):
               total = total + [r1]
               index1 += 1
            else:
               total = total + [r2]
               index2 += 1

        for i in range(index1,len(table1),1):
            total = total + [table1[i]]

        for i in range(index2,len(table2),1):
            total = total + [table2[i]]
\end{verbatim}

Finally, we define a main function to tie it all together:

\begin{verbatim}
    def main():
        table1 = readTable("salaries.1")
        table2 = readTable("salaries.2")
        mergedTable = merge(table1,table2,SALARY) #SALARY is the key
        printTable(mergedTable)
\end{verbatim}

Notice how the main function follows the standard main
pattern:

\begin{itemize}
\item
        get the data
\item
        process the data
\item
        write the result
\end{itemize}

\section{Merge sort}

%\include{ch19}
%\include{ch20}
%\include{ch21}
%\include{ch22}

\end{document}
