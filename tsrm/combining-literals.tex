\chapter{Combining Literals}
\label{CombiningLiterals}

Like the literals themselves, combinations of literals are also
expressions. For example, suppose you have forgotten your times table
and aren't quite sure whether 8 times 7 is 54 or 56. We can ask Scam,
presenting the interpreter with the expression:

\begin{verbatim}
    >>> 8 * 7
    56
    >>> 8*7
    56
\end{verbatim}

As before, the semicolon signals the end of the expression. The
multiplication sign * is known as an {\it operator}, as it {\it operates} on the 8
and the 7, producing an equivalent literal value. The 8 and the 7 are
known as {\it operands}. It seems that the actual names of various operands are
not being taught anymore, so for nostalgia's sake, here they are. The
operand to the left of the multiplication sign (in this case the 8) is
known as the {\it multiplicand}. The operand to the right (in this case the 7)
is known as the {\it multiplier}. The result is known as the {\it product}.

The operands of the other basic operators
have special names too. For addition, the left operand is known as the
{\it augend} and the right operand is known as the {\it addend}.
The result is known as the {\it sum}.
For subtraction,
the left operand is the {\it minuend}, the right the {\it subtrahend}, and
the result as the {\it difference}.
Finally
for division (and I think this is still taught), the left operand is
the {\it dividend}, the right operand is the {\it divisor}, and the 
result is the {\it quotient}.

In general, we will separate
operators from their operands by
spaces, tabs,
or newlines, collectively known as {\it whitespace}.\footnote{
Computer Scientists, when they have to write their annual reports,
often refer to the things they are reporting on as
{\it darkspace}. It's always good to have a lot of darkspace in
your annual report!
}
It's not necessary to do so, but it  makes your code
easier to read.

Scam always takes in an expression and displays an equivalent
literal expression ({\it e.g.}, integer or real). All Scam operators are
binary, meaning they operate on exactly two operands. We first look at
the numeric operators.

\section{Numeric operators}

If it makes sense to add two things together, you can probably do it in
Scam using the + operator. For example:

\begin{verbatim}
    >>> 2 + 3
    5
    
    >>> 1.9 + 3.1 
    5.0
\end{verbatim}
    
One can see that if one adds two integers, the result is an integer. If
one does the same with two reals, the result is a real.

You can even add strings and lists:

\begin{verbatim}
    >>> "hello" + "world"
    'helloworld'

    >>> [1, 3, 5] + [2, 4, 6]
    [1, 3, 5, 2, 4, 6]
\end{verbatim}

The process of joining strings and lists together is known as
{\it concatenation}.

Things get more interesting when
you add things having different types. Adding an integer and a real (in
any order) always yields a real.

\begin{verbatim}
    >>> 2 + 3.3
    5.3
    
    >>> 3.3 + 2
    5.3
\end{verbatim}
    
Adding an string to an integer
(with an augend integer) yields an error;
the types are not `close' enough, like they are with
integers and reals:

\begin{verbatim}
    >>> 2 + "hello"
    TypeError: unsupported operand types(s) for +: 'int' and 'str'
\end{verbatim}

In general, when adding two things,
the types must match or nearly match.
    
You can multiply strings and lists with numbers:

\begin{verbatim}
    >>> "hello" * 3
    'hellohellohello'

    >>> [1, 2] * 3
    [1, 2, 1, 2, 1, 2]
\end{verbatim}

Subtraction and division of numbers follow the
same rules as addition. However, these operators, as defined,
do not work for strings and lists.

Of special note is the division operator with respect to integer
operands. Consider evaluating the following expression:

\begin{verbatim}
    15 / 2
\end{verbatim}

If one asked the Scam interpreter to perform this task, the result
would be 7.5, as expected. However, often we wish for just
the quotient without the remainder. In this case, the
quotient is 7 and the remainder is 0.5. 
The double forward slash operator is Scam's quotient
operator; if we ask the interpreter to  evaluate

\begin{verbatim}
    14 // 5
\end{verbatim}

the result would be 2, not 2.8. 
Use of the quotient operator is known as {\it integer} {\it division}.

The complement to integer division is the modulus operator \%. While the
result of integer division is the quotient, the result of the modulus
operator is the remainder. Thus

\begin{verbatim}
    14 % 5
\end{verbatim}

evaluates to 4 since 4 is left over when 5 is divided into 14. To check
if this is true, one can ask the interpreter to evaluate:

\begin{verbatim}
    (14 // 5 * 5) + (14 % 5) == 14
\end{verbatim}

This complicated expression asks the question `is it true that the
quotient times the divisor plus the remainder is equal to the original
dividend?'. The Scam interpreter will respond that, indeed, it is
true. The reason for the parentheses is delineate
the quotient and the remainder within the addition.
The parentheses can also be used to change the {\it precedence}
of operators; this is
is explained in more detail in the
next chapter.

\section{Comparing things}

Remember the {\sc boolean} literals, {\tt True} and {\tt False}?
We can use the {\sc boolean}
comparison operators to generate such values. For example, we can ask
if 3 is less than 4:

\begin{verbatim}
    >>> 3 < 4
    True
\end{verbatim}

The interpreters response says that, indeed, 3 is less than 4. If it were
not, the interpreter would respond with {\tt False}.
Besides
{\tt <}
(less than),
there are other {\sc boolean} comparison operators:
{\tt <=}
(less than or equal to),
{\tt >}
(greater than),
{\tt >=}
(greater than or equal to),
{\tt ==}
(equal to), and
{\tt !=}
(not equal to).

Besides integers, we can compare reals with reals, strings with strings,
and lists with lists
using the comparison operators:

\begin{verbatim}
    >>> "apple" < "banana"
    True
    
    >>> [1, 2, 3] < [1, 2, 4]
    True
\end{verbatim}
    
In general, it is illegal
to compare integers or reals with strings.

Any Scam type can be compared with any other type with the
{\tt ==}
and
{\tt !=}
comparison operators.
If an integer is compared with a real with these
operators, the integer is converted into a real before the comparison
is made. In other cases, comparing different types with
{\tt ==}
will yield
a value of {\tt False}. Conversely, comparing different types with
{\tt !=}
will yield
{\tt True}
(the exception, as above, being integers compared with reals).
If the types match,
{\tt ==}
will yield true only if the values
match as well. The operator
{\tt !=}
behaves accordingly.

\section{Combining comparisons}

We can combine comparisons with the {\sc boolean} logical connectives
{\tt and} and {\tt or}:

\begin{verbatim}
    >>> 3 < 4 and 4 < 5
    True
    
    >>> 3 < 4 or 4 < 5
    True
    
    >>> 3 < 4 and 5 < 4
    False
    
    >>> 3 < 4 or 5 < 4
    True
\end{verbatim}

The first interaction asks if both the expression
{\tt 3 < 4} and the expression
{\tt 4 < 5} are true. Since both are, the
interpreter responds with {\tt True}. The second interaction
asks if at least one of the expressions is true. Again, the
interpreter responds with {\tt True}. The difference between {\tt and}
and {\tt or} is illustrated in the last two interactions. Since
only one expression is true (the latter expression being
false) only the {\tt or} operator yields a true value.

There is one more {\sc boolean} logic operation, called
{\it not}. It simply reverses the value of the expression
to which it is attached. The {\it not} operator can only be called
as a function (since it is not a binary operator). Since
you do not yet know about functions, I'll show you what
it looks like but won't yet explain its actions.

\begin{verbatim}
    >>> not(3 < 4 and 4 < 5)
    False
    
    >>> not(3 < 4 or 4 < 5)
    False
    
    >>> not(3 < 4 and 5 < 4)
    True
    
    >>> not(3 < 4 or 5 < 4)
    False
\end{verbatim}

Note that we attached {\it not} to each of the previous expressions involving
the logical connectives. Note also that the response of the interpreter
is reversed from before in each case.
