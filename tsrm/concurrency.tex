\chapter{Concurrency}
\label{Concurrency}

Scam provides for concurrency using lightweight threads.
The following subsections describes the 
built-in concurrency and concurrency control functions and give details on 
their use.

\section{Built-in support for threads}

The following functions can be used to control threads and their
interactions:

\begin{description}
\item[thread]
This one-argument function takes in an expression and evaluates in
parallel to the calling thread and returns the thread id of the new
thread.  The code below:

\begin{verbatim}
    (thread (println "Hello World 1"))
    (thread (println "Hello World 2"))
    (thread (println "Hello World 3"))
\end{verbatim} 

yields something similar to the following output:

\begin{verbatim}
    HellHelo Worllod  3
    World 1
    Hello World 2
\end{verbatim} 

The problem here is that threads are executing in parallel.  This means
that when you print out from one thread to the console another thread
could be as well; and you get the overlap seen above.  To get around
you can either use a mutex or the built-in function {\it displayAtomic}.

\item[getTID]
This no-argument function returns the thread ID of the current thread.
The code below:

\begin{verbatim}
    (thread (println (getTID)))
\end{verbatim}

yields something similar to the following output:

\begin{verbatim}
    2
\end{verbatim}

\item[lock]
This no-argument function acquires the built-in mutex.

\item[unlock]
This no-argument function releases the built-in mutex.
Only the thread that acquired the mutex can release it.
The code below demonstrates both {\it lock} and {\it unlock}:

\begin{verbatim}
    (thread (begin (lock) (println "Hello World 1") (unlock)))
    (thread (begin (lock) (println "Hello World 2") (unlock)))
    (thread (begin (lock) (println "Hello World 3") (unlock)))
\end{verbatim} 

yields something similar to the following output:

\begin{verbatim}
    Hello World 2
    Hello World 3
    Hello World 1
\end{verbatim} 

Only the thread that locked the mutex can unlock it.

\item[tjoin]
This one-argument function causes the current thread to wait until
a particular thread terminates. If the desired thread has already
terminated, the function immediately returns. The desired thread is
specified by passing its thread ID to the function.  The code below:

\begin{verbatim}
    (define firstTID (thread (println (fib 25))))
    (tjoin firstTID)
    (thread (println (fib 10)))
\end{verbatim} 

yields something similar to the following output:

\begin{verbatim}
   75025
   55
\end{verbatim} 

Note that the thread which is evaluating fibonacci of 10 must wait until
the thread which is evaluating fibonacci of 25 has finished.

\item[displayAtomic]
This variadic function is similar to {\it display}; however, it ensures
that there will be no overlap of output if the user only uses {\it
displayAtomic} for printing.  The code below:

\begin{verbatim}
    (thread (displayAtomic "Hello World 1\n"))
    (thread (displayAtomic "Hello World 2\n"))
    (thread (displayAtomic "Hello World 3\n"))
\end{verbatim} 

yields something similar to the following output:

\begin{verbatim}
    Hello World 2
    Hello World 1
    Hello World 3
\end{verbatim} 

\end{description}

\section{Thread pools}
In order to avoid the sometimes inevitable system-level restrictions
on the maximum number of lightweight threads allowed per process, Scam
supports the creation of thread pools.  A thread pool pre-allocates
a fixed number of threads, and maintains a queue of expressions.
Expressions may be pushed onto the queue, and the thread pool
will automatically pop the expressions off in order to be executed
concurrently.  The code below
illustrates the use of a pool:

    \begin{verbatim}
        (define (myprint msg) (displayAtomic "Result is " msg "\n"))
        (define pool (tpool 10))
        ((pool 'push) (fib 10))
        ((pool 'push) (fib 11))
        ((pool 'push) (fib 12) myprint)
        ((pool 'push*) (fib 14) this)
        ((pool 'push*) (fib 14) this myprint)
        ((pool 'shutdown))
    \end{verbatim} 


The following constructors and methods are associated with pools:

\begin{description}
\item[tpool]
This one-argument constructor creates a new thread pool object.
The single argument specifies the number of concurrent threads allowed.

\item[push]
This variadic function takes in an expression and an optional
number of call-back functions.  The call-back functions are called
with the result of the evaluated expression.

\item[push*]
This variadic function takes in an expression, a environment,
and an optional number of expressions. 
The required expression is evaluated under the given environment.
The optional expressions are then evaluated with the return value
of the evaluated required expression.
\item[empty?]

This no-argument function returns true if there are no expressions
in the work queue or in the running queue, otherwise it returns
false.

\item[join]
This no-argument function turns off acceptance of new expressions
until the running queue is empty.

\item[shutdown]
This no-argument function waits for the active threads to finish
before letting the thread pool expire.
\end{description}

\section{Parallel execution of expressions}

For compatibility with MIT Scheme, 
expressions in Scam can be evaluated in parallel with the variadic function
{\it pexecute}:

\begin{verbatim}
    (pexecute expr1 expr2 .... exprN)
\end{verbatim}

which is equivalent to the following:

\begin{verbatim}
    (begin
        (thread expr1)
        (thread expr2)
        ...
        (thread exprN)
        )
\end{verbatim}

Each of the expressions will execute in parallel in separate processes.
Conventionally, the expressions are calls to no-argument functions, as in:

\begin{verbatim}
    (pexecute f g (lambda () ...))
\end{verbatim}

In the above example, the functions {\it f} and {\it g}
and the body of the lambda
will all be executed in parallel.

Another function, {\it pexecute*}, is similar to {\it pexecute}, except
that it serializes each expression, in the order given. The call:

\begin{verbatim}
    (pexecute* expr1 expr2 .... exprN)
\end{verbatim}

is equivalent to the following:

\begin{verbatim}
    (begin
        (expr1)
        (expr2)
        ...
        (exprN)
        )
\end{verbatim}

The {\it pexecute*} function is useful for debugging concurrency problems.

\subsection{Debugging concurrency problems}

Locking and unlocking of threads can be difficult to debug.  For this reason, 
Scam has a built-in function for determining which thread has the current lock.
This function, {\it debugMutex} enables and disables the debugging mechanism.

\begin{verbatim}
    (debugMutex #t)
    (debugMutex #f)
\end{verbatim}

The first call turns debugging on while the second turns debugging off. When
on, attempts to acquire the mutex produce output (on stderr) of
the form:

\begin{verbatim}
    thread XXXX is acquiring...
\end{verbatim}

where XXXX is replaced by the process id of the acquiring process.
If the mutex is actually acquired, debugging emits:

\begin{verbatim}
    thread XXXX has acquired.
\end{verbatim}

On the releasing side, debugging emits messages of the form:

\begin{verbatim}
    thread XXXX is releasing...
    thread XXXX has released.
\end{verbatim}

[We need something like this to simulate 
When a process executing in parallel throws an exception, {\it pexecute} will
produce an error message similar to:

\begin{verbatim}
    file philosophers.scm,line 356: parallel execution of thread 3 failed
    try using ????? for more information
\end{verbatim}

If a thread terminates with an exception,
calling ???? may reveal the
exception that caused the failure.
The ???? call simulates concurrency, but actually runs the
given expressions sequentially in the parent process.
