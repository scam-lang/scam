\chapter{Conditionals}
\label{Conditionals}

Conditionals implement decision points in a computer program.
Suppose you have a program that performs some task on an
image. You may well have a point in the program where you
do one thing if the image is a JPEG and quite another
thing if the image is a GIF file. Likely, at this point,
your program will include a conditional expression to make
this decision.

Before learning about conditionals, it is important to
learn about logical expressions. Such expressions are the
core of conditionals and loops.\footnote{
We will learn about loops in the next chapter.
}

\section{Logical expressions}

A logical expression evaluates to a truth value, in essence true or
false. For example, the expression $x > 0$
resolves to true if {\it x} is positive
and false if {\it x} is negative or zero. In Scam, truth is represented by
the symbol {\tt True} and falsehood
by the symbol {\tt False}.
Together,
these two symbols are known as {\it {\sc boolean}} values.

One can assign truth values to variables:

\begin{verbatim}
      >>> c = -1
      >>> z = c > 0;

      >>> z
      False
\end{verbatim}

Here, the variable {\it z} would be assigned a value of {\tt True}
if {\it c} is positive;
since {\it c} is negative, it is assigned a value of {\tt False}.

\section{Logical operators}

Scam has the following logical operators.
 
\begin{tabular}{cl}
    {\tt ==}    & equal to \\
    {\tt !=}    & not equal to \\
    {\tt >=}    & greater than or equal to \\
    {\tt >}     & greater than \\
    {\tt <}     & less than \\
    {\tt <=}    & less than or equal to \\
    {\tt and}   & and \\
    {\tt or}    & or \\
\end{tabular}

The first five operators are used for comparing two things,
while the last two operators are the glue that joins up simpler
logical expressions into more complex ones.

\section{Short circuiting}

When evaluating a logical expression,
Scam evaluates the expression from left to right and
stops evaluating as soon as it finds out that the expression
is definitely true or definitely false.
For example, when encountering the expression:

\begin{verbatim}
      x != 0 and y / x > 2
\end{verbatim}

if {\it x} has a value of 0, the subexpression on the left side of the
{\tt and}
connective resolves to false. At this point, there is no way for the
entire expression to be true (since both the left hand side and the right
hand side must be true for an
{\tt and}
expression to be true), so the right
hand side of the expression is not evaluated. Note that this expression
protects against a divide-by-zero error.

\section{If expressions}

Scam's {\it if} expressions are used to conditionally execute code,
depending on the truth value of what is known as the
{\it test} expression. One version of {\it if} has a block of
code following the test expression:

Here is an example:

\begin{verbatim}
    if (name == "John"):
        print("What a great name you have!")
\end{verbatim}

In this version, when the test expression is true ({\it i.e.}, 
the string {\tt "John"} is bound to the variable {\it name}), 
then the code that is indented under the {\tt if} is evaluated 
(i.e., the compliment
is printed). The indented code is known as a {\it block}.
If the test expression is false, however the
block is not evaluated.
In this text, we will enclose the test expression in parentheses
even though it is not required by Scam. We do that because
some important programming languages require the parentheses
and we want to get you into the habit.

Here is another form of {\it if}:

\begin{verbatim}
    if (major == "Computer Science"):
        print("Smart choice!")
    else:
        print("Ever think about changing your major?")
\end{verbatim}

In this version, {\it if} has two blocks, one following the
test expression and one following the {\tt else} keyword.
Note the colons that follow the test expression and the else;
these are required by Scam.
As before, the first block is evaluated if the test expression
is true. If the test expression is false, however,
the second block is evaluated instead.

\section{if-elif-else chains}

You can chain {\tt if} statements together, as in:

\begin{verbatim}
    if (bases == 4):
        print("HOME RUN!!!")
    elif (bases == 3):
        print("Triple!!")
    elif (bases == 2):
        print("double!")
    elif (bases == 1) 
        print("single")
    else:
        print("out")
\end{verbatim}

The block that is eventually evaluated is
directly underneath the first test expression
that is true, reading from top to bottom.
If no test expression is true, the block associated
with the else is evaluated.

What is the difference between {\tt if-elif-else}
chains and a sequence of
unchained {\it if}s? Consider this rewrite of the
above code:

\begin{verbatim}
    if (bases == 4):
        print("HOME RUN!!!");
    if (bases == 3):
        print("Triple!!");
    if (bases == 2):
        print("double!");
    if (bases == 1):
        print("single");
    else:
        print("out");
\end{verbatim}

In the second version, there are four if statements and
the else belongs to the last if. Does this behave exactly
the same? The answer is, it depends. Suppose the value
of the variable {\it bases} is 0. Then both versions print:

\begin{verbatim}
    out
\end{verbatim}

However, if the value of {\it bases} is 3, for example, the first
version prints:

\begin{verbatim}
    triple!!
\end{verbatim}

while the second version prints:

\begin{verbatim}
    triple!!
    out
\end{verbatim}

Why the difference? In the first version, a subsequent test
expression is evaluated {\it only} if all previous test expressions
evaluated to false. Once a test expression evaluates to true in
an {\tt if-elif-else} chain, the associated block is evaluated and
no more processing of the chain is performed. Like the {\tt and} and
{\tt or} {\sc boolean} connectives,
an {\tt if-elif-else} chain short-circuits.

In contrast, the sequences of {\tt if}s are independent; there is no
short-circuiting. When the test expression of the first if
fails, the test expression of the second {\tt if} succeeds
and {\tt triple!!} is
printed. Now the test expression of the third if is tested and
fails as well as the test expression of the fourth if. But since
the fourth if has an else, the {\tt else} block is evaluated and
{\tt out}
is printed.

It is important to know when to use an {\tt if-elif-else} chain and
when to use a sequence of independent {\tt if}s.
If there should be only
one outcome, then use an {\tt if-elif-else} chain. Otherwise,
use a sequence of {\tt if}s.
