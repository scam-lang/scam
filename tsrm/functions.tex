\chapter{Functions}
\label{Functions}

Recall from 
\link*{the chapter on assignment}[Chapter~\Ref]{Assignment}
the series of expressions we evaluated to find
the {\it y}-value of a point on the line:

\begin{verbatim}
    y = 5x - 3
\end{verbatim}
    
First, we assigned values to the slope,
the {\it x}-value, and the {\it y}-intercept:

\begin{verbatim}
    >>> m = 5
    >>> x = 9  
    >>> b = -3
\end{verbatim}

Once those variables have been assigned,
we can compute the value of {\it y}:

\begin{verbatim}
    >>> y = m * x + b

    >>> y
    42
\end{verbatim}

Now, suppose we wished to find the {\it y}-value corresponding to
a different {\it x}-value or, worse yet, for a different {\it x}-value
on a different line. All the work we did would have to be
repeated. A {\it function} is a way to encapsulate all these operations
so we can repeat them with a minimum of effort.

\section{Encapsulating a series of operations}

First, we will define a not-too-useful function that
calculates {\it y} give a slope of 5,
a {\it y}-intercept of -3, and an
{\it x}-value of 9 (exactly
as above). We do this by wrapping a function around
the sequence of operations above.
The return value of this function is the computed {\it y} value:

\begin{verbatim}
def y():
    m = 5
    x = 9
    b = -3
    return m * x + b
\end{verbatim}

There are a few things to note. The keyword {\tt def} indicates
that a function definition is occurring. The name of this
particular function is {\it y}. The names of the things
being sent to the function are given between the parentheses;
since there is nothing between the parentheses, we don't
need to send any information to this function when we call it.
Together, the first line is known as the {\it function signature},
which tells you the name of the function and how many values
it expects to be sent when called.

The stuff indented from the first line of the function definition
is called the {\it function body} and
is the code that will be evaluated (or executed) when the
function is called. You must remember this: {\it the function body
is not evaluated until the function is called}.

Once the function is defined, we can find the value of {\it y} repeatedly.
Let's assume the function was entered into the file named
{\it line.py}.

First we import the code in {\it line.py} with the from statement:

\begin{verbatim}
    >>> from line import *   # not line.py!
\end{verbatim}

This makes the python interpreter behave as if we had typed
in the function definition residing in {\it line.py} directly into
the interpreter. Note, we omit the {\it .py} extension in the import
statement.

After importing the {\it y} function, the next thing we do
is call it:

\begin{verbatim}
    >>> y()
    42
    >>> y()
    42
\end{verbatim}
    
The parentheses after the {\it y} indicate that we wish to call
the {\it y} function and get its value. Because we designed the
function to take no values when called, we do not place any
values between the parentheses.

Note that when we call the {\it y} function again,
we get the exact same answer.

The {\it y} function, as written,
is not too useful in that we cannot use it to compute
similar things, such as the {\it y}-value for a different value of
{\it x}.
This is because we `hard-wired' the values of {\it b}, {\it x}, and {\it m},
We can improve this function by passing in the value of {\it x}
instead of hard-wiring the value to 9.

\section{Passing arguments}

A hallmark of a good function is that it lets you compute
more than one thing. We can modify our {\it y} function to {\it take in} the
value of {\it x} in which we are interested.
In this way,
we can compute more than one value of {\it y}.
We do this by {\it passing} in 
an {\it argument}\footnote{
The information that is passed into a function is collectively
known as {\it arguments}.}, in this case, the value of {\it x}.

\begin{verbatim}
def y(x):
    slope = 5
    intercept = -3
    return slope * x + intercept
\end{verbatim}

Note that we have moved {\it x} from the body of the function
to between the parentheses. We have also refrained from
giving it a value since its value is to be sent to the function
when the function is called.
What we have done is to {\it parameterize} the function to make it more
general and more useful. The variable {\it x} is now called a
{\it formal parameter} since it sits between the parentheses in
the first line of the function definition.

Now we can compute {\it y} for an infinite number of {\it x}'s:

\begin{verbatim}
    >>> from line.py import *
    >>> y(9)
    42
    
    >>> y(0)
    -3
    
    >>> y(-2)
    -13
\end{verbatim}

What if we wish to
compute a {\it y}-value for a given {\it x} for a different
line? One approach would be to pass in the {\it slope} and {\it intercept}
as well as {\it x}:

\begin{verbatim}
def y(x,m,b):
    return m * x + b
\end{verbatim}

Now we need to pass even more information to {\it y} when we call it:
    
\begin{verbatim}
    >>> from line.py import *
    >>> y(9,5,-3)
    42
     
    >>> y(0,5,-3)
    -3
\end{verbatim}

If we wish to calculate using a different line, we just pass in the
new {\it slope} and {\it intercept} along with our value of {\it x}.
This certainly works as intended, but is not the best way. One problem
is that we keep on having to type in the slope and intercept even if
we are computing {\it y}-values on the same line. Anytime you
find yourself doing the same tedious thing over and over,
be assured that
someone has thought of a way to avoid that particular tedium.
If so, how do we
customize our function so that we only have to enter the slope
and intercept once per particular line? We will explore
one way for doing this. In reading further,
it is not important if you understand all that is going on.
What is important is that you know you can use functions
to run similar code over and over again.

\section{Creating functions on the fly}

Since creating functions is hard work (lots of typing) and
Computer Scientists avoid unnecessary work like the plague, somebody
early on got the idea of writing a function that itself 
creates functions! Brilliant! We can do this for our line problem.
We will tell our creative function to create a {\it y} function
for a particular slope and intercept! While we are at it,
let's change the variable names {\it m} and {\it b} to {\it slope}
and {\it intercept}, respectively:

\begin{verbatim}
def makeLine(slope,intercept):
    def y(x):
        return slope * x + intercept
    return y    # the value of y is returned, y is NOT CALLED!
\end{verbatim}

The {\it makeLine} function creates a {\it y} function
and then returns it. Note that this returned function {\it y} takes
one value when called, the value of {\it x}.

So our creative {\it makeLine} function
simply defines a {\it y} function and then
returns it. Now we can create a bunch of different lines:

\begin{verbatim}
    >>> from line.py import *
    >>> a = makeLine(5,-3)
    >>> b = makeLine(6,2)

    >>> a(9)
    42
    
    >>> b(9)
    56

    >>> a(9)
    42
\end{verbatim}

Notice how lines {\it a} and {\it b} remember
the slope and intercept supplied
when they were created.\footnote{
The local function {\it y} does not really remember these values,
but for an introductory course, this is a good enough explanation.}.
While this is decidedly cool, the problem is many languages (C, C++, and Java
included\footnote{
C++ and Java, as well as Scam, give you another approach, {\it objects}.
We will not go into objects in this course, but
you will learn all about them in your next programming course.})
do not allow you to define functions that create other functions.
Fortunately, Scam does allow this.

While this might seem a little mind-boggling, don't worry. The
things you should take away from this are:

\begin{itemize}
\item
    functions encapsulate calculations
\item
    functions can be parameterized
\item
    functions can be called
\item
    functions return values
\end{itemize}

\section{The Function and Procedure Patterns}

When a function calculates (or obtains) a value and returns it, we say
that it implements the {\it function} pattern. If a function
does not have a return value, we say it implements the
{\it procedure} pattern.

Here is an example of the {\it function} pattern:

\begin{verbatim}
def square(x):
    return x * x
\end{verbatim}

This function takes a value, stores it in {\it x}, computes the square
of {\it x} and return the result of the computation.

Here is an example of the {\it procedure} pattern:

\begin{verbatim}
def greeting(name):
    print("hello,",name)
\end{verbatim}

Almost always,
a function that implements the {\it function} pattern does not print
anything, while a function that implements the procedure
pattern often does\footnote{Many times, the printing is
done to a file, rather than  the console.}.
A common function that implements the procedure pattern
is the {\it main} function.
A common mistake made by beginning programmers is
to print a calculated value rather than to return it. So, when defining
a function, you should ask yourself, should I implement the function
pattern or the procedure pattern?

Most of the function you implement in this class follow the function
pattern.

Another common mistake is to inadvertently implement a {\it procedure} pattern
when a {\it function} pattern is called for. This happens when the {\it return}
keyword is omitted.

\begin{verbatim}
def psquare(x):
    x * x
\end{verbatim}

While the above code looks like the function pattern, it is actually
a procedure pattern. What happens is the value $x * x$ is calculated,
but since it is not returned, the newly calculated value is thrown
away (remember the {\it throw-away} pattern?).

Calling this kind of function yields a surprising result:

\begin{verbatim}
    >>> psquare(4)
    >>>

    >>>print(psquare(6))
    None
\end{verbatim}

When you do not specify a return value, but you use
the return value anyway (as in the printing example),
the return value is set to {\it None}.

Usually, the procedure pattern causes some side-effect to happen
(like printing). A procedure like {\it psquare}, which has no side-effect
is a useless function.
