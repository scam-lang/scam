\documentclass{book}  
\usepackage[usenames]{color}
\usepackage{graphicx}
\usepackage{hyperlatex}
\usepackage{upquote}
\usepackage[margin=1.0in]{geometry}
\usepackage[compact,small]{titlesec}
\usepackage{booktabs}
\usepackage{makeidx}

\W\newcommand{\not}{\xmlent{##172}}
\W\newcommand{\pm}{\xmlent{##177}}

\W\newcommand{\alpha}{\xmlent{##945}}
\W\newcommand{\beta}{\xmlent{##946}}
\W\newcommand{\gamma}{\xmlent{##947}}
\W\newcommand{\delta}{\xmlent{##948}}
\W\newcommand{\epsilon}{\xmlent{##949}}
\W\newcommand{\varepsilon}{\xmlent{##949}}
\W\newcommand{\zeta}{\xmlent{##950}}
\W\newcommand{\eta}{\xmlent{##951}}
\W\newcommand{\theta}{\xmlent{##977}}
\W\newcommand{\vartheta}{\xmlent{##952}}
\W\newcommand{\iota}{\xmlent{##953}}
\W\newcommand{\kappa}{\xmlent{##954}}
\W\newcommand{\lambda}{\xmlent{##955}}
\W\newcommand{\mu}{\xmlent{##956}}
\W\newcommand{\nu}{\xmlent{##957}}
\W\newcommand{\xi}{\xmlent{##958}}
\W\newcommand{\pi}{\xmlent{##960}}
\W\newcommand{\varpi}{\xmlent{##982}}

\W\newcommand{\rho}{\xmlent{##961}}
\W\newcommand{\varrho}{\xmlent{##961}}
\W\newcommand{\sigma}{\xmlent{##963}}
\W\newcommand{\varsigma}{\xmlent{##963}}
\W\newcommand{\tau}{\xmlent{##964}}
\W\newcommand{\upsilon}{\xmlent{##965}}
\W\newcommand{\phi}{\xmlent{##966}}
\W\newcommand{\varphi}{\xmlent{##966}}
\W\newcommand{\chi}{\xmlent{##967}}
\W\newcommand{\psi}{\xmlent{##968}}
\W\newcommand{\omega}{\xmlent{##969}}


\W\newcommand{\Gamma}{\xmlent{##915}}
\W\newcommand{\Delta}{\xmlent{##916}}
\W\newcommand{\Theta}{\xmlent{##920}}
\W\newcommand{\Gamma}{\xmlent{##915}}
\W\newcommand{\Lambda}{\xmlent{##923}}
\W\newcommand{\Xi}{\xmlent{##926}}
\W\newcommand{\Pi}{\xmlent{##928}}
\W\newcommand{\Sigma}{\xmlent{##931}}
\W\newcommand{\Upsilon}{\xmlent{##933}}
\W\newcommand{\Phi}{\xmlent{##934}}
\W\newcommand{\Omega}{\xmlent{##937}}

\W\newcommand{\Re}{\xmlent{##8476}}
\W\newcommand{\Im}{\xmlent{##8465}}
\W\newcommand{\aleph}{\xmlent{##8501}}
\W\newcommand{\partial}{\xmlent{##8706}}
\W\newcommand{\forall}{\xmlent{##8704}}
\W\newcommand{\partial}{\xmlent{##8706}}
\W\newcommand{\exists}{\xmlent{##8707}}
\W\newcommand{\emptyset}{\xmlent{##8709}}

\W\newcommand{\nabla}{\xmlent{##8711}}
\W\newcommand{\in}{\xmlent{##8712}}
\W\newcommand{\ni}{\xmlent{##8715}}

\W\newcommand{\prod}{\xmlent{##8719}}
\W\newcommand{\sum}{\xmlent{##8721}}
\W\newcommand{\sqrt}{\xmlent{##8730}}
%\newcommand{\htmlsqrt}[1]{\xmlent{##8730} (#1)}

\W\newcommand{\propto}{\xmlent{##8733}}
\W\newcommand{\infin}{\xmlent{##8734}}
\W\newcommand{\angle}{\xmlent{##8736}}
\W\newcommand{\wedge}{\xmlent{##8743}}
\W\newcommand{\vee}{\xmlent{##8744}}
\W\newcommand{\cup}{\xmlent{##8745}}
\W\newcommand{\cap}{\xmlent{##8746}}
\W\newcommand{\int}{\xmlent{##8747}}
\W\newcommand{\sim}{\xmlent{##8764}}
\W\newcommand{\approx}{\xmlent{##8776}}
\W\newcommand{\neq}{\xmlent{##8800}}
\W\newcommand{\equiv}{\xmlent{##8801}}
\W\newcommand{\leq}{\xmlent{##8804}}
\W\newcommand{\geq}{\xmlent{##8805}}


\W\newcommand{\cdot}{\xmlent{##8901}}


\W\usepackage{rhxpanel}
\W\newcommand{\HlxPanelHome}{0}
\htmlcss{lusth.css}
\newcommand{\HlxIcons}{}
\htmlpanelfield{Contents}{hlxcontents}
\W\newcommand\sf \bf
\W\newcommand\sc \it

\W\newcommand\highlight{white}
\T\newcommand\highlight{black}

\htmldepth{2}
\htmltitle{The Scam Reference Manual}
\htmladdress{lusth@cs.ua.edu}

\T\setlength\parskip{10 pt}
\T\setlength\parindent{0 in}

\title{The Scam Reference Manual}

\author{by John C. Lusth}

\makeindex
\begin{document}

\maketitle

\W\htmlrule
\W\xlink{Printable Version}{book.pdf}
\W\htmlrule


\begin{rawxml}

<script
    type="text/javascript">
    // Google Internal Site Search script-
    // By JavaScriptKit.com (http://www.javascriptkit.com)
    // For this and over 400+ free scripts,
    // visit JavaScript Kit- http://www.javascriptkit.com/
    // This notice must stay intact for use

    //Enter domain of site to search.
    var domainroot="beastie.cs.ua.edu/scam/tsrm/"

    function Gsitesearch(curobj)
        {
        curobj.q.value="site:"+domainroot+" "+curobj.qfront.value
        }
    </script>
<form
    action="http://www.google.com/search"
    method="get" onSubmit="Gsitesearch(this)">

    <p>Search the encyclop&aelig;dia: 
    <input name="q" type="hidden" />
    <input name="qfront" type="text" style="width: 180px" />
    <input type="submit" value="Google Search" /></p>
    </form>
<p style="font: normal 11px Arial">This free search script provided by<br />
<a href="http://www.javascriptkit.com">JavaScript Kit</a></p>

<h2>Chapters</h2>
\end{rawxml}

\tableofcontents
\setcounter{tocdepth}{2} 

\chapter{Notes on Terminology}
\label{NotesOnTerminology}

This document assumes familiarity with basic programming,
especially the idea of expressions and built-in functions
and placing source code in files.

\verb!Code! is displayed in a \verb!typewriter font!, while {\it variables}
and {\it filenames} are shown in an {\it italic font}.

Sometimes an expression will be given as an example, as in:

\begin{verbatim}
    (+ 2 3 4)
\end{verbatim}

If the value of the expression is of interest, it is shown
on the next line, introduced by the symbol \verb!->!, as in:

\begin{verbatim}
    (+ 2 3 4)
    -> 9
\end{verbatim}

For the 
{\it display}, {\it print}, {\it println}, and {\it inspect} functions,
the response is the print value, not the return value.

\begin{verbatim}
    (inspect (+ 2 3 4))
    -> (+ 2 3 4) is 9
\end{verbatim}

Finally, this document assumse a familiarity with the command-line in
Linux or the Mac OS.

\chapter{Starting Out}
\label{StartingOut}

A word of warning: if you require fancy graphically oriented development
environments, you will be sorely disappointed in the Scam Programming
Language. Scam is programming at its simplest: a prompt at which you type
in expressions which the Scam interpreter evaluates. Of course, you can
create Scam programs using your favorite text editor (mine is {\it vim}).
Scam can be used as a scripting language, as well.

\section{Running Scam}

In a terminal window, simply type the command:

\begin{verbatim}
    scam <fileName>
\end{verbatim}

where \verb!<fileName>! is replaced by the name of the file contining your
Scam program ({\it i.e.} Scam source code).
You should be rewarded with the output from your Scam program.

For example, create a file named hello.s. In it, place the line:

\begin{verbatim}
    (println "hello, world!")
\end{verbatim}

Save the file and exit. Now run your program:

\begin{verbatim}
    $ scam hello.s
    hello, world!
    $
\end{verbatim}

The \verb!$! represents the system prompt.

\section{Scam options}

Scam takes a number of options:

\begin{description}
\item[-M]
    
    display current memory size, then exit
    
\item[-m NNN]
    
    set the memory size to {\it NNN}
    
\item[-v]

    display the Scam version number

\item[-t]

    display a full error trace for an uncaught exception

\end{description}

At this point, you are ready to proceed to next chapter.

\chapter{Comments}
\label{Comments}

Comments should be used both sparingly and effectively in code.
Scam allows for three kinds of comments:

\begin{description}
\item[file]
    all text in a file after the
    two character combination \verb!;$! is ignored.
\item[block]
    all text between the
    two character combinations \verb!;{! and \verb!;}! is ignored.
\item[line]
    all text on a line following the character \verb!;! is ignored.
\end{description}

\chapter{Literals}
\label{Literals}

Scam works by figuring out the meaning or value of some code.
This is true for the tiniest pieces of code to the largest
programs. The process of finding out the meaning of code
is known as {\it evaluation}.

The things whose values are the things themselves are known as
{\it literals}. The literals of Scam can be categorized by the following
types:
{\it integers}, {\it real} {\it numbers}, {\it strings}, {\sc Booleans},
{\it symbols}, and {\it lists}.

Scam (or more correctly, the Scam interpreter) responds to literals
by echoing back the literal itself.
Here are examples of each of the types:

\begin{verbatim}
    (inspect 3)
    -> 3 is 3
     
    (inspect -4.9
    -> -4.900000 is -4.900000
     
    (inspect "hello")
    -> hello is hello
     
    (inspect #t)
    -> #t is #t

    (inspect (list 3 -4.9 "hello"))
    -> (list 3 -4.9 "hello") is (3, -4.9, "hello")
\end{verbatim}
Let's examine the
five types in more detail.

\section{Integers}

Integers are numbers without any fractional parts.
Examples of integers are:

\begin{verbatim}
    (inspect 3)
    -> 3 is 3
    
    (inspect -5)
    -> -5 is -5
    
    (inspect 0)
    -> 0 is 0
\end{verbatim}

Integers must begin with a digit or a minus sign. The initial minus sign
must immediately be followed by a digit.

\section{Real Numbers}

Reals are numbers that do have a fractional part (even if that fractional
part is zero!). Examples of real numbers are:

\begin{verbatim}
    (inspect 3.2)
    -> 3.200000 is 3.200000
    
    (inspect 4.0)
    -> 4.000000 is 4.000000
       
    (inspect 5.)
    -> 5.000000 is 5.000000
       
    (inspect 0.3)
    -> 0.300000 is 0.300000
       
    (inspect .3)
    -> 0.300000 is 0.300000
    
    (inspect 3.0e-4)
    -> 0.000300 is 0.000300
    
    (inspect 3e4)
    -> 30000.000000 is 30000.000000
    
    (inspect .000000987654321)
    -> 0.000001 is 0.000001
\end{verbatim}

Real numbers must start with a digit or a minus sign or a decimal
point. An initial minus sign must immediately be followed by a digit or a
decimal point. An initial decimal point must immediately be followed by
a digit. Scam accepts real numbers in scientific notation. For example,
$3.0 * 10^{-11}$ would be entered as 3.0e-11. The `e' stands for exponent and
the 10 is understood, so e-11 means multiply whatever precedes the
e by $10^{-11}$.

The Scam interpreter can hold huge numbers,
limited by only the amount of memory available to the
interpreter,
but holds only 15 digits after the decimal point:

\begin{verbatim}
    (inspect 1.2345678987654329
    -> 1.234568 is 1.234568
\end{verbatim}

Note that Scam rounds up or rounds down, as necessary.

Numbers greater than $10^6$ and
less than $10^{-6}$ are displayed in
scientific notation.

\section{Strings}

Strings are sequences of characters delineated by double quotation marks:

\begin{verbatim}
    (println "hello, world!")
    -> hello, world!
    
    (println "x\nx")
    -> x
       x
    
    (println "\"z\"")
    -> "z" 
\end{verbatim}

Scam accepts both double quotes and single quotes to
delineate strings. In this text, we will use the convention
that double quotes are used for strings of multiple 
characters and single quotes for strings consisting of
a single character.

Characters in a string can be {\it escaped} (or quoted)
with the backslash character,
which changes the meaning of some characters. For example, the character
{\it n}, in a string refers to the letter {\it n} while the character sequence
{\it $\backslash$n}
refers
to the {\it newline} character. A backslash also changes the meaning of the
letter {\it t},
converting it into a tab character.
You can also quote single and double quotes with backslashes.
When other characters are escaped,
it is assumed the backslash is a character of the
string and it is escaped (with a backslash) in the result:

\begin{verbatim}
    (println "\z")
    -> z
\end{verbatim}

Note that Scam, when asked
the value of strings that contain newline and tab characters, displays
them as escaped characters. When newline and tab characters in a string
are printed in a program, however, they are displayed as actual newline
and tab characters, respectively.
As already noted,
double and single quotes can be embedded in a
string by quoting them with backslashes. A string with no characters
between the double quotes is known as an empty string.

Unlike some languages, there is no character type in Scam. A single
character {\verb+a+}, for example, is entered as the string
{\verb+"a"+}.

\section{Symbols}

A symbol is a set of characters, much like a string. Like strings,
symbols evaluate to themselves. Unlike strings,
symbols are not formed using a beginning quotation mark and an
ending quotation mark. They are also limited
in the characters that compose them. For example, a symbol cannot
contain a space character while a string can. A symbol is introduced
with a single quotation mark:

\begin{verbatim}
    (print 'a)
    -> a

    (print 'hello)
    -> hello
\end{verbatim}

We we learn more about symbols and their relationship to entities
called {\it variables} in a later chapter.

\section{True, False, and nil}

There are two special literals, \verb!#t!
and \verb!#f!.
These literals are known as the {\sc Boolean} values;
\verb!#t! is true and \verb!#f! is false.
Boolean values are used to guide the flow of a program.
The term {\sc Boolean} is derived from the last name of George Boole, who,
in his 1854 paper {\it An Investigation of the Laws of Thought, on which are
founded the Mathematical Theories of Logic and Probabilities}, laid one
of the cornerstones of the modern digital computer. The so-called {\sc Boolean}
logic or {\sc Boolean} algebra is concerned with the rules of combining truth
values (i.e., true or false). As we will see, knowledge of such rules will
be important for making Scam programs behave properly. In particular,
{\sc Boolean} expressions will be used to control conditionals and loops.

Another special literal is \verb!nil!.
This literal is used to
indicate an empty list or an empty string; it also is used
to indicate something that has not yet been
created. More on \verb!nil! when we cover lists and
objects.

\section{Lists}

Lists are just collections of entities.
The simplest list is the empty list:

\begin{verbatim}
    (inspect ())
    -> nil
\end{verbatim}

Since the empty list looks kind of strange, Scam uses the symbol \verb!nil!
to represent an empty list.

One creates non-empty list by
using the built-in {\it list function}. 
Here, we make a list containing the numbers
10, 100, and 1000:

\begin{verbatim}
    (list 10 100 1000)
    -> (10 100 1000)
\end{verbatim}

Lists can contain values besides numbers:

\begin{verbatim}
    (list 'a "help me" length)
    -> (a "help me" <built-in length(items)>)
\end{verbatim}

The first value is an integer, the second a string,
and the third item is also a function. 
The built-in {\it length} function
is used to tell us how many items are in a list:

\begin{verbatim}
    (length (list 'a "help me" length))
    -> 3
\end{verbatim}

As expected, the {\it length} function tells us that the list
\verb!('a "help me" length)! has three items in it.

Lists can even contain lists!

\begin{verbatim}
    (list 0 (list 3 2 1) 4)
    -> (0 (3 2 1) 4)
\end{verbatim}

A list is something known as a {\it data structure};
data structures are extremely important in writing
sophisticated programs.

We will see more of lists in a later chapter.

\chapter{Combining Literals}
\label{CombiningLiterals}

Like the literals themselves, combinations of literals are also
expressions. For example, suppose you have forgotten your times table
and aren't quite sure whether 8 times 7 is 54 or 56. We can ask Scam,
presenting the interpreter with the expression:

\begin{verbatim}
    (* 8 7)
    -> 56
\end{verbatim}

The
multiplication sign * is known as an {\it operator}, as it {\it operates} on the 8
and the 7, producing an equivalent literal value.
As with all LISP/Scheme-like languages, operators like \\verb!*!
are true functions and thus prefix notation is used in function calls.

The 8 and the 7 in the above expressiion are
known as {\it operands}. It seems that the actual names of various operands are
not being taught anymore, so for nostalgia's sake, here they are. The
operand to the left of the multiplication sign (in this case the 8) is
known as the {\it multiplicand}. The operand to the right (in this case the 7)
is known as the {\it multiplier}. The result is known as the {\it product}.

The operands of the other basic operators
have special names too. For addition of two operands, the left operand is known as the
{\it augend} and the right operand is known as the {\it addend}.
The result is known as the {\it sum}.
For subtraction,
the left operand is the {\it minuend}, the right the {\it subtrahend}, and
the result as the {\it difference}.
Finally
for division (and I think this is still taught), the left operand is
the {\it dividend}, the right operand is the {\it divisor}, and the 
result is the {\it quotient}.

We separate
operators from their operands by
spaces, tabs,
or newlines, collectively known as {\it whitespace}.\footnote{
Computer Scientists, when they have to write their annual reports,
often refer to the things they are reporting on as
{\it darkspace}. It's always good to have a lot of darkspace in
your annual report!
}

Scam always takes in an expression and returns an equivalent
literal expression ({\it e.g.}, integer or real). All Scam operators are
variadic, meaning they operate on exactly on any number of operands:

\begin{verbatim}
    (+ 1 2 3 4 5)
    -> 15
\end{verbatim}

\section{Numeric operators}

If it makes sense to add two things together, you can probably do it in
Scam using the + operator. For example:

\begin{verbatim}
    (+ 2 3)
    -> 5
    
    (+ 1.9 3.1)
    -> 5.000000
\end{verbatim}
    
One can see that if one adds two integers, the result is an integer. If
one does the same with two reals, the result is a real.
Things get more interesting when
you add things having different types. Adding an integer and a real (in
any order) always yields a real.

\begin{verbatim}
    (+ 2 3.3)
    -> 5.3
    
    (+ 3.3 2)
    -> 5.3
\end{verbatim}
    
Adding an string to an integer
(with an augend integer) yields an error;
the types are not `close' enough, like they are with
integers and reals:

\begin{verbatim}
    (+ 2 "hello")
    -> EXCEPTION: generalException
       wrong types for '+': INTEGER and STRING
\end{verbatim}

In general, when adding two things,
the types must match or nearly match.
    
Of special note is the division operator with respect to integer
operands. Consider evaluating the following expression:

\begin{verbatim}
    15 / 2
\end{verbatim}

If one asked the Scam interpreter to perform this task, the result
will not be 7.5, as expected, but rather 7, as the division operator
performs {\it integer division}:

\begin{verbatim}
    (/ 15 2)
    -> 7
\end{verbatim}

However, we wish for a real result, we can convert one of
the operands to a real, as in:

\begin{verbatim}
    (/ (real 15) 2)
    -> 7.500000
\end{verbatim}

The complement to integer division is the modulus operator \%. While the
result of integer division is the quotient, the result of the modulus
operator is the remainder. Thus

\begin{verbatim}
    (% 14 5)
    -> 4
\end{verbatim}

evaluates to 4 since 4 is left over when 5 is divided into 14. To check
if this is true, one can ask the interpreter to evaluate:

\begin{verbatim}
    (== (+ (* (/ 14  5) 5) (% 14 5)) 14)
    > #t
\end{verbatim}

This complicated expression asks the question `is it true that the
quotient times the divisor plus the remainder is equal to the original
dividend?'. The Scam interpreter will respond that, indeed, it is
true. 

\section{Comparing things}

Remember the {\sc Boolean} literals, {\tt #t} and {\tt #t}?
We can use the {\sc Boolean}
comparison operators to generate such values. For example, we can ask
if 3 is less than 4:

\begin{verbatim}
    (< 3 4)
    -> #t
\end{verbatim}

Evaluating this expression shows that, indeed, 3 is less than 4. If it were
not, the result would be {\tt #f}.
Besides
{\tt <}
(less than),
there are other {\sc Boolean} comparison operators:
{\tt <=}
(less than or equal to),
{\tt >}
(greater than),
{\tt >=}
(greater than or equal to),
{\tt ==}
(equal to), and
{\tt !=}
(not equal to).

Besides integers, we can compare reals with reals, strings with strings,
and lists with lists
using the comparison operators:

\begin{verbatim}
    (< "apple" < "banana"
    True
    
    ([1, 2, 3] < [1, 2, 4]
    True
\end{verbatim}
    
In general, it is illegal
to compare integers or reals with strings.

Any Scam type can be compared with any other type with the
{\tt ==}
and
{\tt !=}
comparison operators.
If an integer is compared with a real with these
operators, the integer is converted into a real before the comparison
is made. In other cases, comparing different types with
{\tt ==}
will yield
a value of {\tt False}. Conversely, comparing different types with
{\tt !=}
will yield
{\tt True}
(the exception, as above, being integers compared with reals).
If the types match,
{\tt ==}
will yield true only if the values
match as well. The operator
{\tt !=}
behaves accordingly.

\section{Combining comparisons}

We can combine comparisons with the {\sc Boolean} logical connectives
{\tt and} and {\tt or}:

\begin{verbatim}
    (3 < 4 and 4 < 5
    True
    
    (3 < 4 or 4 < 5
    True
    
    (3 < 4 and 5 < 4
    False
    
    (3 < 4 or 5 < 4
    True
\end{verbatim}

The first interaction asks if both the expression
{\tt 3 < 4} and the expression
{\tt 4 < 5} are true. Since both are, the
interpreter responds with {\tt True}. The second interaction
asks if at least one of the expressions is true. Again, the
interpreter responds with {\tt True}. The difference between {\tt and}
and {\tt or} is illustrated in the last two interactions. Since
only one expression is true (the latter expression being
false) only the {\tt or} operator yields a true value.

There is one more {\sc Boolean} logic operation, called
{\it not}. It simply reverses the value of the expression
to which it is attached. The {\it not} operator can only be called
as a function (since it is not a binary operator). Since
you do not yet know about functions, I'll show you what
it looks like but won't yet explain its actions.

\begin{verbatim}
    (not(3 < 4 and 4 < 5)
    False
    
    (not(3 < 4 or 4 < 5)
    False
    
    (not(3 < 4 and 5 < 4)
    True
    
    (not(3 < 4 or 5 < 4)
    False
\end{verbatim}

Note that we attached {\it not} to each of the previous expressions involving
the logical connectives. Note also that the response of the interpreter
is reversed from before in each case.

\chapter{Lists, Strings, and Arrays}
\label{ListsStringsArrays}

Recall that the built-in function {\it list} is used to construct a list.
A similar function, called {\it array}, is used to construct an array populated
with the given elements:

\begin{verbatim}
    (array 1 "two" 'three)
    -> [1,"two",three]
\end{verbatim}

while a function named {\it allocate} is used to allocate an array
of the given size:

\begin{verbatim}
    (allocate 5)
    -> [nil,nil,nil,nil,nil]
\end{verbatim}

Note that the elements of an allocated array are initialized to {\it nil}.

The functions for manipulating lists, arrays, and strings are quite
similar. In the following sections, we will use the term {\it collection}
to stand for lists, arrays, and strings.

\section{Comparing collections}

The {\it equal?} function is used to compare two collections of the same type:

\begin{verbatim}
    (equal? (list 1 (array "23" 'hello)) (list 1 (array "23" 'hello)))
    -> #t

    (eq? (list 1 (array "23" 'hello)) (list 1 (array "23" 'hello)))
    -> #f
\end{verbatim}

Note that the {\it eq?} function tests for pointer equality and thus fails
when comparing two separate lists, even though they look similar.

Strings can be compared with the {\it string-compare} function.
Assuming a lexigraphical ordering based upon the ASCII code,
then a string compare of two strings, {\it a} and {\it b},
returns a negative number if {\it a}
appears lexigraphically before {\it b}, returns zero if a and
b are lexigraphically equal,
and returns a positive number otherwise.

\begin{verbatim}
    (string-compare "abc" "bbc")
    -> -1
\end{verbatim}

\section{Extracting elements}

You can pull out an item from a collection
by using the
{\it getElement} function.
With {\it getElement},
you specify exactly which element
you wish to extract. This specification is called an
{\it index}. The first element of a collection has index 0, the second
index 1, and so on. This concept of having the first element having
an index of zero is known as {\it zero-based counting}, a common concept
in Computer Science. Here is some code that extracts the first element
of a list:

\begin{verbatim}
    (getElement (list "a" #t 7) 0)
    -> "a"

    (getElement (array "b" #f 11) 1)
    -> #f

    (getElement "howdy" 2)
    -> "w"
\end{verbatim}

What happens if the index is too large?

\begin{verbatim}
    (getElement (list "a" #t 7) 3)
    EXCEPTION: generalException
    index (3) is too large
\end{verbatim}

Not surprisingly, an error is generated.
In Scam, as with many programming languages, an error is known
as an {\it exception}.

As with Scheme, the built-in {\it car} and {\it cdr} functions
returns the first item and the tail of a collection, respectively:

\begin{verbatim}
    (car (list 3 5 7))
    -> 3

    (cdr "howdy")
    -> "owdy"

    (cdr (array 2 4 6))
    -> [4,6]

    (car "bon jour")
    -> "b"
\end{verbatim}

\section{Setting elements of a collection}

The {\it set-car!} function can be used to set the first element of
a collection:

\begin{verbatim}
    (set-car! (list 3 5 7) 0 11)
    -> (11 5 7)
\end{verbatim}

More generally, the {\it setElement} function can be used to set a new value
at any legal index:

\begin{verbatim}
    (setElement collection index newValue)
\end{verbatim}

For strings, the new value must be a non-empty string. If the
value is composed of multiple characters, the characters after
the first character are ignored.

\chapter{Variables}
\label{Variables}

One defines variables with the {\it define} function:

\begin{verbatim}
    (define x 13)
\end{verbatim}

The above expression creates a variable named {\it x} in the current
scope and initializes it to the value 13.
If the initializer is missing:

\begin{verbatim}
    (define y)

    (inspect y)
    -> y is nil
\end{verbatim}

the variable is initialized to nil.

\section{Functions}

A function definition is another way to create a variable.
There are two ways to define a function. The first is through a 
regular variable definition, where the initializer is a lambda
expression:

\begin{verbatim}
    (define square (lambda (x) (* x x)))

    (square 3)
    -> 9

    (inspect square)
    -> square is <function square(x)>
\end{verbatim}

The above expression defines a function that returns the square of
its argument and emphasizes that the name of a function is
simply a variable. The second method uses a special syntax:

\begin{verbatim}
    (define (square x) (* x x))
    
    (inspect square)
    -> square is <function square(x)>
\end{verbatim}

In either case, a variable is created and bound to some entity
that knows how to compute values.

\section{Scopes, Environments and Objects}

When one defines a variable, the variable name and value are
inserted into the current scope. In actuality,
the name-value pair is
stored in a table called an {\it environment}. For the predefined
variable {\it this}, its value is the current scope or environment.
For example, consider the following interaction:

\begin{verbatim}
    (define n 10)

    (pp this)
    -> <object 8393>
                 __label  : environment
               __context  : <object 4495>
                 __level  : 0
           __constructor  : nil
                    this  : <object 8393>
                       n  : 10
\end{verbatim}

The {\it pp} function will print out the list of variables
and their values for the given environment.
Among other information stored in the current environment,
we see an entry for {\it n} and its value is indeed 10.

The Scam object system is based upon environments. We will
learn about objects in a later chapter.

\section{Defining Variables Programatically}

The function {\it addSymbol} is used to define variables on the 
fly. For example, to define a variable named {\it x} in the current
scope and to initialized it to 13, one might use the
following expression:

\begin{verbatim}
    (addSymbol 'x 13 this)
\end{verbatim}

You can also define functions this way:

\begin{verbatim}
    (addSymbol 'square (lambda (x) (* x x)) this)
\end{verbatim}

Since {\it addSymbol} evaluates all its arguments, the first
argument can be any expression that resolves to a symbol,
the second argument can be any expression that resolves
to an appropriate value, and the third argument can
be any expression that resolves to an environment or
object.


\section{Variable naming}

Unlike many languages,
Scam is quite liberal in regards to legal variable names.
A variable can't begin with any of the these characters:
\verb!0123456789;,`'"()! nor whitespace and cannot contain any of these
characters: \verb!;,`'"()! nor whitespace. Typically,
variable names start with a letter or underscore, but
they do not have to. This flexibility allows Scam programmers
to easily define new functions that have appropriate names.
Here is a function that increments the value of its argument:

\begin{verbatim}
    (define (+1 n) (+ n 1))
\end{verbatim}

While Scam lets you name variables in wild ways:

\begin{verbatim}
    (define $#1_2!3iiiiii@ 7)
\end{verbatim}

you should temper your
creativity if it gets out of hand.
While the name \verb-$#1_2!3iiiiii@-
is a perfectly good variable name from Scam's point of
view,
it is a particularly poor name from the point of making your Scam
programs readable by you and others.
It is important that your variable
names reflect their purpose.



\chapter{Assignment}
\label{Assignment}

Once a variable has been created, it is possible to change its value,
or {\it binding},
We have already seen a form of assignment in
Section {\link ListsArraysStringsSetting};
Assignment to variables proceeds in a similar fasion.
Consider the following interaction with
the Scam interpreter:

\begin{verbatim}
    (define eyeColor 'black)    ; creation
    
    (inspect eyeColor)          ; reference
    -> (eyeColor is black)
    
    (assign eyeColor 'green)    ; assignment
    
    (eq? eyeColor 'black)       ; equality
    -> #f
    
    (== eyeColor 'brown)        ; equality (alternate)
    -> #f

    (eq? eyeColor 'green)       ; equality
    -> #t

    (set! eyecolor BROWN)       ; assignment (alternate)
\end{verbatim}

The assignment function is not like the arithmetic operators.
Recall that {\tt +} evaluates all its arguments
before performing the addition.
For {\it assign},
the leftmost operand is not evaluated:
If it were, the assignment

\begin{verbatim}
    (define x 1)
    (assign x 3)
\end{verbatim}
    
would be equivalent to:

\begin{verbatim}
    (assign 1 3)
\end{verbatim}

In general, an operator which does not evaluate
all its arguments is known as a {\it special form}\footnote{
In Scam, there are no special forms. As such, {\it assign} is
a true function and can be given a new meaning.}.
For {\it assign}, the evaluation of the first argument
is suppressed.

\section{Other functions for changing the value of a variable}

Scam has two more functions to change the value of a
varible. The first is the Scheme way:

\begin{verbatim}
    (set! x 5)
\end{verbatim}

which changes the current value of {\it x} to 5.
It is equivalent to the {\it assign} function.
Sometimes, however, it is useful to derive the variable name
to be modified programmatically.
A function for doing so is named {\it set}. Unlike {\it assign} and {\it set!},
whose first argument is not evaluated, all arguments to
{\it set} are evaluated. Thus, the following call to {\it set} is
equivalent to the previous call to {\it set!}:

\begin{verbatim}
    (set 'x 5)
\end{verbatim}

\section{Assignment and Collections}

In the chapter on \link{lists and other collections}{ListsStringsArraysSetting},
we saw how to change an element in a collection. Recall the
the functions for doing so:

\begin{verbatim}
    (setElement collection index newValue)
    (set-car! collection newValue)
\end{verbatim}

Also, it is possible to reset the tail of a list:

\begin{verbatim}
    (set-cdr! items newTail)
\end{verbatim}

\section{Assignment and Environments/Objects}

The assignment functions can take an environment as an
optional third argument.
Because the predefined variable
{\it this} always points to the current environment,
the following
four expressions are equivalent:

\begin{verbatim}
        (assign x 5)
        (assign x 5 this)
        (set! x 5 this)
        (set (symbol "x") 5 this)
\end{verbatim}

The {\it symbol} function is used to create a variable name from
a string.
Since environments form the basis for objects in Scam,
{\it assign}, {\it set!}, and {\it set} 
can be used to update the instance variables
of objects.

\chapter{Conditionals}
\label{Conditionals}

Conditionals implement decision points in a computer program.
Suppose you have a program that performs some task on an
image. You may well have a point in the program where you
do one thing if the image is a JPEG and quite another
thing if the image is a GIF file. Likely, at this point,
your program will include a conditional expression to make
this decision.

Before learning about conditionals, it is important to
learn about logical expressions. Such expressions are the
core of conditionals and loops.\footnote{
We will learn about loops in the next chapter.
}

\section{Logical expressions}

A logical expression evaluates to a truth value, in essence true or
false. For example, the expression \verb!(> x 0)!
resolves to true if {\it x} is positive
and false if {\it x} is negative or zero. In Scam, truth is represented by
the symbol \verb!#t! and falsehood
by the symbol \verb!#f!.
Together,
these two symbols are known as {\it {\sc Boolean}} values.

One can assign truth values to variables:

\begin{verbatim}
      (define c -1)
      (define z (> c 0))

      (inspect z)
      -> z is #f
\end{verbatim}

Here, the variable {\it z} would be assigned true
if {\it c}  was positive;
since {\it c} is negative, however, it is assigned false.

\section{Logical operators}

Scam has the following logical operators.
 
\begin{tabular}{cl}
    {\tt =}     & numeric equal to \\
    {\tt !=}    & not equal to \\
    {\tt >}     & greater than \\
    {\tt >=}    & greater than or equal to \\
    {\tt <}     & less than \\
    {\tt <=}    & less than or equal to \\
    {\tt ==}    & pointer equality \\
    {\tt neq?}  & pointer inequality \\
    {\tt eq?}   & pointer equality \\
    {\tt equal?}& structural equality \\
    {\tt and}   & and \\
    {\tt or}    & or \\
    {\tt not}   & not \\
\end{tabular}

The first ten operators are used for comparing two  (or more) things,
while the last three operators are the glue that joins up simpler
logical expressions into more complex ones.

\section{Short circuiting}

When evaluating a logical expression,
Scam evaluates the expression from left to right and
stops evaluating as soon as it finds out that the expression
is definitely true or definitely false.
For example, when encountering the expression:

\begin{verbatim}
      (and (!= x 0) (> (/ y x) 2))
\end{verbatim}

if {\it x} has a value of 0, the subexpression on the left side of the
{\it and}
connective resolves to false. At this point, there is no way for the
entire expression to be true (since both the left hand side and the right
hand side must be true for an
{\it and}
expression to be true), so the right
hand side of the expression is not evaluated. Note that this expression
protects against a divide-by-zero error.

\section{If expressions}

Scam's {\it if} expressions are used to conditionally execute code,
depending on the truth value of what is known as the
{\it test} expression. One version of {\it if} has a block of
code following the test expression:

Here is an example:

\begin{verbatim}
    (if (== name "John")
        (println "What a great name you have!")
        )
\end{verbatim}

In this version, when the test expression is true ({\it i.e.}, 
the string {\tt "John"} is bound to the variable {\it name}), 
then the following expression is evaluated 
(i.e., the compliment
is printed). The indented code is known as a {\it block}.
If the test expression is false, however the
expression following the test expression is not evaluated.

Here is another form of {\it if}:

\begin{verbatim}
    (if (== major "Computer Science")
        (println "Smart choice!")
        (println "Ever think about changing your major?")
        )
\end{verbatim}

In this version, {\it if} has two expressions following
the test.
As before, the first expression is evaluated if the test expression
is true. If the test expression is false, however,
the second expression is evaluated instead.

\section{if-elif-else chains and the {\it cond} function}

You can chain {\tt if} statements together, as in:

\begin{verbatim}
    (if (== bases 4)
        (print "HOME RUN!!!")
        (if (== bases 3)
            (print "Triple!!")
            (if (== bases 2)
                (print "double!")
                (if (== bases 1) 
                    (print "single")
                    (print "out")
                    )
                )
            )
        )
\end{verbatim}

The expression that is eventually evaluated is
directly underneath the first test expression
that is true, reading from top to bottom.
If no test expression is true, the second expression associated
with the most nested if is evaluated.

The {\it cond} function takes care  of the awkward indentation
of the above construct:

\begin{verbatim}
    (cond
        ((== bases 4) (print "HOME RUN!!!"))
        ((== bases 3) (print "Triple!!"))
        ((== bases 2) (print "double!"))
        ((== bases 1) (print "single"))
        (else (print "out"))
        )
\end{verbatim}

The general form of the cond function call is:

\begin{verbatim}
    (cond (expr1 action1) (expr2 action2) ... (else actionN))
\end{verbatim}

where expr1, expr2, and so on are Boolean expressions. In addition
to its compactness, another advantage of a cond is each action
portion of a clause is really an implied block. For example,
suppose we wish to debug an if statement and print out a
message if the test resolves to true. We are required to insert
a begin block, so:

\begin{verbatim}
    (if (alpha a b c)
        (beta y)
        (gamma z)
        )
\end{verbatim}

becomes:

\begin{verbatim}
    (if (alpha a b c)
        (begin (println "it's true!") (beta y))
        (gamma z)
        )
\end{verbatim}

On the other hand:

\begin{verbatim}
    (cond
        ((alpha a b c)
            (beta y))
        (else
            (gamma z))
        )
\end{verbatim}

becomes:

\begin{verbatim}
    (cond
        ((alpha a b c)
            (println "it's true!")
            (beta y))
        (else
            (gamma z))
        )
\end{verbatim}

Note the lack of a {\it begin} block for {\it cond}.

\chapter{Functions}
\label{Functions}

Recall from 
\link*{the chapter on assignment}[Chapter~\Ref]{Assignment}
the series of expressions we evaluated to find
the {\it y}-value of a point on the line:

\begin{verbatim}
    y = 5x - 3
\end{verbatim}
    
First, we assigned values to the slope,
the {\it x}-value, and the {\it y}-intercept:

\begin{verbatim}
    >>> m = 5
    >>> x = 9  
    >>> b = -3
\end{verbatim}

Once those variables have been assigned,
we can compute the value of {\it y}:

\begin{verbatim}
    >>> y = m * x + b

    >>> y
    42
\end{verbatim}

Now, suppose we wished to find the {\it y}-value corresponding to
a different {\it x}-value or, worse yet, for a different {\it x}-value
on a different line. All the work we did would have to be
repeated. A {\it function} is a way to encapsulate all these operations
so we can repeat them with a minimum of effort.

\section{Encapsulating a series of operations}

First, we will define a not-too-useful function that
calculates {\it y} give a slope of 5,
a {\it y}-intercept of -3, and an
{\it x}-value of 9 (exactly
as above). We do this by wrapping a function around
the sequence of operations above.
The return value of this function is the computed {\it y} value:

\begin{verbatim}
def y():
    m = 5
    x = 9
    b = -3
    return m * x + b
\end{verbatim}

There are a few things to note. The keyword {\tt def} indicates
that a function definition is occurring. The name of this
particular function is {\it y}. The names of the things
being sent to the function are given between the parentheses;
since there is nothing between the parentheses, we don't
need to send any information to this function when we call it.
Together, the first line is known as the {\it function signature},
which tells you the name of the function and how many values
it expects to be sent when called.

The stuff indented from the first line of the function definition
is called the {\it function body} and
is the code that will be evaluated (or executed) when the
function is called. You must remember this: {\it the function body
is not evaluated until the function is called}.

Once the function is defined, we can find the value of {\it y} repeatedly.
Let's assume the function was entered into the file named
{\it line.py}.

First we import the code in {\it line.py} with the from statement:

\begin{verbatim}
    >>> from line import *   # not line.py!
\end{verbatim}

This makes the python interpreter behave as if we had typed
in the function definition residing in {\it line.py} directly into
the interpreter. Note, we omit the {\it .py} extension in the import
statement.

After importing the {\it y} function, the next thing we do
is call it:

\begin{verbatim}
    >>> y()
    42
    >>> y()
    42
\end{verbatim}
    
The parentheses after the {\it y} indicate that we wish to call
the {\it y} function and get its value. Because we designed the
function to take no values when called, we do not place any
values between the parentheses.

Note that when we call the {\it y} function again,
we get the exact same answer.

The {\it y} function, as written,
is not too useful in that we cannot use it to compute
similar things, such as the {\it y}-value for a different value of
{\it x}.
This is because we `hard-wired' the values of {\it b}, {\it x}, and {\it m},
We can improve this function by passing in the value of {\it x}
instead of hard-wiring the value to 9.

\section{Passing arguments}

A hallmark of a good function is that it lets you compute
more than one thing. We can modify our {\it y} function to {\it take in} the
value of {\it x} in which we are interested.
In this way,
we can compute more than one value of {\it y}.
We do this by {\it passing} in 
an {\it argument}\footnote{
The information that is passed into a function is collectively
known as {\it arguments}.}, in this case, the value of {\it x}.

\begin{verbatim}
def y(x):
    slope = 5
    intercept = -3
    return slope * x + intercept
\end{verbatim}

Note that we have moved {\it x} from the body of the function
to between the parentheses. We have also refrained from
giving it a value since its value is to be sent to the function
when the function is called.
What we have done is to {\it parameterize} the function to make it more
general and more useful. The variable {\it x} is now called a
{\it formal parameter} since it sits between the parentheses in
the first line of the function definition.

Now we can compute {\it y} for an infinite number of {\it x}'s:

\begin{verbatim}
    >>> from line.py import *
    >>> y(9)
    42
    
    >>> y(0)
    -3
    
    >>> y(-2)
    -13
\end{verbatim}

What if we wish to
compute a {\it y}-value for a given {\it x} for a different
line? One approach would be to pass in the {\it slope} and {\it intercept}
as well as {\it x}:

\begin{verbatim}
def y(x,m,b):
    return m * x + b
\end{verbatim}

Now we need to pass even more information to {\it y} when we call it:
    
\begin{verbatim}
    >>> from line.py import *
    >>> y(9,5,-3)
    42
     
    >>> y(0,5,-3)
    -3
\end{verbatim}

If we wish to calculate using a different line, we just pass in the
new {\it slope} and {\it intercept} along with our value of {\it x}.
This certainly works as intended, but is not the best way. One problem
is that we keep on having to type in the slope and intercept even if
we are computing {\it y}-values on the same line. Anytime you
find yourself doing the same tedious thing over and over,
be assured that
someone has thought of a way to avoid that particular tedium.
If so, how do we
customize our function so that we only have to enter the slope
and intercept once per particular line? We will explore
one way for doing this. In reading further,
it is not important if you understand all that is going on.
What is important is that you know you can use functions
to run similar code over and over again.

\section{Creating functions on the fly}

Since creating functions is hard work (lots of typing) and
Computer Scientists avoid unnecessary work like the plague, somebody
early on got the idea of writing a function that itself 
creates functions! Brilliant! We can do this for our line problem.
We will tell our creative function to create a {\it y} function
for a particular slope and intercept! While we are at it,
let's change the variable names {\it m} and {\it b} to {\it slope}
and {\it intercept}, respectively:

\begin{verbatim}
def makeLine(slope,intercept):
    def y(x):
        return slope * x + intercept
    return y    # the value of y is returned, y is NOT CALLED!
\end{verbatim}

The {\it makeLine} function creates a {\it y} function
and then returns it. Note that this returned function {\it y} takes
one value when called, the value of {\it x}.

So our creative {\it makeLine} function
simply defines a {\it y} function and then
returns it. Now we can create a bunch of different lines:

\begin{verbatim}
    >>> from line.py import *
    >>> a = makeLine(5,-3)
    >>> b = makeLine(6,2)

    >>> a(9)
    42
    
    >>> b(9)
    56

    >>> a(9)
    42
\end{verbatim}

Notice how lines {\it a} and {\it b} remember
the slope and intercept supplied
when they were created.\footnote{
The local function {\it y} does not really remember these values,
but for an introductory course, this is a good enough explanation.}.
While this is decidedly cool, the problem is many languages (C, C++, and Java
included\footnote{
C++ and Java, as well as Scam, give you another approach, {\it objects}.
We will not go into objects in this course, but
you will learn all about them in your next programming course.})
do not allow you to define functions that create other functions.
Fortunately, Scam does allow this.

While this might seem a little mind-boggling, don't worry. The
things you should take away from this are:

\begin{itemize}
\item
    functions encapsulate calculations
\item
    functions can be parameterized
\item
    functions can be called
\item
    functions return values
\end{itemize}

\section{The Function and Procedure Patterns}

When a function calculates (or obtains) a value and returns it, we say
that it implements the {\it function} pattern. If a function
does not have a return value, we say it implements the
{\it procedure} pattern.

Here is an example of the {\it function} pattern:

\begin{verbatim}
def square(x):
    return x * x
\end{verbatim}

This function takes a value, stores it in {\it x}, computes the square
of {\it x} and return the result of the computation.

Here is an example of the {\it procedure} pattern:

\begin{verbatim}
def greeting(name):
    print("hello,",name)
\end{verbatim}

Almost always,
a function that implements the {\it function} pattern does not print
anything, while a function that implements the procedure
pattern often does\footnote{Many times, the printing is
done to a file, rather than  the console.}.
A common function that implements the procedure pattern
is the {\it main} function.
A common mistake made by beginning programmers is
to print a calculated value rather than to return it. So, when defining
a function, you should ask yourself, should I implement the function
pattern or the procedure pattern?

Most of the function you implement in this class follow the function
pattern.

Another common mistake is to inadvertently implement a {\it procedure} pattern
when a {\it function} pattern is called for. This happens when the {\it return}
keyword is omitted.

\begin{verbatim}
def psquare(x):
    x * x
\end{verbatim}

While the above code looks like the function pattern, it is actually
a procedure pattern. What happens is the value $x * x$ is calculated,
but since it is not returned, the newly calculated value is thrown
away (remember the {\it throw-away} pattern?).

Calling this kind of function yields a surprising result:

\begin{verbatim}
    >>> psquare(4)
    >>>

    >>>print(psquare(6))
    None
\end{verbatim}

When you do not specify a return value, but you use
the return value anyway (as in the printing example),
the return value is set to {\it None}.

Usually, the procedure pattern causes some side-effect to happen
(like printing). A procedure like {\it psquare}, which has no side-effect
is a useless function.

\chapter{Input and Output}
\label{InputAndOutput}

Scam uses a {\it port} system for input and output.
When Scam starts up, the current input port defaults to
{\it stdin} (the keyboard) and the current output
port defaults to {\it stdout} (the screen).

To change these ports, one first creates new port
and then sets the port.
For example, to read from a file (say "data")
instead of the keyboard,
first create a file port:

\begin{verbatim}
    (define p (open "data" 'read))   ; p points to a port
    (define oldInput (setPort p))
    ...                              ; read stuff from the file data
    (setPort oldInput)               ; restore the old input port
\end{verbatim}

Once the port is set, all input will come from the new port.
The {\it setPort} function, in addition to setting the port, returns
the old port so that it eventually can be restored.

To change the output port, the procedure is similar, except
the symbol \verb!'write! is used instead.

\begin{verbatim}
    (define p (open "data" 'write))   ; p points to a port
    (define oldOutput (setPort p))
    ...                               ; write stuff to the file data
    (setPort oldOutput)               ; restore the old output port
\end{verbatim}

Opening a file in \verb!'write! mode overwrites the file;
to append content to an existing file, use the \verb!'append!
symbol instead.

Scam only allows a limited number of ports to be open at
any given time. If you no longer need a port, close it with
the built-in function {\it close}, which takes a port as its
sole argument:

\begin{verbatim}
    (close p)
\end{verbatim}

\section{Reading}

Scam supplies built-in functions for reading characters,
integers, reals,
strings, and whitespace delimited tokens:

\begin{verbatim}
    (assign s (readChar))
    (assign i (readInt))
    (assign r (readReal))
    (assign s (readString))
    (assign t (readToken))
    (assign s (readRawChar))
    (assign u (readUntil stopCharacterString))
    (assign w (readWhile continueCharacterString))
\end{verbatim}

The first five functions listed skip any whitespace preceeding the
entity they are to read. The last three functions do not skip whitespace.

Both the {\it readChar} and the {\it readToken} functions return strings.
Scam uses the same rules as the C programming language
for what characters constitute an integer and a real.
None of these functions take an argument; they use the current
input port.

To read a symbol, use the {\it symbol} function in conjunction
with the {\it readToken} function:

\begin{verbatim}
    s = symbol(readToken());
\end{verbatim}

To read a line of text, use the built-in {\it readLine} function:

\begin{verbatim}
    (assign l (readLine))
\end{verbatim}

The {\it readLine} function reads up to, and including, the next
newline character, but the newline is not part of the
returned string.

The {\it pause} function always reads from {\it stdin},
regardless of the current input port.
It reads (and discards) a line of text (up to and including the newline).
Its purpose is to pause execution of a program for debugging
purposes.

Three other reading functions are useful for scanning text.
The first is {\it readRawChar}, which returns a string containing
the next character in the file, regardless of whether that
character is whitespace or not.
The second is {\it readUntil}, which is passed a string of characters
that is used to control the read. For example,

\begin{verbatim}
    (readUntil " \t\n")
\end{verbatim}

will start reading at the current point in the file
and return a string of all characters read up to point
where a character in a given string is encountered.
The character that caused the read to stop is pushed
back into the input stream and will be the next character
read.

The {\it readWhile} function is analogous, stopping when
a character not in the given string is encountered.

\section{Writing}

Most output functions write to the current output port.

The simplest output function is {\it display}. It takes a single
argument, which can be any Scam object:

\begin{verbatim}
    (display "Hello, world!\n")
\end{verbatim}

The character sequence \verb!\! followed by
\verb!n! indicate that
a newline is to be displayed.

More useful than display are the functions {\it print} and
{\it println} in that they take any number of arguments:

\begin{verbatim}
    (print "(f x) is " (f x) "\n")
    (println "(f x) is " (f x))
\end{verbatim}

The {\it println} function is just like {\it print}, except it
outputs a newline after the displaying the last argument.
Thus, the two calls above produce the same output.

When a string is printed, the quote marks are not displayed.
Likewise, when a symbol is printed, the quote mark is not displayed.

The {\it inspect} function 
prints out the unevaluated
version of its argument followed by the arguments evaluation value:

\begin{verbatim}
    (inspect (f x))
    -> (f x) is 3
\end{verbatim}

The {\it inspect} function always prints to {\it stdout},
regardless of the current output port.

\section{Pretty printing}

The function {\it pp} acts much like {\it display}, unless it is
passed an environment/object. In such cases, it prints out a table
listing the variables defined in that scope.
Since functions, thunks, exceptions, and errors are all encoded
as objects, pp can be used to inspect them in greater detail.
For example, consider this definition of square:

\begin{verbatim}
    (define (square x)
        (* x x)
        )
\end{verbatim}

Printing the value of square using \verb!(print square)! yields:

\begin{verbatim}
    <function square(x)>
\end{verbatim}

In contrast, using \verb!(pp square)! yields:

\begin{verbatim}
    <object 8573>
               label  : closure
             context  : <object 8424>
                name  : square
          parameters  : (x)
                code  : (begin (* x x))
\end{verbatim}

\section{Formatting}

The {\it fmt} function can be used to format numbers and strings
if the default formatting is not acceptable. It uses the C
programming language formatting scheme, taking a formatting
specification as a string, and the item to be formatted.
The function returns a string.

For example,

\begin{verbatim}
    sway>"<" + fmt("%6d",3) + ">";
    STRING: "<     3>"

    sway>"<" + fmt("%-6d",3) + ">";
    STRING: "<3     >"
\end{verbatim}

A format specification begins with a percent sign and is usually followed
by a number representing the width (in number of characters)
of the resulting string. If the width is positive, the
item is right justified in the resulting string. If the width
is negative, the item is left justified.
After any width specification is a character specifying the
type of the item to be formatted:
\verb!d! for an integer,
\verb!f! for a real number, and
\verb!s! for a string.

The format specification is quite a bit more sophisticated
than shown here. You can read more on a Linux system by
typing the command \verb!man 3 printf! at the system prompt.

\section{Testing for end of file}

The {\it eof?} function can be used to test whether the last
read was successful or not. The function is NOT used to
test if the {\it next} read would be successful. Here is a typical
use of {\it eof?} in tokenizing a file:

\begin{verbatim}
    (define t (readToken))
    (while (not(eof?))
        (store t)
        (assign t (readToken))
        )
\end{verbatim}

\section{Pushing back a character}

Sometimes, it is necessary to read one character too
many from the input. This happens in cases like
advancing past whitespace in the input.
Here is a typical whitespace-clearing loop:

\begin{verbatim}
    (define ch (readRawChar))
    (while (space? ch))
        (assign ch (readRawChar))
        )

    ; last character read wasn't whitespace
    ; so push it back to be read again later

    (pushBack ch)
\end{verbatim}

The {\it pushBack} function takes a string as its
sole argument, but only pushes back the first
character of the string; subsequent characters in
the string are ignored.

\chapter{Scope}
\label{Scope}

A {\it scope} holds the current set of variables and their values.
In Scam, there is something called the {\it global scope}.
The global scope holds all the values of the built-in variables
and functions (remember, a function name is just a variable).

When you enter the Scam interpreter, either by running it
interactively or by using it to evaluate a program in
a file, you start out in the global scope.
As you define variables, they and their values are added
to the global scope.

This interaction adds the variable {\it x} to the global scope:

\begin{verbatim}
    >>> x = 3
\end{verbatim}

This interaction adds two more variables to the global scope:

\begin{verbatim}
    >>> y = 4
    >>> def negate(z):
    ...     return -z;
    ...
    >>>
\end{verbatim}

What are the two variables?
\T The two variables added to the global scope are {\it y} and {\it negate}.
\W Highlight the following line to see the answer:

\W\begin{quote}
\W {\it The two variables added to the global scope are} {\color{white} y and negate.}
\W\end{quote}

Indeed, since the name of a function is a variable and the variable {\it negate}
is being bound to a function object, it becomes clear that
this binding is occurring in the
global scope, just like {\it y} being bound to 4.

Scopes in Scam can be identified by their indentation level.
The global scope holds all variables defined with an indentation
level of zero.
Recall that when functions are defined, the body of the function
is indented. This implies that variables defined in the function body
belong to a different scope and this is indeed the case.
Thus we can identify to which scope a variable belongs by
looking at the pattern of indentations.
In particular,
we can label variables as 
either {\it local} or {\it non-local}
with respect to a particular scope.
Moreover, non-local variables may be {\it in scope} or
or {\it out of scope}.

\section{In Scope or Out}

The indentation pattern of a program can tells us where
variables are visible (in scope) and where they are
not (out of scope).
We begin by first learning to recognizing the scopes in which variables
are defined.

\subsection{The Local Variable Pattern}

All variables {\it defined} at a particular
indentation level or scope are considered
{\it local} to that indentation level or scope.
In Scam, if one assigns a value to a variable, that variable
must be local to that scope.
The only exception is if the variable was
explicitly declared {\it global} (more on that later).
Moreover, the formal parameters of a function definition
belong to the scope that is identified with
the function body. 
So within a function body, the local variables are the formal
parameters plus any variables defined in the function body.

Let's look at an example.
Note,
you do not need to completely understand the examples presented in
the rest of the chapter in order 
to identify the local and non-local variables.

\begin{verbatim}
def f(a,b):
    c = a + b
    c = g(c) + X
    d = c * c + a
    return d * b
\end{verbatim}

In this example, we can immediately say the 
formal parameters,
{\it a} and {\it b},
are local with respect to the scope of the body of function {\it f}.
Furthermore, variables
{\it c} and {\it d}
are defined in the function body
so they are local as well,
with respect to the scope of the body of function {\it f}.
It is rather wordy to say ``local with respect to the
scope of the body of the function {\it f}'', so Computer Scientists
will almost always shorten this to ``local with respect to {\it f}''
or just ``local''
if it is clear the discussion is about a particular function or scope.
We will use this shortened phrasing from here on out.
Thus {\it a}, {\it b}, {\it c}, and {\it d} are local with respect to {\it f}.
The variable {\it f} is local to the global scope since the function
{\it f} is defined in the global scope.

\subsection{The Non-local Variable Pattern}

In the previous section, we determined the local
variables of the function.
By the process 
of elimination, that means the variables
{\it g}, and {\it X} are non-local.
The name of function itself is non-local with respect
to its body, {\it f} is non-local as well.

Another way of making this determination is
that
neither {\tt g} nor {\tt X} are assigned values
in the function body. Therefore, they must be non-local.
In addition, should a variable be explicitly declared {\it global},
it is non-local even if it is assigned a value.
Here again is an example:

\begin{verbatim}
def h(a,b):
    global c
    c = a + b
    c = g(c) + X
    d = c * c + a
    return d * b
\end{verbatim}

In this example, variables {\it a}, {\it b}, and {\it d} are local with respect
to {\it h} while {\it c}, {\it g}, and {\it X} are non-local.
Even though {\it c} is assigned a value, the declaration:

\begin{verbatim}
    global c
\end{verbatim}

means that {\it c} belongs to a different scope (the global scope) and thus
is non-local.

\subsection{The Accessible Variable Pattern}

A variable is accessible with respect to
a particular scope if it is {\it in scope}.
A variable is in scope if it is local or
was defined in a scope that
{\it encloses} the particular scope.
Some scope
{\it A} encloses some other scope {\it B}
if, by moving (perhaps repeatedly) leftward from
scope B, scope A can be reached.
Here is example:

\begin{verbatim}
Z = 5

def f(x):
   return x + Z

print(f(3))
\end{verbatim}

The variable {\it Z} is local with respect to the global scope
and is non-local with respect to {\it f}. However, we can
move leftward from the scope of {\it f} one indentation level and
reach the global scope where {\it Z} is defined.
Therefore, the global scope encloses the scope of {\it f} and
thus {\it Z} is accessible from {\it f}.
Indeed, the global scope encloses all other scopes and this
is why the built-in functions are accessible at any indentation
level.

Here is another example that has two enclosing scopes:

\begin{verbatim}
X = 3
def g(a)
   def m(b)
      return a + b + X + Y
   Y = 4
   return m(a % 2)

print(g(5))
\end{verbatim}

If we look at function {\it m}, we see that there is only
one local variable, {\it b}, and that {\it m} references three
non-local variables,
{\it a}, {\it X}, and {\it Y}. 
Are these non-local variables accessible?
Moving leftward from the body of {\it m}, we reach the body of {\it g},
so the scope of {\it g} encloses the scope of {\it m}. The local variables
of {\it g} are {\it a}, {\it m}, and {\it Y}, so both {\it a} and {\it Y}
are accessible in the scope of {\it m}.
If we move leftward again, we reach the global scope,
so the global scope encloses the scope of {\it g}, which in turn encloses
the scope of {\it m}. By transitivity, the global scope encloses
the scope of {\it m}, so {\it X}, which is defined in the global scope
is accessible to the scope of {\it m}.
So, all the non-locals of {\it m} are accessible to {\it m}.

In the next section, we explore how a variable can be
inaccessible.

\subsection{The Tinted Windows Pattern}

The scope of local variables is like a car with tinted
windows, with the variables defined within riding in
the back seat.
If you are outside the scope, you cannot
peer through the car windows  and see those variables.
You might try and buy some x-ray glasses, but they
probably wouldn't work.
Here is an example:

\begin{verbatim}
    z = 3

    def f(a):
        c = a + g(a)
        return c * c

    print("the value of a is",a) #x-ray!
    f(z);
\end{verbatim}

The print statement causes an error:

\begin{verbatim}
    Traceback (most recent call last):
      File "xray.py", line 7, in <module>
          print("the value of a is",a) #x-ray!
          NameError: name 'a' is not defined
\end{verbatim}

If we also tried to print the value of {\it c},
which is a local variable of function {\it f}, at that
same point in the program, we would get a similar error.

The rule for figuring out which variables are in scope and
which are not is:
{\it you} {\bf cannot} {\it see into an enclosed scope}.
Contrast this with the non-local pattern:
{\it you} {\bf can} {\it see variables
declared in enclosing outer scopes}.

\subsection{Tinted Windows with Parallel Scopes}

The tinted windows pattern also applies to parallel scopes.
Consider this code:

\begin{verbatim}
    z = 3

    def f(a):
        return a + g(a)

    def g(x):
        # starting point 1
        print("the value of a is",a) #x-ray!
        return x + 1

    f(z);
\end{verbatim}

Note that the global scope encloses both the scope of {\it f} and
the scope of {\it g}. However, the scope of {\it f} does
not enclose the scope of {\it g}. Neither does
the scope of {\it g} enclose the scope of {\it f}.

One of these functions references a variable that is not in scope.
Can you guess which one?
\T The function {\it g} references a  variable not in scope.
\W Highlight the following line to see the answer:

\W\begin{quote}
\W {\it The function} {\color{white} g} {\it references variable not in scope.}
\W\end{quote}

Let's see why by first examining {\it f} to see whether or
not its non-local references are in scope.
The only local variable of function {\it f} is
{\it a}. The only referenced non-local is {\it g}.
Moving leftward from the body of {\it f}, we reach the
global scope where where both {\it f} and {\it g} are defined.
Therefore, {\it g}
is visible with respect to {\it f} since it is defined in a scope
(the global scope) that encloses {\it f}.

Now to investigate {\it g}. The only local variable of
{\it g} is {\it x}
and the only non-local that {\it g} references is {\it a}.
Moving outward to the global scope, we see that there is
no variable {\it a} defined there,
therefore the variable {\it a} is not in scope with
respect to {\it g}.

When we actually run the code,
we get an error similar to the following when running this program:

\begin{verbatim}
    Traceback (most recent call last):
      File "xray.py", line 11, in <module>
        f(z);
      File "xray.py", line 4, in f
        return a + g(a)
      File "xray.py", line 8, in g
        print("the value of a is",a) #x-ray!
    NameError: global name 'a' is not defined
\end{verbatim}

The lesson to be learned here is
that we cannot see into
the local scope of the body of function {\it f},
{\it even if we are at a similar nesting level}.
Nesting level doesn't matter. We can only see variables
in our own scope and those in {\it enclosing} scopes.
All other variables cannot be seen.

Therefore, if you ever see a variable-not-defined error,
you either have spelled the variable name wrong, you haven't
yet created the variable, or you are trying to use x-ray vision
to see somewhere you can't. 

\section{Alternate terminology}

Sometimes, enclosed scopes are referred to as {\it inner} scopes while
enclosing scopes are referred to as {\it outer} scopes. In addition,
both locals and any non-locals found in enclosing scopes are considered
{\it visible} or {\it in scope}, while non-locals that are not
found in an enclosing scope are considered {\it out of scope}.
We will use all these terms in the remainder of the text book.

\section{Three Scope Rules}

Here are three simple rules you can use to help you
figure out the scope of a particular variable:

\begin{itemize}
\item
        Formal parameters belong in
\item
        The function name belongs out
\item
        You can see out but you can't see in (tinted windows).
\end{itemize}

The first rule is shorthand for the fact that formal parameters
belong to the scope of the function body. Since the function body
is `inside' the function definition, we can say the formal parameters
belong in.

The second rule reminds us that as we move outward from a function body,
we find the enclosing scope holds the function definition. That is to
say, the function name is bound to a function object in the scope
enclosing the function body.

The third rule tells us all the variables that belong to 
ever-enclosing scopes are accessible and therefore
can be referenced by the innermost scope. The opposite is
not true. A variable in an enclosed scope can not be referenced
by an enclosing scope. If you forget the directions of this
rule, think of tinted windows. You can see out of a tinted
window, but you can't see in.

\section{Shadowing}

The formal parameters of a function can be thought of
as variable definitions that are only in effect when
the body of the function is being evaluated. That is,
those variables are only visible in the body and no where
else. This is why formal parameters are considered to be
{\it local} variable definitions, since they only have local
effect (with the locality being
the function body). Any direct reference to those
particular variables outside the body of the function
is not allowed (Recall that you can't see in).
Consider the following interaction with
the interpreter:

\begin{verbatim}
    >>> def square(a):
    ...     return a * a
    ...
    >>>
     
    >>> square(4)
    16
     
    >>> a
    NameError: name 'a' is not defined
\end{verbatim}


In the above example, the scope of variable {\it a} is restricted
to the body of the function {\it square}.
Any reference to
{\it a} other than in the context of {\it square} is invalid. Now
consider a slightly different interaction with the
interpreter:

\begin{verbatim}
    >>> a = 10
    >>> def almostSquare(a):
    ...     return a * a + b
    ...
    >>> b = 1
     
    >>> almostSquare(4)
    17

    >>> a
    10
\end{verbatim}

In this dialog, the global scope has three
variables added, {\it a}, {\it almostSquare} and
{\it b}.
In addition, the variable serving as the formal parameter
of {\it almostSquare} has the same name as the first variable
defined in the dialog. Moreover, the body of {\it almostSquare}
refers to both variables {\it a} and {\it b}. To which {\it a} does
the body of almostSquare refer? The global {\it a} or the local {\it a}?
Although it seems confusing at first,
the Scam interpreter has no difficulty in figuring out
what's what. From the responses of the interpreter,
the {\it b} in the body must refer to the variable that was
defined with an initial value of one. This is consistent with
our thinking, since {\it b} belongs to the enclosing scope
and is accessible within the body of {\it almostSquare}.
The {\it a} in the function body must refer to the
formal parameter whose value was set to 4 by the call to
the function (given the output of the interpreter).

When a local variable has the same name as a non-local variable
that is also in scope, the local variable is said to
{\it shadow} the non-local version. The term shadowed refers to the fact
that the other variable is in the shadow of the local
variable and cannot be seen. Thus, when the
interpreter needs the value of the variable, the value of
the local variable is retrieved.
It is also possible for a non-local variable to shadow
another non-local variable. In this case, the variable
in the nearest outer scope shadows the variable in
the scope further away.

In general, when a variable is referenced,
Scam first looks in the local
scope.
If the variable is not found there,
Scam looks in the enclosing scope.
If the variable is not there,
it looks in the scope enclosing the enclosing scope, 
and so on.

In the particular example,
a reference to {\it a} is made when the body of
{\it almostSquare} is executed. The value of {\it a}
is immediately found in the local scope.
When the value of {\it b} is required,
it is not found in the local scope. The interpreter
then searches the enclosing scope (which in this case happens
to be the global scope).
The global scope does hold {\it b} and its value, so a value
of 1 is retrieved.

Since {\it a} has a value of 4 and {\it b} has a value of 1, the
value of 17 is returned by the function. Finally, the
last interaction with the interpreter illustrates the
fact that the initial binding of {\it a} was unaffected by the
function call.

\section{Modules}

Often, we wish to use code that has already been written.
Usually, such code contains handy functions that have
utility for many different projects. In Scam, such
collections of functions are known as modules.
We can include modules into our current project with
the {\it import} statement, which we saw in
\link*{the chapter on functions}[Chapters~\Ref]{Functions}
and
\link*{the chapter on programs and files}[\Ref]{ScamPrograms}.

The import statement has two forms. The first is:

\begin{verbatim}
    from ModuleX import *
\end{verbatim}

This statement imports all the definitions from ModuleX and places
them into the global scope. At this point, those definitions
look the same as the built-ins, but if any of those definitions
have the same name as a built-in, the built-in is shadowed.

The second form looks like this:

\begin{verbatim}
    import ModuleX
\end{verbatim}

This creates a new scope that is separate from the global scope
(but is enclosed by the global scope).
Suppose
{\it ModuleX} has a definition for variable {\it a}, with a value
of 1.
Since {\it a} is in a scope
enclosed by the global scope, it is inaccessible from the global
scope (you can't see in):

\begin{verbatim}
    >>> import ModuleX
    >>> a
    NameError: name 'a' is not defined
\end{verbatim}

The direct reference to {\it a} failed, as expected.
However, one can get to {\it a} and its value {\it indirectly}:

\begin{verbatim}
    >>> import ModuleX
    >>> ModuleX . a
    1
\end{verbatim}

This new notation is known as {\it dot} notation and is commonly
used in object-oriented programming systems to references pieces
of an object. For our purposes, {\it ModuleX} can be thought
of as
a {\it named} scope and the {\it dot} operator is used to look
up variable {\it a} in the scope named ModuleX.

This second form of import is used when the possibility that
some of your functions or variables have the same name as
those in the included module. Here is an example:

\begin{verbatim}
    >>> a = 0
    >>> from ModuleX import *

    >>> a
    1
\end{verbatim}

Note that the {\it ModuleX}'s variable {\it a} has showed the previously
defined {\it a}. With the second form of import, the two versions
of {\it a} can each be referenced:

\begin{verbatim}
    >>> a = 0
    >>> import ModuleX

    >>> a
    0
    >>> ModuleX . a
    1
\end{verbatim}

\chapter{Objects}
\label{Objects}

In the Scam world, an object is a simple a collection of related variables.
You've already been exposed to objects, although you may not
have realized it. When you created a variable, you modified
the environment, which is an object. When you defined a
function, you created 
an object. To view an object, we use the predefined function
{\it pp}.\footnote{
The {\it pp} in the function names stands for {\it pretty print} which
means to print out something so it is 'pretty looking'.
} Evaluating the code:

\begin{verbatim}
    (define (square x)
        (* x x)
        )
    
    (pp square)
\end{verbatim}

yields something similar to the following output:

\begin{verbatim}
    <object 10435>
             __label  : closure
           __context  : <object 10381>
                name  : square
          parameters  : (x)
                code  : (begin (* x x))
\end{verbatim}

We see that the {\it square} function is made up of five {\it fields}.\footnote{
Some people use the term {\it component} or {\it instance variable} instead
of {\it field}.
Also, if you try this, you may see different numbers than 10435 and 10381.
These numbers represent the address of the object in memory.
}
These fields are: {\it \_\_label}, {\it \_\_context}, {\it name},
{\it parameters},
and {\it code}.

Usually, an object lets you look at its individual components.
For example:

\begin{verbatim}
    (println "square's formal parameters are: " (dot square parameters))
\end{verbatim}

yields:

\begin{verbatim}
    square's formal parameters are: (x)
\end{verbatim}
    
We use the function {\it dot} (usually called the {\it dot operator}) to extract
the fields of an object. The first argument to dot is the object, while
the second is the field name. The field name does not have to be quoted.

It is easy to create your own objects. First you must make a
{\it constructor}. A constructor is just a function that returns 
the predefined variable {\it this}. Suppose you want a constructor to
create an object with fields {\it name} and {\it age}. Here is 
one possibility:

\begin{verbatim}
    (define (person)
        (define name)
        (define age)
        this
        )
\end{verbatim}

We can create an object simply by calling the constructor:

\begin{verbatim}
    (define p (person))
\end{verbatim}

The variable {\it p} now points to a {\it person} object and we
can use {\it p} and the {\it dot} operator to
set the fields of the person object:

\begin{verbatim}
    (assign (dot p name) "Boris")
    (assign (dot p age) 33)
    (inspect (dot p name))
\end{verbatim}

Evaluating this code yields the following output:

\begin{verbatim}
    (dot p name) is Boris
\end{verbatim}

It is often convenient to give initial values to the fields of
an object. Here is another version of {\it person} that allows us
to do just that when we create the object:

\begin{verbatim}
    (define (person name age) this)
        
    (define p (person "Boris" 33))
        
    (inspect (dot p name))
\end{verbatim}

The output is the same:

\begin{verbatim}
    (dot p name) is Boris
\end{verbatim}

In general, if a field is to be initialized when the object
is constructed, make that field a formal parameter. If not,
make the field a locally declared variable.

\section{Adding Methods}

Objects can have methods as well.\footnote{
A method is just another name for a local function.
}
Here's a version of the
{\it person} constructor that has a {\it birthday} method.

\begin{verbatim}
    (define (person name age)
        (define (birthday)
            (println "Happy Birthday, " name "!")
            (++ age)
            )
        this
        )
        
    (define p (person "Boris" 33))
    ((dot p birthday))
    (inspect (dot p age))
\end{verbatim}

The output of this code is:

\begin{verbatim}
    Happy Birthday, Boris!
    (dot p age) is 34
\end{verbatim}

In summary, one turns a function into a constructor by making returning

\begin{verbatim}
    this
\end{verbatim}

from a function. The local variables, including
formal parameters, become the fields of the function while
any local functions serve as methods.

\section{Objects and Types}

If you were to ask an object, "What are you?", most
would respond, "I am an environment!". The {\it type} function is
used to ask such questions:

\begin{verbatim}
    (define p (person "betty" 19))
    (inspect (type p))
\end{verbatim}

yields:

\begin{verbatim}
    (type p) is environment
\end{verbatim}

This is because the predefined variable {\it this} always points to
the current environment and when we return this from a function
we are returning an environment. Since environments are objects
and vice versa, this is how making objects in Scam is so easy.

While the type function is often useful, we sometimes 
would like to know what kind of specific object an object is.
For example, we might like to 
know whether or not {\it p} is a {\it person} object.
That is to say,
was {\it p} created by the {\it person} function/constructor?.
One way to do this
is to ask the constructor of the object if it is the person function.
Luckily, all objects carry along a pointer to the function
that constructed them:

\begin{verbatim}
    (define p (person "veronica" 20))
    (inspect (dot p __constructor name))
\end{verbatim}

yields:

\begin{verbatim}
    (dot p __constructor name) is person
\end{verbatim}

Note that:
\verb!(dot p __constructor name)!
is equivalent to
\verb!(dot (dot p __constructor) name)!.
So, to ask if {\it p} is a person, we would use the following
expression:

\begin{verbatim}
    (if (and (eq? (type p) 'environment)
             (eq? (dot p __constructor name) 'person)) ...
\end{verbatim}

Since this construct is rather wordy, there
is a simple function, named {\it is?}, that you can use instead:

\begin{verbatim}
    (if (is? p 'person) ...
\end{verbatim}

The {\it is} function works for non-objects too. All of the following
expressions are true:

\begin{verbatim}
    (is? 3 'INTEGER)
    (is? 3.4 'REAL)
    (is? "hello" 'STRING)
    (is? 'blue 'SYMBOL)
    (is? (list 1 2 3) 'CONS)
    (is? (array "a" "b" "c") 'ARRAY)
    (is? (person 'veronica 20) 'object)
    (is? (person 'veronica 20) 'environment)
    (is? (person 'veronica 20) 'person)
\end{verbatim}

\section{A formal view of object-orientation}

Scam is a fully-featured object-oriented language. What does
that mean exactly? Well, to begin with, a programming language
is considered object-oriented if it has these
three features:

\begin{enumerate}
\item
    encapsulation
\item
    inheritance
\item
    polymorphism
\end{enumerate}

Encapsulation in this sense means that a programmer
can bundle {\it data} and {\it methods} into a single entity.
We've seen that a Scam function can
have local variables and local functions.
So, if we consider local variables (including the formal
parameters) as data and local
functions as methods, we see that Scam can encapsulate
in the object-oriented sense.

Inheritance is the ability to use the data and methods
of one kind of object by another as if they were defined
in the other object to begin with.
The next chapter deals with inheritance.

Polymorphism means that an object that inherits appears
to be both kinds of object, the kind of object it is
itself and the kind of object from which it inherits.
We will learn more about
polymorphism in the next chapter as well.

\section{Other objects in Scam}

While environments constitute the bulk of objects in
Scam, two other object types are built into Scam. They are closures
(seen the the first section) and error objects. The
main library adds in a third object type known as a thunk.

An error object is generated when an exception is caught.
The fields of an error object are code, value, and trace.
A thunk is an expression and an environment bundled
together. Thunks are used to delay evaluation of
an expression for a later time. The fields of a thunk
are {\it code} and {\it \_\_context}.
You can learn more about error objects and thunks
in subsequent chapters.

\section{Fun with {\it objects}}

Because of the flexibility of Scam, one can
add Java-like display behavior to objects.
In Java, if an object has a method named {\it toString},
then if one attempts to print the object, the {\it toString}
method is called to generate the print value of the object.

Here is an example:

\begin{verbatim}
    (define (person name age)
        (define (birthday)
            (println "Happy Birthday, " name "!")
            (++ age)
            )
        (define (toString)
            (string+ name "(age " age ")")
            )
        this
        )

    (define p (person "boris" 33))
\end{verbatim}

Given the above definition of {\it person}, printing the value of {\it p}:

\begin{verbatim}
    (println p)
\end{verbatim}

yields:

\begin{verbatim}
    <object 23452>
\end{verbatim}

Uh oh. The {\it toString} method wasn't called! This is because
the current version of {\it println}, defined in {\it main.lib},
does not understand the {\it toString}
method. No problem here, we'll just reassign {\it display},
since {\it println} calls {\it print} and {\it print} calls {\it display}
to do the actual printing:

\begin{verbatim}
    (define (display  item)
        (if (and (object? item) (local? 'toString item))
            (__display ((dot item toString)))
            (__display item)
            )
        )
\end{verbatim}

The main library binds the original version of {\it display}
to the symbol {\it \_\_display} for
safe-keeping.

Note that this new version of {\it display} is only found in the local
environment, so the current versions of {\it println} and {\it print}, defined
in an outer scope, cannot see the new version of {\it display} in the current
scope.
To do so would be a scope violation.
We solve this problem by {\it cloning} the two printing functions.
The process of cloning produces new closures with 
the local environment as the definition environment.
In all other respects, the cloned function is identical.
Thus, when the new {\it print} calls {\it display}, the new, local version 
will be found.

\begin{verbatim}
    (include "reflection.lib")

    (define print (clone print))
    (define println (clone println))

    (println p)
\end{verbatim}

The {\it reflection} library must be included
to access the {\it clone} function.

Now, printing the object {\it p} yields:

\begin{verbatim}
    boris (age 33)
\end{verbatim}

\chapter{Encapsulation, Inheritance and Polymorphism}
\label{Inheritance}

\section{A formal view of object-orientation}

Scam is a fully-featured object-oriented language. What does
that mean exactly? Well, to begin with, a programming language
is considered object-oriented if it has these
three features:

\begin{enumerate}
\item
    encapsulation
\item
    inheritance
\item
    polymorphism
\end{enumerate}

Encapsulation in this sense means that a programmer
can bundle {\it data} and {\it methods} into a single entity.
We've seen that a Scam function can
have local variables and local functions.
So, if we consider local variables (including the formal
parameters) as data and local
functions as methods, we see that Scam can encapsulate
in the object-oriented sense.

Inheritance is the ability to use the data and methods
of one kind of object by another as if they were defined
in the other object to begin with.

Polymorphism means that an object that inherits appears
to be both kinds of object, the kind of object it is
itself and the kind of object from which it inherits.

\section{Simple encapsulation}

The previous chapter was concerned with encapsulation;
let us review.

A notion that simplifies encapsulation in Scam is to use
environments themselves as objects. Since an environment can be thought
of as a table of the variable names currently in scope, along with their
values, and an object can be thought of as a table of instance variables
and method names, along with their values, the association of these two
entities is not unreasonable.

Thus, to create an object, we need only cause a new scope to come into
being. A convenient way to do this is to make a function call. The call
causes a new environment to be created, in which the arguments to the
call are bound to the formal parameters and under which the function body
is evaluated. Our function need only return a pointer to the current
execution environment to create an object.  Under such a scenario, we
can view the the function definition as a class definition with the
formal parameters serving as instance variables and locally defined
functions serving as instance methods.  

Scam 
allows the current execution environment
to be referenced and returned.
Here is an example of object creation in Scam:

\begin{verbatim}
    (define (bundle a b)
        (define (total base) (+ base a b))
        (define (toString) (string+ "a:" a ", b:" b))
        this    ;return the execution environment
        )

    (define obj (bundle 3 4))

    (inspect ((dot obj display)))  ;call the display function
    (inspect ((dot obj total) 0))  ;call the total function
\end{verbatim}

The variable {\it this} is always bound to the current execution
enironment or scope. Since, in Scam, objects and environments the same,
this can be roughly thought of as a self reference to an object. The
{\it dot} function (equivalent to the dot operator in Java) is used to
retrieve the values of the given instance variable from the given object.
The {\it inspect} function prints the unevaluated argument followed by
its evaluation.

Running the above program yields the following output:

\begin{verbatim}
    ((dot obj display)) is a:3, b:4
    ((dot obj total) 0) is 7
\end{verbatim}

It can be seen from the code and the output that encapsulation via this
method produces objects that can be manipulated in an intuitive manner.

It should be stated that encapsulation is considered merely a device for
holding related data together; whether the capsule is transparent or
not is not considered important for the purposes of this paper. Thus,
in the above example, all components are publicly visible. 

\section{Three common types of inheritance}

Any specification of inheritance semantics must be (relatively)
consistent with the afore-mentioned intuition about inheritance.
With regards to inheritance behavior pragmatics, there seems to be three
forms of inheritance behavior that make up this intuition.  Taking the
names given by Bertrand Meyer in ``The Many Faces of Inheritance: 
A Taxonomy of Taxonomies'',
the three are {\it extension},
{\it reification}, and {\it variation}.  In extension inheritance,
the heir simply adds features in addition to the features of the
ancestor; the heir is indistinguishable from the ancestor, modulo the
original features.  In reification inheritance, the heir completes,
at least partially, an incompletely specified ancestor.  An example of
reification inheritance is the idea of an abstract base class in Java.
In variation inheritance, the heir adds no new features but overrides
some of the ancestor's features. Unlike extension inheritance, the heir
is distinguishable from the ancestor, modulo the original features. The
three inheritance types are not mutually exclusive; as a practical matter,
all three types of inheritance could be exhibited in a single instance
of general inheritance.  Any definition of inheritance should capture
the intent of these forms.
As it turns out, it is very easy to implement these three forms
of inheritance in Scam.

Scam uses a novel approach to inheritance. Other languages 
processors pass
a pointer to the object in question to all object methods.
This pointer is known as a self-reference. This passing of
a self-reference may be hidden from the programmer or may
be made explicit. In any case, Scam displenses with self-references
and implements inheritance through the manipulation of scope.

\subsection{Extension inheritance}

In order to provide inheritance by manipulating scope, it must be
possible to both get and set the static scope, at runtime, of objects
and function closures.  There are two functions that will help us
perform those tasks. They are
{\it getEnclosingScope} and {\it setEnclosingScope} and are defined
in the supplemental library, {\it inherit.lib}.
While at first
glance it may seem odd to change a static scope at runtime,
these functions translate into getting and setting
the{\it  \_\_context} pointer of an environment (or closure).

Recall that in extension inheritance, the subclass strictly adds
new features to a superclass and that a subclass object and a
superclass object are indistinguishable, behavior-wise, with regards
to the features provided by the superclass.  
Consider two objects, {\it child} and {\it parent}. The extension
inheritance of {\it child} from {\it parent} can be implemented with
the following pseudocode:

\begin{verbatim}
    setEnclosingScope(parent,getEnclosingScope(child));
    setEnclosingScope(child,parent);
\end{verbatim}

As a concrete example, consider the following Scam program:

\begin{verbatim}
    (include "inherit.lib")

    (define (c) "happy")
    (define (parent)
        (define (b) "slap")
        this
        )
    (define (child)
        (define (a) "jacks")
        (define temp (parent))
        (setEnclosingScope temp (getEnclosingScope this))
        (setEnclosingScope this temp)
        this
        )

    (define obj (child))

    (inspect ((dot obj b)))
    (inspect ((dot obj a)))
    (inspect ((dot obj c)))
\end{verbatim}

Running this program yields the following output:

\begin{verbatim}
    ((dot obj a)) is jacks
    ((dot obj b)) is slap
    ((dot obj c)) is happy
\end{verbatim}

The call to {\it a} immediately finds the child's method.  The call to
{\it b} results in a search of the child. Failing to find a binding
for {\it b} in {\it child}, the enclosing scope of {\it child} is
searched. Since the enclosing scope of {\it child} has been reset to {\it
parent}, {\it parent} is searched for {\it b} and a binding is found.
In the final call to {\it c}, a binding is not found in either the
child or the parent, so the enclosing scope of {\it parent} is searched.
That has been reset to {\sc child}'s enclosing scope. There a binding
is found. So even if the parent object is created in a scope different
from the child, the correct behavior ensues.

For an arbitrarily long inheritance chain, {\it p1} inherits from {\it
p2}, which inherits from {\it p3} and so on, the most distant ancestor
of the child object receives the child's enclosing scope:

\begin{verbatim}
    setEnclosingScope(pN,getEnclosingScope(p1))
    setEnclosingScope(p1,p2);
    setEnclosingScope(p2,p3);
    ...
    setEnclosingScope(pN-1,pN)
\end{verbatim}

It should be noted that the code examples in this and the next subsections
hard-wire the inheritance manipulations. As will be seen further on,
Scam automates these tasks.

\subsection{Reification inheritance}

As stated earlier, reification inheritance concerns a subclass fleshing
out a partially completed implementation by the superclass. A consequence
of this finishing aspect is that, unlike extension inheritance, the
superclass must have access to subclass methods.  A typical approach
to handling this problem is rather inelegant, passing a reference to
the original object as hidden, or not so hidden,
parameter to all methods.  Within method
bodies, method calls are routed through this reference. Inheritance in
Python is done just this way; the object reference is bound to the
first formal parameter in all object methods.

That said, our approach for extension inheritance
does not work for reification inheritance. Suppose a parent method
references a method provided by the child. In Scam,
when a function definition is encountered, a closure is
created and this closure holds a pointer to the definition environment. It
is this pointer that implements static scoping in such interpreters.

For parent methods, then, the enclosing scope is the parent.  When the
function body of the method is being evaluated, the reference to the
method supplied by the child goes unresolved, since it is not found in
the parent method. The enclosing scope of the parent method, the parent
itself, is searched next.  The reference remains unresolved. Next the
enclosing scope of the parent is searched, which has been reset to the
enclosing scope of the child. Again, the reference goes unresolved (or
resolved by happenstance should a binding appear in some enclosing scope
of the child).

The solution to this problem is to reset the enclosing scopes of the
parent methods to the child. In pseudocode:

\begin{verbatim}
    setEnclosingScope(parent,getEnclosingScope(child));
    setEnclosingScope(child,parent);
    for each method m of parent
        setEnclosingScope(m,child)
\end{verbatim}

Now, reification inheritance works as expected. Here is an example:

\begin{verbatim}
    (include "inherit.lib")

    (define (parent)
        (define (ba) (string+ (b) (a)))
        (define (b) "slap")
        this
        )
    (define (child)
        (define (a) "jacks")
        (define temp (parent))
        (setEnclosingScope temp (getEnclosingScope this))
        (setEnclosingScope this temp)
        (setEnclosingScope (dot temp ba) this)
        this
        )

    (define obj (child))

    (inspect ((dot obj ba)))
\end{verbatim}

The output of this program is:

\begin{verbatim}
    ((dot obj ba)) is "slapjacks"
\end{verbatim}

As can be seen, the reference to {\it a} in the function {\it ba} is
resolved correctly, due to the resetting of {\it ba}'s enclosing scope
by {\it child}.

For longer inheritance chains, the pseudocode of the previous subsection
is modified accordingly:

\begin{verbatim}
    setEnclosingScope(pN,getEnclosingScope(p1))
    setEnclosingScope(p1,p2);
    for each method m of p2: setEnclosingScope(m,p1)
    setEnclosingScope(p2,p3);
    for each method m of p3: setEnclosingScope(m,p1)
    ...
    setEnclosingScope(pN-1,pN)
    for each method m of pN: setEnclosingScope(m,p1)
\end{verbatim}

All ancestors of the child has the enclosing scopes of their methods reset.

\subsection{Variation inheritance}

Variation inheritance captures the idea of a subclass overriding a
superclass method. If functions are naturally virtual (as in Java),
then the overriding function is always called preferentially over the
overridden function.

If {\it child} is redefined as follows:

\begin{verbatim}
    (define (child)
        (define (b) "jumping")
        (define (a) "jacks")
        (define temp (parent))
        (setEnclosingScope temp (getEnclosingScope this))
        (setEnclosingScope this temp)
        (setEnclosingScope (dot temp ab) this)
        this
        )
\end{verbatim}

then the new version of {\it b} overrides the parent version.  The output
now becomes:

\begin{verbatim}
    ((dot obj ba)) is jumpingjacks 
\end{verbatim}

This demonstrates that both reification and variation inheritance can be
implemented using the same mechanism.  Another benefit is that instance
variables and instance methods are treated uniformly. Unlike virtual
functions in Java and C++, instance variables in those languages shadow
superclass instance variables of the same name, but only for subclass
methods.  For superclass methods, the superclass version of the instance
variable is visible, while the subclass version is shadowed.  With this
approach, both instance variables and instance methods are virtual,
eliminating the potential error of shadowing a superclass instance
variable. Here is an example:

\begin{verbatim}
    (include "inherit.lib")

    (define (parent)
        (define x 0)
        (define (toString) (string+ "x:" x))
        this
        )
    (define (child)
        (define x 1)
        (define temp (parent))
        (setEnclosingScope temp (getEnclosingScope this))
        (setEnclosingScope this temp)
        (setEnclosingScope (dot temp toString) this)
        this
        )

    (define p-obj (parent))
    (define c-obj (child))

    (inspect ((dot p-obj toString)))
    (inspect ((dot c-obj toString)))
\end{verbatim}

The output:

\begin{verbatim}
    ((dot p-obj toString)) is x:0
    ((dot c-obj toString)) is x:1
\end{verbatim}

demonstrates the virtuality of the instance variable \emph{x}.  Even
though the program calls the superclass version of \emph{toString},
the subclass version of \emph{x} is referenced.

\subsection{Implementing Inheritance in Scam}

Since environments are objects in Scam, implementing the {\it
getEnclosingScope} and {\it setEnclosingScope} functions are trivial:

\begin{verbatim}
    (define (setEnclosingScope a b) (assign (dot a __context) b))
    (define (getEnclosingScope a) (dot a __context))
\end{verbatim}

Moreover, the task of resetting the enclosing scopes of the parties
involved can be automated. Scam provides a library, 
named {\it inherit.lib}, written in Scam that
provides a number of inheritance mechanisms. The first (and simplest)
is ad-hoc inheritance. Suppose we have objects {\it a}, {\it b}, and{\it  c}
and we wish
{\it a} to inherit from {\it b} and {\it c}
(and if both {\it b} and {\it c} provide functionality,
we prefer {\it b}'s implementation).
To do so, we call the {\it mixin} function:

\begin{verbatim}
    (mixin a b c)
\end{verbatim}

A definition of {\it mixin} could be:

\begin{verbatim}
    (define (mixin object @)  ; @ points to a list of remaining args
        (define outer (getEnclosingScope object))
        (define spot object)
        (while (not (null? (cdr @)))
            (define current (car @))
            (resetClosures current object)
            (setEnclosingScope spot current)
            (assign spot current)
            (assign @ (cdr @)
            )
        (setEnclosingScope (car @) outer)
        (resetClosures (car @) object)
        )
\end{verbatim}

The other type of inheritance emulates the {\it extends} operation in
Java.  For this type of inheritance, the convention is that an object
must declare a parent. In the constructor for an object, the parent
instance variable is set to the parent object, usually obtained via the
parent constructor.  Here is an example:

\begin{verbatim}
    (define (b)
        (define parent nil)
        ...
        this
        )
    (define (a)
        (define parent (b))     ;setting the parent 
        ...
        this
        )
\end{verbatim}

Now, to instantiate an object, the {\it new} function is called:

\begin{verbatim}
    (define obj (new (a)))
\end{verbatim}

The {\it new} function follows the parent pointers to reset the enclosing
scopes appropriately. Here is a possible implementation of {\it new},
which follows the definition of {\it mixin} closely:

\begin{verbatim}
    (define (new object)
        (define outer (getEnclosingScope object))
        (define spot object)
        (define current (dot spot parent))
        (while (not (null? current))
            (resetClosures current object)
            (setEnclosingScope spot current)
            (assign spot current)
            (define current (dot spot parent))
            )
        (setEnclosingScope spot outer)
        (resetClosures spot object)
        )
\end{verbatim}

Other forms of inheritance are possible as well. The flexibility of this
approach does not require inheritance to be built into the language.

\subsection{Darwinian versus Lamarckian Inheritance}

The behavior of the inheritance scheme implemented in this paper differs
from the inheritance schemes of the major industrial-strength languages
in one important way.  In Java, for example, if a superclass method
references a variable defined in an outer scope (this can happen
with nested classes), those references are resolved the same way,
{\it whether or not} an object of that class was instantiated as a
stand-alone object or as part of an instantiation of a subclass object.
This is remeniscent of the inheritance theory of Jean-Baptiste Lamarck,
who postulated that the environment influences inheritance. In Java,
the superclass retains traces of its environment which can influence
the behavior of a subclass object.

With selfless inheritance, the static scopes of the superclass objects
are replaced with the static scope of the subclass object, a purely
Darwinian outcome. The superclass objects contributes the methods
and instance variables (say, the genes of the superclass) but none of
the environmental influences.  Thus, the subclass object must provide
bindings for the non-local references either through its own definitions
or in its definition environment.

\section{Polymorphism}

Polymorphism is a word that literally means ``having multiple shapes''.
With regards to object-orientation, polymorphism means one kind of object
can look like another kind of object. One concrete example of this
involves inheritance: if object {\it child} inherits from object {\it parent},
does the {\it child} look like a {\it parent} object
as well as a {\it child} object? In
other words, can a variable that points to a parent object also
point to a child object? This question is of critical importance
for statically typed languages such as C++ and Java, but is not
so important for dynamically-typed languages like Scam.
This is because a Scam variable can point to any type of entity,
so the question of whether a variable can point to either a child
or a parent is moot.

That said, it is often useful in an dynamically-typed, object-oriented
language to ask whether or not a variable points to an object that
looks like some other object. The {\it is?} function, introduced in the
previous chapter, can answer these questions. Consider this set
of constructors:

\begin{verbatim}
    (include "inherit.lib")

    (define (p)
        (define parent nil)
        this
        )

    (define (c)
        (define parent (p))
        this
        )
\end{verbatim}

Here, we have {\it b} inheriting from {\it a}.
If we create an {\it a} object and a {\it b} object using {\it new}:

\begin{verbatim}
    (define p-obj (new (p)))
    (define c-obj (new (c)))
\end{verbatim}

we can now ask what kinds of objects they are:

\begin{verbatim}
    (inspect (is? p-obj 'p))
    (inspect (is? c-obj 'c))
\end{verbatim}

As expected, the output is:

\begin{verbatim}
    (is? p-obj 'p) is #t
    (is? c-obj 'c) is #t
\end{verbatim}

However, a typical view in the object-oriented world is that a
child object {\it is also} a parent object,
since it inherits all the fields and
methods of the {\it parent}. The {\it is?} function conforms to this idea:

\begin{verbatim}
    (is? c-obj 'p)
\end{verbatim}

evaluates to true.
Conversely, the typical view is that the parent object {\it is not}
a child object. The expression:

\begin{verbatim}
    (is? p-obj 'c)
\end{verbatim}

evaluates to false.

%\chapter{Parameter Passing}
\label{ParameterPassing}

There are (at least) six historical and current
methods of passing arguments to a function when a function call
is made. They are:

\begin{itemize}
\item
    {\it call-by-value}
\item
    {\it call-by-reference}
\item
    {\it call-by-value-result}
\item
    {\it call-by-name}
\item
    {\it call-by-need}
\item
    {\it generalized delayed evaluation}
\end{itemize}

Let's examine these six methods in turn. After which, we will
investigate variadic functions in Scam.

\section*{Call-by-value}

This method is the only method of parameter passing allowed by C, Java,
Scam, and Scheme. In this method, the formal parameters are set up as local
variables that contain the value of the expressions that were passed as
arguments to the function. Changes to local variables are not reflected
in the actual arguments. For example, an attempt to define a function
for exchanging the values
of two variables passed to it might look like:

\begin{verbatim}
    (define (swap a b)
        (define temp a)
        (set! a b)
        (set! b temp)
        )
\end{verbatim}

Consider this code which uses {\it swap}:

\begin{verbatim}
    (define x 3)
    (define y 4)

    (swap x y)

    (inspect x)
    (inspect y)
\end{verbatim}

Under {\it call-by-value},
this function would not yield the intended semantics.
The output of the above code is:

\begin{verbatim}
    x is 3
    y is 4
\end{verbatim}

This is because only the values of the local variables {\it a} and {\it b}
were swapped; the variables {\it x} and {\it y} remain unchanged
as only their values were passed to the swapping function.
In
general, one cannot get a swap routine to work under {\it call-by-value}
unless the addresses of the variables are somehow sent. One way of
using addresses is to pass an array (in C and Scam, when an array name
is used as an argument, the address of the first element is sent. In Java,
the address of the array
object is sent). For example,
the code fragment:

\begin{verbatim}
    (define x (array 1))
    (define y (array 0))

    (swap x y)  ;address of beginning element is sent

    (println "x[0] is " (getElement x 0) " and y[0] is " (getElement y 0))
\end{verbatim}

with {\it swap} defined as...

\begin{verbatim}
    (define (swap a b)
        (define temp (getElement a 0))
        (setElement a 0 (getElement b 0))
        (setElement b 0 temp)
        )
\end{verbatim}

would print out:

\begin{verbatim}
    x[0] is 0 and y[0] is 1
\end{verbatim}

In this case, the addresses of arrays {\it x} and {\it y}
are stored in the local variables {\it a} and
{\it b}, respectively.
This is
still call-by-value since even if the address stored in {\it a}, for example,
is modified,
{\it x} still "points" to the same array as before. Here is an example:

\begin{verbatim}
    (define (change a)
       (assign a (array 13))
       )

    (define x (array 42))

    (inspect (getElement x 0))
\end{verbatim}

yields:

\begin{verbatim}
    (getElement x 0) is 42)
\end{verbatim}

Note that C has an operator that extracts the address of a variable, the
\& operator. By using \&, one can write a swap in C that does
not depend on arrays:

\begin{verbatim}
    void swap(int *a,int *b)
        {
        int temp;
        temp = *a;
        *a = *b;
        *b = temp;
        }
\end{verbatim}

The call to {\it swap} would look like:

\begin{verbatim}
    int x = 3;
    int y = 4;

    swap(&x,&y);

    printf("x is %d and y is %d\n",x,y);
\end{verbatim}

with output:

\begin{verbatim}
    x is 4 and y is 3
\end{verbatim}

as desired.

Note that this is still {\it call-by-value} since the {\it value} of the
address of {\it x} (and {\it y}) is being passed to the swapping function.

\section*{Call-by-reference}

This second method differs from the first in that changes to
the formal parameters during execution of the function body are
immediately reflected in actual arguments. Both C++ and Pascal allow
for call-by-reference. Normally, this is accomplished,
under the hood, by passing the
address of the actual argument (assuming it has an address) rather than
the value of the actual argument. References to the analogous formal
parameter are translated to references to the memory location stored in
the formal parameter. In C++, {\it swap} could be defined as:

\begin{verbatim}
    void swap(int &a, int &b) // keyword & signifies
        {                     // call-by-reference
        int temp = a;
        a = b;
        b = temp;
        }
\end{verbatim}

Now consider the code fragment:

\begin{verbatim}
    var x = 3;  //assume x at memory location 1000
    var y = 4;  //assume y at memory location 1008

    //location 1000 holds a 3
    //location 1008 holds a 4

    swap(x,y);
    cout << "x is " << x << " and y is " << y << "\n";
\end{verbatim}

When the swapping function starts executing, the value 1000 is stored
in the local variable {\it a} and 1008 is stored in local variable {\it b}.
The line:

\begin{verbatim}
    temp = a;
\end{verbatim}

is translated, not into store the value of {\it a} (which is 1000) in
variable {\it temp}, but rather store the value at memory location 1000 (which
is 3) in variable {\it temp}. Similar translations are made for the remaining
statements in the function body. Thus, the code fragment prints out:

\begin{verbatim}
    x is 4 and y is 3
\end{verbatim}

The swap works! When trying to figure out what happens under
{\it call-by-reference}, it is often useful to draw pictures of the various
variables and their values and locations, then update them as the function
body executes.

It is possible to simulate {\it call-by-reference} in Scam
with delayed evaluation.

\section*{Call-by-value-result}

This method is a combination of the first two. Execution of the function
body proceeds as in {\it call}-by-value. That is, no updates of the actual
arguments are made. However, after execution of the body, but just before
the function returns, the actual arguments are updated with the final
values of their associated formal parameters. This method of parameter
passing is often used for Ada in-out parameters. Would swap work under
call-by-value-result? 

Like {\it call-by-reference},
it is possible to simulate {\it call-by-value-result} in Scam
with delayed evaluation.

\section*{Call-by-name}

{\it Call-by-name} was used in Algol implementations. In essence, functions
are treated as macros. Under {\it call-by-name}, the fragment:

\begin{verbatim}
    (define (swap a b)
        (define temp a)
        (set! a b)
        (set! b temp)
        )

    (define x 3)
    (define y 4)

    (swap x y)

    (println "x is "  x " and y is " y)
\end{verbatim}

...would be translated into:

\begin{verbatim}
    (define (swap a b)
        (define temp a)
        (set! a b)
        (set! b temp)
        )

    (define x 3)
    (define y 4)

    ;substitute the body of the function for the call, 
    ;renaming the references to formal parameters with the names of 
    ;the actual args

    (scope
        (define temp x)
        (assign x y)
        (assign y temp)
        )

    (println "x is "  x " and y is " y)
\end{verbatim}

Under {\it call-by-name}, the {\it swap} works as desired,
so why is {\it call-by-name}
a method that has fallen into relative disuse? One reason is complexity.
What happens if a local parameter happens to have the same name as one of
the actual args. Suppose {\it swap} had been written as:

\begin{verbatim}
    (define (swap a b)
        (define x a) 
        (set! a b)
        (set! b x)
        )
\end{verbatim}

Then a naive substitution and renaming would have produced:

\begin{verbatim}
    (scope
        (define x x)
        (assign x y)
        (assign y x)
        )
\end{verbatim}

which is surely incorrect.
Further problems occur if the body of the function references
globals which have been shadowed in the calling function. This requires
a complicated renaming scheme. Finally, {\it call-by-name} makes treating
functions as first-class objects problematic (being difficult to recover
the static environment of the called function). {\it Call-by-name}
exists today in C++, where it is possible to {\it inline} function calls
for performance reasons, and in macro processors.

\section*{Call-by-need}

In {\it call-by-value}, the arguments in a function call are evaluated and
the results are bound to the formal parameters of the function. In
{\it call-by-need}, the arguments themselves are literally bound
to the formal
parameters, as in {\it call-by-name}. A major difference is
that the calling environment is also bound to the formal
parameters as well. This bundle of literal argument and 
evaluation environment is known as a {\it thunk}.
The actual values of the arguments
are determined only when such values are
needed; when such a need occurs, the thunk is
evaluated, causing the literal argument
in the thunk to be
evaluated in the stored (calling) environment.
For example, consider this code:

\begin{verbatim}
    (define z 5)
    (f (+ z 3)) 
\end{verbatim}

with {\it f} defined as:

\begin{verbatim}
    (define (f x)
        (define y x)  ;x needed! x is fixed to 8 under call-by-need
        (set! z (* z 2))
        (+ x y)       ;x needed! x was already evaluated under call-by-need
        )
\end{verbatim}

in the same scope as {\it z}.
Under {\it call-by-name}, the return value is 21, but under
{\it call-by-need}, the return value is 16.
This is because the
value of {\it z} changed {\it after} the point when the value of {\it x}
(really \verb!(+ z 3)! was needed and the value of {\it x} was fixed from
the prior evaluation of {\it x}. Under {\it call-by-name}, the second
reference to {\it x} causes a fresh, new evaluation of {\it z},
the yielding the result of 21.

{\it Call-by-need}
is exactly the method used to implement streams in the 
textbook
{\it The Structure and Interpretation of Computer Programs}.
It is important to remember that the evaluation of a
{\it call-by-need} argument is done only once, 
with the result stored in the thunk for future requests.

\section*{Generalized delayed evaluation}

{\it Call-by-need} illustrates an example of delayed evaluation.
An unevaluated argument is stored in the thunk along with
its execution environment. The execution environment
is also known as the {\it calling environment},
due to the fact that the execution environment is where the
function call was made. Generalized delayed evaluation
loosens the restrictions of call-by-need in that the
function definer controls which arguments are delayed and
whether or not a fresh evaluation of a delayed argument
is needed.
Differences between {\it call-by-need} and
{\it generalized delayed evaluation}
arise when a variable making up a delayed argument 
experiences a change of state in between references in
the function body.

\section*{Simulating {\it call-by-reference} in Scam}

It is possible to simulate {\it call-by-reference} in Scam by 
delaying the evaluation of function call arguments and
manipulating the calling environment.
Before that can happen, however, the called function needs to
be able to access the calling environment.

To obtain the
calling environment in Scam,
one simply adds a formal parameter with the name {\it \#}.
The calling environment is then passed silently to the
function when a function call is made. This means that
the {\it \#} formal parameter is not matched to any actual argument
and can appear anywhere in the parameter list (except after
the variadic parameters {\it @} and {\it \$} - more on them later).

In addition to grabbing a handle to the calling environment,
a swapping function also needs to delay the evaluation of
the variables passed in.
One delays the evaluation of an argument by naming the formal
parameter matched to the argument in a special way. If the
formal parameter name begins with a {\it \$}, then
the corresponding argument is delayed.

With the ability to grab the calling environment,
delay the evaluation of arguments, and access the
bindings in an environment (see the chapter on Objects),
we can now define a {\it swap} function that works as intended.

\begin{verbatim}
    (define (swap # $a $b)
        (define temp (get $a #))
        (set $a (get $b #) #)
        (set $b temp #)
        )
\end{verbatim}

Note that the local variables {\it \$a} and {\it \$b} are regular variables;
they happen to point to unevaluated fragments
of Scam code.
Note also that {\it set} was used instead of {\it set!}
to change variable values.
If {\it set!} had been used, only the value of the local
variables {\it \$a} and {\it \$b} would change.
Unlike {\it set!}, {\it set} evaluates all its arguments,
so the values of {\it \$a} and {\it \$b} are passed to {\it set}, rather than
their names.
Finally note, that we pass the calling environment, {\it \#}, as the third
parameter to set, so that changes to the symbols to which
{\it \$a} and {\it \$b}
are bound
happen in the calling environment.

\section{Variadic functions}

A variadic function is a function that can take a different
number of arguments from call to call.
Scam allows this via two special formal parameter names.
They are {\it @} and {\it \$}.
If the last formal parameter is {\it @}, then all remaining
(evaluated) arguments not matched to any previous formal
parameters are gathered up in a list and {\it @} is bound to this
list.  For example, consider these definitions:

\begin{verbatim}
    (define (variadic first @)
        (println "the first argument is " first)
        (println "the remaining arguments are:")
        (while (valid? @)
            (println "    " (car @))
            (set! @ (cdr @))
            )
        )
    (define x 1)
    (define y 2)
\end{verbatim}

The call \verb!(variadic x)! produces:

\begin{verbatim}
    the first argument is 1
    the remaining arguments are:
\end{verbatim}

while the call \verb!(variadic x y (+ x y))! produces:

\begin{verbatim}
    the first argument is 1
    the remaining arguments are:
         2
         3
\end{verbatim}

Similar to {\it @,} the formal parameter {\it \$} is expected to be the
last formal parameter. The difference is that the arguments
bundled up into a list
are delayed. Suppose one replaced all occurrences of {\it @} with
{\it \$} in the definition of {\it variadic}.
Then, the call \verb!(variadic x y (+ x y))! would produce:

\begin{verbatim}
    the first argument is x
    the remaining arguments are:
        x
        (+ x y)
\end{verbatim}

%\chapter{Overriding Functions}
\label{OverridingFunctions}

Suppose one wishes to count how many additions are performed when
code in a module is executed. One way to do this is to override
the built-in addtion function:

\begin{verbatim}
    (define plus-count 0)
    (define (+ a b)
        (assign plus-count (+ plus-count 1))
        (+ a b)
        )
\end{verbatim}

The problem here is that the original binding of {\it +} is lost when the 
new version is defined.
Calling {\it +} now will result in an infinite recursive
loop.

One solution is to save the original binding of {\it +} before the new version
is defined:

\begin{verbatim}
    (define old+ +)
    (define (+ a b)
        (assign plus-count (old+ plus-count 1))
        (old+ a b)
        )
\end{verbatim}

With the original version of {\it +} bound to {\it old+},
now {\it a} and {\it b} can be added
together properly.
    
Scam automates this process with two functions, {\it redefine} and {\it prior}.
If the new version of a function is ``redefined'' rather than defined,
the previous
binding of the function is saved in the function closure that results.
The {\it prior} function 
is then used to retrieve this binding. Here is a rewrite
of the above code using these functions:

\begin{verbatim}
    (include "reflection.lib")

    (redefine (+ a b)
        (assign plus-count ((prior) plus-count 1))
        ((prior) a b)
        )
\end{verbatim}

The {\it redefine} and {\it prior} functions can be accessed by including
{\it reflection.lib}.
    
\section{Implementing {\it redefine} and {\it prior}}

Recall that closures are objects in Scam.
The {\it redefine} function works by adding a field to the closure
object generated by a function definition. It begins by
delaying evaluation of the paramater list and the function
body and then extracting the function name from the parameter list.

\begin{verbatim}
    (define (redefine # $params $)
        ;obtain the function name
        (define f-name (car $params))
        ;find the previous binding of the function name
        ;if no prior binding, use the variadic identity function 
        (if (defined? f-name #)
            (define f-prior (get f-name #))
            (define f-prior (lambda (@) @))
            )
        ;now generate the function closure
        (define f (eval (cons 'define (cons $params $)) #))
        ;add the previous binding to the closure
        (addSymbol '__prior f-prior f)
        f
        )
\end{verbatim}

It continues by looking up the function name in the calling scope, {\it \#},
binding that value to the symbol {\it f-prior}. If no binding exists,
an identity function is bound to {\it f-prior}.
Next, the desired function definition is processed by building
the code for a function definition from the delayed parameter list
and the delayed body.
Finally, a new field is added to the function closure and bound
to the prior function.

The {\it prior} function then looks for the added symbol and returns
its binding. It does so by extracting the constructor of the calling
environment and then retrieving the value of the symbol that was 
added by {\it redefine}:

\begin{verbatim}
    (define (prior #)
        (define f (dot # __constructor))
        (get '__prior f)
        )
\end{verbatim}

\section{Cloning functions}

If you override defined in an enclosing scope, only functions that
call the overridden function in the current scope see the new definition.
Functions in an enclosing scope that call the overridden function,
see the old version. This would be a scope violation otherwise.
To solve this problem, one can override the offending function
or more simply, clone it. Cloning a function creates a new
definition in the current scope. The only difference is the
definition environment is changed to the current environment,
the function parameter list and the body remain unchanged.

To clone a function, one calls the clone function, passing
in the function to be cloned. Consider this code:

\begin{verbatim}
    (include "reflection.lib")

    (define (f x) x)
    (inspect this)
    (inspect (dot f __context))

    (scope
        (inspect (local? 'f this))
        (define f (clone f))
        (inspect (local? 'f this))
        (inspect this)
        (inspect (dot f __context))
        )
\end{verbatim}

The output generated will be something like:

\begin{verbatim}
    this is <object 11698>
    (dot f __context) is <object 11698>
    (local? (quote f) this) is #f
    (local? (quote f) this) is #t
    this is <object 13317>
    (dot f __context) is <object 13317>
\end{verbatim}

The first two calls to {\it inspect} show that {\it f}'s definition
environment is the outer scope. The next call to {\it inspect} shows that
{\it f} is not defined in the inner scope. The last three calls show that
{\it f} is now defined with the proper definition environment.

For an example of using clone, see the chapter on Object.

%\chapter{More about Functions}
\label{MoreAboutFunctions}

We have already seen some examples of functions,
some user-defined and some built-in.
For example, we have used the built-in functions,
such as 
{\tt *} and defined our own functions,
such as {\it square}.
In reality, {\it square} is not a function, per se, but a variable
that is bound to the function that multiplies two numbers
together. It is tedious to say `the function bound to
the variable {\it square}', however,
so we say the more concise (but technically incorrect)
phrase `the {\it square} function'.

\section{Built-in Functions}

Scam has many built-in, or {\it predefined}, functions.
No one, however,
can anticipate all possible tasks that someone might want to perform,
so most programming languages allow the user to define new functions.
Scam is no exception and provides
for the creation of new and novel functions.
Of course,
to be useful,
these functions should be able to call
built-in functions as well as other programmer created
functions.

For example, a function that determines whether a given
number is odd or even is not built into Scam but can be
quite useful in certain situations.
Here is a definition
of a function named {\it isEven} which returns true if the
given number is even, false otherwise:

\begin{verbatim}
    >>> def isEven(n):
    ...     return n % 2 == 0
    ...
    >>>

    >>> isEven(42)
    True

    >>> isEven(3)
    False

    >>> isEven(3 + 5)
    True
\end{verbatim}

We could spend days talking about about what's going on in these
interactions with the interpreter. First, let's talk
about the syntax of a function definition. Later, we'll
talk about the purpose of a function definition. Finally,
will talk about the mechanics of a function definition
and a function call.

\section{Function syntax}

Recall that the words of a programming language include its
primitives, keywords and variables. A function definition
corresponds to a sentence in the language in that it is
built up from the words of the language. And like human
languages, the sentences must follow a certain form. This
specification of the form of a sentence is known as its
{\it syntax}. Computer Scientists often use a special way
of describing syntax of a programming language called the
Backus-Naur form (or {\sc bnf}). Here is a high-level description
of the syntax of a Scam function definition using {\sc bnf}:

\begin{verbatim}
    functionDefinition : signature ':' body

    signature : 'def' variable '(' optionalParameterList ')'

    body : block

    optionalParameterList : *EMPTY*
                          | parameterList
    
    parameterList : variable
                  | variable ',' parameterList
    
    block:  definition 
          | definition block
          | statement
          | statement block
\end{verbatim}

The first {\sc bnf} {\it rule} says that a function definition is
composed of two pieces, a signature and a body, separated
by the colon character
(parts of the rule that appear verbatim appear within single quotes).
The signature starts
with the keyword {\it def}
followed by a variable,
followed by an open parenthesis, followed by something
called an {\it optionalParameterList}, and finally followed by a close
parenthesis.
The body of a function 
something called a {\it block},
which is composed of {\it definitions} and {\it statements}.
The {\it optionalParameterList} rule tells us that
the list of formal parameters can possibly be empty,
but if not, is composed of a list of variables
separated by commas.

As we can see from the {\sc bnf} rules,
parameters are variables that will be bound
to the values supplied in the function call.
In the particular case of {\it isEven}, from the
previous section,
the variable {\it x} will be bound to the number whose
evenness is to be determined. As noted earlier,
it is customary to call {\it x}
a {\it formal parameter} of the function {\it isEven}.
In function calls, the values to be bound to the
formal parameters are called {\it arguments}.

\section{Function Objects}

Let's look more closely at the body of {\it isEven}:

\begin{verbatim}
    def isEven(x):
        return x % 2 == 0
\end{verbatim}

The \% operator is bound to the remainder or modulus
function. The {\tt ==} operator is bound to the equality function
and determines whether the value of the left operand
expression is equal to the value of the right operand
expression, yielding true or false as appropriate. The
{\sc boolean} value produced by {\tt ==} is then immediately returned as the
value of the function.

When given a function definition like that above, Scam
performs a couple of tasks. The first is to create the
internal form of the function, known as a {\it function object},
which holds the function's signature and body.
The second task is to bind
the function name to the function object so that it
can be called at a later time.
Thus, the name
of the function is simply a variable that happens to be
bound to a function object. As noted before, we often say
'the function {\it isEven}' even though we really mean 'the
function object bound to the variable {\it even}?'.

The value of a function definition is the
function object; you can see this by
printing out the value of {\it isEven}:

\begin{verbatim}
    >>> print(isEven)
    <function isEven at 0x9cbf2ac>

    >>> isEven = 4
    >>> print(isEven)
    4

\end{verbatim}

Further interactions with the interpreter provide evidence
that {\it isEven} is indeed a variable; we can reassign its value,
even though it is considered in poor form to do so.

\section{Calling Functions}

Once a function is created,
it is used by {\it calling} the
function with {\it arguments}.
A function is called by supplying
the name of the function followed by a parenthesized,
comma separated, list of expressions.
The arguments are
the values that the formal parameters will receive.
In Computer Science speak, we say that the values
of the arguments are to be bound to
the
formal parameters.
In general, if there are {\it n} formal parameters,
there should be {\it n} arguments.\footnote{
For {\it variadic} functions, which Scam
allows for, the number of arguments
may be more or less than the number of formal parameters.
}
Furthermore, the value of the
first argument is bound to the first formal parameter, the
second argument is bound to the second formal parameter,
and so on. Moreover, all the arguments are evaluated
before being bound to any of the parameters.

Once the evaluated arguments are bound to the parameters,
then the body of the function is evaluated. Most times,
the expressions in the body of the function will reference
the parameters. If so, how does the interpreter find the
values of those parameters? That question is answered in
the next chapter.

\section{Returning from functions}

The return value of a function is the value of the expression
following the {\tt return} keyword.
For a function to return this  expression, however,
the return has to be {\it reached}.
Look at this example:

\begin{verbatim}
def test(x,y):
    if (y == 0):
        return 0
    else:
        print("good value for y!")
        return x / y

    print("What?")
    return 1
\end{verbatim}

Note that the {\it ==} operator returns true if the
two operands have the same value.
In the function, if {\it y} is zero, then the
\begin{verbatim}
    return 0
\end{verbatim}

statement is reached.
This causes an
immediate return from the function and
no other expressions in the function body are evaluated.
The return value, in this case, is zero.
If {\it y} is not equal to zero,
a message is printed and 
the second return is reached, again causing
an immediate return. In this case,
a quotient is returned.

Since both parts of the if statement have
returns, then the last two lines of the
function:

\begin{verbatim}
    print("What?")
    return 1
\end{verbatim}

are {\it unreachable}. Since they
are unreachable, they cannot be
executed under any conditions and
thus serve no purpose and can be deleted.

%\chapter{Loops}
\label{Loops}

In the previous chapter, you learned how recursion can
solve a problem by breaking it in to smaller versions
of the same problem. Another approach is to use
{\it iterative} {\it loops}. In some programming
languages, loops are preferred as they use much
less computer memory as compared to recursions.
In other languages, this is not the case at all.
In general, there
is no reason to prefer recursions over loops or vice versa,
other than this memory issue.
Any loop can be written as a recursion and any recursion
can be written as a loop.
Use a recursion if that makes the implementation
more clear, otherwise, use an iterative loop.

The most
basic loop structure in Scam is the {\it while} loop, an example of
which is:

\begin{verbatim}
    while (i < 10):
        print(i,end="")
        i = i + 1
\end{verbatim}

We see a {\tt while} loop looks much like an {\tt if}
statement.
The difference
is that blocks belonging to {\tt if}s are evaluated at most once whereas
blocks associated with loops may be evaluated many many times.
Another difference in nomenclature is that the block of a loop
is known as the {\it body} (like blocks associated with function
definitions). Furthermore, the loop test expression is known
as the {\it loop condition}.

As Computer Scientists hate to type extra characters if they can
help it, you will often see:

\begin{verbatim}
    i = i + 1
\end{verbatim}

written as

\begin{verbatim}
    i += 1
\end{verbatim}

The latter version is read as ``increment {\it i}''.

A {\it while} loop tests its condition before the body of the loop is
executed. If the initial test fails, the body is not executed at all. For
example:

\begin{verbatim}
    i = 10
    while (i < 10):
        print(i,end="")
        i += 1
\end{verbatim}

never prints out anything since the test immediately fails. In this example,
however:

\begin{verbatim}
    i = 0;
    while (i < 10):
        print(i,end="")
        i += 1
\end{verbatim}

the loop prints out the digits 0 through 9:

\begin{verbatim}
    0123456789
\end{verbatim}
    
A {\tt while} loop repeatedly evaluates its body
as long as the loop condition remains true.

To write an infinite loop, use {\tt :true} as the condition:

\begin{verbatim}
    while (True):
        i = getInput()
        print("input is",i)
        process(i)
        }
\end{verbatim}

\section{Other loops}

There are many kinds of loops in Scam, in this text
we will only refer to {\tt while} loops and {\tt for}
loops that count,
as these are commonly found in other programming languages.
The {\tt while} loop we have seen; here is an example of a counting
{\tt for} loop:

\begin{verbatim}
    for i in range(0,10,1):
        print(i)
\end{verbatim}

This loop is exactly equivalent to:

\begin{verbatim}
    i = 0
    while (i < 10):
        print(i)
        i += 1
\end{verbatim}

In fact, a while loop of the general form:

\begin{verbatim}
    i = INIT
    while (i < LIMIT):
        # body
        ...
        i += STEP
\end{verbatim}

can be written as a \\verb!for! loop:

\begin{verbatim}
    for i in range(INIT,LIMIT,STEP):
        # body
        ...
\end{verbatim}

The {\it range} function counts from
{\tt INIT} to {\tt LIMIT} (non-inclusive)
by {\tt STEP} and these
values are assigned
to {\it i}, in turn. After each assignment to {\it i},
the loop body is evaluated.
After the last value is assigned to {\it i} and the
loop body evaluated on last time, the \\verb!for! loop ends.

In Scam, the {\it range} function assumes 1 for the step
if the step is omitted and assumes 0 for the initial
value and 1 for the step if both the initial value and
step are omitted.
However, in this text, we will always give the initial
value and step of the \\verb!for! loop explicitly.

For loops are commonly used to sweep through each element of an list:

\begin{verbatim}
     for i in range(0,len(items),1):
         print(items[i]) 
\end{verbatim}

Recall the items in a list of $n$ elements are located at
indices $0$ through $n - 1$. These are exactly the values
produced by the {\it range} function. So, this loop accesses
each element, by its index, in turn, and thus prints out
each element, in turn.
Since using an index of {\it n} in a list of {\it n} items produces an
error, the {\it range} function conveniently makes its given
limit non-inclusive.

As stated earlier, there are other kinds of loops in
Scam, some of which, at times, are more convenient
to use than a {\tt while} loop or a counting {\tt for} loop. However,
anything that can be done with those other loops can be
done with the loops presented here.
Like recursion and lists, loops and lists go very well
together.
The next sections detail some common loop patterns involving
lists.

\section{The {\it counting} pattern}

The counting pattern is used to count the number of items
in a collection. Note that the built-in function len already
does this for Scam lists, but many programming languages
do not have lists as part of the language; the programmer
must supply lists. For this example, assume that the {\it start}
function gets the first item in the given list,
returning {\tt None}
if there are no items in the list. The {\it next}
function returns the next item in the given list,
returning {\tt None}
if there are no more items. This {\tt while} loop counts the number
of items in the list:

\begin{verbatim}
    count = 0
    i = start(items)
    while (i != None)
         count += 1
         i = next(items)
\end{verbatim}

When the loop finishes, the variable count holds the number of
items in the list.

The counting pattern increments a counter everytime the loop
body is evaluated.

\section{The {\it filtered-count} pattern}

A variation on counting pattern involves filtering. When {\it filtering},
we use an {\tt if} statement to decide whether we should count an item
or not. Suppose we wish to count the number of even items in
a list:

\begin{verbatim}
    count = 0
    for i in range(0,len(items),1):
        if (items[i] % 2 == 0):
            count += 1
\end{verbatim}

When this loop terminates, the variable {\it count} will hold the
number of even integers in the list of items since the count
is incremented only when the item of interest is even.

\section{The {\it accumulate} pattern}

Similar to the counting pattern, the {\it accumulate} pattern
updates a variable, not by increasing its value by one, but by the value of
an item. This loop, sums all the values in a list:

\begin{verbatim}
    total = 0
    for i in range(0,len(items),1):
        total += items[i]
\end{verbatim}

By convention, the variable {\it total} is used to accumulate the item
values. When accumulating a sum, total is initialized to zero. When
accumulating a product, total is initialized to one.

\section{The {\it filtered-accumulate} pattern}

Similar to the {\it accumulate} pattern, the {\it filtered-accumulate} pattern
updates a variable only if some test is passed.
This function sums all the even values in a given list, returning
the final sum:

\begin{verbatim}
    def sumEvens(items):
        total = 0
        for i in range(0,len(items),1):
            if (items[i] % 2 == 0)
                total += items[i]
        return total
\end{verbatim}

As before, the variable {\it total} is used to accumulate the item
values. As with a regular accumulating, {\it total} is initialized to zero when
accumulating a sum. The initialization value is one when
accumulating a product and the initialization value is
the empty list when accumulating a list (see {\it filtering} below).

\section{The {\it search} pattern}

The {\it search} pattern is a slight variation of {\it filtered-counting}.
Suppose we wish to see if a value is present in a list. We can
use a filtered-counting approach and if the count is greater than
zero, we know that the item was indeed in the list.

\begin{verbatim}
    count = 0
    for i in range(0,len(items),1):
        if (items[i] == target):
            count += 1
    found = count > 0
\end{verbatim}

This pattern is so common, it is often encapsulated in a function.
Moreover, we can improve the efficiency by short-circuiting the
search. Once the target item is found, there is no need to 
search the remainder of the list:

\begin{verbatim}
    def find(target,items):
        found = False:
        i = 0
        while (not(found) and i < len(items)):
            if (items[i] == target):
                found = True
            i += 1
        return found
\end{verbatim}

We presume the target item is not in the list and 
as long as it is not found, we continue to search the list.
As soon as we find the item, we set the variable found
to True and then the loop condition fails, since
not true is false.

Experienced programmers would likely define this function
to use an immediate return once the target item is found:

\begin{verbatim}
    def find(target,items):
        for i in range(0,len(items),1):
            if (items[i] == target):
                return True
        return False
\end{verbatim}

As a beginning programmer, however, you should avoid returns
from the body of a loop. The reason is most beginners end up
defining the function this way instead:

\begin{verbatim}
    def find(target,items):
        for i in range(0,len(items),1):
            if (items[i] == target):
                return True
            else:
                return False
\end{verbatim}

The behavior of this latter version of {\it find} is incorrect,
but unfortunately, it appears to work correctly under some
conditions. If you cannot figure out why this version
fails under some conditions and appears to succeed under
others, you most definitely should stay away from placing
returns in loop bodies.
        
\section{The {\it filter} pattern}

Recall that a special case of a filtered-accumulation is the {\it filter}
pattern.
A loop version  of filter starts out by initializing an accumulator variable to
an empty list. In the loop body, the accumulator variable
gets updated with those items from the original list that 
pass some test.

Suppose we wish to extract the even numbers from a list.
Our test, then, is to see if the current element is even.
If so, we add it to our growing list:

\begin{verbatim}
    def extractEvens(items):
        evens = []
        for i in range(0,len(items),1):
            if (items[i] % 2 == 0):
                evens = evens + [items[i]]
        return evens
\end{verbatim}

Given a list of integers, {\it extractEvens} returns a (possibly empty)
list of the even numbers:

\begin{verbatim}
    >>> extractEvens([4,2,5,2,7,0,8,3,7])
    [4, 2, 2, 0, 8]

    >>> extractEvens([1,3,5,7,9])
    []
\end{verbatim}

\section{The {\it extreme} pattern}

Often, we wish to find the largest or smallest value
in a list. Here is one approach, which assumes that the
first item is the largest and then corrects that assumption
if need be:

\begin{verbatim}
    largest = items[0]
    for i in range(0,len(items),1):
        if (items[i] > largest):
            largest = items[i]
\end{verbatim}

When this loop terminates, the variable {\it largest} holds the
largest value. We can improve the loop slightly by noting
that the first time the loop body evaluates, we compare the
putative largest value against itself, which is a worthless
endeavor. To fix this, we can start the index variable {\it i}
at 1 instead:

\begin{verbatim}
    largest = items[0]
    for i in range(1,len(items),1): #start comparing at index 1
        if (items[i] > largest):
            largest = items[i]
\end{verbatim}

Novice programmers often make the mistake of initialing setting
{\it largest} to zero and then comparing all values against {\it largest},
as in:

\begin{verbatim}
    largest = 0
    for i in range(0,len(items),1):
        if (items[i] > largest):
            largest = items[i]
\end{verbatim}

This code appears to work in some cases, namely if the largest
value in the list is greater than or equal to zero. If not,
as is the case when all values in the list are negative,
the code produces an erroneous result of zero as the largest
value.

\section{The {\it extreme-index} pattern}

Sometimes, we wish to find the index of the most extreme
value in a list rather than the actual extreme value.
In such cases, we assume index zero holds the extreme value:

\begin{verbatim}
    ilargest = 0
    for i in range(1,len(items),1):
        if (items[i] > items[ilargest]):
            ilargest = i
\end{verbatim}

Here, we successively store the index of the largest value
see so far in the variable {\it ilargest}.

\section{The {\it shuffle} pattern}

Recall, the {\it shuffle} pattern from the previous chapter.
Instead of using recursion, we can use a version of
the loop accumulation pattern instead. As before,
let's assume the lists are exactly the same size:

\begin{verbatim}
    def shuffle(list1,list2):
        list3 = []
        for i in range(0,len(list1),1):
            list3 = list3 + [list1[i],list2[i]]
        return list3
\end{verbatim}

Note how we initialized the resulting list {\it list3} to
the empty list. Then, as we walked the first list, we
pulled elements from both lists, adding them into the
resulting list.

When we have walked past the end of {\it list1} is empty,
we know we have also walked past the end of {\it list2}, since the
two lists have the same size.

If the incoming lists do not have the same length,
life gets more complicated:

\begin{verbatim}
    def shuffle2(list1,list2):
        list3 = []
        if (len(list1) < len(list2)):
            for i in range(0,len(list1),1):
                list3 = list3 + [list1[i],list2[i]]
            return list3 + list2[i:]
        else:
            for i in range(0,len(list2),1):
                list3 = list3 + [list1[i],list2[i]]
            return list3 + list1[i:]
\end{verbatim}

We can also use a {\it while} loop that goes until one of the lists
is empty. This has the effect of removing the redundant code
in {\it shuffle2}:

\begin{verbatim}
    def shuffle3(list1,list2):
        list3 = []
        i = 0
        while (i < len(list2) and i < len(list2)):
            list3 = [list1[i],list2[i]]
            i = i + 1
        ...
\end{verbatim}

When the loop ends, one or both of the lists have been exhausted,
but we don't know which one or ones. A simple solution is to
add both remainders to {\it list3} and return.

\begin{verbatim}
    def shuffle3(list1,list2):
        list3 = []
        i = 0
        while (i < len(list2) and i < len(list2)):
            list3 = [list1[i],list2[i]]
            i = i + 1
        return list3 + list1[i:] + list2[i:]
\end{verbatim}

Suppose {\it list1} is empty. Then the expression {\tt list1[i:]} will
generate the empty list. Adding the empty list to {\it list3} will
have no effect, as desired. The same is true if {\it list2}
(or both {\it list1} and {\it list2} are empty).

\section{The {\it merge} pattern}

We can also merge using a loop. Suppose we have two
ordered lists (we will assume increasing order)
and we wish to merge them into one ordered
list. We start by keeping two index variables,
one pointing to the smallest element in {\it list1} and
one pointing to the smallest element in {\it list2}.
Since the lists are ordered, we know the that the smallest
elements are at the head of the lists:

\begin{verbatim}
   i = 0  # index variable for list1
   j = 0  # index variable for list2
\end{verbatim}

Now, we loop, similar to {\it shuffle3}:

\begin{verbatim}
    while (i < len(list1) and j < len(list2)):
\end{verbatim}

Inside the loop, we test to see if the smallest element
in list1 is smaller than the smallest element in list2:

\begin{verbatim}
        if (list1[i] < list2[j]):
\end{verbatim}

If it is, we add the element from {\it list1} to {\it list3} and increase
the index variable {\it i} for {\it list1} since we have `used up' the value
at index {\it i}.

\begin{verbatim}
            list3 = list3 + [list1[i]]
            i = i + 1
\end{verbatim}

Otherwise, {\it list2} must have the smaller element and we do likewise:

\begin{verbatim}
            list3 = list3 + [list2[j]]
            j = j + 1
\end{verbatim}

Finally, when the loop ends ({\it i} or {\it j} has gotten too large),
we add the remainders of both lists to {\it list3} and return:

\begin{verbatim}
    return list3 + list1[i:] + list2[j:]
\end{verbatim}

In the case of merging, one of the lists will be exhausted
and the other will not. As with shuffle3, we really don't
care which list was exhausted.

Putting it all together yields:

\begin{verbatim}
    def merge(list1,list2):
        list3 = []
        i = 0
        j = 0
        while (i < len(list1) and j < len(list2)):
            if (list1[i] < list2[j]):
                list3 = list3 + [list1[i]]
                i = i + 1
            else:
                list3 = list3 + [list2[j]]
                j = j + 1
        return list3 + list1[i:] + list2[j:]
\end{verbatim}

\section{The {\it fossilized} Pattern}

Sometimes, a loop is so ill-specified that it never ends. This
is known as an {\it infinite loop}. Of the
two loops we are investigating, the {\it while} loop is the most
susceptible to infinite loop errors. One common mistake
is the {\it fossilized} pattern, in which the index variable never
changes so that the loop condition never becomes false:

\begin{verbatim}
    i = 0
    while (i < n):
        print(i)
\end{verbatim}

This loop keeps printing until you terminate the program with
prejudice. The reason is that {\it i} never changes; presumably a
statement to increment {\it i} at the bottom of the loop body has
been omitted.

\section{The {\it missed-condition} pattern}

Related to the bottomless pattern of recursive functions
is the missed condition pattern of loops.
With missed condition, the index variable is updated, but
it is updated in such a way that the loop condition is
never evaluates to false.

\begin{verbatim}
    i = n
    while (i > 0):
        print(i)
        i += 1
\end{verbatim}

Here, the index variable {\it i} needs to be decremented rather than
incremented. If {\it i} has an initial value greater than zero,
the increment pushes {\it i} further and further above zero.
Thus, the loop condition never fails and the loop
becomes infinite.

%\chapter{Comparing Recursion and Looping}
\label{RecursionLoop}

In the previous two chapters, we learned about repeatedly
evaluating the same code using both recursion and loops.
Now we compare and contrast the two techniques by
implementing the three mathematical functions from
\link*{the chapter on assignment}[Chapter~\Ref]{Recursion}:
{\it factorial}, {\it fibonacci}, and {\it gcd}, with loops.

\section{Factorial}

Recall that the factorial function, written recursively,
looks like this:

\begin{verbatim}
    def factorial(n):
        if (n == 0):
            return 1
        else:
            return n * factorial(n - 1)
\end{verbatim}

We see that is a form of the {\it accumulate} pattern. So our factorial
function using a loop should look something like this:

\begin{verbatim}
    def factorial(n):
        total = ???
        for i in range(???):
            total *= ???
        return total
\end{verbatim}

Since we are accumulating a product, total should
be initialized to 1.

\begin{verbatim}
    def factorial(n):
        total = 1
        for i in range(???):
            total *= ???
        return total
\end{verbatim}

Also, the loop variable should take on all values in
the factorial, from 1 to {\it n}:

\begin{verbatim}
    def factorial(n):
        total = 1
        for i in range(1,n+1,1):
            total *= ???
        return total
\end{verbatim}

Finally, we accumulate {\it i} into the total:

\begin{verbatim}
    def factorial(n):
        total = 1
        for i in range(1,n+1,1):
            total *= i
        return total
\end{verbatim}

The second argument to range is set to $n+1$ instead of $n$ because
we want $n$ to be included in the total.

Now, compare the loop version to the recursive version. Both contain
about the same amount of code, but the recursive version is easier
to ascertain as correct.

\section{The greatest common divisor }

Here is a slightly different version of the gcd function, built using
the following recurrence:

\begin{center}
\begin{tabular}{lcll}
\T\toprule
    $gcd$($a$,$b$) & is & $a$ & if $b$ is zero\\
    $gcd$($a$,$b$) & is & $gcd$($b$,$a$ \% $b$) & otherwise \T\\
\T\bottomrule
\end{tabular}
\end{center}

The function allows one more recursive call than the other. By doing
so, we eliminate the need for the local variable {\it remainder}. Here is
the implementation:

\begin{verbatim}
    def gcd(a,b):
        if (b == 0):
            return a
        else:
            return gcd(b,a % b)
\end{verbatim}

Let's turn it into a looping function. This style of
recursion doesn't fit any of the patterns we know, so
we'll have to start from scratch. We do know that 
{\it b} becomes the new value of {\it a} and {\it a} \% {\it b} becomes 
the new value of {\it b}
on every recursive call,
so the same thing must happen on every evaluation of
the loop body.
We stop when {\it b} is equal to zero so we should continue looping
while {\it b} is  not equal to zero. These observations lead us
to this implementation:

\begin{verbatim}
    def gcd(a,b):
        while (b != 0):
            a = b
            b = a % b
        return a
\end{verbatim}

Unfortunately, this implementation is faulty, since we've lost
the original value of {\it a} by the time we perform the modulus 
operation. Reversing the two statements in the body of the loop:

\begin{verbatim}
    def gcd(a,b):
        while (b != 0):
            b = a % b
            a = b
        return a
\end{verbatim}

is no better; we lose the original value of {\it b} by the time we
assign it to {\it a}. What we need to do is temporarily save the
original value of {\it b} before we assign {\it a}'s value. Then
we can assign the saved value to {\it a} after {\it b} has been reassigned:

\begin{verbatim}
    def gcd(a,b):
        while (b != 0):
            temp = b
            b = a % b
            a = temp
        return a
\end{verbatim}

Now the function is working correctly. But why did we temporarily
need to save a value in the loop version and not in the recursive
version? The answer is that the recursive call does not perform
any assignments so no values were lost. On the recursive call,
new versions of the formal parameters {\it a} and {\it b} received the
computations performed for the function call. The old versions
were left untouched.

It should be noted that Scam allows simultaneous assignment that
obviates the need for the temporary variable:

\begin{verbatim}
    def gcd(a,b):
        while (b != 0):
            a,b = b,a % b
        return a
\end{verbatim}

While this code is much shorter, it is a little more difficult to
read. Moreover, other common languages do not share this feature
and you are left using a temporary variable to preserve needed
values when using those languages.

\section{The Fibonacci sequence}

Recall the recursive implementation of Fibonacci:

\begin{verbatim}
    def fib(n):
        if (n < 2)
            return n
        else
            return fib(n - 1) + fib(n - 2)
\end{verbatim}

For brevity, we have collapsed the two base cases into
a single base case. If {\it n} is zero, zero is returned and if
{\it n} is one, one is returned, as before.

Let's So let's try to compute
using an iterative loop. As before, this doesn't seem
to fit a pattern, so we start by reasoning about this.
If we let {\it a} be the first Fibonacci number, zero, and {\it b}
be the second Fibonacci number, one, then the third fibonacci
number would be $a + b$, which we can save in a variable 
named {\it c}.
At  this point, the fourth Fibonacci number would be $b + c$,
but since we are using a loop, we need to have the code be
the same for each iteration of the loop. If we let $a$ have the
value of $b$ and $b$ have the value of $c$, then the fourth Fibonacci
number would be $a + b$ again.
This leads to our implementation:

\begin{verbatim}
    def fib(n):
        a = 0    # the first Fibonacci number
        b = 1    # the second Fibonacci number
        for i in range(0,n,1):
            c = a + b
            a = b
            b = c
        return a
\end{verbatim}

In the loop body, we see that {\it fib} is much like {\it gcd};
the second number becomes the first number and some combination of
the first and second number becomes the second number.
In the case of {\it gcd}, the combination was the remainder and, in the
case of {\it fib}, the combination is sum.
A rather large question remains, why does the function return {\it a}
instead of {\it b} or {\it c}? The reason is, suppose {\it fib} was
called with a value of 0, which is supposed to generate
the first Fibonacci number. The loop does not run in this case
and the value of {\it a} is returned, zero, as required.
If a value of 1 is passed to {\it fib}, then the loop runs exactly
once and {\it a} gets the original value of {\it b}, one. The loop expects and
this time, one is returned, as required. So, empirically, it
appears that the value of a is the correct choice of return value.
As with factorial, hitting on the right way to proceed iteratively
is not exactly straightforward, while the recursive version
practically wrote itself.

\section{CHALLENGE: Transforming loops into recursions}

To transform an iterative loop into a
recursive loop, one first identifies those variables
that exist outside the loop but are changing in the loop body;
these variable will become formal parameters in the recursive
function.
For example, the {\it fib} loop above has three (not two!)
variables that
are being changed during each iteration of the loop:
{\it a}, {\it b}, and {\it i}.\footnote{The loop variable
is considered an outside variable changed by the loop.} 
The variable {\it c} is used only inside the loop and thus is
ignored.

Given this, we start out
our recursive function like so:

\begin{verbatim}
    def loop(a,b,i):
        ...
\end{verbatim}

The loop test becomes an {\it if} test in the body of
the {\it loop} function:

\begin{verbatim}
    def loop(a,b,i)
        if (i < n):
            ...
        else:
            ...
\end{verbatim}

The {\it if-true} block becomes the recursive call.
The arguments to the recursive call encode the updates
to the loop variables 
The {\it if-false} block becomes the value the loop attempted to
calculate:

\begin{verbatim}
    def loop(a,b,i):
        if (i < n):
            return loop(b,a + b,i + 1)
        else:
            return a
\end{verbatim}

Remember, a gets the value of b and b gets the value of
{\it c} which is $a + b$. Since we are performing recursion
with no assignments, we don't need the variable {\it c} anymore.
The loop variable {\it i} is incremented
by one each time.

Next, we replace the loop with the the {\it loop} function in the
function
containing the original loop. That way, any non-local variables
referenced in the test or body of the original loop will
be visible to the {\it loop} function:

\begin{verbatim}
    def fib(n):
        def (a,b,i):
            if (i < n)
                return loop(b,a + b,i + 1)
            else:
                return a
        ...
\end{verbatim}

Finally, we call the {\it loop} function with the initial
values of the loop variables:

\begin{verbatim}
    def fib(n):
        def (a,b,i):
            if (i < n)
                return loop(b,a + b,i + 1)
            else:
                return a
        return loop(0,1,0)
\end{verbatim}

Note that this recursive function looks nothing like our
original {\it fib}. However, it suffers from none of the inefficiencies
of the original version and yet it performs no assignments.\footnote{
A style of programming that uses no assignments is called {\it functional}
programming and is very important in theorizing about the nature
of computation.} The reason for its efficiency is that it performs
that exact calculations and number of calculations as the
loop based function.

For more practice, let's convert the iterative version of
{\it factorial} into a recursive function using this method.
We'll again end up with a different recursive function
than before. For convenience, here is the loop version:

\begin{verbatim}
    def fact(n):
        total = 1
        for i in range(1,n+1,1):
            total *= i
        return total
\end{verbatim}

We start, as before, by working on the {\it loop} function.
In this case,
only two variables are changing in the loop:
{\it total} and {\it i}.

\begin{verbatim}
    def loop(total,i):
        ...
\end{verbatim}

Next, we write the {\it if} expression:

\begin{verbatim}
    def loop(total,i):
        if (i < n + 1):
            return loop(total * i,i + 1)
        else:
            return total
\end{verbatim}

Next, we embed the {\it loop} function and call it:

\begin{verbatim}
    def fact(n):
        def loop(total,i):
            if (i < n + 1):
                return loop(total * i,i + 1)
            else:
                return total
        return loop(1,1)
\end{verbatim}

The moral of this story is that any iterative loop can be rewritten
as a recursion and any recursion can be rewritten as
an iterative loop. Moreover, in {\it good} languages,\footnote{
Scam is one of these good languages!}
there is no reason to prefer one way over the other,
either in terms of the time it takes or the space used
in execution. To reiterate,
use a recursion if that makes the implementation
more clear, otherwise, use an iterative loop.

%\chapter{More on Input}
\label{MoreOnInput}

Now that we have learned how to loop, we can perform
more sophisticated types of input.

\section{Converting command line arguments en mass}

Suppose all the command-line arguments are numbers that
need to be converted from their string versions stored
in {\it sys.argv}.
We can use a loop and the accumulate pattern to accumulate
the converted string elements:

\begin{verbatim}
    def convertArgsToNumbers():
        total = []
        # start at 1 to skip over program file name
        for i in range(1,len(sys.argv),1):
            num = eval(sys.argv[i])
            total = total + [num]
        return total
\end{verbatim}

The accumulator, total, starts out as the empty list. For each
element of sys.argv beyond the program file name, we convert
it and store the result in num. We then turn that number
into a list (by enclosing it in brackets) and then add it
to the growing list.

With a program file named {\it convert.py} as follows:

\begin{verbatim}
    import sys

    def main():
        ints = convertArgsToNumbers()
        print("original args are",sys.argv[1:])
        print("converted args are",ints)

    def convertArgsToNumbers():
        ...

    main()
\end{verbatim}

we get the following behavior:

\begin{verbatim}
   $ python convert.py 1 34 -2
   original args are ['1', '34', '-2']
   converted args are [1, 34, -2]
\end{verbatim}

Note the absence of quotation marks in the converted list,
signifying that the elements are indeed numbers.

\section{Reading individual items from files}

Instead of reading all of the file at once using the {\it read} function,
we can read it one item at a time. When we read an
item at a time, we always follow this pattern:

\begin{verbatim}
    open the file
    read the first item
    while the read was good
        process the item
        read the next item
    close the file
\end{verbatim}

In Scam, we tell if the read was good by checking the
value of the variable that points to the value read.
Usually, the empty string is used to indicate the
read failed.

\section*{Processing files a line at a time}

Here is another version of the {\it copyFile} function from
\link*{the chapter on input and output}[Chapter~\Ref]{InputAndOutput}.
This version reads and writes
one line at a time. In addition, the function returns
the number of lines processed:

\begin{verbatim}
    def copyFile(inFile,outFile):
        in = open(inFile,"r")
        out = open(outFile,"w")
        count = 0
        line = in.readline()
        while (line != ""):
            out.write(line)
            count += 1
            line = in.readline()
        in.close()
        out.close()
        return count
\end{verbatim}
            
Notice we used the counting pattern.

\section*{Using a Scanner}

A scanner is a reading subsystem that allows you
to read whitespace-delimited tokens from a file.
To get a scanner for Scam, issue this command:

\begin{verbatim}
    wget beastie.cs.ua.edu/cs150/projects/scanner.py
\end{verbatim}

To use a scanner, you will need to import it into your program:

\begin{verbatim}
   from scanner import *
\end{verbatim}

Typically, a scanner is used with a loop. Suppose we wish to count the
number of 
short tokens (a token is a series of characters surrounded by empty space)
in a file. Let's assume a short token is one whose length is less
than or equal to some limit.
Here is a loop that does that:

\begin{verbatim}
    def countShortTokens(fileName):
        s = Scanner(fileName)              #create the scanner
        count = 0
        token = s.readtoken()              #read the first token
        while token != "":                 #check if the read was good
            if (len(token) <= SHORT_LIMIT):
                count += 1
            token = s.readtoken()          #read the next token
        s.close()                          #always close the scanner when done
        return count
\end{verbatim}

Note that the use of the scanner follows the standard reading pattern:
opening (creating the scanner),
making the first read, testing if the read was good, processing
the item read (by counting it), reading the next item, and finally
closing the file (by closing the scanner) after the loop terminates.
Using a scanner always means performing the five steps as given
in the comments.
This code also incorporates the filtered-counting pattern, as expected.

\section{Reading Tokens into a List}

Note that the {\it countShortTokens} function is doing two things, reading
the tokens and also counting the number of short tokens. It is said that this
function has two {\it concerns}, reading and counting. A
fundamental principle of Computer Science is {\it separation of
concerns}.
To separate the concerns, we have one function read the tokens,
storing them into a list (reading and storing is considered to
be a single concern).
We then have another function count the tokens. Thus, we will
have separated the two concerns into separate functions, each
with its own concern.
Here is the reading (and storing) function, which implements
the accumulation pattern:

\begin{verbatim}
    def readTokens(fileName):
        s = Scanner(fileName)              #create the scanner
        items = []
        token = s.readtoken()              #read the first token
        while token != "":                 #check if the read was good
            items = items + [token]
            token = s.readtoken()          #read the next token
        s.close()                          #always close the scanner when done
        return items
\end{verbatim}

Next, we implement the filtered-counting function. Instead of passing
the file name, as before, we pass the list of tokens that
were read:

\begin{verbatim}
    def  countTokens(items):
        count = 0
        for i in range(0,len(items),1)
            if (len(items[i]) <= SHORT_LIMIT):
                count += 1
        return count
\end{verbatim}

Each function is now simpler than the original function. This makes
it easier to fix any errors in a function since you can concentrate
on the single concern implemented by that function.

\section{Reading Records into a List}

Often, data in a file is organized as {\it records}, where
a record is just a collection of consecutive tokens.
Each token in a record is known as a {\it field}.
Suppose every four tokens in a file comprises a record:

\begin{verbatim}
    Smith    President 32  87000
    Jones    Assistant 15  99000
    Thompson    Hacker  2 147000
\end{verbatim}

Typically, we define a function to read one collection of tokens
at a time.
Here is a function that reads a single record:

\begin{verbatim}
    def readRecord(s):                   # we pass the scanner in
        name = s.readtoken()
        if name == "":
            return None                  # no record, returning None
        title = s.readtoken()
        service = eval(s.readtoken())
        salary = eval(s.readtoken())
        return [name,title,service,salary]
\end{verbatim}

Note that we return either a record as a list or None if
no record was read.
Since years of service and salary are numbers, we
convert them appropriately with {\it eval}.

To total up all the salaries, for example, we can use an accumulation
loop (assuming the salary data resides in a file named
{\it salaries}).
We do so by repeatedly calling {\it readrecord}:

\begin{verbatim}
    function totalPay(fileName):
        s = Scanner(fileName)
        total = 0
        record = readRecord(s)
        while (record != None):
            total += record[3]
            record = readRecord(s)
        s.close()            
        print("total salaries:",total)
\end{verbatim}

Note that it is the job of the caller of {\it readRecord} to
create the scanner, repeatedly send the scanner to
{\it readRecord}, and close
the scanner when done.
Also note that we tell if the read was good by checking
to see if {\it readRecord} return {\tt None}.

The above function has two stylistic flaws. It uses those
magic numbers we read about in 
\link*{the chapter on assignment}[Chapter~\Ref]{Assignment}.
It is not clear from the code that the field at index
three is the salary.
To make the code more readable, we can set up some ``constants''
in the global scope (so that they will be visible everywhere):
The second issue is that
that the function has two concerns (reading and accumulating).
We will fix the magic number problem first.

\begin{verbatim}
    NAME = 0
    TITLE = 1
    SERVICE = 2
    SALARY = 3
\end{verbatim}

Our accumulation loop now becomes:

\begin{verbatim}
    total = 0
    record = readRecord(s)
    while record != None:
        total += record[SALARY]
        record = readRecord(s)
\end{verbatim}

We can also rewrite our {\it readRecord} function so
that it only needs to know the number of fields:

\begin{verbatim}
    def readRecord(s):                   # we pass the scanner in
        name = s.readtoken()
        if name == "":
            return None                  # no record, returning None
        title = s.readtoken()
        service = eval(s.readtoken())
        salary = eval(s.readtoken())

        # create an empty record

        result = [0,0,0,0]               

        # fill out the elements

        result[NAME] = name
        result[TITLE] = title
        result[SERVICE] = service
        result[SALARY] = salary

        return result
\end{verbatim}

Even if someone changes the constants to:

\begin{verbatim}
    NAME = 3
    TITLE = 2
    SERVICE = 1
    SALARY = 0
\end{verbatim}

The code still works correctly. Now, however, the salary resides
at index 0, but the accumulation loop is still accumulating the
salary due to its use of the constant to access the salary.

\section{Creating a List of Records}

We can separate the two concerns of the {\it totalPay} function
by having one function read the records into a list
and having another total up the salaries.
A list of
records is known as a {\it table}.
Creating the table
is just like accumulating the salary, but instead
we accumulate the entire record into a list:

\begin{verbatim}
    def readTable(fileName):
        s = Scanner(fileName)
        table = []
        record = readRecord(s)
        while record != None:
            table += [record]       #brackets around record!
            record = readRecord(s)
        s.close()            
\end{verbatim}

Now the table holds all the records in the file.
We must remember to enclose the record in square brackets 
before we accumulate it into the growing table. The
superior student will try this code without the brackets
and ascertain the difference.

The accumulation function is straightforward:

\begin{verbatim}
    def totalPay(fileName):
        table = readTable(fileName)
        total = 0
        for i in range(0,len(table),1):
            record = table[i]
            total += record[SALARY]
        return total
\end{verbatim}

We can simply this function by removing the temporary variable
{\it record}:

\begin{verbatim}
    def totalPay(fileName):
        table = readTable(fileName)
        total = 0
        for i in range(0,len(table),1):
            total += table[i][SALARY]
        return total
\end{verbatim}

Since a table is just a list, so we can walk it, accumulate
items in each record (as we just did with salary), filter it and so on.

\section{Other Scanner Methods}

A scanner object has other methods for reading. They are

\begin{description}
\item[{\tt readline()}]
    read a line from a file, like Scam's {\it readline}.
\item[{\tt readchar()}]
    read the next non-whitespace character
\item[{\tt readrawchar()}]
    read the next character, whitespace or no
\item[{\tt readstring()}]
    read a string - if a string is not pending, '' is returned
\item[{\tt readint()}]
    read an integer - if an integer is not pending, '' is returned
\item[{\tt readfloat()}]
    read a floating point number - if an float is not pending, '' is returned
\end{description}

You can also use a scanner to read from the keyboard. Simply
pass an empty string as the file name:

\begin{verbatim}
    s = Scanner("")
\end{verbatim}

You can scan tokens and such from a string as well by first
creating a keyboard scanner, and then setting the input
to the string you wish to scan:

\begin{verbatim}
    s = Scanner("")
    s.fromstring(str)
\end{verbatim}

%\chapter{Arrays}
\label{Arrays}

In this chapter, we will study {\it arrays}, a device
used for bringing related bits of data together underneath one
roof, so to speak. Such a device is known as a data structure;
an array is one of the most basic of data structures.

Unfortunately, arrays are not built in to Scam and
third party array modules are practically or semantically 
deficient. Therefore,
this chapter is based on a custom array module you can
download (sorry, Linux only).

\section{Getting the array and list modules}

If you are running Linux on a 32-bit Intel machine,
run these commands to download the modules:

\begin{verbatim}
    wget beastie.cs.ua.edu/cs150/array-i386/nakarray.so
\end{verbatim}

If you are running Linux on a 64-bit machine, use this
command instead:

\begin{verbatim}
    wget beastie.cs.ua.edu/cs150/array-amd64/nakarray.so
\end{verbatim}

Place the {\it nakarray.so} file
in the same directory as
your program that uses arrays. One would
import the array module
with the following line:

\begin{verbatim}
    from nakarray import *
\end{verbatim}

\section{Arrays as data structures}

A {\it data}
{\it structure} is simply a collection of
bits of information\footnote{
Bits in the informal sense, not zeros and ones.
}
that are
somehow glued together into a single whole. Each of these bits
can be be accessed individually. Usually, the bits are somehow
related, so the data structure is a convenient way
to keep all these related bits nicely packaged together.

At its most basic, a data structure supports the following actions:

\begin{itemize}
\item
       creating the structure
\item
       putting data into the structure
\item
       taking data out of the structure
\end{itemize}

We will study these actions with an eye to how long each
action is expected to take.

An array is a data structure with the property that each
individual piece of data can be accessed just as quickly as any of the
others. In other words, the process of putting data in and
taking data out takes a constant amount of time regardless of
the size of the array and where the data being accessed is
placed in the array\footnote{
Not all data structures have this
property (but they do have other advantages over arrays.}.

\section{Array creation}

To create an array, one uses the {\it Array} function:

\begin{verbatim}
    a = Array(10)
\end{verbatim}

The function call \verb!Array(10)!
creates an array with room for ten items or {\it elements}.
The variable {\it a} is created and set to point to
the array.

In many languages, array creation can take anywhere from {\it constant} time
(if elements are not initialized) or {\it linear} time (if they are).
What do we mean by constant time? We mean, in this particular
case, the amount of time it takes to create an array of 10 elements
takes the same amount of time it takes to create an array of 
1,000,000 elements. In other words, the time it takes to create an
array is independent of the size of the array.
By the same token, linear time means the time it takes to create
an array is proportional to the size of the array. If array creation
takes linear time, we would expect the time it takes to create
a 1,000,000 element array would be, roughly, 100,000,000 times
longer than the time to create a 10 element array. 

At the end of the chapter, we will attempt to figure out whether
array creation takes constant or linear time.

\section{Setting and getting array elements}

Most data structures allow you to both add and remove data.
For arrays, adding data means to {\it set} the value of an array element,
while removing data means to {\it get} the value of an array element.
It does not mean that elements themselves are being created and
destroyed (although this can be the case in other data structures).

To set an individual element, one uses {\it square bracket notation}.
For example, here's how to set the
first item in the array

\begin{verbatim}
    >>> a[0] = 42
\end{verbatim}

The number between the square brackets is known as an
{\it index}.
As with Scam lists, note that the index of the first element is zero.
This means
that the index
of the second element is one, and so on.
In other words, zero-based counting is used to
number array elements.

Once an element has been set, you can retrieve it from the array
using a similar notation. The following code illustrates the
retrieval of the first (index zero) element.

\begin{verbatim}
    >>> e = a[0]

    >>> e
    42
\end{verbatim}

Of course, there is no need to assign the value of an array
element to a variable; you can access it directly:

\begin{verbatim}
    >>> a[0]
    42
\end{verbatim}

The elements of an array are initialized to \verb!None!, so if you
access an element that hasn't been set, you get \verb!None! as a
result:

\begin{verbatim}
    >>> print(a[1])
    None

    >>>print(a)
    [42, None, None, None, None, None, None, None, None, None]
\end{verbatim}

To find out the number of elements in an array, you can use
the {\it len} function:

\begin{verbatim}
    >>> len(a)
    10
\end{verbatim}

Because of zero-based counting, the index of the last element
is always one less the number of elements in the array. For
a ten element array then, the last legal index is nine:

\begin{verbatim}
    >>> a[9] = 13

    >>>print(a)
    [42, None, None, None, None, None, None, None, None, 13]
\end{verbatim}

Trying to access an element beyond the $9^{th}$ results in
an error:

\begin{verbatim}
    >>> a[10] = 99
    Traceback (most recent call last):
      File "<stdin>", line 1, in <module>
      IndexError: Index '10' is out of range.
\end{verbatim}

\section{Limitations on our arrays}

Unlike built-in Scam strings and lists, the {\it nakarray} module
does not allow:

\begin{itemize}
\item
        `slicing' an array.
\item
        negative indices
\item
        concatenation two arrays using the plus operator
\end{itemize}

The reasons for these restrictions is that other common languages
do not support these features. In those languages, you need
to implement these features yourself. The superior student
will attempt to add these features by defining special purpose
functions for our arrays.

\section{Advantages of our arrays}

The array module that comes with the
Scam distribution has two major disadvantages:

\begin{itemize}
\item
    you cannot create an array of a given size - you must
    instead create an empty array and then repeatedly append
    elements to the array.
\item
    you must specify the type of element to  be stored in
    the array at array creation time - such arrays are known
    as {\it homogeneous} arrays.
\end{itemize}
        
As seen in the previous sections,
arrays made with the {it nakarray} can
be made of any size quite easily.
As to the second disadvantage,
we feel that arrays, 
like Scam lists, should be
{\it heterogeneous}.
Homogeneous, in Computer Science speak, means ``all of one type''
so a homogeneous array can only hold all integers or all strings,
but not both. Heterogeneous arrays can hold a mixture of types.

\section{The utility of arrays}

Due to their constant time properties, arrays are
a useful data structure for many applications involving
large amounts of data. One of the most useful applications
is searching for presence of certain elements in an array.
The next chapter discusses how arrays can make this
search process quite efficient.

%\chapter{Sorting}
\label{Sorting}

\section*{Sorted Tables}

Recall that a table is a list of records where
each record is a list of the fields incorporating
the record.

Sometimes, you need to merge two sorted tables into
one table that remains sorted. First, you have
to decide which field is used for the
sorting. In our example, the records in the data file
could be sorted on NAME or on SALARY or any other
field.

Suppose we had two data files that are sorted on SALARY,
{\it salaries.1} and {\it salaries.2}.  We wish to merge the data
in both files, printing  out the merged data, again in
sorted order.

First, we need to read the data into tables:

\begin{verbatim}
    table1 = readTable("salaries.1")
    table2 = readTable("salaries.2")
\end{verbatim}

Our strategy is to compare the first unaccumulated
record in {\it table1} to the first unaccumulated record
in {\it table2}. Let's call these records {\it r1} and {\it r2}, respectively.
If the salary of {\it r1} is less than that of {\it r2}, we accumulate
{\it r1}. Otherwise we accumulate {\it r2}.
We will repeat this process using a loop.

It is clear we need two variables, the first points to
the index of the first unaccumulated record in the table1,
while the second variable points to the first unaccumulated
record in the second table.
We start out both variables at zero, meaning no records
have been accumulated yet:

\begin{verbatim}
    index1 = 0
    index2 = 0
\end{verbatim}

How do we know when to stop accumulating? When we
run out of records to compare. This happens
when {\it index1} has passed
the index of the last record in {\it table1} or {\it index2} has passed the
index of the last record in {\it table2}.
We reverse that logic for a while loop, because it runs
while the test condition is true. The reversed logic is
``as long as index1 has not passed the {\it index} of the last record
in {\it table1} AND {\it index2} has not passed the index of the last
record in {\it table2}''.

\begin{verbatim}
    total = []
    while (index1 < len(table1) and index2 < len(table2):
        r1 = table1[index1]
        r2 = table2[index2]
        ...
\end{verbatim}

We also must advance {\it index1} and {\it index2} to that the loop will
finally end. When do we advance {\it index1}? When we accumulate
a record from {\it table1}. When do we advance {\it index2}? Likewise,
when we accumulate a record from {\it table2}.


\begin{verbatim}
    total = []
    while (index1 < len(table1) and index2 < len(table2):
        r1 = table1[index1]
        r2 = table2[index2]
        if (r1[SALARY] < r2[SALARY]):
           total = total + [r1]
           index1 += 1
        else:
           total = total + [r2]
           index2 += 1
\end{verbatim}

When will this loop end? When one of the indices gets too high\footnote{
Only one will be too high. Why is that?}.
This means we will have accumulated all the
records from one of the tables, but we don't know which one.
So, we add two more loops to accumulate any left over records:

\begin{verbatim}
    for i in range(index1,len(table1),1):
        total = total + [table1[i]]

    for i in range(index2,len(table2),1):
        total = total + [table2[i]]
\end{verbatim}

Finally, we encapsulate all of our merging code into
a function, passing in the index of the field that was
used to sort the data. This field is
known as the {\it key}:

\begin{verbatim}
    def merge(table1,table2,key):
        total = []
        while (index1 < len(table1) and index2 < len(table2):
            r1 = table1[index1]
            r2 = table2[index2]
            if (r1[key] < r2[key]):
               total = total + [r1]
               index1 += 1
            else:
               total = total + [r2]
               index2 += 1

        for i in range(index1,len(table1),1):
            total = total + [table1[i]]

        for i in range(index2,len(table2),1):
            total = total + [table2[i]]
\end{verbatim}

Finally, we define a main function to tie it all together:

\begin{verbatim}
    def main():
        table1 = readTable("salaries.1")
        table2 = readTable("salaries.2")
        mergedTable = merge(table1,table2,SALARY) #SALARY is the key
        printTable(mergedTable)
\end{verbatim}

Notice how the main function follows the standard main
pattern:

\begin{itemize}
\item
        get the data
\item
        process the data
\item
        write the result
\end{itemize}

\section{Merge sort}

%\include{ch19}
%\include{ch20}
%\include{ch21}
%\include{ch22}

\T\printindex
\W\chapter*{Index}
\W\htmlprintindex
\end{document}
