\chapter{Input and Output}
\label{InputAndOutput}

Scam uses a {\it port} system for input and output.
\index{port}
When Scam starts up, the current input port defaults to
{\it stdin} (the keyboard) and the current output
\index{stdin}
port defaults to {\it stdout} (the screen).
\index{stdout}

To change these ports, one first creates new port
and then sets the port.
For example, to read from a file (say "data")
instead of the keyboard,
first create a file port:

\begin{verbatim}
    (define p (open "data" 'read))   ; p points to a port
    (define oldInput (setPort p))
    ...                              ; read stuff from the file data
    (setPort oldInput)               ; restore the old input port
\end{verbatim}

Once the port is set, all input will come from the new port.
The {\it setPort} function, in addition to setting the port, returns
the old port so that it eventually can be restored, if needed.

To change the output port, the procedure is similar, except
the symbol \verb!'write! is used instead of \verb!'read!.

\begin{verbatim}
    (define p (open "data" 'write))   ; p points to a port
    (define oldOutput (setPort p))
    ...                               ; write stuff to the file data
    (setPort oldOutput)               ; restore the old output port
\end{verbatim}

Opening a file in \verb!'write! mode overwrites the file;
to append content to an existing file, use the \verb!'append!
symbol instead.

Scam only allows a limited number of ports to be open at
any given time. If you no longer need a port, close it with
the built-in function {\it close}, which takes a port as its
sole argument:

\begin{verbatim}
    (close p)
\end{verbatim}

\section{Reading}

Scam supplies built-in functions for reading characters,
integers, reals,
strings, and whitespace delimited tokens:

\begin{verbatim}
    (assign s (readChar))
    (assign i (readInt))
    (assign r (readReal))
    (assign s (readString))
    (assign t (readToken))
    (assign s (readRawChar))
    (assign u (readUntil stopCharacterString))
    (assign w (readWhile continueCharacterString))
\end{verbatim}

The first five functions listed skip any whitespace preceeding the
entity they are to read. The last three functions do not skip whitespace.

Both the {\it readChar} and the {\it readToken} functions return strings.
Scam uses the same rules as the C programming language
for what characters constitute an integer and a real.
None of these functions take an argument; they use the current
input port.

To read a symbol, use the {\it symbol} function in conjunction
with the {\it readToken} function:

\begin{verbatim}
    s = symbol(readToken());
\end{verbatim}

To read a line of text, use the built-in {\it readLine} function:

\begin{verbatim}
    (assign l (readLine))
\end{verbatim}

The {\it readLine} function reads up to, and including, the next
newline character, but the newline is not part of the
returned string.

The {\it pause} function always reads from {\it stdin},
regardless of the current input port.
It reads (and discards) a line of text (up to and including the newline).
Its purpose is to pause execution of a program for debugging
purposes.

Three other reading functions are useful for scanning text.
The first is {\it readRawChar}, which returns a string containing
the next character in the file, regardless of whether that
character is whitespace or not.
The second is {\it readUntil}, which is passed a string of characters
that is used to control the read. For example,

\begin{verbatim}
    (readUntil " \t\n")
\end{verbatim}

will start reading at the current point in the file
and return a string of all characters read up to point
where a character in a given string is encountered.
The character that caused the read to stop is pushed
back into the input stream and will be the next character
read.

The {\it readWhile} function is analogous, stopping when
a character not in the given string is encountered.

\section{Writing}

Most output functions write to the current output port.

The simplest output function is {\it display}. It takes a single
argument, which can be any Scam object:

\begin{verbatim}
    (display "Hello, world!\n")
\end{verbatim}

The character sequence \verb!\! followed by
\verb!n! indicate that
a newline is to be displayed.

More useful than {\it display} are the functions {\it print} and
{\it println}, in that they take any number of arguments:

\begin{verbatim}
    (print "(f x) is " (f x) "\n")
    (println "(f x) is " (f x))
\end{verbatim}

The {\it println} function is just like {\it print}, except it
outputs a newline after the displaying the last argument.
Thus, the two calls above produce the same output.

When a string is printed, the quote marks are not displayed.
Likewise, when a symbol is printed, the quote mark is not displayed.

The {\it inspect} function 
prints out the unevaluated
version of its argument followed by the arguments evaluation value:

\begin{verbatim}
    (inspect (f x))
    -> (f x) is 3
\end{verbatim}

The {\it inspect} function always prints to {\it stdout},
regardless of the current output port. Like {\it pause}, it is
useful for debugging.

\section{Pretty printing}

The function {\it pp} acts much like {\it display}, unless it is
passed an environment/object. In such cases, it prints out a table
listing the variables defined in that scope.
Since functions, thunks, exceptions, and errors are all encoded
as objects, pp can be used to inspect them in greater detail.
For example, consider this definition of square:

\begin{verbatim}
    (define (square x)
        (* x x)
        )
\end{verbatim}

Printing the value of square using \verb!(print square)! yields:

\begin{verbatim}
    <function square(x)>
\end{verbatim}

In contrast, using \verb!(pp square)! yields:

\begin{verbatim}
    <object 8573>
               label  : closure
             context  : <object 8424>
                name  : square
          parameters  : (x)
                code  : (begin (* x x))
\end{verbatim}

Yet a third way to look at a functgion is with the {\it pretty} function:

\begin{verbatim}
    (include "pretty.lib")

    (pretty square)
\end{verbatim}

outputs:

\begin{verbatim}
    (define (square x)
        (* x x)
        )
\end{verbatim}

To use the {\it pretty} function, you must include the {\it pretty.lib} library.
You can use the {\it pretty} function to display any arbitrary piece of 
Scam code:

\begin{verbatim}
    (pretty '(begin (println "testing square...") (square 3)))
\end{verbatim}

produces:

\begin{verbatim}
    (begin
        (println "testing square...")
        (square 3)
        )
\end{verbatim}

To change the level of indent, set the variable {\it \_\_pretty-indent} 
appropriately:

\begin{verbatim}
    (assign __pretty-indent "  ")
    (pretty square)
\end{verbatim}

produces:

\begin{verbatim}
    (define (square x)
      (* x x)
      this
      )
\end{verbatim}

The initial indentation defaults to no indentation.
To change this, call the {\it ppCode} function:

\begin{verbatim}
    (ppCode square "    ")
\end{verbatim}

\section{Formatting}

The {\it fmt} function can be used to format numbers and strings
if the default formatting is not acceptable. It uses the C
programming language formatting scheme, taking a formatting
specification as a string, and the item to be formatted.
The function returns a string.

For example,

\begin{verbatim}
    (string+ "<" (fmt "%6d" 3) ">")
    -> "<     3>"

    (string+ "<" (fmt "%-6d" 3) ">")
    -> "<3     >"
\end{verbatim}

A format specification begins with a percent sign and is usually followed
by a number representing the width (in number of characters)
of the resulting string. If the width is positive, the
item is right justified in the resulting string. If the width
is negative, the item is left justified.
After any width specification is a character specifying the
type of the item to be formatted:
\verb!d! for an integer,
\verb!f! for a real number, and
\verb!s! for a string.

The format specification is quite a bit more sophisticated
than shown here. You can read more on a Linux system by
typing the command \verb!man 3 printf! at the system prompt.

\section{Testing for end of file}

The {\it eof?} function can be used to test whether the last
read was successful or not. The function is NOT used to
test if the {\it next} read would be successful. Here is a typical
use of {\it eof?} in tokenizing a file:

\begin{verbatim}
    (define t (readToken))
    (while (not(eof?))
        (store t)
        (assign t (readToken))
        )
\end{verbatim}

\section{Pushing back a character}

Sometimes, it is necessary to read one character too
many from the input. This happens in cases like
advancing past whitespace in the input.
Here is a typical whitespace-clearing loop:

\begin{verbatim}
    (define ch (readRawChar))
    (while (space? ch))
        (assign ch (readRawChar))
        )

    ; last character read wasn't whitespace
    ; so push it back to be read again later

    (pushBack ch)
\end{verbatim}

The {\it pushBack} function takes a string as its
sole argument, but only pushes back the first
character of the string; subsequent characters in
the string are ignored.
