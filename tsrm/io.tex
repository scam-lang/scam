\chapter{Input and Output}
\label{InputAndOutput}

Scam distinguishes between reading from the keyboard
and writing to the console as opposed to reading and writing
to a file. First we cover reading from the keyboard.

\section{Reading from the keyboard}

To read from the keyboard, one uses the
{\it input} function:

\begin{verbatim}
    name = input("What is your name? ")
\end{verbatim}

The {\it input} function prints the given message to  the console
(this message is known as a {\it prompt})
and waits until a response is typed.
The {\it input} function (as of Scam3) always returns a string.
If you wish to read an {\it integer}, you can wrap the call to {\it input}
in a call to {\it int}:

\begin{verbatim}
    age = int(input("How old are you? "))
\end{verbatim}

Other conversion functions similar to {\it int} are {\it float}, which
converts the string {\it input} returns to a real number, and
{\it eval}, which converts a string into its Scam equivalent.
For example, we could substitute {\it eval} for {\it int} or {\it float} and
we would get the exact same result, {\it provided} an integer
or a real number, respectively, were typed in response to
{\it input} prompt. The {\it eval} function can even to some math for you:

\begin{verbatim}
    >>> eval("3 + 7")
    10
\end{verbatim}

\section{Writing to the Console}

One uses the {\it print} function to display text on the console:
The {\it print} function is {\it variadic}, which means
it can take a variable number
of arguments.
The {\it print} function has lots of options, but we will be
interested in only two, {\it sep} and {\it end}.
The {\it sep} (for separator) option specifies what
is printed between the arguments
sent to be printed. If the {\it sep} option is missing, a
space is printed between the values received by {\it print}:

\begin{verbatim}
    >>> print(1,2,3)
    1 2 3
    >>>
\end{verbatim}

If we wish to use commas as the separator, we would do this:

\begin{verbatim}
    >>> print(1,2,3,sep=",")
    1,2,3
    >>>
\end{verbatim}

If we wish to have no separator, we bind {\it sep} to an empty string:

\begin{verbatim}
    >>> print(1,2,3,sep="")
    123
    >>>
\end{verbatim}

The {\it end} option specifies what be printed after the arguments
are printed. If you don't supply a {\it end} option, a newline
is printed by default. This call to {\it print} prints an exclamation
point and then a newline at the end:

\begin{verbatim}
    >>> print(1,2,3,end="!\n")
    1 2 3!
    >>>
\end{verbatim}

If you don't want anything printed at the end, bind {\it end} to an
empty string:

\begin{verbatim}
    >>> print(1,2,3,end="")
    1 2 3>>>
\end{verbatim}

Notice how the Scam prompt ends up on the same line as
the values printed.

You can combine the {\it sep} and {\it end} options.

\section{Reading from the Command Line}

The command line is the line typed in a terminal window
that runs a python program (or any other program).
Here is a typical command line on a Linux system:

\begin{verbatim}
    lusth@sprite:~/l1/activities$ python3 prog3.py
\end{verbatim}

Everything up to and including the dollar sign is
the system prompt. As with all prompts, it is used
to signify that the system is waiting for input.
The user of the system (me) has typed in the command:

\begin{verbatim}
    python3 prog3.py
\end{verbatim}

in response to the prompt. Suppose {\it prog3.py} is a file
with the following code:

\begin{verbatim}
    import sys

    def main():
        print("command-line arguments:")
        print("    ",sys.argv)

    main()
\end{verbatim}

In this case, the output of this program would be:

\begin{verbatim}
    command-line arguments:
        ['prog3.py']
\end{verbatim}

Note that the program imports the {\it sys} module and when
the value of the variable {\it sys.argv} is printed, we
see its value is:

\begin{verbatim}
    ['prog3.py']
\end{verbatim}

This tells us that {\it sys.argv} points to a list (because of the
square brackets) and that the program file name, as a string,
is found in this list.

\section*{Command-line arguments}

The second way to pass information to a program is
through {\it command-line arguments}.
Any whitespace-delimited tokens following
the program file name are stored in {\it sys.argv} 
along with the name of the program being run
by the Scam interpreter.
For example, suppose we run {\it prog3.py} with the this
command:

\begin{verbatim}
    python3 prog3.py 123 123.4 True hello, world
\end{verbatim}

Then the output would be:

\begin{verbatim}
    command-line arguments:
        ['prog3.py', '123', '123.4', 'True', 'hello,', 'world']
\end{verbatim}

From this result, we can see that all of the tokens are stored
in {\it sys.argv} and that they are stored as strings, {\it regardless} of
whether they look like some other entity, such as integer,
real number, or Boolean.

If we wish for {\tt "hello, world"} to be a single token,
we would need to enclose the tokens in quotes:

\begin{verbatim}
    python3 prog3.py 123 123.4 True "hello, world"
\end{verbatim}

In this case, the output is:

\begin{verbatim}
    command-line arguments:
        ['prog3.py', '123', '123.4', 'True', 'hello, world']
\end{verbatim}

There are certain characters that have special meaning to
the system. A couple of these are '*' and ';'. To include
these characters in a command-line argument, they need to
be {\it escaped} by inserting a backslash prior to the 
character.
Here is an example:

\begin{verbatim}
    python3 prog.py \; \* \\
\end{verbatim}

To insert a backslash, one escapes it with a backslash.
The output from this command is:

\begin{verbatim}
    command-line arguments:
        ['prog3.py', ';', '*', '\\']
\end{verbatim}

Although it looks as if there are two backslashes in the last
token, there is but a single backslash. Scam uses two
backslashes to indicate a single backslash.

\subsection*{Counting the command line arguments}

The number of command-line arguments (including the program file name)
can be found by using the {\it len} (length) function. If we modify
{\it prog3.py}'s main function to be:

\begin{verbatim}
    def main():
        print("command-line argument count:",len(sys.argv))
        print("command-line arguments:")
        print("    ",sys.argv)
\end{verbatim}

and enter the following command at the system prompt:

\begin{verbatim}
    python3 prog3.py 123 123.4 True hello world
\end{verbatim}

we get this output:

\begin{verbatim}
    command-line argument count: 6
    command-line arguments:
         ['prog3.py', '123', '123.4', 'True', 'hello', 'world']
\end{verbatim}

As expected, we see that there are six command-line arguments,
including the program file name.

\section*{Accessing individual arguments}

As with most programming languages and with Scam lists,
the individual tokens in {\it sys.argv} are accessed with
zero-based indexing. To access the program file name, we would
use the expression:

\begin{verbatim}
    sys.argv[0]
\end{verbatim}

To access the first command-line argument after the program
file name, we would use the expression:

\begin{verbatim}
    sys.argv[1]
\end{verbatim}

Let's now modify {\it prog3.py} so that it prints out each argument
individually. We will using a construct called a loop to do this. You
will learn about looping later, but for now, the statement
starting with {\tt for} generates all the legal indices for
{\it sys.argv}, from zero to the length of {\it sys.argv} minus one.
Each index, stored in the variable {\it i}, is then used to print the argument
stored at that location in sys.argv:

\begin{verbatim}
    def main():
        print("command-line argument count:",len(sys.argv))
        print("command-line arguments:")
        for i in range(0,len(sys.argv),1):
            print("   ",i,":",sys.argv[i])
\end{verbatim}
    
Given this command:

\begin{verbatim}
    python3 prog3.py 123 123.4 True hello world
\end{verbatim}

we get the following output:

\begin{verbatim}
    command-line argument count: 6
    command-line arguments:
        0 : prog3.py
        1 : 123
        2 : 123.4
        3 : True
        4 : hello
        5 : world
\end{verbatim}

This code works no matter how many command line arguments are sent.
The superior student will ascertain that this is true.

\subsection{What command-line arguments are}

The command line arguments  are stored as strings.
Therefore, you must use the {\it int}, {\it float}, or {\it eval} functions
if you
wish to use any of the command line arguments as numbers.

\section{Reading from Files}

The third way to get data to a program is to read the
data that has been previously stored in a file.

Scam uses a {\it file pointer} system in reading from a
file. To read from a file, the first
step is to obtain a pointer to the file. This is
known as {\it opening} a file. Suppose
we wish to read from a file named {\it data}. We
would open the file like this:

\begin{verbatim}
    fp = open("data","r")
\end{verbatim}

The {\it open} command takes two arguments, the name of the
file and the kind of file pointer to return. In this
case, we wish for a {\it reading} file pointer, so we pass
the string {\tt "r"}.

Next, we can read the entire file into a single string with
the {\it read} method:

\begin{verbatim}
    text = fp.read()
\end{verbatim}

After this statement is evaluated, the variable {\it text} would
point to a string containing every character in the file {\it data}.
We call {\it read} a method, rather than a function (which it is), to
indicate that it is a function that belongs to a file pointer
object, which {\it fp} is. You will learn about objects 
in a later class, but the `dot' operator is a clue that
the thing to the left of the dot is an object and the
thing to the right is a method (or simple variable) that
belongs to the object.

When we are done reading a file, we {\it close} it:

\begin{verbatim}
    fp.close()
\end{verbatim}

\section{Writing to a File}

Scam also requires a file pointer to write to 
a file. The {\it open} function is again used to obtain
a file pointer, but this time we desire a {\it writing}
file pointer, so we send the string {\tt "w"} as the
second argument to {\it open}:

\begin{verbatim}
    fp = open("data.save","w")
\end{verbatim}

Now the variable {\it fp} points to a writing file object, which
has a method named {\it write}. The only argument {\it write} takes
is a string. That string is then written to the file. Here is
a function that copies the text from one file into another:

\begin{verbatim}
    def copyFile(inFile,outFile):
        in = open(inFile,"r")
        out = open(outFile,"w")
        text = in.read();
        out.write(text);
        in.close()
        out.close()
\end{verbatim}
            
Opening a file in order to write to
it has the effect of emptying the file of its contents
soon as it is opened.
The following code deletes
the {\it contents} of a file (which is different than
deleting the file):

\begin{verbatim}
    # delete the contents
    fp = open(fileName,"w")
    fp.close()
\end{verbatim}

We can get rid of the variable {\it fp} by simply treating the
call to {\it open} as the object {\it open} returns, as in:

\begin{verbatim}
    # delete the contents
    open(fileName,"w").close()
\end{verbatim}

If you wish to start writing to a file, but save what was
there previously, call the {\it open} function to obtain
an {\it appending} file pointer:

\begin{verbatim}
    fp = open(fileName,"a")
\end{verbatim}

Subsequent writes to {\it fp} will append text to what is already there.
