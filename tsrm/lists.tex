\chapter{Lists, Strings, and Arrays}
\label{ListsStringsArrays}

Recall that the built-in function {\it list} is used to construct a list.
A similar function, called {\it array}, is used to construct an array populated
with the given elements:

\begin{verbatim}
    (array 1 "two" 'three)
    -> [1,"two",three]
\end{verbatim}

while a function named {\it allocate} is used to allocate an array
of the given size:

\begin{verbatim}
    (allocate 5)
    -> [nil,nil,nil,nil,nil]
\end{verbatim}

Note that the elements of an allocated array are initialized to {\it nil}.

The functions for manipulating lists, arrays, and strings are quite
similar. In the following sections, we will use the term {\it collection}
to stand for lists, arrays, and strings.

\section{Comparing collections}

The {\it equal?} function is used to compare two collections of the same type:

\begin{verbatim}
    (equal? (list 1 (array "23" 'hello)) (list 1 (array "23" 'hello)))
    -> #t

    (eq? (list 1 (array "23" 'hello)) (list 1 (array "23" 'hello)))
    -> #f
\end{verbatim}

Note that the {\it eq?} function tests for pointer equality and thus fails
when comparing two separate lists, even though they look similar.

Strings can be compared with the {\it string-compare} function.
Assuming a lexigraphical ordering based upon the ASCII code,
then a string compare of two strings, {\it a} and {\it b},
returns a negative number if {\it a}
appears lexigraphically before {\it b}, returns zero if a and
b are lexigraphically equal,
and returns a positive number otherwise.

\begin{verbatim}
    (string-compare "abc" "bbc")
    -> -1
\end{verbatim}

\section{Extracting elements}

You can pull out an item from a collection
by using the
{\it getElement} function.
With {\it getElement},
you specify exactly which element
you wish to extract. This specification is called an
{\it index}. The first element of a collection has index 0, the second
index 1, and so on. This concept of having the first element having
an index of zero is known as {\it zero-based counting}, a common concept
in Computer Science. Here is some code that extracts the first element
of a list:

\begin{verbatim}
    (getElement (list "a" #t 7) 0)
    -> "a"

    (getElement (array "b" #f 11) 1)
    -> #f

    (getElement "howdy" 2)
    -> "w"
\end{verbatim}

What happens if the index is too large?

\begin{verbatim}
    (getElement (list "a" #t 7) 3)
    EXCEPTION: generalException
    index (3) is too large
\end{verbatim}

Not surprisingly, an error is generated.
In Scam, as with many programming languages, an error is known
as an {\it exception}.

As with Scheme, the built-in {\it car} and {\it cdr} functions
returns the first item and the tail of a collection, respectively:

\begin{verbatim}
    (car (list 3 5 7))
    -> 3

    (cdr "howdy")
    -> "owdy"

    (cdr (array 2 4 6))
    -> [4,6]

    (car "bon jour")
    -> "b"
\end{verbatim}

\section{Setting elements of a collection}

The {\it set-car!} function can be used to set the first element of
a collection:

\begin{verbatim}
    (set-car! (list 3 5 7) 0 11)
    -> (11 5 7)
\end{verbatim}

More generally, the {\it setElement} function can be used to set a new value
at any legal index:

\begin{verbatim}
    (setElement collection index newValue)
\end{verbatim}

For strings, the new value must be a non-empty string. If the
value is composed of multiple characters, the characters after
the first character are ignored.
