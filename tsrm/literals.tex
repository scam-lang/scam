\chapter{Literals}
\label{Literals}

Scam works by figuring out the meaning or value of some code.
This is true for the tiniest pieces of code to the largest
programs. The process of finding out the meaning of code
is known as {\it evaluation}.

The things whose values are the things themselves are known as
{\it literals}. The literals of Scam can be categorized by the following
types:
{\it integers}, {\it real} {\it numbers}, {\it strings}, {\sc Booleans},
{\it symbols}, and {\it lists}.

Scam (or more correctly, the Scam interpreter) responds to literals
by echoing back the literal itself.
Here are examples of each of the types:

\begin{verbatim}
    (inspect 3)
    -> 3 is 3
     
    (inspect -4.9
    -> -4.900000 is -4.900000
     
    (inspect "hello")
    -> hello is hello
     
    (inspect #t)
    -> #t is #t

    (inspect (list 3 -4.9 "hello"))
    -> (list 3 -4.9 "hello") is (3, -4.9, "hello")
\end{verbatim}
Let's examine the
five types in more detail.

\section{Integers}

Integers are numbers without any fractional parts.
Examples of integers are:

\begin{verbatim}
    (inspect 3)
    -> 3 is 3
    
    (inspect -5)
    -> -5 is -5
    
    (inspect 0)
    -> 0 is 0
\end{verbatim}

Integers must begin with a digit or a minus sign. The initial minus sign
must immediately be followed by a digit.

\section{Real Numbers}

Reals are numbers that do have a fractional part (even if that fractional
part is zero!). Examples of real numbers are:

\begin{verbatim}
    (inspect 3.2)
    -> 3.200000 is 3.200000
    
    (inspect 4.0)
    -> 4.000000 is 4.000000
       
    (inspect 5.)
    -> 5.000000 is 5.000000
       
    (inspect 0.3)
    -> 0.300000 is 0.300000
       
    (inspect .3)
    -> 0.300000 is 0.300000
    
    (inspect 3.0e-4)
    -> 0.000300 is 0.000300
    
    (inspect 3e4)
    -> 30000.000000 is 30000.000000
    
    (inspect .000000987654321)
    -> 0.000001 is 0.000001
\end{verbatim}

Real numbers must start with a digit or a minus sign or a decimal
point. An initial minus sign must immediately be followed by a digit or a
decimal point. An initial decimal point must immediately be followed by
a digit. Scam accepts real numbers in scientific notation. For example,
$3.0 * 10^{-11}$ would be entered as 3.0e-11. The `e' stands for exponent and
the 10 is understood, so e-11 means multiply whatever precedes the
e by $10^{-11}$.

The Scam interpreter can hold huge numbers,
limited by only the amount of memory available to the
interpreter,
but holds only 15 digits after the decimal point:

\begin{verbatim}
    (inspect 1.2345678987654329
    -> 1.234568 is 1.234568
\end{verbatim}

Note that Scam rounds up or rounds down, as necessary.

Numbers greater than $10^6$ and
less than $10^{-6}$ are displayed in
scientific notation.

\section{Strings}

Strings are sequences of characters delineated by double quotation marks:

\begin{verbatim}
    (println "hello, world!")
    -> hello, world!
    
    (println "x\nx")
    -> x
       x
    
    (println "\"z\"")
    -> "z" 
\end{verbatim}

Scam accepts both double quotes and single quotes to
delineate strings. In this text, we will use the convention
that double quotes are used for strings of multiple 
characters and single quotes for strings consisting of
a single character.

Characters in a string can be {\it escaped} (or quoted)
with the backslash character,
which changes the meaning of some characters. For example, the character
{\it n}, in a string refers to the letter {\it n} while the character sequence
{\it $\backslash$n}
refers
to the {\it newline} character. A backslash also changes the meaning of the
letter {\it t},
converting it into a tab character.
You can also quote single and double quotes with backslashes.
When other characters are escaped,
it is assumed the backslash is a character of the
string and it is escaped (with a backslash) in the result:

\begin{verbatim}
    (println "\z")
    -> z
\end{verbatim}

Note that Scam, when asked
the value of strings that contain newline and tab characters, displays
them as escaped characters. When newline and tab characters in a string
are printed in a program, however, they are displayed as actual newline
and tab characters, respectively.
As already noted,
double and single quotes can be embedded in a
string by quoting them with backslashes. A string with no characters
between the double quotes is known as an empty string.

Unlike some languages, there is no character type in Scam. A single
character {\verb+a+}, for example, is entered as the string
{\verb+"a"+}.

\section{Symbols}

A symbol is a set of characters, much like a string. Like strings,
symbols evaluate to themselves. Unlike strings,
symbols are not formed using a beginning quotation mark and an
ending quotation mark. They are also limited
in the characters that compose them. For example, a symbol cannot
contain a space character while a string can. A symbol is introduced
with a single quotation mark:

\begin{verbatim}
    (print 'a)
    -> a

    (print 'hello)
    -> hello
\end{verbatim}

We we learn more about symbols and their relationship to entities
called {\it variables} in a later chapter.

\section{True, False, and nil}

There are two special literals, \verb!#t!
and \verb!#f!.
These literals are known as the {\sc Boolean} values;
\verb!#t! is true and \verb!#f! is false.
Boolean values are used to guide the flow of a program.
The term {\sc Boolean} is derived from the last name of George Boole, who,
in his 1854 paper {\it An Investigation of the Laws of Thought, on which are
founded the Mathematical Theories of Logic and Probabilities}, laid one
of the cornerstones of the modern digital computer. The so-called {\sc Boolean}
logic or {\sc Boolean} algebra is concerned with the rules of combining truth
values (i.e., true or false). As we will see, knowledge of such rules will
be important for making Scam programs behave properly. In particular,
{\sc Boolean} expressions will be used to control conditionals and loops.

Another special literal is \verb!nil!.
This literal is used to
indicate an empty list or an empty string; it also is used
to indicate something that has not yet been
created. More on \verb!nil! when we cover lists and
objects.

\section{Lists}

Lists are just collections of entities.
The simplest list is the empty list:

\begin{verbatim}
    (inspect ())
    -> nil
\end{verbatim}

Since the empty list looks kind of strange, Scam uses the symbol \verb!nil!
to represent an empty list.

One creates non-empty list by
using the built-in {\it list function}. 
Here, we make a list containing the numbers
10, 100, and 1000:

\begin{verbatim}
    (list 10 100 1000)
    -> (10 100 1000)
\end{verbatim}

Lists can contain values besides numbers:

\begin{verbatim}
    (list 'a "help me" length)
    -> (a "help me" <built-in length(items)>)
\end{verbatim}

The first value is an integer, the second a string,
and the third item is also a function. 
The built-in {\it length} function
is used to tell us how many items are in a list:

\begin{verbatim}
    (length (list 'a "help me" length))
    -> 3
\end{verbatim}

As expected, the {\it length} function tells us that the list
\verb!('a "help me" length)! has three items in it.

Lists can even contain lists!

\begin{verbatim}
    (list 0 (list 3 2 1) 4)
    -> (0 (3 2 1) 4)
\end{verbatim}

A list is something known as a {\it data structure};
data structures are extremely important in writing
sophisticated programs.

\section{Indexing into Lists}

You can pull out an item from a list by using another function
named {\it getElement}.
With the {\it getElement} function,
you specify exactly which element (or elements)
you wish to extract from the list. This specification is called an
{\it index}. The first element of the list has index 0, the second
index 1, and so on. This concept of having the first element having
an index of zero is known as {\it zero-based counting}, a common concept
in Computer Science. Here is some code that extracts the first element
of a list:

\begin{verbatim}
    (getElement (list "a" #t 7) 0)
    -> "a"

    (getElement (list "a" #t 7) 1)
    -> #t

    (getElement (list "a" #t 7) 2)
    -> 7
\end{verbatim}

What happens if our index is too large?

\begin{verbatim}
    (getElement (list "a" #t 7) 3)
    EXCEPTION: generalException
    index (3) is too large
\end{verbatim}

Not surprisingly, we get an error.
In Scam, as with many programming languages, an error is known
as an {\it exception}.

As with Scheme, the built-in {\it car} and {\it cdr} functions
returns the first item and the tail of a given list, respectively.

We will see more of lists in a later chapter.
