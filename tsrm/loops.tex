\chapter{Loops}
\label{Loops}

In the previous chapter, you learned how recursion can
solve a problem by breaking it in to smaller versions
of the same problem. Another approach is to use
{\it iterative} {\it loops}. In some programming
languages, loops are preferred as they use much
less computer memory as compared to recursions.
In other languages, this is not the case at all.
In general, there
is no reason to prefer recursions over loops or vice versa,
other than this memory issue.
Any loop can be written as a recursion and any recursion
can be written as a loop.
Use a recursion if that makes the implementation
more clear, otherwise, use an iterative loop.

The most
basic loop structure in Scam is the {\it while} loop, an example of
which is:

\begin{verbatim}
    while (i < 10):
        print(i,end="")
        i = i + 1
\end{verbatim}

We see a {\tt while} loop looks much like an {\tt if}
statement.
The difference
is that blocks belonging to {\tt if}s are evaluated at most once whereas
blocks associated with loops may be evaluated many many times.
Another difference in nomenclature is that the block of a loop
is known as the {\it body} (like blocks associated with function
definitions). Furthermore, the loop test expression is known
as the {\it loop condition}.

As Computer Scientists hate to type extra characters if they can
help it, you will often see:

\begin{verbatim}
    i = i + 1
\end{verbatim}

written as

\begin{verbatim}
    i += 1
\end{verbatim}

The latter version is read as ``increment {\it i}''.

A {\it while} loop tests its condition before the body of the loop is
executed. If the initial test fails, the body is not executed at all. For
example:

\begin{verbatim}
    i = 10
    while (i < 10):
        print(i,end="")
        i += 1
\end{verbatim}

never prints out anything since the test immediately fails. In this example,
however:

\begin{verbatim}
    i = 0;
    while (i < 10):
        print(i,end="")
        i += 1
\end{verbatim}

the loop prints out the digits 0 through 9:

\begin{verbatim}
    0123456789
\end{verbatim}
    
A {\tt while} loop repeatedly evaluates its body
as long as the loop condition remains true.

To write an infinite loop, use {\tt :true} as the condition:

\begin{verbatim}
    while (True):
        i = getInput()
        print("input is",i)
        process(i)
        }
\end{verbatim}

\section{Other loops}

There are many kinds of loops in Scam, in this text
we will only refer to {\tt while} loops and {\tt for}
loops that count,
as these are commonly found in other programming languages.
The {\tt while} loop we have seen; here is an example of a counting
{\tt for} loop:

\begin{verbatim}
    for i in range(0,10,1):
        print(i)
\end{verbatim}

This loop is exactly equivalent to:

\begin{verbatim}
    i = 0
    while (i < 10):
        print(i)
        i += 1
\end{verbatim}

In fact, a while loop of the general form:

\begin{verbatim}
    i = INIT
    while (i < LIMIT):
        # body
        ...
        i += STEP
\end{verbatim}

can be written as a \\verb!for! loop:

\begin{verbatim}
    for i in range(INIT,LIMIT,STEP):
        # body
        ...
\end{verbatim}

The {\it range} function counts from
{\tt INIT} to {\tt LIMIT} (non-inclusive)
by {\tt STEP} and these
values are assigned
to {\it i}, in turn. After each assignment to {\it i},
the loop body is evaluated.
After the last value is assigned to {\it i} and the
loop body evaluated on last time, the \\verb!for! loop ends.

In Scam, the {\it range} function assumes 1 for the step
if the step is omitted and assumes 0 for the initial
value and 1 for the step if both the initial value and
step are omitted.
However, in this text, we will always give the initial
value and step of the \\verb!for! loop explicitly.

For loops are commonly used to sweep through each element of an list:

\begin{verbatim}
     for i in range(0,len(items),1):
         print(items[i]) 
\end{verbatim}

Recall the items in a list of $n$ elements are located at
indices $0$ through $n - 1$. These are exactly the values
produced by the {\it range} function. So, this loop accesses
each element, by its index, in turn, and thus prints out
each element, in turn.
Since using an index of {\it n} in a list of {\it n} items produces an
error, the {\it range} function conveniently makes its given
limit non-inclusive.

As stated earlier, there are other kinds of loops in
Scam, some of which, at times, are more convenient
to use than a {\tt while} loop or a counting {\tt for} loop. However,
anything that can be done with those other loops can be
done with the loops presented here.
Like recursion and lists, loops and lists go very well
together.
The next sections detail some common loop patterns involving
lists.

\section{The {\it counting} pattern}

The counting pattern is used to count the number of items
in a collection. Note that the built-in function len already
does this for Scam lists, but many programming languages
do not have lists as part of the language; the programmer
must supply lists. For this example, assume that the {\it start}
function gets the first item in the given list,
returning {\tt None}
if there are no items in the list. The {\it next}
function returns the next item in the given list,
returning {\tt None}
if there are no more items. This {\tt while} loop counts the number
of items in the list:

\begin{verbatim}
    count = 0
    i = start(items)
    while (i != None)
         count += 1
         i = next(items)
\end{verbatim}

When the loop finishes, the variable count holds the number of
items in the list.

The counting pattern increments a counter everytime the loop
body is evaluated.

\section{The {\it filtered-count} pattern}

A variation on counting pattern involves filtering. When {\it filtering},
we use an {\tt if} statement to decide whether we should count an item
or not. Suppose we wish to count the number of even items in
a list:

\begin{verbatim}
    count = 0
    for i in range(0,len(items),1):
        if (items[i] % 2 == 0):
            count += 1
\end{verbatim}

When this loop terminates, the variable {\it count} will hold the
number of even integers in the list of items since the count
is incremented only when the item of interest is even.

\section{The {\it accumulate} pattern}

Similar to the counting pattern, the {\it accumulate} pattern
updates a variable, not by increasing its value by one, but by the value of
an item. This loop, sums all the values in a list:

\begin{verbatim}
    total = 0
    for i in range(0,len(items),1):
        total += items[i]
\end{verbatim}

By convention, the variable {\it total} is used to accumulate the item
values. When accumulating a sum, total is initialized to zero. When
accumulating a product, total is initialized to one.

\section{The {\it filtered-accumulate} pattern}

Similar to the {\it accumulate} pattern, the {\it filtered-accumulate} pattern
updates a variable only if some test is passed.
This function sums all the even values in a given list, returning
the final sum:

\begin{verbatim}
    def sumEvens(items):
        total = 0
        for i in range(0,len(items),1):
            if (items[i] % 2 == 0)
                total += items[i]
        return total
\end{verbatim}

As before, the variable {\it total} is used to accumulate the item
values. As with a regular accumulating, {\it total} is initialized to zero when
accumulating a sum. The initialization value is one when
accumulating a product and the initialization value is
the empty list when accumulating a list (see {\it filtering} below).

\section{The {\it search} pattern}

The {\it search} pattern is a slight variation of {\it filtered-counting}.
Suppose we wish to see if a value is present in a list. We can
use a filtered-counting approach and if the count is greater than
zero, we know that the item was indeed in the list.

\begin{verbatim}
    count = 0
    for i in range(0,len(items),1):
        if (items[i] == target):
            count += 1
    found = count > 0
\end{verbatim}

This pattern is so common, it is often encapsulated in a function.
Moreover, we can improve the efficiency by short-circuiting the
search. Once the target item is found, there is no need to 
search the remainder of the list:

\begin{verbatim}
    def find(target,items):
        found = False:
        i = 0
        while (not(found) and i < len(items)):
            if (items[i] == target):
                found = True
            i += 1
        return found
\end{verbatim}

We presume the target item is not in the list and 
as long as it is not found, we continue to search the list.
As soon as we find the item, we set the variable found
to True and then the loop condition fails, since
not true is false.

Experienced programmers would likely define this function
to use an immediate return once the target item is found:

\begin{verbatim}
    def find(target,items):
        for i in range(0,len(items),1):
            if (items[i] == target):
                return True
        return False
\end{verbatim}

As a beginning programmer, however, you should avoid returns
from the body of a loop. The reason is most beginners end up
defining the function this way instead:

\begin{verbatim}
    def find(target,items):
        for i in range(0,len(items),1):
            if (items[i] == target):
                return True
            else:
                return False
\end{verbatim}

The behavior of this latter version of {\it find} is incorrect,
but unfortunately, it appears to work correctly under some
conditions. If you cannot figure out why this version
fails under some conditions and appears to succeed under
others, you most definitely should stay away from placing
returns in loop bodies.
        
\section{The {\it filter} pattern}

Recall that a special case of a filtered-accumulation is the {\it filter}
pattern.
A loop version  of filter starts out by initializing an accumulator variable to
an empty list. In the loop body, the accumulator variable
gets updated with those items from the original list that 
pass some test.

Suppose we wish to extract the even numbers from a list.
Our test, then, is to see if the current element is even.
If so, we add it to our growing list:

\begin{verbatim}
    def extractEvens(items):
        evens = []
        for i in range(0,len(items),1):
            if (items[i] % 2 == 0):
                evens = evens + [items[i]]
        return evens
\end{verbatim}

Given a list of integers, {\it extractEvens} returns a (possibly empty)
list of the even numbers:

\begin{verbatim}
    >>> extractEvens([4,2,5,2,7,0,8,3,7])
    [4, 2, 2, 0, 8]

    >>> extractEvens([1,3,5,7,9])
    []
\end{verbatim}

\section{The {\it extreme} pattern}

Often, we wish to find the largest or smallest value
in a list. Here is one approach, which assumes that the
first item is the largest and then corrects that assumption
if need be:

\begin{verbatim}
    largest = items[0]
    for i in range(0,len(items),1):
        if (items[i] > largest):
            largest = items[i]
\end{verbatim}

When this loop terminates, the variable {\it largest} holds the
largest value. We can improve the loop slightly by noting
that the first time the loop body evaluates, we compare the
putative largest value against itself, which is a worthless
endeavor. To fix this, we can start the index variable {\it i}
at 1 instead:

\begin{verbatim}
    largest = items[0]
    for i in range(1,len(items),1): #start comparing at index 1
        if (items[i] > largest):
            largest = items[i]
\end{verbatim}

Novice programmers often make the mistake of initialing setting
{\it largest} to zero and then comparing all values against {\it largest},
as in:

\begin{verbatim}
    largest = 0
    for i in range(0,len(items),1):
        if (items[i] > largest):
            largest = items[i]
\end{verbatim}

This code appears to work in some cases, namely if the largest
value in the list is greater than or equal to zero. If not,
as is the case when all values in the list are negative,
the code produces an erroneous result of zero as the largest
value.

\section{The {\it extreme-index} pattern}

Sometimes, we wish to find the index of the most extreme
value in a list rather than the actual extreme value.
In such cases, we assume index zero holds the extreme value:

\begin{verbatim}
    ilargest = 0
    for i in range(1,len(items),1):
        if (items[i] > items[ilargest]):
            ilargest = i
\end{verbatim}

Here, we successively store the index of the largest value
see so far in the variable {\it ilargest}.

\section{The {\it shuffle} pattern}

Recall, the {\it shuffle} pattern from the previous chapter.
Instead of using recursion, we can use a version of
the loop accumulation pattern instead. As before,
let's assume the lists are exactly the same size:

\begin{verbatim}
    def shuffle(list1,list2):
        list3 = []
        for i in range(0,len(list1),1):
            list3 = list3 + [list1[i],list2[i]]
        return list3
\end{verbatim}

Note how we initialized the resulting list {\it list3} to
the empty list. Then, as we walked the first list, we
pulled elements from both lists, adding them into the
resulting list.

When we have walked past the end of {\it list1} is empty,
we know we have also walked past the end of {\it list2}, since the
two lists have the same size.

If the incoming lists do not have the same length,
life gets more complicated:

\begin{verbatim}
    def shuffle2(list1,list2):
        list3 = []
        if (len(list1) < len(list2)):
            for i in range(0,len(list1),1):
                list3 = list3 + [list1[i],list2[i]]
            return list3 + list2[i:]
        else:
            for i in range(0,len(list2),1):
                list3 = list3 + [list1[i],list2[i]]
            return list3 + list1[i:]
\end{verbatim}

We can also use a {\it while} loop that goes until one of the lists
is empty. This has the effect of removing the redundant code
in {\it shuffle2}:

\begin{verbatim}
    def shuffle3(list1,list2):
        list3 = []
        i = 0
        while (i < len(list2) and i < len(list2)):
            list3 = [list1[i],list2[i]]
            i = i + 1
        ...
\end{verbatim}

When the loop ends, one or both of the lists have been exhausted,
but we don't know which one or ones. A simple solution is to
add both remainders to {\it list3} and return.

\begin{verbatim}
    def shuffle3(list1,list2):
        list3 = []
        i = 0
        while (i < len(list2) and i < len(list2)):
            list3 = [list1[i],list2[i]]
            i = i + 1
        return list3 + list1[i:] + list2[i:]
\end{verbatim}

Suppose {\it list1} is empty. Then the expression {\tt list1[i:]} will
generate the empty list. Adding the empty list to {\it list3} will
have no effect, as desired. The same is true if {\it list2}
(or both {\it list1} and {\it list2} are empty).

\section{The {\it merge} pattern}

We can also merge using a loop. Suppose we have two
ordered lists (we will assume increasing order)
and we wish to merge them into one ordered
list. We start by keeping two index variables,
one pointing to the smallest element in {\it list1} and
one pointing to the smallest element in {\it list2}.
Since the lists are ordered, we know the that the smallest
elements are at the head of the lists:

\begin{verbatim}
   i = 0  # index variable for list1
   j = 0  # index variable for list2
\end{verbatim}

Now, we loop, similar to {\it shuffle3}:

\begin{verbatim}
    while (i < len(list1) and j < len(list2)):
\end{verbatim}

Inside the loop, we test to see if the smallest element
in list1 is smaller than the smallest element in list2:

\begin{verbatim}
        if (list1[i] < list2[j]):
\end{verbatim}

If it is, we add the element from {\it list1} to {\it list3} and increase
the index variable {\it i} for {\it list1} since we have `used up' the value
at index {\it i}.

\begin{verbatim}
            list3 = list3 + [list1[i]]
            i = i + 1
\end{verbatim}

Otherwise, {\it list2} must have the smaller element and we do likewise:

\begin{verbatim}
            list3 = list3 + [list2[j]]
            j = j + 1
\end{verbatim}

Finally, when the loop ends ({\it i} or {\it j} has gotten too large),
we add the remainders of both lists to {\it list3} and return:

\begin{verbatim}
    return list3 + list1[i:] + list2[j:]
\end{verbatim}

In the case of merging, one of the lists will be exhausted
and the other will not. As with shuffle3, we really don't
care which list was exhausted.

Putting it all together yields:

\begin{verbatim}
    def merge(list1,list2):
        list3 = []
        i = 0
        j = 0
        while (i < len(list1) and j < len(list2)):
            if (list1[i] < list2[j]):
                list3 = list3 + [list1[i]]
                i = i + 1
            else:
                list3 = list3 + [list2[j]]
                j = j + 1
        return list3 + list1[i:] + list2[j:]
\end{verbatim}

\section{The {\it fossilized} Pattern}

Sometimes, a loop is so ill-specified that it never ends. This
is known as an {\it infinite loop}. Of the
two loops we are investigating, the {\it while} loop is the most
susceptible to infinite loop errors. One common mistake
is the {\it fossilized} pattern, in which the index variable never
changes so that the loop condition never becomes false:

\begin{verbatim}
    i = 0
    while (i < n):
        print(i)
\end{verbatim}

This loop keeps printing until you terminate the program with
prejudice. The reason is that {\it i} never changes; presumably a
statement to increment {\it i} at the bottom of the loop body has
been omitted.

\section{The {\it missed-condition} pattern}

Related to the bottomless pattern of recursive functions
is the missed condition pattern of loops.
With missed condition, the index variable is updated, but
it is updated in such a way that the loop condition is
never evaluates to false.

\begin{verbatim}
    i = n
    while (i > 0):
        print(i)
        i += 1
\end{verbatim}

Here, the index variable {\it i} needs to be decremented rather than
incremented. If {\it i} has an initial value greater than zero,
the increment pushes {\it i} further and further above zero.
Thus, the loop condition never fails and the loop
becomes infinite.
