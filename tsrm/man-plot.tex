\section{Plot}

The {\it plot} library uses one linux application:
{\it gnuplot}. A {\it plot}-based program uses {\it gnuplot}
to generate the plot and display it.

Here is a sample program that plots a simple graph:
\color{CodeGreen}
\begin{codesize}
\begin{verbatim}
     include("plot");
     var x = array(0,1,2,3,4,5);
     var y = array(0,1,2,3,4,5);
     openPlot();
     addData(x,y);
     addPlotPoints(1,2);
     showPlot();
     closePlot();     
\end{verbatim}
\end{codesize}
\color{black}

The first line of the  program causes the {\it plot} library to be 
loaded. The second and third lines alloate arrays to be used. The fourth 
line opens a plot file named {\it data}. The fifth line adds the arrays 
to the data file. The sixth writes to the data file that column 1 should 
be plotted versus column 2. The seventh line displays the plot while the 
eigth closes the file.

The next section, in man page format, describes the public interface of 
the {\it plot} library.

\subsection*{The Sway Plot Library}

\color{CodeGreen}
\begin{codesize}
\begin{verbatim}

			Sway Plot Library



NAME
	plot - a Sway plotting library

SYNOPSIS
	include("plot");

DESCRIPTION
	plot is a library for Sway and is based upon the gnuplot 
	package. There are functions for most of the gnuplot primitives 
	as well as functions for higher abstraction and data manipulation. 
	There is also a degree of data control not available in gnuplot.

    TOP-LEVEL FUNCTIONS
	There are three top-level functions. The open and close image 
	functions are required, whereas the third is purely aesthetic.
	
	function plotOpen()
		Open a basic plot file. This merely initializes all 
		files and filenames.
	
	function plotClose(name)
		Writes the image data out to the file name constructed 
		by appending the image name with its type.

	function showPlot()
		Displays the image constructed at that point.

    PLOTTING FUNCTIONS
	In the following lines xs and ys are arrays of numbers.
	
	function trendPoly(xs,ys,degree,precision)
		Plots a trendline of specified degree using xs and ys as 
		the x and y values to be plotted. This function also 
		produces an equation to the screen representing the 
		computed tendline with decimal precision up to the 
		specified precision.

	function addData(xs,ys,...)
		Adds data to the document used to plot points.

	function addPlotPoints(X,Y)
		Plots points in the data file using the columns 
		specified. X represents the column containing the 
		x-values and Y represents the column containing the 
		y-values.

    SETTINGS FUNCTIONS

	function xlabel(name)
		Labels the x-axis of the plot with the title name.

	function ylabel(name)
		Labels the y-axis of the plot with the title name.
	
	function xRange(x1,x2)
		Sets the x-axis to range from point x1 to point x2 in 
		automatically determined increments.	
	
	function yRange(y1,y2)
		Sets the y-axis to range from point y1 to y2 in 
		automatically determined increments.

	function autoscale()
		Automatically determines the x-axis and y-axis scaling.

	function moveKey(x,y)
		Move the key location to point (x,y) on the graph.

	function removeKey()
		Removes the key.
	
	function labelPoint(x,y,label)
		Place a label titled label at point (x,y).

	function removeLabels()
		Removes all labels placed on the plot.

	function logscale()
		Converts the x and y-axis to logarithmic scales.

	function xLogscale()
		Converts the x-axis to logarithmic scale.
		
	function yLogscale()
		Converts the y-axis to logarithmic scale.

	function setxTics(xs)
		Specifically places every tic at a set location along 
		the x-axis.
	
	function setyTics(ys)
		Specifically places every tic at a set location along 
		the y-axis.

	function removexTics()
		Resets the x-axis back to its automatic scale.

	function removeyTics()
		Resets the y-axis back to its automatic scale.

    AUTHOR
		Written by James M Tacey, November, 2009.


1.0			15 November 2009
\end{verbatim}
\end{codesize}
\color{black}
\end{document}
