\section{Polynomial}

The {\it polynomial} library uses no additional linux packages beyond 
sway.

Here is a simple polynomial program that merely forms the equation 
x^2+x+1.

\color{CodeGreen}
\begin{codesize}
\begin{verbatim}
     include("polynomial");
     var p = contant(1);
     var x = variable("x");
     p = poly(term(1,x,2),p);
     p = poly(term(1,x,1),p);
     inspect(p . toString());
\end{verbatim}
\end{codesize}
\color{black}

The first line of the program causes the {\it polynomial} library to be 
loaded. The second line of the program creates a polynomial variable and 
stores a constant 1 in it. The second line creates a variable x that 
represents the variable 'x'. the third and fourth lines add an x^2 and x 
to the polynomial variable respectively. The final line outputs the 
string product to standar output.

The next section, in man page format, describes the public interface of 
the {\it polynomial} library.

\subsection*{The Sway Polynomial Library}

\color{CodeGreen}
\begin{codesize}
\begin{verbatim}


polynomial		Sway Polynomial Library


NAME
	polynomial - a Sway polynmial library

SYNOPSIS
	include("polynomial")

DECRIPTION
	polynomial is a library for sway that easily allows for the 
	creation and use of polynomials as strings. These functions can 
	be concatenated to form extremely long polynomials for use in 
	other libraries and functions.

POLYNOMIAL FUNCTIONS

	function constant(value)
		Creates a string that represents a constant that is 
		passed.

	function variable(name)
		Creates a string that represents a variable with the 
		name passed.
	
	function term(coeffecient,symbol,exponent)
		Creates a string that represents a single term of a 
		polynomial. This string is formatted such that the 
		coeffecient is multiplied by the symbol and the symbol 
		is raised to the power of the exponent.
	
	function poly(term1,term2)
		Creates a larger polynomial object by concatonating two 
		polynomial terms together.
	
	AUTHOR
		Written by James M. Tacey, November, 2009


1.0				29 November 2009
\end{verbatim}
\end{codesize}
\color{black}
