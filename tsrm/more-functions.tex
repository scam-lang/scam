\chapter{More about Functions}
\label{MoreAboutFunctions}

We have already seen some examples of functions,
some user-defined and some built-in.
For example, we have used the built-in functions,
such as 
{\tt *} and defined our own functions,
such as {\it square}.
In reality, {\it square} is not a function, per se, but a variable
that is bound to the function that multiplies two numbers
together. It is tedious to say `the function bound to
the variable {\it square}', however,
so we say the more concise (but technically incorrect)
phrase `the {\it square} function'.

\section{Built-in Functions}

Scam has many built-in, or {\it predefined}, functions.
No one, however,
can anticipate all possible tasks that someone might want to perform,
so most programming languages allow the user to define new functions.
Scam is no exception and provides
for the creation of new and novel functions.
Of course,
to be useful,
these functions should be able to call
built-in functions as well as other programmer created
functions.

For example, a function that determines whether a given
number is odd or even is not built into Scam but can be
quite useful in certain situations.
Here is a definition
of a function named {\it isEven} which returns true if the
given number is even, false otherwise:

\begin{verbatim}
    >>> def isEven(n):
    ...     return n % 2 == 0
    ...
    >>>

    >>> isEven(42)
    True

    >>> isEven(3)
    False

    >>> isEven(3 + 5)
    True
\end{verbatim}

We could spend days talking about about what's going on in these
interactions with the interpreter. First, let's talk
about the syntax of a function definition. Later, we'll
talk about the purpose of a function definition. Finally,
will talk about the mechanics of a function definition
and a function call.

\section{Function syntax}

Recall that the words of a programming language include its
primitives, keywords and variables. A function definition
corresponds to a sentence in the language in that it is
built up from the words of the language. And like human
languages, the sentences must follow a certain form. This
specification of the form of a sentence is known as its
{\it syntax}. Computer Scientists often use a special way
of describing syntax of a programming language called the
Backus-Naur form (or {\sc bnf}). Here is a high-level description
of the syntax of a Scam function definition using {\sc bnf}:

\begin{verbatim}
    functionDefinition : signature ':' body

    signature : 'def' variable '(' optionalParameterList ')'

    body : block

    optionalParameterList : *EMPTY*
                          | parameterList
    
    parameterList : variable
                  | variable ',' parameterList
    
    block:  definition 
          | definition block
          | statement
          | statement block
\end{verbatim}

The first {\sc bnf} {\it rule} says that a function definition is
composed of two pieces, a signature and a body, separated
by the colon character
(parts of the rule that appear verbatim appear within single quotes).
The signature starts
with the keyword {\it def}
followed by a variable,
followed by an open parenthesis, followed by something
called an {\it optionalParameterList}, and finally followed by a close
parenthesis.
The body of a function 
something called a {\it block},
which is composed of {\it definitions} and {\it statements}.
The {\it optionalParameterList} rule tells us that
the list of formal parameters can possibly be empty,
but if not, is composed of a list of variables
separated by commas.

As we can see from the {\sc bnf} rules,
parameters are variables that will be bound
to the values supplied in the function call.
In the particular case of {\it isEven}, from the
previous section,
the variable {\it x} will be bound to the number whose
evenness is to be determined. As noted earlier,
it is customary to call {\it x}
a {\it formal parameter} of the function {\it isEven}.
In function calls, the values to be bound to the
formal parameters are called {\it arguments}.

\section{Function Objects}

Let's look more closely at the body of {\it isEven}:

\begin{verbatim}
    def isEven(x):
        return x % 2 == 0
\end{verbatim}

The \% operator is bound to the remainder or modulus
function. The {\tt ==} operator is bound to the equality function
and determines whether the value of the left operand
expression is equal to the value of the right operand
expression, yielding true or false as appropriate. The
{\sc boolean} value produced by {\tt ==} is then immediately returned as the
value of the function.

When given a function definition like that above, Scam
performs a couple of tasks. The first is to create the
internal form of the function, known as a {\it function object},
which holds the function's signature and body.
The second task is to bind
the function name to the function object so that it
can be called at a later time.
Thus, the name
of the function is simply a variable that happens to be
bound to a function object. As noted before, we often say
'the function {\it isEven}' even though we really mean 'the
function object bound to the variable {\it even}?'.

The value of a function definition is the
function object; you can see this by
printing out the value of {\it isEven}:

\begin{verbatim}
    >>> print(isEven)
    <function isEven at 0x9cbf2ac>

    >>> isEven = 4
    >>> print(isEven)
    4

\end{verbatim}

Further interactions with the interpreter provide evidence
that {\it isEven} is indeed a variable; we can reassign its value,
even though it is considered in poor form to do so.

\section{Calling Functions}

Once a function is created,
it is used by {\it calling} the
function with {\it arguments}.
A function is called by supplying
the name of the function followed by a parenthesized,
comma separated, list of expressions.
The arguments are
the values that the formal parameters will receive.
In Computer Science speak, we say that the values
of the arguments are to be bound to
the
formal parameters.
In general, if there are {\it n} formal parameters,
there should be {\it n} arguments.\footnote{
For {\it variadic} functions, which Scam
allows for, the number of arguments
may be more or less than the number of formal parameters.
}
Furthermore, the value of the
first argument is bound to the first formal parameter, the
second argument is bound to the second formal parameter,
and so on. Moreover, all the arguments are evaluated
before being bound to any of the parameters.

Once the evaluated arguments are bound to the parameters,
then the body of the function is evaluated. Most times,
the expressions in the body of the function will reference
the parameters. If so, how does the interpreter find the
values of those parameters? That question is answered in
the next chapter.

\section{Returning from functions}

The return value of a function is the value of the expression
following the {\tt return} keyword.
For a function to return this  expression, however,
the return has to be {\it reached}.
Look at this example:

\begin{verbatim}
def test(x,y):
    if (y == 0):
        return 0
    else:
        print("good value for y!")
        return x / y

    print("What?")
    return 1
\end{verbatim}

Note that the {\it ==} operator returns true if the
two operands have the same value.
In the function, if {\it y} is zero, then the
\begin{verbatim}
    return 0
\end{verbatim}

statement is reached.
This causes an
immediate return from the function and
no other expressions in the function body are evaluated.
The return value, in this case, is zero.
If {\it y} is not equal to zero,
a message is printed and 
the second return is reached, again causing
an immediate return. In this case,
a quotient is returned.

Since both parts of the if statement have
returns, then the last two lines of the
function:

\begin{verbatim}
    print("What?")
    return 1
\end{verbatim}

are {\it unreachable}. Since they
are unreachable, they cannot be
executed under any conditions and
thus serve no purpose and can be deleted.
