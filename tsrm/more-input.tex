\chapter{More on Input}
\label{MoreOnInput}

Now that we have learned how to loop, we can perform
more sophisticated types of input.

\section{Converting command line arguments en mass}

Suppose all the command-line arguments are numbers that
need to be converted from their string versions stored
in {\it sys.argv}.
We can use a loop and the accumulate pattern to accumulate
the converted string elements:

\begin{verbatim}
    def convertArgsToNumbers():
        total = []
        # start at 1 to skip over program file name
        for i in range(1,len(sys.argv),1):
            num = eval(sys.argv[i])
            total = total + [num]
        return total
\end{verbatim}

The accumulator, total, starts out as the empty list. For each
element of sys.argv beyond the program file name, we convert
it and store the result in num. We then turn that number
into a list (by enclosing it in brackets) and then add it
to the growing list.

With a program file named {\it convert.py} as follows:

\begin{verbatim}
    import sys

    def main():
        ints = convertArgsToNumbers()
        print("original args are",sys.argv[1:])
        print("converted args are",ints)

    def convertArgsToNumbers():
        ...

    main()
\end{verbatim}

we get the following behavior:

\begin{verbatim}
   $ python convert.py 1 34 -2
   original args are ['1', '34', '-2']
   converted args are [1, 34, -2]
\end{verbatim}

Note the absence of quotation marks in the converted list,
signifying that the elements are indeed numbers.

\section{Reading individual items from files}

Instead of reading all of the file at once using the {\it read} function,
we can read it one item at a time. When we read an
item at a time, we always follow this pattern:

\begin{verbatim}
    open the file
    read the first item
    while the read was good
        process the item
        read the next item
    close the file
\end{verbatim}

In Scam, we tell if the read was good by checking the
value of the variable that points to the value read.
Usually, the empty string is used to indicate the
read failed.

\section*{Processing files a line at a time}

Here is another version of the {\it copyFile} function from
\link*{the chapter on input and output}[Chapter~\Ref]{InputAndOutput}.
This version reads and writes
one line at a time. In addition, the function returns
the number of lines processed:

\begin{verbatim}
    def copyFile(inFile,outFile):
        in = open(inFile,"r")
        out = open(outFile,"w")
        count = 0
        line = in.readline()
        while (line != ""):
            out.write(line)
            count += 1
            line = in.readline()
        in.close()
        out.close()
        return count
\end{verbatim}
            
Notice we used the counting pattern.

\section*{Using a Scanner}

A scanner is a reading subsystem that allows you
to read whitespace-delimited tokens from a file.
To get a scanner for Scam, issue this command:

\begin{verbatim}
    wget beastie.cs.ua.edu/cs150/projects/scanner.py
\end{verbatim}

To use a scanner, you will need to import it into your program:

\begin{verbatim}
   from scanner import *
\end{verbatim}

Typically, a scanner is used with a loop. Suppose we wish to count the
number of 
short tokens (a token is a series of characters surrounded by empty space)
in a file. Let's assume a short token is one whose length is less
than or equal to some limit.
Here is a loop that does that:

\begin{verbatim}
    def countShortTokens(fileName):
        s = Scanner(fileName)              #create the scanner
        count = 0
        token = s.readtoken()              #read the first token
        while token != "":                 #check if the read was good
            if (len(token) <= SHORT_LIMIT):
                count += 1
            token = s.readtoken()          #read the next token
        s.close()                          #always close the scanner when done
        return count
\end{verbatim}

Note that the use of the scanner follows the standard reading pattern:
opening (creating the scanner),
making the first read, testing if the read was good, processing
the item read (by counting it), reading the next item, and finally
closing the file (by closing the scanner) after the loop terminates.
Using a scanner always means performing the five steps as given
in the comments.
This code also incorporates the filtered-counting pattern, as expected.

\section{Reading Tokens into a List}

Note that the {\it countShortTokens} function is doing two things, reading
the tokens and also counting the number of short tokens. It is said that this
function has two {\it concerns}, reading and counting. A
fundamental principle of Computer Science is {\it separation of
concerns}.
To separate the concerns, we have one function read the tokens,
storing them into a list (reading and storing is considered to
be a single concern).
We then have another function count the tokens. Thus, we will
have separated the two concerns into separate functions, each
with its own concern.
Here is the reading (and storing) function, which implements
the accumulation pattern:

\begin{verbatim}
    def readTokens(fileName):
        s = Scanner(fileName)              #create the scanner
        items = []
        token = s.readtoken()              #read the first token
        while token != "":                 #check if the read was good
            items = items + [token]
            token = s.readtoken()          #read the next token
        s.close()                          #always close the scanner when done
        return items
\end{verbatim}

Next, we implement the filtered-counting function. Instead of passing
the file name, as before, we pass the list of tokens that
were read:

\begin{verbatim}
    def  countTokens(items):
        count = 0
        for i in range(0,len(items),1)
            if (len(items[i]) <= SHORT_LIMIT):
                count += 1
        return count
\end{verbatim}

Each function is now simpler than the original function. This makes
it easier to fix any errors in a function since you can concentrate
on the single concern implemented by that function.

\section{Reading Records into a List}

Often, data in a file is organized as {\it records}, where
a record is just a collection of consecutive tokens.
Each token in a record is known as a {\it field}.
Suppose every four tokens in a file comprises a record:

\begin{verbatim}
    Smith    President 32  87000
    Jones    Assistant 15  99000
    Thompson    Hacker  2 147000
\end{verbatim}

Typically, we define a function to read one collection of tokens
at a time.
Here is a function that reads a single record:

\begin{verbatim}
    def readRecord(s):                   # we pass the scanner in
        name = s.readtoken()
        if name == "":
            return None                  # no record, returning None
        title = s.readtoken()
        service = eval(s.readtoken())
        salary = eval(s.readtoken())
        return [name,title,service,salary]
\end{verbatim}

Note that we return either a record as a list or None if
no record was read.
Since years of service and salary are numbers, we
convert them appropriately with {\it eval}.

To total up all the salaries, for example, we can use an accumulation
loop (assuming the salary data resides in a file named
{\it salaries}).
We do so by repeatedly calling {\it readrecord}:

\begin{verbatim}
    function totalPay(fileName):
        s = Scanner(fileName)
        total = 0
        record = readRecord(s)
        while (record != None):
            total += record[3]
            record = readRecord(s)
        s.close()            
        print("total salaries:",total)
\end{verbatim}

Note that it is the job of the caller of {\it readRecord} to
create the scanner, repeatedly send the scanner to
{\it readRecord}, and close
the scanner when done.
Also note that we tell if the read was good by checking
to see if {\it readRecord} return {\tt None}.

The above function has two stylistic flaws. It uses those
magic numbers we read about in 
\link*{the chapter on assignment}[Chapter~\Ref]{Assignment}.
It is not clear from the code that the field at index
three is the salary.
To make the code more readable, we can set up some ``constants''
in the global scope (so that they will be visible everywhere):
The second issue is that
that the function has two concerns (reading and accumulating).
We will fix the magic number problem first.

\begin{verbatim}
    NAME = 0
    TITLE = 1
    SERVICE = 2
    SALARY = 3
\end{verbatim}

Our accumulation loop now becomes:

\begin{verbatim}
    total = 0
    record = readRecord(s)
    while record != None:
        total += record[SALARY]
        record = readRecord(s)
\end{verbatim}

We can also rewrite our {\it readRecord} function so
that it only needs to know the number of fields:

\begin{verbatim}
    def readRecord(s):                   # we pass the scanner in
        name = s.readtoken()
        if name == "":
            return None                  # no record, returning None
        title = s.readtoken()
        service = eval(s.readtoken())
        salary = eval(s.readtoken())

        # create an empty record

        result = [0,0,0,0]               

        # fill out the elements

        result[NAME] = name
        result[TITLE] = title
        result[SERVICE] = service
        result[SALARY] = salary

        return result
\end{verbatim}

Even if someone changes the constants to:

\begin{verbatim}
    NAME = 3
    TITLE = 2
    SERVICE = 1
    SALARY = 0
\end{verbatim}

The code still works correctly. Now, however, the salary resides
at index 0, but the accumulation loop is still accumulating the
salary due to its use of the constant to access the salary.

\section{Creating a List of Records}

We can separate the two concerns of the {\it totalPay} function
by having one function read the records into a list
and having another total up the salaries.
A list of
records is known as a {\it table}.
Creating the table
is just like accumulating the salary, but instead
we accumulate the entire record into a list:

\begin{verbatim}
    def readTable(fileName):
        s = Scanner(fileName)
        table = []
        record = readRecord(s)
        while record != None:
            table += [record]       #brackets around record!
            record = readRecord(s)
        s.close()            
\end{verbatim}

Now the table holds all the records in the file.
We must remember to enclose the record in square brackets 
before we accumulate it into the growing table. The
superior student will try this code without the brackets
and ascertain the difference.

The accumulation function is straightforward:

\begin{verbatim}
    def totalPay(fileName):
        table = readTable(fileName)
        total = 0
        for i in range(0,len(table),1):
            record = table[i]
            total += record[SALARY]
        return total
\end{verbatim}

We can simply this function by removing the temporary variable
{\it record}:

\begin{verbatim}
    def totalPay(fileName):
        table = readTable(fileName)
        total = 0
        for i in range(0,len(table),1):
            total += table[i][SALARY]
        return total
\end{verbatim}

Since a table is just a list, so we can walk it, accumulate
items in each record (as we just did with salary), filter it and so on.

\section{Other Scanner Methods}

A scanner object has other methods for reading. They are

\begin{description}
\item[{\tt readline()}]
    read a line from a file, like Scam's {\it readline}.
\item[{\tt readchar()}]
    read the next non-whitespace character
\item[{\tt readrawchar()}]
    read the next character, whitespace or no
\item[{\tt readstring()}]
    read a string - if a string is not pending, '' is returned
\item[{\tt readint()}]
    read an integer - if an integer is not pending, '' is returned
\item[{\tt readfloat()}]
    read a floating point number - if an float is not pending, '' is returned
\end{description}

You can also use a scanner to read from the keyboard. Simply
pass an empty string as the file name:

\begin{verbatim}
    s = Scanner("")
\end{verbatim}

You can scan tokens and such from a string as well by first
creating a keyboard scanner, and then setting the input
to the string you wish to scan:

\begin{verbatim}
    s = Scanner("")
    s.fromstring(str)
\end{verbatim}
