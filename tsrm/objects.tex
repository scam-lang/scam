\chapter{Objects}
\label{Objects}

In the Scam world, an object is a simple a collection of related variables.
You've already been exposed to objects, although you may not
have realized it. When you created a variable, you modified
the environment, which is an object. When you defined a
function, you created 
an object. To view an object, we use the predefined function
{\it pp}.\footnote{
The {\it pp} in the function names stands for {\it pretty print} which
means to print out something so it is 'pretty looking'.
} Evaluating the code:

\begin{verbatim}
    (define (square x)
        (* x x)
        )
    
    (pp square)
\end{verbatim}

yields something similar to the following output:

\begin{verbatim}
    <object 10435>
             __label  : closure
           __context  : <object 10381>
                name  : square
          parameters  : (x)
                code  : (begin (* x x))
\end{verbatim}

We see that the {\it square} function is made up of five {\it fields}.\footnote{
Some people use the term {\it component} or {\it instance variable} instead
of {\it field}.
Also, if you try this, you may see different numbers than 10435 and 10381.
These numbers represent the address of the object in memory.
}
These fields are: {\it \_\_label}, {\it \_\_context}, {\it name},
{\it parameters},
and {\it code}.

Usually, an object lets you look at its individual components.
For example:

\begin{verbatim}
    (println "square's formal parameters are: " (dot square parameters))
\end{verbatim}

yields:

\begin{verbatim}
    square's formal parameters are: (x)
\end{verbatim}
    
We use the function {\it dot} (usually called the {\it dot operator}) to extract
the fields of an object. The first argument to dot is the object, while
the second is the field name. The field name does not have to be quoted.

It is easy to create your own objects. First you must make a
{\it constructor}. A constructor is just a function that returns 
the predefined variable {\it this}. Suppose you want a constructor to
create an object with fields {\it name} and {\it age}. Here is 
one possibility:

\begin{verbatim}
    (define (person)
        (define name)
        (define age)
        this
        )
\end{verbatim}

We can create an object simply by calling the constructor:

\begin{verbatim}
    (define p (person))
\end{verbatim}

The variable {\it p} now points to a {\it person} object and we
can use {\it p} and the {\it dot} operator to
set the fields of the person object:

\begin{verbatim}
    (assign (dot p name) "Boris")
    (assign (dot p age) 33)
    (inspect (dot p name))
\end{verbatim}

Evaluating this code yields the following output:

\begin{verbatim}
    (dot p name) is Boris
\end{verbatim}

It is often convenient to give initial values to the fields of
an object. Here is another version of {\it person} that allows us
to do just that when we create the object:

\begin{verbatim}
    (define (person name age) this)
        
    (define p (person "Boris" 33))
        
    (inspect (dot p name))
\end{verbatim}

The output is the same:

\begin{verbatim}
    (dot p name) is Boris
\end{verbatim}

In general, if a field is to be initialized when the object
is constructed, make that field a formal parameter. If not,
make the field a locally declared variable.

\section{Adding Methods}

Objects can have methods as well.\footnote{
A method is just another name for a local function.
}
Here's a version of the
{\it person} constructor that has a {\it birthday} method.

\begin{verbatim}
    (define (person name age)
        (define (birthday)
            (println "Happy Birthday, " name "!")
            (++ age)
            )
        this
        )
        
    (define p (person "Boris" 33))
    ((dot p birthday))
    (inspect (dot p age))
\end{verbatim}

The output of this code is:

\begin{verbatim}
    Happy Birthday, Boris!
    (dot p age) is 34
\end{verbatim}

In summary, one turns a function into a constructor by making returning

\begin{verbatim}
    this
\end{verbatim}

from a function. The local variables, including
formal parameters, become the fields of the function while
any local functions serve as methods.

\section{Objects and Types}

If you were to ask an object, "What are you?", most
would respond, "I am an environment!". The {\it type} function is
used to ask such questions:

\begin{verbatim}
    (define p (person "betty" 19))
    (inspect (type p))
\end{verbatim}

yields:

\begin{verbatim}
    (type p) is environment
\end{verbatim}

This is because the predefined variable {\it this} always points to
the current environment and when we return this from a function
we are returning an environment. Since environments are objects
and vice versa, this is how making objects in Scam is so easy.

While the type function is often useful, we sometimes 
would like to know what kind of specific object an object is.
For example, we might like to 
know whether or not {\it p} is a {\it person} object.
That is to say,
was {\it p} created by the {\it person} function/constructor?.
One way to do this
is to ask the constructor of the object if it is the person function.
Luckily, all objects carry along a pointer to the function
that constructed them:

\begin{verbatim}
    (define p (person "veronica" 20))
    (inspect (dot p __constructor name))
\end{verbatim}

yields:

\begin{verbatim}
    (dot p __constructor name) is person
\end{verbatim}

Note that:
\verb!(dot p __constructor name)!
is equivalent to
\verb!(dot (dot p __constructor) name)!.
So, to ask if {\it p} is a person, we would use the following
expression:

\begin{verbatim}
    (if (and (eq? (type p) 'environment)
             (eq? (dot p __constructor name) 'person)) ...
\end{verbatim}

Since this construct is rather wordy, there
is a simple function, named {\it is?}, that you can use instead:

\begin{verbatim}
    (if (is? p 'person) ...
\end{verbatim}

The {\it is} function works for non-objects too. All of the following
expressions are true:

\begin{verbatim}
    (is? 3 'INTEGER)
    (is? 3.4 'REAL)
    (is? "hello" 'STRING)
    (is? 'blue 'SYMBOL)
    (is? (list 1 2 3) 'CONS)
    (is? (array "a" "b" "c") 'ARRAY)
    (is? (person 'veronica 20) 'object)
    (is? (person 'veronica 20) 'environment)
    (is? (person 'veronica 20) 'person)
\end{verbatim}

\section{A formal view of object-orientation}

Scam is a fully-featured object-oriented language. What does
that mean exactly? Well, to begin with, a programming language
is considered object-oriented if it has these
three features:

\begin{enumerate}
\item
    encapsulation
\item
    inheritance
\item
    polymorphism
\end{enumerate}

Encapsulation in this sense means that a programmer
can bundle {\it data} and {\it methods} into a single entity.
We've seen that a Scam function can
have local variables and local functions.
So, if we consider local variables (including the formal
parameters) as data and local
functions as methods, we see that Scam can encapsulate
in the object-oriented sense.

Inheritance is the ability to use the data and methods
of one kind of object by another as if they were defined
in the other object to begin with.
The next chapter deals with inheritance.

Polymorphism means that an object that inherits appears
to be both kinds of object, the kind of object it is
itself and the kind of object from which it inherits.
We will learn more about
polymorphism in the next chapter as well.

\section{Other objects in Scam}

While environments constitute the bulk of objects in
Scam, two other object types are built into Scam. They are closures
(seen the the first section) and error objects. The
main library adds in a third object type known as a thunk.

An error object is generated when an exception is caught.
The fields of an error object are code, value, and trace.
A thunk is an expression and an environment bundled
together. Thunks are used to delay evaluation of
an expression for a later time. The fields of a thunk
are {\it code} and {\it \_\_context}.
You can learn more about error objects and thunks
in subsequent chapters.

\section{Fun with {\it objects}}

Because of the flexibility of Scam, one can
add Java-like display behavior to objects.
In Java, if an object has a method named {\it toString},
then if one attempts to print the object, the {\it toString}
method is called to generate the print value of the object.

Here is an example:

\begin{verbatim}
    (define (person name age)
        (define (birthday)
            (println "Happy Birthday, " name "!")
            (++ age)
            )
        (define (toString)
            (string+ name "(age " age ")")
            )
        this
        )

    (define p (person "boris" 33))
\end{verbatim}

Given the above definition of {\it person}, printing the value of {\it p}:

\begin{verbatim}
    (println p)
\end{verbatim}

yields:

\begin{verbatim}
    <object 23452>
\end{verbatim}

Uh oh. The {\it toString} method wasn't called! This is because
the current version of {\it println}, defined in {\it main.lib},
does not understand the {\it toString}
method. No problem here, we'll just reassign {\it display},
since {\it println} calls {\it print} and {\it print} calls {\it display}
to do the actual printing:

\begin{verbatim}
    (define (display  item)
        (if (and (object? item) (local? 'toString item))
            (__display ((dot item toString)))
            (__display item)
            )
        )
\end{verbatim}

The main library binds the original version of {\it display}
to the symbol {\it \_\_display} for
safe-keeping.

Note that this new version of {\it display} is only found in the local
environment, so the current versions of {\it println} and {\it print}, defined
in an outer scope, cannot see the new version of {\it display} in the current
scope.
To do so would be a scope violation.
We solve this problem by {\it cloning} the two printing functions.
The process of cloning produces new closures with 
the local environment as the definition environment.
In all other respects, the cloned function is identical.
Thus, when the new {\it print} calls {\it display}, the new, local version 
will be found.

\begin{verbatim}
    (include "reflection.lib")

    (define print (clone print))
    (define println (clone println))

    (println p)
\end{verbatim}

The {\it reflection} library must be included
to access the {\it clone} function.

Now, printing the object {\it p} yields:

\begin{verbatim}
    boris (age 33)
\end{verbatim}
