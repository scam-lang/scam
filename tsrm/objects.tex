\chapter{Objects}
\label{Objects}

In the Scam world, an object is simply a collection of related variables.
You've already been exposed to objects, although you may not
have realized it. When you created a variable, you modified
the environment, which is an object. When you defined a
function, you created 
an object. To view an object, we use the predefined function
{\it pp}.\footnote{
The {\it pp} in the function names stands for {\it pretty print} which
means to print out something so it is 'pretty looking'.
} Evaluating the code:

\begin{verbatim}
    (define (square x)
        (* x x)
        )
    
    (pp square)
\end{verbatim}

yields something similar to the following output:

\begin{verbatim}
    <object 10435>
             __label  : closure
           __context  : <object 10381>
                name  : square
          parameters  : (x)
                code  : (begin (* x x))
\end{verbatim}

We see that the {\it square} function is made up of five {\it fields}.\footnote{
Some people use the term {\it component} or {\it instance variable} instead
of {\it field}.
Also, if you try this, you may see different numbers than 10435 and 10381.
These numbers represent the address of the object in memory.
}
These fields are: {\it \_\_label}, {\it \_\_context}, {\it name},
{\it parameters},
and {\it code}.

Usually, an object lets you look at its individual components.
For example:

\begin{verbatim}
    (println "square's formal parameters are: " (get 'parameters square))
\end{verbatim}

yields:

\begin{verbatim}
    square's formal parameters are: (x)
\end{verbatim}
    
We use the function {\it get} to extract
the fields of an object. The first argument to {\it get} is an expression that
should resolve to a symbol to be gotten,
while
the second is the object holding the field. 

\section{Creating objects}

It is easy to create your own objects. First you must make a
{\it constructor}. A constructor is just a function that returns 
the predefined variable {\it this}. Suppose you want a constructor to
create an object with fields {\it name} and {\it age}. Here is 
one possibility:

\begin{verbatim}
    (define (person)
        (define name)
        (define age)
        this
        )
\end{verbatim}

We can create an object simply by calling the constructor:

\begin{verbatim}
    (define p (person))
\end{verbatim}

The variable {\it p} now points to a {\it person} object and we
can use {\it p} and the {\it set} operator to
set the fields of the person object:

\begin{verbatim}
    (set 'name "Boris" p)
    (set 'age 33 p)
    (inspect (get 'name p))
\end{verbatim}

The {\it set} function takes three arguments. The first should evaluate
to the field name, the second to the new field value, and the third
to the object to update.
Evaluating this code yields the following output:

\begin{verbatim}
    (get 'name  p) is Boris
\end{verbatim}

Scam allows an alternative form for getting values from an object.
This form uses function call syntax. For example, the following two expressions
are equivalent:

\begin{verbatim}
    (get 'name p)
    (p 'name)
\end{verbatim}

The advantage of the latter form is that if the retrieved field is
itself an object, you can chain field names together.
For example if {\it n} points
to a linked list:

\begin{verbatim}
    (define (node value next) this)
    (define n (node 'a (node 'b (node 'c nil))))
\end{verbatim}
    
you can get the value of the third node in the
list with
either expression:

\begin{verbatim}
    (get 'value (get 'next (get 'next n)))
    (n 'next 'next 'value)
\end{verbatim}

To set the value of a field using a similar syntax, one calls the {\it set*}
function. Here, both calls set the value of the third node in the list to 5:

\begin{verbatim}
    (set 'value 5 (get 'next (get 'next n)))
    (set* n 'next 'next 'value 5)
\end{verbatim}

Note that for {\it set}, the object is the last argument while for {\it set*},
the object is the first.
    
\section{Initializing fields}

It is often convenient to give initial values to the fields of
an object. Here is another version of {\it person} that allows us
to do just that when we create the object:

\begin{verbatim}
    (define (person name age) this)
        
    (define p (person "Boris" 33))
        
    (inspect (p 'name))
\end{verbatim}

The output is the same:

\begin{verbatim}
    (p 'name) is Boris
\end{verbatim}

In general, if a field is to be initialized when the object
is constructed, make that field a formal parameter. If not,
make the field a locally declared variable.

\section{Adding Methods}

Objects can have methods as well.\footnote{
A method is just another name for a local function.
}
Here's a version of the
{\it person} constructor that has a {\it birthday} method.

\begin{verbatim}
    (define (person name age)
        (define (birthday)
            (println "Happy Birthday, " name "!")
            (++ age)
            )
        this
        )
        
    (define p (person "Boris" 33))
    ((p 'birthday))
    (inspect (p 'age))
\end{verbatim}

The output of this code is:

\begin{verbatim}
    Happy Birthday, Boris!
    (p 'age) is 34
\end{verbatim}

In summary, one turns a function into a constructor by returning
{\it this}
from a function. The local variables, including
formal parameters, become the fields of the function while
any local functions serve as methods.

\section{The variable {\it this} is not {\bf this}}

It is tempting to think that the predefined variable {\it this} is
equivalent to {\bf this} in Java.
In Scam, {\it this} always refers to the current scope/environment,
while in Java, {\bf this} refers to the current object. The difference
is noticable in object methods. For example, consider the following
constructor:

\begin{verbatim}
    (define (f x)
        (define (g y)
            this
            )
        this
        )
\end{verbatim}

If we create an {\it f} object:

\begin{verbatim}
    (define a (f 0))
\end{verbatim}

and then save the return value of the {\it g} method:

\begin{verbatim}
    (define b ((f 'g) 1))
\end{verbatim}
    
then {\it a} and {\it b} are not the same objects; {\it b} would
be a `sub-object', as it were, of {\it a}. In fact:

\begin{verbatim}
    (eq? a b)
\end{verbatim}

returns false, while:

\begin{verbatim}
    (eq? a (b '__context))
\end{verbatim}

evaluates to true. To have the {\it g} method return a pointer to the object
created by {\it f}, we would have {\it g} return its context:

\begin{verbatim}
    (define (f x)
        (define (g y)
            __context
            )
        this
        )
\end{verbatim}

\section{Objects and Types}

If you were to ask an object, "What are you?", most
would respond, "I am an environment!". The {\it type} function is
used to ask such questions:

\begin{verbatim}
    (define p (person "betty" 19))
    (inspect (type p))
\end{verbatim}

yields:

\begin{verbatim}
    (type p) is environment
\end{verbatim}

This is because the predefined variable {\it this} always points to
the current environment and when we return {\it this} from a function,
we are returning an environment. Because environments are objects
and vice versa, making objects in Scam is quite easy.

While the {\it type} function is often useful, we sometimes 
would like to know what kind of specific object an object is.
For example, we might like to 
know whether or not {\it p} is a {\it person} object.
That is to say,
was {\it p} created by the {\it person} function/constructor?.
One way to do this
is to ask the constructor of the object if it is the {\it person} function.
Luckily, all objects carry along a pointer to the function
that constructed them:

\begin{verbatim}
    (define p (person "veronica" 20))
    (inspect (p '__constructor 'name))
\end{verbatim}

yields:

\begin{verbatim}
    (p '__constructor 'name) is person
\end{verbatim}

So, to ask if {\it p} is a person, we would use the following
expression:

\begin{verbatim}
    (if (and (eq? (type p) 'environment)
             (eq? (p '__constructor 'name) 'person)) ...
\end{verbatim}

Since this construct is rather wordy, there
is a simple function, named {\it is?}, that you can use instead:

\begin{verbatim}
    (if (is? p 'person) ...
\end{verbatim}

The {\it is} function works for non-objects too. All of the following
expressions are true:

\begin{verbatim}
    (is? 3 'INTEGER)
    (is? 3.4 'REAL)
    (is? "hello" 'STRING)
    (is? 'blue 'SYMBOL)
    (is? (list 1 2 3) 'CONS)
    (is? (array "a" "b" "c") 'ARRAY)
    (is? (person 'veronica 20) 'object)
    (is? (person 'veronica 20) 'environment)
    (is? (person 'veronica 20) 'person)
\end{verbatim}

\section{Other objects in Scam}

While environments constitute the bulk of objects in
Scam, two other object types are built into Scam. They are closures
(seen the the first section) and error objects. The
main library adds in a third object type known as a thunk.

An error object is generated when an exception is caught.
The fields of an error object are code, value, and trace.
A thunk is an expression and an environment bundled
together. Thunks are used to delay evaluation of
an expression for a later time. The fields of a thunk
are {\it code} and {\it \_\_context}.
You can learn more about error objects and thunks
in subsequent chapters.

\section{Fun with {\it objects}}

Because of the flexibility of Scam, one can
add Java-like display behavior to objects.
In Java, if an object has a method named {\it toString},
then if one attempts to print the object, the {\it toString}
method is called to generate the print value of the object.

Here is an example:

\begin{verbatim}
    (define (person name age)
        (define (birthday)
            (println "Happy Birthday, " name "!")
            (++ age)
            )
        (define (toString)
            (string+ name "(age " age ")")
            )
        this
        )

    (define p (person "boris" 33))
\end{verbatim}

Given the above definition of {\it person}, printing the value of {\it p}:

\begin{verbatim}
    (println p)
\end{verbatim}

yields:

\begin{verbatim}
    <object 23452>
\end{verbatim}

Uh oh. The {\it toString} method wasn't called! This is because
the current version of {\it println}, defined in {\it main.lib},
does not understand the {\it toString}
method. No problem here, we'll just reassign {\it display},
since {\it println} calls {\it print} and {\it print} calls {\it display}
to do the actual printing:

\begin{verbatim}
    (define (display  item)
        (if (and (object? item) (local? 'toString item))
            (__display ((item 'toString)))
            (__display item)
            )
        )
\end{verbatim}

The main library binds the original version of {\it display}
to the symbol {\it \_\_display} for
safe-keeping.

Note that this new version of {\it display} is only found in the local
environment, so the current versions of {\it println} and {\it print}, defined
in an outer scope, cannot see the new version of {\it display} in the current
scope.
To do so would be a scope violation.
We solve this problem by {\it cloning} the two printing functions.
The process of cloning produces new closures with 
the local environment as the definition environment.
In all other respects, the cloned function is identical.
Thus, when the new {\it print} calls {\it display}, the new, local version 
will be found.

\begin{verbatim}
    (include "reflection.lib")

    (define print (clone print))
    (define println (clone println))

    (println p)
\end{verbatim}

The {\it reflection} library must be included
to access the {\it clone} function.

Now, printing the object {\it p} yields:

\begin{verbatim}
    boris (age 33)
\end{verbatim}
