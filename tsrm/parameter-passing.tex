\chapter{Parameter Passing}
\label{ParameterPassing}

There are (at least) six historical and current
methods of passing arguments to a function when a function call
is made. They are:

\begin{itemize}
\item
    {\it call-by-value}
\item
    {\it call-by-reference}
\item
    {\it call-by-value-result}
\item
    {\it call-by-name}
\item
    {\it call-by-need}
\item
    {\it generalized delayed evaluation}
\end{itemize}

Let's examine these six methods in turn. After which, we will
investigate variadic functions in Scam.

\section*{Call-by-value}

This method is the only method of parameter passing allowed by C, Java,
Scam, and Scheme. In this method, the formal parameters are set up as local
variables that contain the value of the expressions that were passed as
arguments to the function. Changes to local variables are not reflected
in the actual arguments. For example, an attempt to define a function
for exchanging the values
of two variables passed to it might look like:

\begin{verbatim}
    (define (swap a b)
        (define temp a)
        (set! a b)
        (set! b temp)
        )
\end{verbatim}

Consider this code which uses {\it swap}:

\begin{verbatim}
    (define x 3)
    (define y 4)

    (swap x y)

    (inspect x)
    (inspect y)
\end{verbatim}

Under {\it call-by-value},
this function would not yield the intended semantics.
The output of the above code is:

\begin{verbatim}
    x is 3
    y is 4
\end{verbatim}

This is because only the values of the local variables {\it a} and {\it b}
were swapped; the variables {\it x} and {\it y} remain unchanged
as only their values were passed to the swapping function.
In
general, one cannot get a swap routine to work under {\it call-by-value}
unless the addresses of the variables are somehow sent. One way of
using addresses is to pass an array (in C and Scam, when an array name
is used as an argument, the address of the first element is sent. In Java,
the address of the array
object is sent). For example,
the code fragment:

\begin{verbatim}
    (define x (array 1))
    (define y (array 0))

    (swap x y)  ;address of beginning element is sent

    (println "x[0] is " (getElement x 0) " and y[0] is " (getElement y 0))
\end{verbatim}

with {\it swap} defined as...

\begin{verbatim}
    (define (swap a b)
        (define temp (getElement a 0))
        (setElement a 0 (getElement b 0))
        (setElement b 0 temp)
        )
\end{verbatim}

would print out:

\begin{verbatim}
    x[0] is 0 and y[0] is 1
\end{verbatim}

In this case, the addresses of arrays {\it x} and {\it y}
are stored in the local variables {\it a} and
{\it b}, respectively.
This is
still call-by-value since even if the address stored in {\it a}, for example,
is modified,
{\it x} still "points" to the same array as before. Here is an example:

\begin{verbatim}
    (define (change a)
       (assign a (array 13))
       )

    (define x (array 42))

    (inspect (getElement x 0))
\end{verbatim}

yields:

\begin{verbatim}
    (getElement x 0) is 42)
\end{verbatim}

Note that C has an operator that extracts the address of a variable, the
\& operator. By using \&, one can write a swap in C that does
not depend on arrays:

\begin{verbatim}
    void swap(int *a,int *b)
        {
        int temp;
        temp = *a;
        *a = *b;
        *b = temp;
        }
\end{verbatim}

The call to {\it swap} would look like:

\begin{verbatim}
    int x = 3;
    int y = 4;

    swap(&x,&y);

    printf("x is %d and y is %d\n",x,y);
\end{verbatim}

with output:

\begin{verbatim}
    x is 4 and y is 3
\end{verbatim}

as desired.

Note that this is still {\it call-by-value} since the {\it value} of the
address of {\it x} (and {\it y}) is being passed to the swapping function.

\section*{Call-by-reference}

This second method differs from the first in that changes to
the formal parameters during execution of the function body are
immediately reflected in actual arguments. Both C++ and Pascal allow
for call-by-reference. Normally, this is accomplished,
under the hood, by passing the
address of the actual argument (assuming it has an address) rather than
the value of the actual argument. References to the analogous formal
parameter are translated to references to the memory location stored in
the formal parameter. In C++, {\it swap} could be defined as:

\begin{verbatim}
    void swap(int &a, int &b) // keyword & signifies
        {                     // call-by-reference
        int temp = a;
        a = b;
        b = temp;
        }
\end{verbatim}

Now consider the code fragment:

\begin{verbatim}
    var x = 3;  //assume x at memory location 1000
    var y = 4;  //assume y at memory location 1008

    //location 1000 holds a 3
    //location 1008 holds a 4

    swap(x,y);
    cout << "x is " << x << " and y is " << y << "\n";
\end{verbatim}

When the swapping function starts executing, the value 1000 is stored
in the local variable {\it a} and 1008 is stored in local variable {\it b}.
The line:

\begin{verbatim}
    temp = a;
\end{verbatim}

is translated, not into store the value of {\it a} (which is 1000) in
variable {\it temp}, but rather store the value at memory location 1000 (which
is 3) in variable {\it temp}. Similar translations are made for the remaining
statements in the function body. Thus, the code fragment prints out:

\begin{verbatim}
    x is 4 and y is 3
\end{verbatim}

The swap works! When trying to figure out what happens under
{\it call-by-reference}, it is often useful to draw pictures of the various
variables and their values and locations, then update them as the function
body executes.

It is possible to simulate {\it call-by-reference} in Scam
with delayed evaluation.

\section*{Call-by-value-result}

This method is a combination of the first two. Execution of the function
body proceeds as in {\it call}-by-value. That is, no updates of the actual
arguments are made. However, after execution of the body, but just before
the function returns, the actual arguments are updated with the final
values of their associated formal parameters. This method of parameter
passing is often used for Ada in-out parameters. Would swap work under
call-by-value-result? 

Like {\it call-by-reference},
it is possible to simulate {\it call-by-value-result} in Scam
with delayed evaluation.

\section*{Call-by-name}

{\it Call-by-name} was used in Algol implementations. In essence, functions
are treated as macros. Under {\it call-by-name}, the fragment:

\begin{verbatim}
    (define (swap a b)
        (define temp a)
        (set! a b)
        (set! b temp)
        )

    (define x 3)
    (define y 4)

    (swap x y)

    (println "x is "  x " and y is " y)
\end{verbatim}

...would be translated into:

\begin{verbatim}
    (define (swap a b)
        (define temp a)
        (set! a b)
        (set! b temp)
        )

    (define x 3)
    (define y 4)

    ;substitute the body of the function for the call, 
    ;renaming the references to formal parameters with the names of 
    ;the actual args

    (scope
        (define temp x)
        (assign x y)
        (assign y temp)
        )

    (println "x is "  x " and y is " y)
\end{verbatim}

Under {\it call-by-name}, the {\it swap} works as desired,
so why is {\it call-by-name}
a method that has fallen into relative disuse? One reason is complexity.
What happens if a local parameter happens to have the same name as one of
the actual args. Suppose {\it swap} had been written as:

\begin{verbatim}
    (define (swap a b)
        (define x a) 
        (set! a b)
        (set! b x)
        )
\end{verbatim}

Then a naive substitution and renaming would have produced:

\begin{verbatim}
    (scope
        (define x x)
        (assign x y)
        (assign y x)
        )
\end{verbatim}

which is surely incorrect.
Further problems occur if the body of the function references
globals which have been shadowed in the calling function. This requires
a complicated renaming scheme. Finally, {\it call-by-name} makes treating
functions as first-class objects problematic (being difficult to recover
the static environment of the called function). {\it Call-by-name}
exists today in C++, where it is possible to {\it inline} function calls
for performance reasons, and in macro processors.

\section*{Call-by-need}

In {\it call-by-value}, the arguments in a function call are evaluated and
the results are bound to the formal parameters of the function. In
{\it call-by-need}, the arguments themselves are literally bound
to the formal
parameters, as in {\it call-by-name}. A major difference is
that the calling environment is also bound to the formal
parameters as well. This bundle of literal argument and 
evaluation environment is known as a {\it thunk}.
The actual values of the arguments
are determined only when such values are
needed; when such a need occurs, the thunk is
evaluated, causing the literal argument
in the thunk to be
evaluated in the stored (calling) environment.
For example, consider this code:

\begin{verbatim}
    (define z 5)
    (f (+ z 3)) 
\end{verbatim}

with {\it f} defined as:

\begin{verbatim}
    (define (f x)
        (define y x)  ;x needed! x is fixed to 8 under call-by-need
        (set! z (* z 2))
        (+ x y)       ;x needed! x was already evaluated under call-by-need
        )
\end{verbatim}

in the same scope as {\it z}.
Under {\it call-by-name}, the return value is 21, but under
{\it call-by-need}, the return value is 16.
This is because the
value of {\it z} changed {\it after} the point when the value of {\it x}
(really \verb!(+ z 3)! was needed and the value of {\it x} was fixed from
the prior evaluation of {\it x}. Under {\it call-by-name}, the second
reference to {\it x} causes a fresh, new evaluation of {\it z},
the yielding the result of 21.

{\it Call-by-need}
is exactly the method used to implement streams in the 
textbook
{\it The Structure and Interpretation of Computer Programs}.
It is important to remember that the evaluation of a
{\it call-by-need} argument is done only once, 
with the result stored in the thunk for future requests.

\section*{Generalized delayed evaluation}

{\it Call-by-need} illustrates an example of delayed evaluation.
An unevaluated argument is stored in the thunk along with
its execution environment. The execution environment
is also known as the {\it calling environment},
due to the fact that the execution environment is where the
function call was made. Generalized delayed evaluation
loosens the restrictions of call-by-need in that the
function definer controls which arguments are delayed and
whether or not a fresh evaluation of a delayed argument
is needed.
Differences between {\it call-by-need} and
{\it generalized delayed evaluation}
arise when a variable making up a delayed argument 
experiences a change of state in between references in
the function body.

\section*{Simulating {\it call-by-reference} in Scam}

It is possible to simulate {\it call-by-reference} in Scam by 
delaying the evaluation of function call arguments and
manipulating the calling environment.
Before that can happen, however, the called function needs to
be able to access the calling environment.

To obtain the
calling environment in Scam,
one simply adds a formal parameter with the name {\it \#}.
The calling environment is then passed silently to the
function when a function call is made. This means that
the {\it \#} formal parameter is not matched to any actual argument
and can appear anywhere in the parameter list (except after
the variadic parameters {\it @} and {\it \$} - more on them later).

In addition to grabbing a handle to the calling environment,
a swapping function also needs to delay the evaluation of
the variables passed in.
One delays the evaluation of an argument by naming the formal
parameter matched to the argument in a special way. If the
formal parameter name begins with a {\it \$}, then
the corresponding argument is delayed.

With the ability to grab the calling environment,
delay the evaluation of arguments, and access the
bindings in an environment (see the chapter on Objects),
we can now define a {\it swap} function that works as intended.

\begin{verbatim}
    (define (swap # $a $b)
        (define temp (get $a #))
        (set $a (get $b #) #)
        (set $b temp #)
        )
\end{verbatim}

Note that the local variables {\it \$a} and {\it \$b} are regular variables;
they happen to point to unevaluated fragments
of Scam code.
Note also that {\it set} was used instead of {\it set!}
to change variable values.
If {\it set!} had been used, only the value of the local
variables {\it \$a} and {\it \$b} would change.
Unlike {\it set!}, {\it set} evaluates all its arguments,
so the values of {\it \$a} and {\it \$b} are passed to {\it set}, rather than
their names.
Finally note, that we pass the calling environment, {\it \#}, as the third
parameter to set, so that changes to the symbols to which
{\it \$a} and {\it \$b}
are bound
happen in the calling environment.

\section{Variadic functions}

A variadic function is a function that can take a different
number of arguments from call to call.
Scam allows this via two special formal parameter names.
They are {\it @} and {\it \$}.
If the last formal parameter is {\it @}, then all remaining
(evaluated) arguments not matched to any previous formal
parameters are gathered up in a list and {\it @} is bound to this
list.  For example, consider these definitions:

\begin{verbatim}
    (define (variadic first @)
        (println "the first argument is " first)
        (println "the remaining arguments are:")
        (while (valid? @)
            (println "    " (car @))
            (set! @ (cdr @))
            )
        )
    (define x 1)
    (define y 2)
\end{verbatim}

The call \verb!(variadic x)! produces:

\begin{verbatim}
    the first argument is 1
    the remaining arguments are:
\end{verbatim}

while the call \verb!(variadic x y (+ x y))! produces:

\begin{verbatim}
    the first argument is 1
    the remaining arguments are:
         2
         3
\end{verbatim}

Similar to {\it @,} the formal parameter {\it \$} is expected to be the
last formal parameter. The difference is that the arguments
bundled up into a list
are delayed. Suppose one replaced all occurrences of {\it @} with
{\it \$} in the definition of {\it variadic}.
Then, the call \verb!(variadic x y (+ x y))! would produce:

\begin{verbatim}
    the first argument is x
    the remaining arguments are:
        x
        (+ x y)
\end{verbatim}
