\chapter{Precedence and Associativity}
\label{PrecedenceAndAssociativity}

\section{Precedence}

Precedence (partially) describes the order in which operators, in
an expression involving different operators, are evaluated. In Scam,
the expression

\begin{verbatim}
    3 + 4 < 10 - 2
\end{verbatim}

evaluates to true. In particular, {\tt 3 + 4} and {\tt 10 - 2}
are evaluated before the {\tt <}, yielding {\tt 7 < 8},
which is indeed true. This implies that {\tt +} and
{\tt -} have higher precedence than {\tt <}. If
{\tt <} had higher precedence, then {\tt 4 < 10}
would be evaluated first, yielding {\tt 3 + :true - 2}, which
is nonsensical.

Note that precedence is only a partial ordering. We cannot tell, for
example whether {\tt 3 + 4} is evaluated before the {\tt 10 - 2},
or vice versa. Upon close examination, we see that it does not
matter which is performed first as long as both are performed before
the expression involving {\tt <} is evaluated.

It is common to
assume that the left operand is evaluated before the right operand. For
the {\sc boolean} connectives {\tt and} and {\tt or},
this is indeed true. But for other operators, such an assumption
can lead you into trouble. You will learn why later. For now, remember
never, never, never depend on the order in which operands are evaluated!

The lowest precedence operator in Scam is the assignment operator which
is described later. Next come the {\sc boolean} connectives {\tt and}
and {\tt or}.  At the next higher level are the {\sc boolean}
comparatives, {\tt <}, {\tt <=}, {\tt >},
{\tt >=}, {\tt ==}, and {\tt !=}.  After
that come the additive arithmetic operators {\tt +}
and{\tt -}. Next comes the
multiplicative operators
{\tt *}, {\tt /} and {\tt \%}.
Higher still is the exponentiation operator {\tt **}.
Finally, at the highest level of precedence is the selection,
or {\it dot}, operator (the dot operator is a period or full-stop). Higher
precedence operations are performed before lower precedence operations.
Functions which are called with operator syntax have the same precedence
level as the mathematical operators.

\section{Associativity}

Associativity describes how multiple expressions connected by operators at the same
precedence level are evaluated. All the operators, with the exception of
the assignment and exponentiation operators, are left associative.
For example the expression
5 - 4 - 3 - 2 - 1 is equivalent to ((((5 - 4) - 3) - 2) - 1). For
a left-associative structure, the equivalent, fully parenthesized,
structure has open parentheses piling up on the left. If the minus
operator was right associative, the equivalent expression would be
(5 - (4 - (3 - (2 - 1)))), with the close parentheses piling up on
the right. For a commutative operator, it does not matter whether it
is left associative or right associative. Subtraction, however, is not
commutative, so associativity does matter. For the given expression, the
left associative evaluation is -5. If minus were right associative, the
evaluation would be 3. 
