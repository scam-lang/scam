\chapter{Scam Programs and Using Files}
\label{ScamPrograms}

After a while, it gets rather tedious to
cut and paste into the Scam interpreter.
A more efficient method is to store
your program in a text file and then 
load the file.

I use {\it vim} as my text editor. {\it Vim} is
an editor that was written by programmers
for programmers ({\it emacs} is another such
editor) and serious Computer Scientists
and Programmers should learn {\it vim} (or {\it emacs}).

\section{Your first program}

Create a text file named {\it hello.py}. The
name really doesn't matter and doesn't
have to end in {\it .py} (the {\it .py} is a convention
to remind us this file contains Scam source
code). Place in the file:

\begin{verbatim}
print("hello, world!");
\end{verbatim}

Save your work and exit the text editor.

Now execute the following command at the
system prompt (not the Scam interpreter prompt!):

\begin{verbatim}
    python3 hello.py
\end{verbatim}

You should see the phrase:

\begin{verbatim}
    hello, world!
\end{verbatim}

displayed on your console. Here's a trace using Linux:

\begin{verbatim}
    lusth@warka:~$ python3 hello.py
    hello, world!
    lusth@warka:~$
\end{verbatim}

The {\tt lusth@warka:~\$} is my system prompt.

\section{Vim and Scam}

Move into your home directory and list the files found there
with this command:

\begin{verbatim}
    ls -al
\end{verbatim}

If you see the file .exrc, then all is well and good.
If you do not, then run the following command to
retrieve it:

\begin{verbatim}
    (cd ; wget beastie.cs.ua.edu/cs150/.exrc)
\end{verbatim}

This configures {\tt vim} to understand Scam syntax and to color
various primitives and keywords in a pleasing manner.

\section{A Neat Macro}

One of the more useful things you can do is set up a {\it vim} 
macro. Edit the file {\it .exrc} in your home directory and
find these lines:

\begin{verbatim}
    map @ :!python %^M
    map # :!python % 
    set ai sm sw=4
\end{verbatim}

If you were unable to download the file in the previous section,
just enter the lines above in the {\it .exrc} file.

The first line makes the {\tt '@'} key,
when pressed,
run the Scam interpreter on
the file you are currently editing (save your work first before
tapping the {\tt @} key). The \verb+^M+ part of the macro
is not a two character sequence (\verb+^+ followed by {\tt M}),
but a single character made by typing \verb+<Ctrl>-v+ followed by
\verb+<Ctrl>-m+.
It's just when you type \verb+<Ctrl>-v <Ctrl>-m+, it will display as
\verb+^M+.
The second line defines a similar macro that pauses to let you enter
command-line arguments to your Scam program.
The third line sets some useful parameters:
{\it autoindent} and {\it showmatch}.
The expression {\tt sw=4} sets the indentation to four spaces.

\section{Writing Scam Programs}

A typical Scam program is composed of two sections. The first
section is composed of variable and function definitions.
The next section is composed of statements, which are Scam
expression. Usually the latter section is reduced to
a single function call (we'll see an example in a bit).

The {\it hello.py} file above was a program with no
definitions and a single statement. A Scam program composed
only of definitions will usually run with no output to the
screen. Such programs are usually written for the express
purpose of being included
into other programs.

Typically, one of the function definitions is a function
named {\it main} (by convention); this function takes no
arguments.  The last line of the program (by convention)
is a call to {\it main}.
Here is a rewrite of {\it hello.py} using that convention.

\begin{verbatim}
def main():
    println("hello, world!")

main()
\end{verbatim}

This version's output is exactly the same as the previous version.
We also can see that {\it main} implements the {\it procedure patterrn} since
it has no explicit return value.

\section{Order of definitions}

A function (or variable) must be created or defined\footnote{
From now on, we will use the word defined.}
before it is used. This 
program will generate an error:

\begin{verbatim}
main()           #undefined variable error

def main():
    y = 3
    x = y * y
    print("x is",x)

\end{verbatim}

since {\it main} can't be called  until it is defined.
This program is legal, however:

\begin{verbatim}
def main():
    x = f(3)
    print("x is",x)
def f(z):
    return z * z
main()
\end{verbatim}

because even though the body of {\it main} refers to
function {\it f} before function {\it f} is defined,
function {\it f} is defined by the time
function {\it main} is called (the last statement of the program).

Don't be alarmed if you don't understand what is going on with
this program. But you should be able to type this program into
a file and then run it. If you do so, you should see the
output:

\begin{verbatim}
    x is 9
\end{verbatim}

\section{Importing code}

One can include one file of Scam code into another.
The included file is known as a module. The {\it import}
statement is used to include a module

\begin{verbatim}
from moduleX.py import *
\end{verbatim}

where {\it moduleX.py} is the name of the file containing the Scam definitions
you wish to include.
Usually import statements are placed at the top of the file.
Including a module imports all the code
in that module, as if you had written it 
the file yourself.

If {\it moduleX.py} has import statements, those modules will be
included in the file as well.

Import statements often are used include the standard Scam libraries.
