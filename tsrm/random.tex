\chapter{Randomness}
\label{Randomness}

Scam has a few functions for generating pseudo-random numbers.
Although the word {\it randomly} is used in the chapter,
{\it pseudo-randomly}
would be a more accurate term.

\section{Built-in Random Functions}

The built-in functions are
{\it randomInt}, {\it randomMax}, and {\it randomSeed}.

\subsection*{{\tt randomMax()}}

This function, which takes no arguments, returns the largest
integer
that can be returned from the {\it randomInt} function.

\subsection*{{\tt randomInt()}}

This function, which takes no arguments,
returns a randomly generated number between 0
and \verb!(randomMax)!, inclusive.

\subsection*{{\tt randomSeed(seed)}}

This function, which takes a positive integer as an
argument, resets the state of Scam's pseudo-random number
generator. If a program uses {\it randomInt} without resetting the
random state, the same sequence of pseudo-random numbers will
be generated each time the program is run. To generate a
different sequence each time, a common trick is to set the
state with the current time at the start of the program:

\begin{verbatim}
    (randomSeed (time))
\end{verbatim}

\section{The {\it random} library}

The random library, included with the expression:

\begin{verbatim}
    (include "random.lib")
\end{verbatim}

adds the following functions:
{\it randomReal}, {\it randomRange}, {\it shuffle}, and {\it flip}.

\subsection*{{\tt randomReal()}}

This function,  which takes no arguments, returns a real number
between 0 (inclusive) and 1 (exclusive).

\subsection*{{\tt randomRange(low,high)}}

This function,  which takes no arguments, returns an integer
between {\it low} (inclusive) and {\it high} (exclusive).
To randomly retrieve a value from a collection named {\it items},
one might use the expression:

\begin{verbatim}
    (getElement items (randomRange 0 (length items)))
\end{verbatim}

\subsection*{{\tt shuffle(items)}}

This function, which takes a list or an array as an argument,
randomly rearranges the values in the given collection.

\subsection*{{\tt flip()}}

This function, which takes no arguments,
randomly returns integer 0 or integer 1.
It is used to simulate coin flips.
