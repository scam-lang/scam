\chapter{Recursion}
\label{Recursion}

In 
\link*{the chapter on conditionals}[Chapter~\Ref]{Conditionals},
we learned about conditionals.
When we combine conditionals with functions that call
themselves, we obtain a powerful programming paradigm
called {\it recursion}.

Recursion is a form of looping; when we loop,
we evaluate code over and over again. To stop the loop,
we use a conditional.
Recursive loops are often
easier to write and understand, as compared to the iterative loops
such as {\it while}s and {\it for}s, which you will learn
about in the next chapter.

Many mathematical functions are easy to implement recursively, so
we will start there. Recall that the factorial of a number
{\it n} is:

    \[ n! = n * (n - 1) * (n - 2) * ... * 2 * 1 \]

Consider writing a function which computes the factorial of a positive
integer. For example, if the function were passed the value of 4, it
should return the value of 24 since 4! is 4*3*2*1 or 24.
To apply
recursion to solve this problem or any problem, for that matter,
it must be possible to state the solution
of a problem so that it references a simpler version
of the problem. For factorial, the factorial of a
number can be stated in terms of a simpler factorial.

    \[ 0! = 1 \]
    \[ n! = n * (n - 1)! \] otherwise

This second formulation states that the factorial of zero is one\footnote{
Mathematicians, being an inclusive bunch, like to invite
zero to the factorial party.
}
and that the factorial of any other (positive) number is obtained by
multiplying the that number by the factorial of one less than that
number. After some study, you should be able to see that both
the first formulation (with the ellipses ...) and this new
formulation are equivalent.\footnote{
The first formulation gets the basic idea of factorial
across but is not very precise. For example, how would you compute
the factorial of three using the first formulation?
}
The second form is particularly well suited for implementation
as a computer program:

\begin{verbatim}
      def factorial(n):
          if (n == 0):
              return 1
          else:
              return n * factorial(n - 1)
\end{verbatim}

Note how  the {\it factorial} function
precisely implements the second formulation.

Convince yourself that the function really works by tracing the function call:

\begin{verbatim}
    factorial(3)
\end{verbatim}

Decomposing the call, we find that:

\begin{verbatim}
    factorial(3) is 3 * factorial(2)
\end{verbatim}

since {\it n}, having a value of 3, is not equal to 0.
and so the second block of the {\it if} is evaluated.
We can replace \verb!factorial(2)!
by \verb!2 * factorial(1)!, yielding:

\begin{verbatim}
    factorial(3) is 3 * 2 * factorial(1)
\end{verbatim}

since {\it n}, now having a value of 2, is still not zero.
Continuing along this vein, we can replace
\verb!factorial(1)! by \verb!1 * factorial(0)!, yielding:

\begin{verbatim}
    factorial(3) is 3 * 2 * 1 * factorial(0)
\end{verbatim}

Now in this call to factorial,
{\it n} does have a value of zero, so we can replace
\verb!factorial(0)! with its immediate return value of zero:

\begin{verbatim}
    factorial(3) is 3 * 2 * 1 * 1
\end{verbatim}

Thus, \verb!factorial(3)! has a value of six:

\begin{verbatim}
    >>> factorial(3)
    6
\end{verbatim}

as expected.

\section{The parts of a recursive function}

Recursive approaches rely on the fact that it is
usually simpler to solve a smaller problem than a
larger one. In the factorial problem, trying to
find the factorial $n - 1$ is a little bit simpler
than finding the factorial of $n$. If finding the
factorial of $n - 1$ is still too hard to solve easily,
then find the factorial of $n - 2$ and so on until
we find a case where the solution is easy.
With regards to factorial, this is when $n$ is equal to
zero. The {\it easy-to-solve} code (and the values that get
you there) is known as the {\it base} case. The
{\it find-the-solution-using-a-simpler-problem} code (and the values
that get you there) is known as the {\it recursive} case.
The recursive case usually contains a call to the very
function being executed. This call is known as a
{\it recursive} call.

Most well-formed recursive functions are composed of
at least one {\it base} case and at least one {\it recursive} case.

\section{The greatest common divisor}

Consider finding the greatest common divisor (gcd) of two numbers. One
The Ancient Greek Philosopher Euclid devised a solution involving
repeated division.
The first division divides the two
numbers in question, saving the remainder. Now the divisor becomes
the dividend and the
remainder becomes the divisor. This process is repeated until
the remainder becomes zero.
At that
point, the current divisor is the gcd.
We can specify this as a recurrence equation, with this last bit about
the remainder becoming zero as our base case:

\begin{center}
\begin{tabular}{lcll}
\T\toprule
    $gcd$($a$,$b$) & is & $b$ & if $a$ divided by $b$ has a remainder of zero\\
    $gcd$($a$,$b$) & is & $gcd$($b$,$a$ \% $b$) & otherwise \T\\
\T\bottomrule
\end{tabular}
\end{center}

where {\it a} and {\it b} are the dividend and the divisor, respectively.
Recall that the modulus operator \verb!%! returns the remainder.
Using the recurrence equation as a guide, we can easily implement
a function for computing the gcd of two numbers.

\begin{verbatim}
    def gcd(dividend,divisor):
        if (dividend % divisor == 0):
            return divisor
        else:
            gcd(divisor,dividend % divisor)
\end{verbatim}

We can improve this function slightly, by noting that
the remainder is computed again to make the recursive call.
Rather than compute it twice, we compute it straight off
and save it in an aptly named variable:

\begin{verbatim}
    def gcd(dividend,divisor):
        remainder = dividend % divisor
        if (remainder == 0):
            return divisor
        else:
            return gcd(divisor,remainder)
\end{verbatim}

Look at how the recursive version turns the {\it divisor} into 
the {\it dividend}
by passing {\it divisor} as the first argument in the recursive
call.
By the same token, {\it remainder} becomes {\it divisor} by nature
of being the second argument in the recursive call.
To convince yourself that the routine really works,
modify {\it gcd} to `visualize' the arguments. On simple
way of visualizing the action of a function is to
add a print statement:

\begin{verbatim}
    def gcd(dividend,divisor):
        remainder = dividend % divisor
        print("gcd:",dividend,divisor,remainder)
        if (remainder == 0):
            return divisor
        else:
            return gcd(divisor,remainder)
\end{verbatim}

After doing so, we get the following output:

\begin{verbatim}
    >>> gcd(66,42)
    gcd: 66 42 24
    gcd: 42 24 18
    gcd: 24 18 6
    gcd: 18 6 0
    INTEGER: 6
\end{verbatim}

Note, how the first remainder, 24, keeps shifting
to the left.
In the first recursive call, the remainder becomes 
{\it second}, so the 24 shifts one spot to the left. On
the second recursive call, the current {\it divisor},
which is 24,
becomes the {\it dividend},
so the 24 shifts once again to
the left.

\section{The Fibonacci sequence}

A third example of recursion is the computation of the
$n^{th}$ Fibonacci number.
The Fibonacci series looks like this:
 
\begin{verbatim}
    n       0   1   2   3   4   5   6   7   8    ...
    Fib(n)  0   1   1   2   3   5   8  13  21    ...
\end{verbatim}

and is found in nature again and again.\footnote{
Pineapples, the golden ratio, chambered nautilus, etc.
}
In general, a Fibonacci number is equal to the sum of the previous
two Fibonacci numbers. The exceptions are the zeroeth and the first
Fibonacci numbers which are equal to 0 and 1 respectively. Voila! The
recurrence case and the two base cases have jumped right out at us! Here,
then is a recursive implementation of a function which computes the $n^{th}$
Fibonacci number.

\begin{center}
\begin{tabular}{lcll}
\T\toprule
    {\it fib}({\it n}) & is & 0 & if {\it n} is zero\\
    {\it fib}({\it n}) & is & 1 & if {\it n} is one\\
    {\it fib}({\it n}) & is & {\it fib}({\it n} - 1) + {\it fib}({\it n} - 2) & otherwise\\
\T\bottomrule
\end{tabular}
\end{center}

Again, it is straightforward to convert the recurrence equation
into a working function:
 
\begin{verbatim}
    # compute the nth Fibonacci number
    # n must be non-negative!
    
    def fibonacci(n):
        if (n == 0):
            return 0
        elif (n == 1):
            return 1
        else:
            return fibonacci(n-1) + fibonacci(n-2)
\end{verbatim}

Our implementation is straightforward and elegant. Unfortunately, it's
horribly inefficient. Unlike our recursive version of
{\it factorial} which
recurred about as many times as the size of the number sent to the function,
our Fibonacci version will recur many, many more times than the size of
its input.  Here's why.

Consider the call to \verb!fib(6)!.
Tracing all the recursive calls to {\it fib}, we get:

\begin{verbatim}
    fib(6) is fib(5) + fib(4)
\end{verbatim}

Replacing \verb!fib(5)! with \verb!fib(4) + fib(3)!,
we get:

\begin{verbatim}
    fib(6) is fib(4) + fib(3) + fib(4)
\end{verbatim}

We can already see a problem, we will compute \verb!fib(4)! twice,
once from the original call to \verb!fib(6)! and again when we
try to find \verb!fib(5)!.
If we write down all the recursive calls generated by \verb!fib(6)!
with each recursive call indented from the previous, we
get a structure that looks like this:

\begin{verbatim}
    fib(6)
        fib(5)
            fib(4)
                fib(3)
                    fib(2)
                        fib(1)
                        fib(0)
                    fib(1)
                fib(2)
                    fib(1)
                    fib(0)
            fib(3)
                fib(2)
                    fib(1)
                    fib(0)
                fib(1)
        fib(4)
            fib(3)
                fib(2)
                    fib(1)
                    fib(0)
                fib(1)
            fib(2)
                fib(1)
                fib(0)
\end{verbatim}

We would expect, based on how the Fibonacci sequence is
generated,
to take about six 'steps' to calculate \verb!fib(6)!. 
Instead,
ultimately there were 13 calls to either
\verb!fib(1)! or \verb!fib(0)!.\footnote{
13 is $7^{th}$ Fibonacci number. Curious!
}
There was a tremendous amount of duplicated
and, therefore, wasted effort. An important part
of Computer Science is how to reduce the wasted effort.
One way is to cache previously computed values.\footnote{
Another way is to use an iterative loop. You will learn
about loops in the next chapter.}

\section{Manipulating lists with recursion}

Recursion and lists go hand-in-hand. What follows
are a number of recursive patterns involving lists
that you should be able to recognize and implement.

For the following discussion, assume the 
{\it head} function returns the first item in the given list,
while the
{\it tail}
function returns a list composed of all the items
in the given list except for the first element.
If the list is empty, it will have a value of 
\verb![]!.

By the way, the head and tail functions are easily implemented
in Scam:

\begin{verbatim}
    def head(items): return items[0]
    def tail(items): return items[1:]  #slicing!
\end{verbatim}

\section{The {\it counting} pattern}

The {\it counting} pattern is used to count the number of items
in a collection. If a list is empty, then its count of items
is zero.
The following function
counts and ultimately returns the number of items in the list:

\begin{verbatim}
    def count(items):
        if (items == []):
            return 0
        else:
            return 1 + tail(items)
\end{verbatim}

The functions works on the observation that if you count
the number of items in the tail of a list, then the number
of items in the entire list is one plus that number. The
extra one accounts for the head item that was not counted when the
tail was counted.

\section{The {\it accumulate} pattern}

The {\it accumulate} pattern is used to combine all the values in
a list.
The following function performs a summation of the list values:

\begin{verbatim}
    def sum(items):
        if (items == []):
            return 0
        else:
            return head(items) + sum(tail(items))
\end{verbatim}

Note that the only difference between the {\it count} function and
the {\it sum} function is the recursive case adds in the value of
the head item, rather than just counting the head item.
That the function count and the function sum fact look similar
is no coincidence. In fact, most recursive functions, especially
those working on lists, look very
similar to one another.

\section{The {\it filtered-count} and {\it filtered-accumulate} patterns}

A variation on the {\it counting} and {\it accumulate} 
patterns involves {\it filtering}. When filtering,
we use an additional \verb!if! statement to decide whether 
or not we should count the item, or in the case of accumulating,
whether or not the item ends up in the accumulation.

Suppose we wish to count the number of even items in
a list:

\begin{verbatim}
    def countEvens(items):
        if (items == []):
            return 0
        elif (head(items) % 2 == 0):
            return 1 + countEvens(tail(items))
        else:
            return 0 + countEvens(tail(items))
\end{verbatim}

The base case states that there are zero even numbers
in an empty list.
The first recursive case simply counts the head item if
it is even and so adds 1 to the count of even items in
the remainder of the list. The second recursive case
does not count the head item as even (because it is not)
and so adds in a 0 to the count of the remaining items.
Of course, the last return would almost always be written
as:

\begin{verbatim}
    return countEvens(tail(items))
\end{verbatim}

As another example of filtered counting, we can
pass in a value and count how many times that
value occurs:

\begin{verbatim}
    def occurrences(target,items):
        if (items == []):
            return 0
        elif (head(items) == target):
            return 1 + occurrences(target,tail(items))
        else:
            return occurrences(target,tail(items))
\end{verbatim}

An example of a filtered-accumulation would be 
to sum the even-numbered integers in a list:

\begin{verbatim}
    def sumEvens(items):
        if (items == []):
            return []
        elif (isEven(head(items))):
            return head(items) + sumEvens(tail(items))
        else:
            return sumEvens(tail(items))
\end{verbatim}

where the {\it isEven} function is defined as:

\begin{verbatim}
    def isEven(x):
        return x % 2 == 0
\end{verbatim}

\section{The {\it filter} pattern}

A special case of a filtered-accumulation is called {\it filter}.
Instead of summing the filtered items (for example), we collect
the filtered items into a list. The new list is said to be
a {\it reduction} of the original list.

Suppose we wish to extract the even numbers from a list. The
code looks very much like the {\it sumEvens} function in the previous
section, but instead of adding in the desired item,
we make a list out of it and concatenate to it the
reduction of the tail of the list:

\begin{verbatim}
    def extractEvens(items):
        if (items == []):
            return []
        elif (isEven(head(items))):
            return [head(items)] + extractEvens(tail(items))
        else:
            return extractEvens(tail(items))
\end{verbatim}

Given a list of integers, {\it extractEvens} returns a (possibly empty)
list of the even numbers:

\begin{verbatim}
    >>> extractEvens([4,2,5,2,7,0,8,3,7])
    [4, 2, 2, 0, 8]

    >>> extractEvens([1,3,5,7,9])
    []
\end{verbatim}

\section{The {\it map} pattern}

{\it Mapping} is a task closely coupled with that
of reduction, but rather than collecting
certain items, as with the {\it filter} pattern, we
collect all the items. As we collect, however,
we transform each item as
we collect it. The basic pattern looks like this:

\begin{verbatim}
    def map(f,items):
        if (items == []): 
            return []
        else:
            return [f(head(items))] + map(f,tail(items))
\end{verbatim}

Here, function {\it f} is used to transform each item in the
recursive step.

Suppose we wish to subtract one from each element in a list.
First we need a transforming function that reduces its argument
by one:

\begin{verbatim}
    def decrement(x): return x - 1
\end{verbatim}

Now we can ``map'' the {\it decrement} function over a list of numbers:

\begin{verbatim}
    >>> map(decrement,[4,3,7,2,4,3,1])
    [3, 2, 6, 1, 3, 2,  0]
\end{verbatim}

\section{The {\it search} pattern}

The {\it search} pattern is a slight variation of {\it filtered-counting}.
Suppose we wish to see if a value is present in a list. We can
use a filtered-counting approach and if the count is greater than
zero, we know that the item was indeed in the list.

\begin{verbatim}
    def find(target,items):
        return occurrences(target,items) > 0
\end{verbatim}

In this case, {\it occurrences} helps {\it find} do its job. We call
such functions, naturally, {\it helper functions}.
We can improve the efficiency of {\it find} by having it
perform the search, but short-circuiting the
search once the target item is found. We do this
by turning the first recursive case into a second base case:

\begin{verbatim}
    def find(target,items):
        if (items == []):
            return False
        elif (head(items) == target):    # short-circuit!
            return True
        else:
            return find(target,tail(items))
\end{verbatim}

When the list is empty, we return false because if
the item had been list, we would have hit the second
base case (and returned true) before hitting the first.
If neither base case hits, we simple search the remainder of
the list (the recursive case).
If the second base case never hits, the first base case 
eventually will.

\section{The {\it shuffle} pattern}

Sometimes, we wish to combine two lists into a third list,
This is easy to do with the concatenation operator, \verb!+!.

\begin{verbatim}
   list3 = list1 + list2
\end{verbatim}

This places the first element in the second list after the last
element in the first list.
However, many times we wish to intersperse the elements from the
first list with the elements in the second list. This is known
as a {\it shuffle}, so named since it is similar to shuffling a deck of
cards. When a deck of cards is shuffled,
the deck is divided in two halves (one half is akin to the first
list and the other half is akin to the second list). Next the 
two halves are interleaved back into a single deck (akin to
the resulting third list).

We can use recursion to shuffle two lists. If both lists are exactly
the same length, the recursive function is easy to implement
using the {\it accumulate} pattern:

\begin{verbatim}
    def shuffle(list1,list2):
        if (list1 == []):
            return []
        else:
            return [head(list1),head(list2)] + shuffle(tail(list1),tail(list2))
\end{verbatim}

If {\it list1} is empty
(which means {\it list2} is empty since they have the same number of elements),
the function returns the empty, since shuffling nothing together
yields nothing.
Otherwise, we take the first elements of each list and make a list
out of the two elements, then appending the shuffle of
the remaining elements to that list.

If you have ever shuffled a deck of cards, you will know that it is
rare for the deck to be split exactly in half prior to the shuffle.
Can we amend our shuffle function to deal with this problem? Indeed,
we can,
by simply placing the extra cards (list items) at the end
of the shuffle. We don't know which list ({\it list1} or {\it list2})
will go empty first,
so we test for each list becoming empty in turn:

\begin{verbatim}
    def shuffle2(list1,list2):
        if (list1 == []):
            return list2
        elif (list2 == []):
            return list1
        else:
            return [head(list1),head(list2)] + shuffle(tail(list1),tail(list2))
\end{verbatim}

If either list is empty, we return the other. Only if both are
not empty do we execute the recursive case.
       
\section{The {\it merge} pattern}

With the {\it shuffle} pattern, we always took the head elements from both
lists at each step in the shuffling process.
Sometimes, we wish to place a constraint of the choice of elements.
For example, suppose the two lists to be combined are sorted
and we wish the resulting list to be sorted as well. The
following example shows that shuffling does not always work:

\begin{verbatim}
    >>> a = [1,4,6,7,8]
    >>> b = [2,3,5,9]

    >>> c = shuffle2(a,b)
    [1, 2, 4, 3, 6, 5, 7, 9, 8]
\end{verbatim}

The {\it merge} pattern is used to ensure the resulting list is
sorted and is based upon the {\it filtered-accumulate} pattern.
the merge. We only accumulate an item {\it if} it is the smallest item
in the two lists:

\begin{verbatim}
    def merge(list1,list2):
        if (list1 == []):
            return list2
        elif (list2 == []):
            return list1
        elif (head(list1) < head(list2)):
            return [head(list1)] + merge(tail(list1),list2)
        else:
            return [head(list2)] + merge(list1,tail(list2))
\end{verbatim}

As with {\it shuffle2}, we don't know which list will become empty
first, so we check both in turn.

In the first recursive case, the first element of the first list
is smaller than the first element of the second list.
So we accumulate the first element of the first list and
recur, sending the tail of the first list because we have
used/accumulated the head of that list. The second list we
pass unmodified, since we did not use/accumulate an element
from the second list.

In the second recursive case, we implement the symmetric version
of the first recursive case, focusing on the second list rather
than the first.

\section{The {\it generic} {\it merge} pattern}

The {\it merge} function in the previous section hard-wired the
comparison operator to \verb!<!. Many times, the elements
of the lists to be merged cannot be compared with \verb!<! or, if
they can, a different operator, such as \verb!>!, might be desired.
The generic merge solves this problem by allowing the caller to
pass in a comparison function as a third argument:

\begin{verbatim}
    def genericMerge(list1,list1,pred):
\end{verbatim}
        
where {\it pred} is the formal parameter that
holds a predicate function\footnote{Recall that a predicate
function returns {\it True} or {\it False}.}.
Now we replace the \verb!<! in merge with a call to {\it pred}.

\begin{verbatim}
    def genericMerge(list1,list1,pred):
        if (list1 == []):
            return list2
        elif (list2 == []):
            return list1
        elif (pred(head(list1),head(list2))):
            return [head(list1)] + genericMerge(tail(list1),list2,pred)
        else:
            return [head(list2)] + genericMerge(list1,tail(list2),pred)
\end{verbatim}

The {\it pred} function, which is passed the two head elements, returns
\verb!True!, if the first element should be accumulated,
and \verb!False!, otherwise.

We can still use {\it genericMerge} to merge two sorted lists of numbers
(which can be compared with <) by using the {\it operator} module. The
{\it operator} module provides function forms of the
operators \verb!+!, \verb!-!,
\verb!<!, and so on.

\begin{verbatim}
    >>> import operator
    >>> a = [1,4,6,7,8]
    >>> b = [2,3,5,9]

    >>> c = genericMerge(a,b,operator.lt)
    [1, 2, 3, 4, 5, 6, 7, 8, 9]
\end{verbatim}

The {\it genericMerge} function is a {\it generalization} of {\it merge}.
When we generalize a function,  we modify it so it can do what it
did before plus new things that it could not. Here, we can still
have it (genericMerge) do what it (merge) did before, by passing
in the correct comparison operator.

\section{The {\it fossilized} pattern}

If a recursive function mistakenly never makes the problem
smaller, the problems is said to be {\it fossilized}.
Without ever smaller problems, the base case is never reached
and the function recurs\footnote{The word is {\it recurs}, not {\it recurses}!}
forever.
This condition is known as an
{\it infinite recursive loop}. Here is an example:

\begin{verbatim}
    def factorial(n):
        if (n == 0):
            return 1
        else:
            return n * factorial(n)
\end{verbatim}

Since factorial is solving the same problem over and over, {\it n}
never gets smaller so it never reaches zero.
Fossilizing the problem is a common error made by both novice
and expert programmers alike.

\section{The {\it bottomless} pattern}

Related to the {\it fossilized} pattern is the {\it bottomless} pattern.
With the {\it bottomless} pattern, the problem gets smaller, but the base
case is never reached. Here is a function that attempts to
divide a positive number by two, by seeing how many times
you can subtract two from the number:\footnote{
Yes, division is just repeated subtraction, just like
multiplication is repeated division.}

\begin{verbatim}
    def div2(n):
        if (n == 0):
            return 0
        else:
            return 1 + div2(n - 2)
\end{verbatim}

Things work great for a while:

\begin{verbatim}
    >>> div2(16)
    8

    >>> div2(6)
    3

    >>> div2(134)
    67
\end{verbatim}

But then, something goes terribly wrong:

\begin{verbatim}
    >>> div2(7)
    RuntimeError: maximum recursion depth exceeded in cmp
\end{verbatim}

What happened? To see, let's {\it visualize} our function,
as we did with the {\it gcd} function previously,
by adding a {\it print} statement:

\begin{verbatim}
    def div2(n):
        print("div2: n is",n)
        if (n == 0):
            return 0
        else:
            return 1 + div2(n - 2)
\end{verbatim}

Now every time the function is called, both originally and
recursively, we can see how the value of {\it n} is changing:

\begin{verbatim}
    >>div2(7)
    div2: n is 7
    div2: n is 5
    div2: n is 3
    div2: n is 1
    div2: n is -1
    div2: n is -3
    div2: n is -5
    div2: n is -7
    ...
    RuntimeError: maximum recursion depth exceeded in cmp
\end{verbatim}

Now we can see why things went wrong, the value of {\it n}
skipped over the value of zero and just kept on going.
The solution is to change the base case to catch odd
(and even) numbers:

\begin{verbatim}
    def div2(n):
        if (n < 2):
            return 0
        else:
            return 1 + div2(n - 2)
\end{verbatim}
    
Remember, when you see a recursion depth exceeded error,
you likely have implemented either
the fossilized or the bottomless pattern.
