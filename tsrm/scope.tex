\chapter{Scope}
\label{Scope}

A {\it scope} holds the current set of variables and their values.
In Scam, there is something called the {\it global scope}.
The global scope holds all the values of the built-in variables
and functions (remember, a function name is just a variable).

When you enter the Scam interpreter, either by running it
interactively or by using it to evaluate a program in
a file, you start out in the global scope.
As you define variables, they and their values are added
to the global scope.

This interaction adds the variable {\it x} to the global scope:

\begin{verbatim}
    >>> x = 3
\end{verbatim}

This interaction adds two more variables to the global scope:

\begin{verbatim}
    >>> y = 4
    >>> def negate(z):
    ...     return -z;
    ...
    >>>
\end{verbatim}

What are the two variables?
\T The two variables added to the global scope are {\it y} and {\it negate}.
\W Highlight the following line to see the answer:

\W\begin{quote}
\W {\it The two variables added to the global scope are} {\color{white} y and negate.}
\W\end{quote}

Indeed, since the name of a function is a variable and the variable {\it negate}
is being bound to a function object, it becomes clear that
this binding is occurring in the
global scope, just like {\it y} being bound to 4.

Scopes in Scam can be identified by their indentation level.
The global scope holds all variables defined with an indentation
level of zero.
Recall that when functions are defined, the body of the function
is indented. This implies that variables defined in the function body
belong to a different scope and this is indeed the case.
Thus we can identify to which scope a variable belongs by
looking at the pattern of indentations.
In particular,
we can label variables as 
either {\it local} or {\it non-local}
with respect to a particular scope.
Moreover, non-local variables may be {\it in scope} or
or {\it out of scope}.

\section{In Scope or Out}

The indentation pattern of a program can tells us where
variables are visible (in scope) and where they are
not (out of scope).
We begin by first learning to recognizing the scopes in which variables
are defined.

\subsection{The Local Variable Pattern}

All variables {\it defined} at a particular
indentation level or scope are considered
{\it local} to that indentation level or scope.
In Scam, if one assigns a value to a variable, that variable
must be local to that scope.
The only exception is if the variable was
explicitly declared {\it global} (more on that later).
Moreover, the formal parameters of a function definition
belong to the scope that is identified with
the function body. 
So within a function body, the local variables are the formal
parameters plus any variables defined in the function body.

Let's look at an example.
Note,
you do not need to completely understand the examples presented in
the rest of the chapter in order 
to identify the local and non-local variables.

\begin{verbatim}
def f(a,b):
    c = a + b
    c = g(c) + X
    d = c * c + a
    return d * b
\end{verbatim}

In this example, we can immediately say the 
formal parameters,
{\it a} and {\it b},
are local with respect to the scope of the body of function {\it f}.
Furthermore, variables
{\it c} and {\it d}
are defined in the function body
so they are local as well,
with respect to the scope of the body of function {\it f}.
It is rather wordy to say ``local with respect to the
scope of the body of the function {\it f}'', so Computer Scientists
will almost always shorten this to ``local with respect to {\it f}''
or just ``local''
if it is clear the discussion is about a particular function or scope.
We will use this shortened phrasing from here on out.
Thus {\it a}, {\it b}, {\it c}, and {\it d} are local with respect to {\it f}.
The variable {\it f} is local to the global scope since the function
{\it f} is defined in the global scope.

\subsection{The Non-local Variable Pattern}

In the previous section, we determined the local
variables of the function.
By the process 
of elimination, that means the variables
{\it g}, and {\it X} are non-local.
The name of function itself is non-local with respect
to its body, {\it f} is non-local as well.

Another way of making this determination is
that
neither {\tt g} nor {\tt X} are assigned values
in the function body. Therefore, they must be non-local.
In addition, should a variable be explicitly declared {\it global},
it is non-local even if it is assigned a value.
Here again is an example:

\begin{verbatim}
def h(a,b):
    global c
    c = a + b
    c = g(c) + X
    d = c * c + a
    return d * b
\end{verbatim}

In this example, variables {\it a}, {\it b}, and {\it d} are local with respect
to {\it h} while {\it c}, {\it g}, and {\it X} are non-local.
Even though {\it c} is assigned a value, the declaration:

\begin{verbatim}
    global c
\end{verbatim}

means that {\it c} belongs to a different scope (the global scope) and thus
is non-local.

\subsection{The Accessible Variable Pattern}

A variable is accessible with respect to
a particular scope if it is {\it in scope}.
A variable is in scope if it is local or
was defined in a scope that
{\it encloses} the particular scope.
Some scope
{\it A} encloses some other scope {\it B}
if, by moving (perhaps repeatedly) leftward from
scope B, scope A can be reached.
Here is example:

\begin{verbatim}
Z = 5

def f(x):
   return x + Z

print(f(3))
\end{verbatim}

The variable {\it Z} is local with respect to the global scope
and is non-local with respect to {\it f}. However, we can
move leftward from the scope of {\it f} one indentation level and
reach the global scope where {\it Z} is defined.
Therefore, the global scope encloses the scope of {\it f} and
thus {\it Z} is accessible from {\it f}.
Indeed, the global scope encloses all other scopes and this
is why the built-in functions are accessible at any indentation
level.

Here is another example that has two enclosing scopes:

\begin{verbatim}
X = 3
def g(a)
   def m(b)
      return a + b + X + Y
   Y = 4
   return m(a % 2)

print(g(5))
\end{verbatim}

If we look at function {\it m}, we see that there is only
one local variable, {\it b}, and that {\it m} references three
non-local variables,
{\it a}, {\it X}, and {\it Y}. 
Are these non-local variables accessible?
Moving leftward from the body of {\it m}, we reach the body of {\it g},
so the scope of {\it g} encloses the scope of {\it m}. The local variables
of {\it g} are {\it a}, {\it m}, and {\it Y}, so both {\it a} and {\it Y}
are accessible in the scope of {\it m}.
If we move leftward again, we reach the global scope,
so the global scope encloses the scope of {\it g}, which in turn encloses
the scope of {\it m}. By transitivity, the global scope encloses
the scope of {\it m}, so {\it X}, which is defined in the global scope
is accessible to the scope of {\it m}.
So, all the non-locals of {\it m} are accessible to {\it m}.

In the next section, we explore how a variable can be
inaccessible.

\subsection{The Tinted Windows Pattern}

The scope of local variables is like a car with tinted
windows, with the variables defined within riding in
the back seat.
If you are outside the scope, you cannot
peer through the car windows  and see those variables.
You might try and buy some x-ray glasses, but they
probably wouldn't work.
Here is an example:

\begin{verbatim}
    z = 3

    def f(a):
        c = a + g(a)
        return c * c

    print("the value of a is",a) #x-ray!
    f(z);
\end{verbatim}

The print statement causes an error:

\begin{verbatim}
    Traceback (most recent call last):
      File "xray.py", line 7, in <module>
          print("the value of a is",a) #x-ray!
          NameError: name 'a' is not defined
\end{verbatim}

If we also tried to print the value of {\it c},
which is a local variable of function {\it f}, at that
same point in the program, we would get a similar error.

The rule for figuring out which variables are in scope and
which are not is:
{\it you} {\bf cannot} {\it see into an enclosed scope}.
Contrast this with the non-local pattern:
{\it you} {\bf can} {\it see variables
declared in enclosing outer scopes}.

\subsection{Tinted Windows with Parallel Scopes}

The tinted windows pattern also applies to parallel scopes.
Consider this code:

\begin{verbatim}
    z = 3

    def f(a):
        return a + g(a)

    def g(x):
        # starting point 1
        print("the value of a is",a) #x-ray!
        return x + 1

    f(z);
\end{verbatim}

Note that the global scope encloses both the scope of {\it f} and
the scope of {\it g}. However, the scope of {\it f} does
not enclose the scope of {\it g}. Neither does
the scope of {\it g} enclose the scope of {\it f}.

One of these functions references a variable that is not in scope.
Can you guess which one?
\T The function {\it g} references a  variable not in scope.
\W Highlight the following line to see the answer:

\W\begin{quote}
\W {\it The function} {\color{white} g} {\it references variable not in scope.}
\W\end{quote}

Let's see why by first examining {\it f} to see whether or
not its non-local references are in scope.
The only local variable of function {\it f} is
{\it a}. The only referenced non-local is {\it g}.
Moving leftward from the body of {\it f}, we reach the
global scope where where both {\it f} and {\it g} are defined.
Therefore, {\it g}
is visible with respect to {\it f} since it is defined in a scope
(the global scope) that encloses {\it f}.

Now to investigate {\it g}. The only local variable of
{\it g} is {\it x}
and the only non-local that {\it g} references is {\it a}.
Moving outward to the global scope, we see that there is
no variable {\it a} defined there,
therefore the variable {\it a} is not in scope with
respect to {\it g}.

When we actually run the code,
we get an error similar to the following when running this program:

\begin{verbatim}
    Traceback (most recent call last):
      File "xray.py", line 11, in <module>
        f(z);
      File "xray.py", line 4, in f
        return a + g(a)
      File "xray.py", line 8, in g
        print("the value of a is",a) #x-ray!
    NameError: global name 'a' is not defined
\end{verbatim}

The lesson to be learned here is
that we cannot see into
the local scope of the body of function {\it f},
{\it even if we are at a similar nesting level}.
Nesting level doesn't matter. We can only see variables
in our own scope and those in {\it enclosing} scopes.
All other variables cannot be seen.

Therefore, if you ever see a variable-not-defined error,
you either have spelled the variable name wrong, you haven't
yet created the variable, or you are trying to use x-ray vision
to see somewhere you can't. 

\section{Alternate terminology}

Sometimes, enclosed scopes are referred to as {\it inner} scopes while
enclosing scopes are referred to as {\it outer} scopes. In addition,
both locals and any non-locals found in enclosing scopes are considered
{\it visible} or {\it in scope}, while non-locals that are not
found in an enclosing scope are considered {\it out of scope}.
We will use all these terms in the remainder of the text book.

\section{Three Scope Rules}

Here are three simple rules you can use to help you
figure out the scope of a particular variable:

\begin{itemize}
\item
        Formal parameters belong in
\item
        The function name belongs out
\item
        You can see out but you can't see in (tinted windows).
\end{itemize}

The first rule is shorthand for the fact that formal parameters
belong to the scope of the function body. Since the function body
is `inside' the function definition, we can say the formal parameters
belong in.

The second rule reminds us that as we move outward from a function body,
we find the enclosing scope holds the function definition. That is to
say, the function name is bound to a function object in the scope
enclosing the function body.

The third rule tells us all the variables that belong to 
ever-enclosing scopes are accessible and therefore
can be referenced by the innermost scope. The opposite is
not true. A variable in an enclosed scope can not be referenced
by an enclosing scope. If you forget the directions of this
rule, think of tinted windows. You can see out of a tinted
window, but you can't see in.

\section{Shadowing}

The formal parameters of a function can be thought of
as variable definitions that are only in effect when
the body of the function is being evaluated. That is,
those variables are only visible in the body and no where
else. This is why formal parameters are considered to be
{\it local} variable definitions, since they only have local
effect (with the locality being
the function body). Any direct reference to those
particular variables outside the body of the function
is not allowed (Recall that you can't see in).
Consider the following interaction with
the interpreter:

\begin{verbatim}
    >>> def square(a):
    ...     return a * a
    ...
    >>>
     
    >>> square(4)
    16
     
    >>> a
    NameError: name 'a' is not defined
\end{verbatim}


In the above example, the scope of variable {\it a} is restricted
to the body of the function {\it square}.
Any reference to
{\it a} other than in the context of {\it square} is invalid. Now
consider a slightly different interaction with the
interpreter:

\begin{verbatim}
    >>> a = 10
    >>> def almostSquare(a):
    ...     return a * a + b
    ...
    >>> b = 1
     
    >>> almostSquare(4)
    17

    >>> a
    10
\end{verbatim}

In this dialog, the global scope has three
variables added, {\it a}, {\it almostSquare} and
{\it b}.
In addition, the variable serving as the formal parameter
of {\it almostSquare} has the same name as the first variable
defined in the dialog. Moreover, the body of {\it almostSquare}
refers to both variables {\it a} and {\it b}. To which {\it a} does
the body of almostSquare refer? The global {\it a} or the local {\it a}?
Although it seems confusing at first,
the Scam interpreter has no difficulty in figuring out
what's what. From the responses of the interpreter,
the {\it b} in the body must refer to the variable that was
defined with an initial value of one. This is consistent with
our thinking, since {\it b} belongs to the enclosing scope
and is accessible within the body of {\it almostSquare}.
The {\it a} in the function body must refer to the
formal parameter whose value was set to 4 by the call to
the function (given the output of the interpreter).

When a local variable has the same name as a non-local variable
that is also in scope, the local variable is said to
{\it shadow} the non-local version. The term shadowed refers to the fact
that the other variable is in the shadow of the local
variable and cannot be seen. Thus, when the
interpreter needs the value of the variable, the value of
the local variable is retrieved.
It is also possible for a non-local variable to shadow
another non-local variable. In this case, the variable
in the nearest outer scope shadows the variable in
the scope further away.

In general, when a variable is referenced,
Scam first looks in the local
scope.
If the variable is not found there,
Scam looks in the enclosing scope.
If the variable is not there,
it looks in the scope enclosing the enclosing scope, 
and so on.

In the particular example,
a reference to {\it a} is made when the body of
{\it almostSquare} is executed. The value of {\it a}
is immediately found in the local scope.
When the value of {\it b} is required,
it is not found in the local scope. The interpreter
then searches the enclosing scope (which in this case happens
to be the global scope).
The global scope does hold {\it b} and its value, so a value
of 1 is retrieved.

Since {\it a} has a value of 4 and {\it b} has a value of 1, the
value of 17 is returned by the function. Finally, the
last interaction with the interpreter illustrates the
fact that the initial binding of {\it a} was unaffected by the
function call.

\section{Modules}

Often, we wish to use code that has already been written.
Usually, such code contains handy functions that have
utility for many different projects. In Scam, such
collections of functions are known as modules.
We can include modules into our current project with
the {\it import} statement, which we saw in
\link*{the chapter on functions}[Chapters~\Ref]{Functions}
and
\link*{the chapter on programs and files}[\Ref]{ScamPrograms}.

The import statement has two forms. The first is:

\begin{verbatim}
    from ModuleX import *
\end{verbatim}

This statement imports all the definitions from ModuleX and places
them into the global scope. At this point, those definitions
look the same as the built-ins, but if any of those definitions
have the same name as a built-in, the built-in is shadowed.

The second form looks like this:

\begin{verbatim}
    import ModuleX
\end{verbatim}

This creates a new scope that is separate from the global scope
(but is enclosed by the global scope).
Suppose
{\it ModuleX} has a definition for variable {\it a}, with a value
of 1.
Since {\it a} is in a scope
enclosed by the global scope, it is inaccessible from the global
scope (you can't see in):

\begin{verbatim}
    >>> import ModuleX
    >>> a
    NameError: name 'a' is not defined
\end{verbatim}

The direct reference to {\it a} failed, as expected.
However, one can get to {\it a} and its value {\it indirectly}:

\begin{verbatim}
    >>> import ModuleX
    >>> ModuleX . a
    1
\end{verbatim}

This new notation is known as {\it dot} notation and is commonly
used in object-oriented programming systems to references pieces
of an object. For our purposes, {\it ModuleX} can be thought
of as
a {\it named} scope and the {\it dot} operator is used to look
up variable {\it a} in the scope named ModuleX.

This second form of import is used when the possibility that
some of your functions or variables have the same name as
those in the included module. Here is an example:

\begin{verbatim}
    >>> a = 0
    >>> from ModuleX import *

    >>> a
    1
\end{verbatim}

Note that the {\it ModuleX}'s variable {\it a} has showed the previously
defined {\it a}. With the second form of import, the two versions
of {\it a} can each be referenced:

\begin{verbatim}
    >>> a = 0
    >>> import ModuleX

    >>> a
    0
    >>> ModuleX . a
    1
\end{verbatim}
