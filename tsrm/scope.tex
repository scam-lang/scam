\chapter{Scopes, Environments, and Objects}
\label{ScopesEnvironmentsObjects}

A {\it scope} holds the current set of variables and their values.
As with Scheme, Scam implements scope using environments.
For all Scam programs, the built-in functions are stored
in the outermost scope, while the functions defined
in the main library, {\it main.lib}, are stored in the next-to-the-outermost
scope. Finally, the user is given a global scope of his or her own.
These three scopes make up the global scope for all 
Scam programs.

One can write a very simple Scam program to see the three
levels of the global scope that are in existence for all programs:

\begin{verbatim}
    (print "USER SCOPE: ")
    (pp this)
    (print "LIBRARY SCOPE: ")
    (pp (. this __context)
    (print "BUILT-IN SCOPE: ")
    (pp (. (. this __context) __context))
\end{verbatim}

Running this program yields the following output:

\begin{verbatim}
    USER SCOPE: <object 8675>
                   __label  : environment
                 __context  : <object 4777>
                   __level  : 0
             __constructor  : nil
                    this  : <object 8675>
                     ...
    LIBRARY SCOPE: <object 4777>
                 __label  : environment
               __context  : <object 62>
                 __level  : 0
           __constructor  : nil
                    this  : <object 4777>
                    code  : <function code($s)>
            stream-null?  : <function stream-null?(s)>
              stream-cdr  : <function stream-cdr(s)>
              stream-car  : <function stream-car(s)>
             cons-stream  : <function cons-stream(# a $b)>
                     ...
    BUILT-IN SCOPE: <object 62>
                 __label  : environment
               __context  : nil
                 __level  : 0
           __constructor  : nil
                    this  : <object 62>
                 ScamEnv  : [ORBIT_SOCKETDIR=/tmp/orbit-lusth,...]
                ScamArgs  : [scam,scope.s]
             stack-depth  : <builtIn stack-depth()>
                set-cdr!  : <builtIn set-cdr!(spot value)>
                set-car!  : <builtIn set-car!(spot value)>
                     ...
\end{verbatim}

We can see from this output that every environment
has five predefined variables. So that they are accessible
and available for manipulation with reduced
likelihood of being trashed, the first four begin with
two underscores. The predefined variables are:

\begin{description}
\item[\_\_label]
    Environments are objects in Scam.
    The {\it \_\_label} field is used to distinguish the native objects from
    each other. For environments, the label value is {\it environment}.
    Other label values are {\it closure}, {\it thunk}, {\it error},
    and {\it exception}.
\item[\_\_context]
    This field holds the enclosing scope. For the outermost scope,
    the value of this field is {\it nil}.
\item[\_\_level]
    This fields holds the nesting level of dynamic scopes or, in 
    other words, the depth of the call tree for a particular function.
\item[\_\_constructor]
    This field holds the closure for which the environment was constructed.
    If the environment did not arise from a function call, the value of
    this field is {\it nil}.
\item[this]
    A self-reference to the current environment.
\end{description}


As variables are defined, their names and values are inserted
into the most local scope at the time of definition.

\section{Nesting scopes}

If a definition occurs in 
a nested begin block, it belongs to the scope in which the
begin block is found, recursively. Thus, when the following
code is evaluated:

\begin{verbatim}
    (define (f w)
        (define x 1)
        (begin
            (define y 2)
            (begin
                (define z 3)
                )
            )
        (pp this)
        )
    (f 0)
\end{verbatim}

We see the following output:

\begin{verbatim}
    <object 9726>
             __label  : environment
           __context  : <object 9024>
             __level  : 1
       __constructor  : <function f(w)>
                this  : <object 9726>
                   z  : 3
                   y  : 2
                   x  : 1
                   w  : 0
\end{verbatim}

Note that the nested defines of {\it y} and {\it z} were promoted
to the scope of the function body, mirroring the behavior of Python.

To force a block to have its scope, as in C, C++, and Java,
one can use the {\it scope} function:

\begin{verbatim}
    (define (g w)
        (define x 1)
        (scope
            (define y 2)
            (begin
                (define z 3)
                )
            )
        (pp this)
        )
    (g 0)
\end{verbatim}

Now the output reflects that fact that {\it y} and {\it z} are in their own
separate scope:

\begin{verbatim}
    <object 9805>
             __label  : environment
           __context  : <object 9024>
             __level  : 1
       __constructor  : <function g(w)>
                this  : <object 9805>
                   x  : 1
                   w  : 0
\end{verbatim}

Because {\it y} and {\it z} are now in an enclosed scope,
they are no longer visible.

