\chapter{Sorting}
\label{Sorting}

\section*{Sorted Tables}

Recall that a table is a list of records where
each record is a list of the fields incorporating
the record.

Sometimes, you need to merge two sorted tables into
one table that remains sorted. First, you have
to decide which field is used for the
sorting. In our example, the records in the data file
could be sorted on NAME or on SALARY or any other
field.

Suppose we had two data files that are sorted on SALARY,
{\it salaries.1} and {\it salaries.2}.  We wish to merge the data
in both files, printing  out the merged data, again in
sorted order.

First, we need to read the data into tables:

\begin{verbatim}
    table1 = readTable("salaries.1")
    table2 = readTable("salaries.2")
\end{verbatim}

Our strategy is to compare the first unaccumulated
record in {\it table1} to the first unaccumulated record
in {\it table2}. Let's call these records {\it r1} and {\it r2}, respectively.
If the salary of {\it r1} is less than that of {\it r2}, we accumulate
{\it r1}. Otherwise we accumulate {\it r2}.
We will repeat this process using a loop.

It is clear we need two variables, the first points to
the index of the first unaccumulated record in the table1,
while the second variable points to the first unaccumulated
record in the second table.
We start out both variables at zero, meaning no records
have been accumulated yet:

\begin{verbatim}
    index1 = 0
    index2 = 0
\end{verbatim}

How do we know when to stop accumulating? When we
run out of records to compare. This happens
when {\it index1} has passed
the index of the last record in {\it table1} or {\it index2} has passed the
index of the last record in {\it table2}.
We reverse that logic for a while loop, because it runs
while the test condition is true. The reversed logic is
``as long as index1 has not passed the {\it index} of the last record
in {\it table1} AND {\it index2} has not passed the index of the last
record in {\it table2}''.

\begin{verbatim}
    total = []
    while (index1 < len(table1) and index2 < len(table2):
        r1 = table1[index1]
        r2 = table2[index2]
        ...
\end{verbatim}

We also must advance {\it index1} and {\it index2} to that the loop will
finally end. When do we advance {\it index1}? When we accumulate
a record from {\it table1}. When do we advance {\it index2}? Likewise,
when we accumulate a record from {\it table2}.


\begin{verbatim}
    total = []
    while (index1 < len(table1) and index2 < len(table2):
        r1 = table1[index1]
        r2 = table2[index2]
        if (r1[SALARY] < r2[SALARY]):
           total = total + [r1]
           index1 += 1
        else:
           total = total + [r2]
           index2 += 1
\end{verbatim}

When will this loop end? When one of the indices gets too high\footnote{
Only one will be too high. Why is that?}.
This means we will have accumulated all the
records from one of the tables, but we don't know which one.
So, we add two more loops to accumulate any left over records:

\begin{verbatim}
    for i in range(index1,len(table1),1):
        total = total + [table1[i]]

    for i in range(index2,len(table2),1):
        total = total + [table2[i]]
\end{verbatim}

Finally, we encapsulate all of our merging code into
a function, passing in the index of the field that was
used to sort the data. This field is
known as the {\it key}:

\begin{verbatim}
    def merge(table1,table2,key):
        total = []
        while (index1 < len(table1) and index2 < len(table2):
            r1 = table1[index1]
            r2 = table2[index2]
            if (r1[key] < r2[key]):
               total = total + [r1]
               index1 += 1
            else:
               total = total + [r2]
               index2 += 1

        for i in range(index1,len(table1),1):
            total = total + [table1[i]]

        for i in range(index2,len(table2),1):
            total = total + [table2[i]]
\end{verbatim}

Finally, we define a main function to tie it all together:

\begin{verbatim}
    def main():
        table1 = readTable("salaries.1")
        table2 = readTable("salaries.2")
        mergedTable = merge(table1,table2,SALARY) #SALARY is the key
        printTable(mergedTable)
\end{verbatim}

Notice how the main function follows the standard main
pattern:

\begin{itemize}
\item
        get the data
\item
        process the data
\item
        write the result
\end{itemize}

\section{Merge sort}
