\chapter{Starting Out}
\label{StartingOut}

A word of warning: if you require fancy graphically oriented development
environments, you will be sorely disappointed in the Scam Programming
Language. Scam is programming at its simplest: a prompt at which you type
in expressions which the Scam interpreter evaluates. Of course, you can
create Scam programs using your favorite text editor (mine is {\it vim}).
Scam can be used as a scripting language, as well.

\section{Running Scam}

In a terminal window, simply type the command:

\begin{verbatim}
    scam <fileName>
\end{verbatim}

where \verb!<fileName>! is replaced by the name of the file contining your
Scam program ({\it i.e.} Scam source code).
You should be rewarded with the output from your Scam program.

For example, create a file named hello.s. In it, place the line:

\begin{verbatim}
    (println "hello, world!")
\end{verbatim}

Save the file and exit. Now run your program:

\begin{verbatim}
    $ scam hello.s
    hello, world!
    $
\end{verbatim}

The \verb!$! represents the system prompt.

\section{Scam options}

Scam takes a number of options:

\begin{description}
\item[-M]
    
    display current memory size, then exit
    
\item[-m NNN]
    
    set the memory size to {\it NNN}
    
\item[-v]

    display the Scam version number

\item[-t]

    display a full error trace for an uncaught exception

\end{description}

At this point, you are ready to proceed to next chapter.
