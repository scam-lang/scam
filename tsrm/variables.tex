\chapter{Variables}
\label{Variables}

Suppose you found an envelope lying on the street and on the front of
the envelope was printed the name {\it numberOfDogsTeeth}.
Suppose further
that you opened the envelope and inside was a piece of paper with
the number 42 written upon it.
What might you conclude from such an
encounter? Now suppose you kept walking and found another envelope
labeled {\it meaningOfLifeUniverseEverything} and,
again, upon opening it
you found a slip of paper with the number 42 on it.
Further down the
road,
you find two more envelopes,
entitled {\it numberOfDotsOnPairOfDice} and
{\it StatuteOfLibertyArmLength},
both of which contain the number 42.

Finally,
you find one last envelope labeled {\it sixTimesNine} and inside you,
yet again,
find the number 42.
At this point,
you're probably thinking `somebody
has an odd affection for the number 42' but then the times table that
is stuck somewhere in the dim recesses of your brain begins yelling at
you saying `54! It's 54!'.
After this goes on for an embarrassingly long
time,
you realize that 6 * 9 is not 42, but 54.
So you cross out the 42
in the last envelope and write 54 instead and put the envelope back
where you found it.

This strange little story,
believe it or not,
has profound implications
for writing programs that both humans and computers can understand.
For
programming languages,
the envelope is a metaphor for something called
a {\it variable},
which can be thought of as a label for a place in memory
where a literal value can reside.
In many programming languages,
one can change the value at that memory location,
much like replacing
the contents of an envelope.\footnote{
Languages that do not allow changes to a variable are called
{\it functional languages}.
Scam is an `impure' functional language
since it is {\it mostly} functional but allows for variable modification.
}
A variable is our first encounter with a
concept known as {\it abstraction},
a concept that is fundamental to the whole
of computer science.\footnote{
Another fundamental concept is analogy and if you understand the
purpose of the envelope story after reading this section,
you're well
on your way to being a computer scientist!
}

\section{Variables}

Most likely,
you've encountered the term {\it variable} before.
Consider the
slope-intercept form of an algebraic equation of a particular line:

    \[  y = 2x - 3 \]

You probably can tell from this equation that the slope of this line
is 2 and that it intercepts the {\it y}-axis at -3.
But what role do
the letters {\it y} and {\it x} actually play?  The names {\it x} and {\it y}
are placeholders and stand for the {\it x}- and {\it y}-coordinates of any
conceivable point on that line.
Without placeholders,
the line would
have to be described by listing every point on the line.
Since there
are an infinite number of points,
clearly an exhaustive list is not
feasible.
As you learned in your algebra class,
the common name for a
place holder for a specific value is the term {\it variable}.

One can generalize the above line resulting in an equation that describes
every line.\footnote{
The third great fundamental concept in computer science
is {\it generalization}.
In particular,
computer scientists are always trying to make things
more abstract and more general (but not overly so).
The reason is that
software/systems/models exhibiting the proper levels of abstraction and
generalization are much much easier to understand and modify.
This is
especially useful when you are required to make a last second
change to the software/system/model.
}

    \[  y = mx + b \]

Here,
the variable {\it m} stands for the slope and {\it b} stands for the
{\it y}-intercept.
Clearly,
this equation was not dreamed up by a computer
scientist since a cardinal rule is to choose good names for variables,
such as {\it s} for slope and {\it i} for intercept.
But alas,
for historical reasons,
we are stuck with {\it m} and {\it b}.

The term {\it variable} is also used in most programming languages,
including
Scam,
and the term has roughly the equivalent meaning.
The difference is
programming languages use the envelope metaphor
while algebraic meaning of variable is an equivalence to a value.\footnote{
Even the envelope metaphor can be confusing since it
implies that two variables having the same value must
each have a copy of that value. Otherwise, how can one value
be in two envelopes at the same time? For simple literals, copying is
the norm.
For more complex objects, the cost of copying would
be prohibitive.
The solution is to storing the {\it address} of the object,
instead of the object itself, in the envelope. Two variables can
now `hold' the same object since it is the address is copied.
}
The difference is purely philosophical and not worth going into at this
time.
Suppose you found three envelopes,
marked {\it m}, {\it x}, and {\it b},
and inside
those three envelopes you found the numbers 6, 9, and -12 respectively.
If
you were asked to make a {\it y} envelope,
what number should you put inside?
If
the number 42 in the {\it sixTimesNine} envelope in the previous story did not
bother you ({\it e.g.}, your internal times table was nowhere to be found),
perhaps you might need a little help in completing your task.
We can
have Scam calculate this number with the following dialog:

\begin{verbatim}
    >>> m = 6
    
    >>> x = 9  
    
    >>> b = -12
    
    >>> y = m * x + b
    
    >>> y
    42
\end{verbatim}

The Scam interpreter,
when asked to compute the value of an expression containing variables,
goes to those envelopes (so to speak) and
retrieves the values stored there.
Note also that Scam
requires the use of the multiplication sign
to multiply the slope {\it m} by the {\it x} value.
In the algebraic equation,
the multiplication sign is elided,
but is required here.

One creates variables in Scam by simply assigning a value
to the variable.\footnote{
In many other languages, you have to declare a variable
with a special syntax before you can assign a value
to it}. If the variable does not exist, it is created;
if it does exist, it's value is updated.
Note that the interpreter does not give a response
when a variable is created or updated.

Here are some more examples of variable creation:

\begin{verbatim}
    >>> dots = 42

    >>> bones = 206

    >>> dots
    42

    >>> bones
    206

    >>> CLXIV = bones - dots
    164
\end{verbatim}

After a variable is created/updated,
the variable and its value can be used interchangeably.
Thus,
one use of variables is to set up constants that
will be used over and over again.
For example,
it is an easy matter to
set up an equivalence between the variable {\sf PI} and the real number 3.14159.

\begin{verbatim}
    PI = 3.14159
    radius = 10
    area = PI * radius * radius
    circumference = 2 * PI * radius
\end{verbatim}

Notice how the expressions used to compute the values of the variables
area and circumference are more readable than if 3.14159 was used
instead of {\sf PI}.
In fact,
that is one of the main uses of variables,
to
make code more readable.
The second is if the value of {\sf PI} should change
(e.g. a more accurate value of {\sf PI} is desired,\footnote{
The believed value of {\sf PI} has changed throughout the centuries and not
always to be more accurate (see
\xlink
{http://en.wikipedia.org/wiki/History\_of\_Pi}
{http://en.wikipedia.org/wiki/History\_of\_Pi}
)
}
we would only need
to change the definition of {\sf PI} (this assumes,
of course,
we can store
those definitions for later retrieval and do not need to type them into
the interpreter again).

\section{Variable naming}

Like many languages,
Scam is quite restrictive in regards to legal variable names.
A variable name must begin with a letter or an underscore
and may be followed by any number of letters, digits, or
underscores.

Variables are the next layer in a programming languages,
resting on
the literal expressions and combinations of expressions (which are
expressions themselves).
In fact,
variables can be thought of as an
abstraction of the literals and collections of literals.
As an analogy,
consider your name.
Your
name is not you,
but it is a convenient (and abstract) way of referring
to you.
In the same way,
variables can be considered as the names of
things.
A variable isn't the thing itself,
but a convenient way to referring
to the thing.

While Scam lets you name variables in wild ways:

\begin{verbatim}
    >>> _1_2_3_iiiiii__ = 7
\end{verbatim}

you should temper your
creativity if it gets out of hand.
For example,
rather than use the
variable {\it m} for the slope,
we could use the name {\it slope} instead:

\begin{verbatim}
    slope = 6
\end{verbatim}

We could have also used a different name:

\begin{verbatim}
    _e_p_o_l_s_ = 6
\end{verbatim}

The name {\tt \_e\_p\_o\_l\_s\_}
is a perfectly good variable name from Scam's point of
view.
It is a particularly poor name from the point of making your Scam
programs readable by you and others.
It is important that your variable
names reflect their purpose.
In the example above,
which is the better name: {\it m}, {\it slope}, or 
{\tt \_e\_p\_o\_l\_s\_}
to represent the slope of a line?
