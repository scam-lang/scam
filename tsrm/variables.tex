\chapter{Variables}
\label{Variables}

One defines variables with the {\it define} function:

\begin{verbatim}
    (define x 13)
\end{verbatim}

The above expression creates a variable named {\it x} in the current
scope and initializes it to the value 13.
If the initializer is missing:

\begin{verbatim}
    (define y)

    (inspect y)
    -> y is nil
\end{verbatim}

the variable is initialized to nil.

\section{Defining functions}

There are two ways to define a function. The first is through a 
regular variable definition, where the initializer is a lambda
expression:

\begin{verbatim}
    (define square (lambda (x) (* x x)))

    (square 3)
    -> 9

    (inspect square)
    -> square is <function square(x)>
\end{verbatim}

The above expression defines a function that returns the square of
its argument. The second method uses a special syntax:

\begin{verbatim}
    (define (square x) (* x x))
    
    (inspect square)
    -> square is <function square(x)>
\end{verbatim}

The two forms are equivalent, so regardless of the method, the
name of the function is simply a variable. The only difference
is its value.

\section{Environments and Objects}

When one defines a variable, the variable name and value are
stored in a table called an {\it environment}. The predefined
variable this always points to the current environment.
For example, consider this interaction:

\begin{verbatim}
    (define n 10)

    (inspect this)
    -> this is <object 8393>
                   label  : environment
                 context  : <object 4495>
                   level  : 0
             constructor  : nil
                    this  : <object 8393>
                       n  : 10
\end{verbatim}

Among other information stored in the current environment,
we see an entry for {\it n} and its value is indeed 10.

The Scam object system is based upon environments. We will
learn about objects in a later chapter.

\section{Defining Variables Programatically}

The function addSymbol is used to define variables on the 
fly. For example, to define a variable named x in the current
scope and to initialized it to 13, one might use the
following expression:

\begin{verbatim}
    (addSymbol 'x 13 this)
\end{verbatim}

You can also define functions this way:

\begin{verbatim}
    (addSymbol 'square (lambda (x) (* x x)) this)
\end{verbatim}

Since {\it addSymbol} evaluates all its arguments, the first
argument can be any expression that resolves to a symbol,
the second argument can be any expression that resolves
to an appropriate value, and the third argument can
be any expression that resolves to an environment or
object.

\section{Variable naming}

Unlike many languages,
Scam is quite liberal in regards to legal variable names.
A variable can't begin with any of the these characters:
\verb!0123456789;,'"()! nor whitespace and cannot contain any of these
characters: \verb!;,'"()! nor whitespace. Typically,
variable names start with a letter or underscore, but
they do not have to. This flexibility allows Scam programmers
to easily define new functions that have appropriate names.
Here is a function that increments the value of its argument:

\begin{verbatim}
    (define (+1 n) (+ n 1))
\end{verbatim}

While Scam lets you name variables in wild ways:

\begin{verbatim}
    (assign $#1_2!3iiiiii@ 7)
\end{verbatim}

you should temper your
creativity if it gets out of hand.
While the name \verb-$#1_2!3iiiiii@-
is a perfectly good variable name from Scam's point of
view,
it is a particularly poor name from the point of making your Scam
programs readable by you and others.
It is important that your variable
names reflect their purpose.


