\documentclass{article}  
\usepackage[margin=1.0in]{geometry}
\usepackage{hyperlatex}
\usepackage{color}

\htmlcss{lusth.css}

\htmldepth{1}
\htmltitle{the scam home page}
%\htmladdress{lusth@cs.ua.edu}

\T\setlength\parskip{10 pt}
\T\setlength\parindent{0 in}

\title{The Scam Programming Language}

\author{written by: John C. Lusth}

\date{Revision Date: \today}

\begin{document}

\maketitle

\W\xlink{Printable Version}{index.pdf}

Scam is a nifty little language
that looks a lot like Scheme, but is so much more.
Scam is:

\begin{itemize}
    \item
        an impure functional language with Scheme-like syntax
    \item
        a language where writing iterators like while and for
        is possible purely through function definition
    \item
        a fully object-oriented language with trivial syntax and
        easy-to-comprehend semantics
    \item
        a language with programmer-controlled delayed
        evaluation
    \item
        a language with built-in garbage collection
    \item
        a that optimizes tail-recurion
\end{itemize}

\section*{Scam versus Scheme}

For Scheme programmers, here is a list of important
differences: \xlink{quick hits}{quick.html}.

\section*{Downloads}

Latest Scam versions:

\begin{description}
    \item
        \xlink{scam-i386.tgz} {scam-i386.tgz}
    \item
        \xlink{scam-amd64.tgz}{scam-amd64.tgz}
    \item
        \xlink{scam-OSX.tgz}{scam-OSX.tgz}
    \item
        ~
    \item
        \xlink{scam-0.1.1\_i386.deb} {scam-0.1.1\_i386.deb}
    \item
        \xlink{scam-0.1.1\_amd64.deb}{scam-0.1.1\_amd64.deb}
\end{description}

%Here is an experimental Microsoft Windows version: \xlink{scam.exe}{scam.exe}.
%You will also need \xlink{readline5.dll}{readline5.dll} and
%\xlink{history5.dll}{history5.dll}. Copy all three files to:
%
%\begin{verbatim}
%    c:\windows\system32
%\end{verbatim}
%
%and you should be good to go. You will need to run Scam in a DOS box (command
%prompt).

%If you want to use the set of Scam libraries (useful functions 
%written in Scam) but don't know how to set up a path on Windows, install the 
%\xlink{libraries}{lib/}
%in same directory as your program, if you want to use any of them.

\section*{Installation Notes}

To install Scam from a deb package
(using {\tt scam\_0.1.1\_i386.deb} as an example), run the command:

\begin{verbatim}
    sudo dpkg -i scam_0.1.1_i386.deb
\end{verbatim}

If you are on a system without {\it sudo}, run the above command
as {\it root} without the {\it sudo} prefix.

To install Scam from a tarball, perform the following steps
(using {\tt scam-i386.tgz} as an example)...

\begin{enumerate}

    \item
        Save the download to a file, noting where it was saved. For the rest 
        of the steps, assume the tar file was saved in a Downloads directory
        that hangs of the your home directory.
    \item
        Make a directory named {\it scam} that hangs off your home directory and then move into that directory


\begin{verbatim}
    mkdir ~/scam
    cd ~/scam
\end{verbatim}

    \item
        Untar the download in the {\it scam} directory:

\begin{verbatim}
    tar xvzf ~/Downloads/scam-i386.tgz
\end{verbatim}

    \item
        copy the {\it scam} executable to a directory in your path:

\begin{verbatim}
    cp scam ~/bin                # if you have your own bin
\end{verbatim}

or

\begin{verbatim}
    sudo cp scam /usr/local/bin  #if you have sudo privleges
\end{verbatim}

or you can put \verb!~/scam! in your path.

\end{enumerate}

If you have trouble installing Scam, send a message to 
\xlink{lusth@cs.ua.edu}{mailto:lusth@cs.ua.edu}.

\section*{Learning more about Scam}

There is an Open Source
reference manual that delves into Scam in detail.
It is called 
\xlink
    {The Scam Reference Manual}
    {http://beastie.cs.ua.edu/scam/tsrm}

The reference manual gives the nuts and bolts on the syntax
and semantics of Scam.

\section*{About this website}
This website is rendered in the \emph{Old School} style.
\emph{Old School} is known for its somewhat spartan,
but clean and crisp, no-nonsense, get-down-to-brass-tacks
appearance. \emph{Old School} was invented by
Tim Berners-Lee during the latter part of the last 
century. It is made a bit more flashy using a
style sheet.

\section*{Acknowledgements}

This material is based upon work supported by the
National Science Foundation, grant numbers \#00244269 and \#0633290.

Any opinions, findings, and conclusions or recommendations
expressed in this material are those of the author and
do not necessarily reflect the views of the National Science Foundation.

\xlink
    {lusth@cs.ua.edu}
    {mailto:lusth@cs.ua.edu}

\end{document}
