\documentclass{article}  
\usepackage{hyperlatex}
\usepackage{color}

\htmlcss{steely.css}

\htmldepth{1}
\htmltitle{scam vs scheme}
\htmladdress{lusth@cs.ua.edu}
\T\oddsidemargin  -.1 true in       % Note that \oddsidemargin=\evensidemargin
\T\evensidemargin -.1 true in
\T\setlength\topmargin{+0.1in}
\T\setlength\textheight{9.1 in}
\T\setlength\textwidth{6.7 in}
\T\setlength\parskip{6 pt}
\T\setlength\parindent{0 in}

\title{The Scam Programming Language\\
Scam vs Scheme}

\author{written by: John C. Lusth}

\date{Revision Date: \today}

\begin{document}

\maketitle

\W\xlink{Printable Version}{quick.pdf}

Things to note for programmers who are trying Scam:

\subsubsection*{variable names}

    Scam variable/function names are case-sensitive.

\subsubsection*{set!}

    The Scam version of set! evaluates all its arguments. To set the variable
    {\it x} to 5, one would call set! this way, quoting the variable to supress
    evaluation:
    
\begin{verbatim}
        (set! 'x 5)
\end{verbatim}

    This above can be accomplished as well with the assign function which 
    does not evaluate its first argument:

\begin{verbatim}
        (assign x 5)
\end{verbatim}

    The set! function also takes an optional environment, where the predefined
    variable this always points to the current environment. Thus, the following
    three expressions are equivalent:

\begin{verbatim}
        (set! 'x 5)
        (set! 'x 5 this)
        (assign x 5)
\end{verbatim}

\subsubsection*{variadic functions}

    If the last formal parameter in a function definition is the
    symbol {\it @}, all remaining arguments not matched with 
    preceding formal parameters are bundled up into a list, with
    the variable @ set to point to that list. Here is a redefinition
    of the built-in function println that has the same semantics:

\begin{verbatim}
        (define (println @)
            (if (null? @)
                (print "\n")
                (begin (print (car @)) (apply println (cdr @)))
                )
            )
\end{verbatim}

\subsubsection*{special forms}

    There are no special forms in Scam and all functions can be
    redefined, including {\it if}, {\it and}, {\it or}, and {\it define}.

\subsubsection*{objects}

    Class and constructor are the same thing in Scam. Any function that
    returns the pre-defined variable {\it this} is considered a class
    definition and a constructor for that class. Here is an example
    {\it Node} class:

\begin{verbatim}
        (define (Node value next)
            this
            )
\end{verbatim}

    There are two methods for extracting object components, \verb!.! and
    {\it get}. The following expressions have identical semantics:

\begin{verbatim}
        (. value (Node 3 nil))
        (get 'value (Node 3 nil))
\end{verbatim}

    Like {\it set!}, {\it get} evaluates all its arguments.

    Note that environments are identified with objects in Scam.

\subsubsection*{object methods}

    Nested functions in a constructor are methods:

\begin{verbatim}
        (define (Node value next)
            (define (toString)
                (+ "Node(" (string value) "," (string next) ")")
                )
            this
            )
\end{verbatim}

    Calling object methods proceeds as expected:

\begin{verbatim}
        ((get 'toString (Node 3 nil))
\end{verbatim}


\subsubsection*{inheritance}

    Inheritance is not native to Scam, but is accomplished by 
    including the inheritance library:
    
\begin{verbatim}
        (include "inherit.lib")
\end{verbatim}

     Once included, the new function is used to perform
     inheritance:

\begin{verbatim}
        (define (A)
            (define parent nil)
            this
            )

        (define (B)
            (define parent (A))
            this
            )
    
        (define obj (new (B)))
\end{verbatim}

    Note that constructors in an inheritance hierarchy must, by
    convention, define a {\it parent} variable.

    Inheritance is similar to Java; every method is virtual.

\end{document}
