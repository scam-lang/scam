\documentclass{article}  
\usepackage{hyperlatex}
\usepackage{color}

\htmlcss{lusth.css}

\htmldepth{1}
\htmltitle{scam vs scheme}
\htmladdress{lusth@cs.ua.edu}
\T\setlength\parskip{6 pt}
\T\setlength\parindent{0 in}

\title{The Scam Programming Language\\
\date{Revision Date: \today}
}

\author{written by: John C. Lusth}


\begin{document}

\maketitle

\W\subsubsection{\xlink{Printable Version}{quick.pdf}}
\W\htmlrule


\section{Scam vs Scheme}

Things to note for programmers who are trying Scam:

\subsubsection*{variable names}

    Scam variable/function names are case-sensitive.

\subsubsection*{assignment}

    The Scam version of {\it set!} has an additional mode, when compared to
    Scheme;
    the function takes an optional environment,
    where the predefined
    variable {\it this} always points to the current environment.
    Thus, the following
    two expressions are equivalent:

\begin{verbatim}
        (set! x 5)
        (set! x 5 this)
\end{verbatim}

    There is a version of {\it set!} that evaluates all its arguments;
    to assign to a variable, one quotes the variable name:

\begin{verbatim}
         (set 'x 5)
\end{verbatim}

    The {\it set} function
    also takes an environment as an optional third argument.

\subsubsection*{variadic functions}

    If the last formal parameter in a function definition is the
    symbol {\it @}, all remaining arguments not matched with 
    preceding formal parameters are bundled up into a list, with
    the variable {\it @} set to point to that list. Here is a redefinition
    of the built-in function {\it println} that has the same semantics:

\begin{verbatim}
        (define (println @)
            (while (valid? @)
                (print (car @))
                (set! @ (cdr @))
                )
            (print "\n")
            )
\end{verbatim}

\subsubsection*{special forms}

    There are no special forms in Scam; all functions can be
    redefined, including {\it if}, {\it and}, {\it or}, and {\it define}.

\subsubsection*{objects}

    Class and constructor are the same thing in Scam. Any function that
    returns the pre-defined variable {\it this} is considered a class
    definition and a constructor for that class. Here is an example
    {\it Node} class:

\begin{verbatim}
        (define (Node value next)
            this
            )
\end{verbatim}

    There are two methods for extracting object components, {\it dot} and
    {\it get}. The latter two expressions have identical semantics:

\begin{verbatim}
        (define n (Node 3 nil))
        (dot n value)
        (get 'value n)
\end{verbatim}

    The {\it dot} function does not evaluate its last argument.
    Like {\it set}, {\it get} evaluates all its arguments.

    Note that environments and objects are equivalent in Scam.

\subsubsection*{object methods}

    Nested functions in a constructor are methods:

\begin{verbatim}
        (define (Node value next)
            (define (toString)
                (+ "Node(" (string value) "," (string next) ")")
                )
            this
            )
\end{verbatim}

    Calling object methods proceeds as expected:

\begin{verbatim}
        (define n (Node 3 nil))
        ((get 'toString n))
\end{verbatim}


\subsubsection*{inheritance}

    Inheritance is not native to Scam, but is accomplished by 
    including the inheritance library:
    
\begin{verbatim}
        (include "inherit.lib")
\end{verbatim}

     Once included, the {\it new} function is used to perform
     inheritance:

\begin{verbatim}
        (define (A)
            (define parent nil)
            this
            )

        (define (B)
            (define parent (A))
            this
            )
    
        (define obj (new (B)))
\end{verbatim}

    Note that constructors in an inheritance hierarchy must, by
    convention, define a {\it parent} variable.

    Inheritance is similar to Java; every method is virtual.
    Unlike Java, ancestor objects `inherit' the enclosing scope
    of the child object. In other words, if an ancestor method
    references a non-local variable, the non-local is resolved
    in the scope of the child object and, if not resolved there,
    in the child's enclosing scope.

\end{document}
